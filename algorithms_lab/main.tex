\documentclass[justified,nobib]{tufte-handout}

\setcounter{secnumdepth}{2}
\setcounter{tocdepth}{2}

% Packages
\usepackage[utf8]{inputenc}
\usepackage[T1]{fontenc}
\usepackage[activate={true,nocompatibility},final,tracking=true,kerning=true,spacing=true,factor=1100,stretch=10,shrink=10]{microtype}

\usepackage{tikz}
\usepackage{pgfplots}
\pgfplotsset{
    colormap={diverging}{
            rgb255=(215,25,28)
            rgb255=(253,174,97)
            rgb255=(255,255,191)
            rgb255=(171,221,164)
            rgb255=(43,131,186)
        },
    colormap={sequential}{
            rgb255=(255,255,204)
            rgb255=(161,218,180)
            rgb255=(65,182,196)
            rgb255=(44,127,184)
            rgb255=(37,52,148)
        },
    compat=1.16,
}

\usetikzlibrary{calc,arrows,shapes,positioning}
\usepgfplotslibrary{fillbetween}

\usepackage{xparse}
\usepackage{xcolor}
\usepackage{xspace}
\usepackage{amsmath,amsfonts,amsthm,amssymb,mathrsfs,bbm,mathtools,nicefrac,bm,centernot,bbold}
\usepackage{enumitem}
\usepackage{hyperref}
\usepackage[font+=footnotesize,justification=justified,singlelinecheck=false,labelfont=bf,labelsep=period,hypcap=false]{caption}
\usepackage[justification=c]{subcaption}
\usepackage{float}
\usepackage{longtable}
\usepackage{derivative}
\usepackage[most]{tcolorbox}
\usepackage[parfill]{parskip}
\usepackage{graphicx}
\usepackage{cleveref}
\usepackage{booktabs}
\usepackage{algpseudocode}
\usepackage{makecell}
\usepackage{mdframed}
\usepackage{thmtools}
\usepackage{natbib}

% Figure support as explained in Gilles Castel's blog post
\usepackage{import}
\usepackage{xifthen}
\usepackage{pdfpages}
\usepackage{transparent}
\newcommand{\incfig}[1]{
    \def\svgwidth{\columnwidth}
    \import{./figures/}{#1.pdf_tex}
}

\geometry{
    paperwidth=210mm,
    paperheight=297mm,
    left=42pt,
    top=72pt,
    textwidth=320pt,
    marginparsep=20pt,
    marginparwidth=180pt,
    textheight=650pt,
    footskip=40pt,
}

% Fix some stuff
% http://tex.stackexchange.com/questions/76273/multiple-pdfs-with-page-group-included-in-a-single-page-warning
\pdfsuppresswarningpagegroup=1

% Theorem environments 
\mdfsetup{skipabove=1em,skipbelow=0em, innertopmargin=10pt, innerbottommargin=10pt}

\declaretheoremstyle[headfont=\bfseries, bodyfont=\normalfont, mdframed={ nobreak }]{box_style}
\declaretheoremstyle[headfont=\bfseries, bodyfont=\normalfont, numbered=no]{nobox_style}
\declaretheoremstyle[headfont=\bfseries, bodyfont=\normalfont, mdframed={ nobreak }, numbered=no]{box_nonumber_style}

\declaretheorem[style=box_style, name=Theorem]{theorem}
\declaretheorem[sibling=theorem, style=box_style, name=Example]{example}
\declaretheorem[sibling=theorem, style=box_style, name=Definition]{definition}
\declaretheorem[sibling=theorem, style=box_style, name=Lemma]{lemma}

\declaretheorem[style=nobox_style, name=Remark]{remark}
\declaretheorem[style=nobox_style, name=Properties]{properties}
\declaretheorem[style=nobox_style, name=Observation]{observation}
\declaretheorem[style=box_nonumber_style, name=Corollary]{corollary}
\newmdenv[nobreak]{important}

% Listing environment
\newcounter{listing}
\newcommand\listingname{Listing}
\makeatletter
\newcommand\listoflistings{
    \section*{List of listings}
    %  \begin{fullwidth}%
    \@starttoc{loe}
    %  \end{fullwidth}%
}
\renewcommand\thelisting
{\@arabic\c@listing}
\def\fps@listing{tbp}
\def\ftype@listing{1}
\def\ext@listing{loe}
\def\fnum@listing{\listingname\nobreakspace\thelisting}
\newenvironment{listing}[1][htbp]
{\begin{@tufte@float}[#1]{listing}{}
        \begin{mdframed}[backgroundcolor=black!5]}
            {\end{mdframed}
    \end{@tufte@float}}
\let\l@listing\l@figure
\makeatother

% Algorithm environment
\newcounter{algorithm}
\newcommand\algorithmname{Algorithm}
\makeatletter
\newcommand\listofalgorithms{
    \section*{List of algorithms}
    \@starttoc{loe}
}
\renewcommand\thealgorithm
{\@arabic\c@algorithm}
\def\fps@algorithm{tbp}
\def\ftype@algorithm{1}
\def\ext@algorithm{loe}
\def\fnum@algorithm{\algorithmname\nobreakspace\thealgorithm}
\newenvironment{algorithm}[1][htbp]
{\begin{@tufte@float}[#1]{algorithm}{}
        \begin{mdframed}[backgroundcolor=white]}
            {\end{mdframed}
    \end{@tufte@float}}
\let\l@algorithm\l@figure
\makeatother

\algdef{SE}[REPEATN]{RepeatN}{End}[1]{\algorithmicrepeat\ #1 \textbf{times}}{\algorithmicend}

\renewcommand{\qedsymbol}{$\blacksquare$}

\usepackage{authoraftertitle}

\renewcommand*{\maketitle}{%
    \begin{titlepage}
        \setlength{\parindent}{0pt}%
        \setlength{\parskip}{4pt}%
        {\LARGE\itshape\MyTitle\par}
        {\Large\itshape\MyAuthor\par}
        {\Large\itshape\today\par}

        \vfill
        {Note that these are not the official lecture notes of the taught course, but only notes
            written by me. As such, it might contain mistakes. The source code can be found at
            \url{github.com/cristianpjensen/eth-cs-notes}. If you find a mistake, please create an
            issue or open a pull request.}
        \vspace{4cm}
    \end{titlepage}
}

% Obscure delimiters
\DeclarePairedDelimiter{\floor}{\lfloor}{\rfloor}
\DeclarePairedDelimiter{\ceil}{\lceil}{\rceil}
\DeclarePairedDelimiter{\ip}{\langle}{\rangle}
\DeclarePairedDelimiter{\abs}{\lvert}{\rvert}
\DeclarePairedDelimiter{\norm}{\lVert}{\rVert}

% Better left and right
\newcommand{\lft}{\mathopen{}\left}
\newcommand{\rgt}{\aftergroup\mathclose\aftergroup{\aftergroup}\right}

% Common sets
\newcommand{\R}{\mathbb{R}}
\newcommand{\Q}{\mathbb{Q}}
\newcommand{\N}{\mathbb{N}}
\newcommand{\Z}{\mathbb{Z}}

% Expected value and variance
\newcommand{\E}{\mathbb{E}}
\newcommand{\Var}{\mathrm{Var}}

% Input and output space
\newcommand{\X}{\mathcal{X}}
\newcommand{\Y}{\mathcal{Y}}

% Vector, matrix, and tensor
\renewcommand{\vec}[1]{\bm{#1}}
\newcommand{\mat}[1]{\bm{#1}}
\newcommand{\tens}[1]{\bm{\mathsf{#1}}}

% Big-O notation
\newcommand{\bigo}[1]{\mathcal{O}\lft(#1\rgt)}

% Common functions
\newcommand{\transpose}[1]{#1^\top}
\newcommand{\inv}[1]{#1^{-1}}
\renewcommand{\det}[1]{\mathrm{det}\lft(#1\rgt)}
\newcommand{\tr}[1]{\mathrm{tr}\lft(#1\rgt)}
\newcommand{\diag}[1]{\mathrm{diag}\lft(#1\rgt)}

% Distributions
\newcommand{\sample}{\sim}
\newcommand{\sampleiid}{\overset{\mathrm{iid}}{\sim}}
\NewDocumentCommand{\Gauss}{O{0}O{1}o}{\mathcal{N}\lft(#1,#2 \IfValueT{#3}{\mid #3}\rgt)}
\newcommand{\MultiGauss}{\Gauss[\vec{0}][\mat{I}]}
\NewDocumentCommand{\Unif}{m}{\mathrm{Unif}\parentheses*{#1}}

% Math comment in smaller font size
\newcommand{\mathcomment}[1]{\mathrm{\footnotesize #1}}

% Notation environment if necessary
\newenvironment{notation}
  {\begin{fullwidth}
   \begin{longtable}{ll} \toprule
    \multicolumn{2}{c}{\centering \smallcaps{Summary of notation}} \\
    \midrule
    \vspace{0.5ex} \\
  }
  {\vspace{0.5ex} \\
   \bottomrule
   \end{longtable}
   \end{fullwidth}
  }

% Add a period to the end of an abbreviation unless there's one
% already, then \xspace
\makeatletter
\DeclareRobustCommand\onedot{\futurelet\@let@token\@onedot}
\def\@onedot{\ifx\@let@token.\else.\null\fi\xspace}

\def\eg{\emph{e.g}\onedot} \def\Eg{\emph{E.g}\onedot}
\def\ie{\emph{i.e}\onedot} \def\Ie{\emph{I.e}\onedot}
\def\cf{\emph{c.f}\onedot} \def\Cf{\emph{C.f}\onedot}
\def\etc{\emph{etc}\onedot} \def\vs{\emph{vs}\onedot}
\def\wrt{w.r.t\onedot} \def\dof{d.o.f\onedot}
\makeatother

% Argument optima
\DeclareMathOperator*{\argmax}{argmax} % no space, limits underneath in displays
\DeclareMathOperator*{\argmin}{argmin} % no space, limits underneath in displays


\lstdefinestyle{mystyle}{
    basicstyle=\ttfamily,
    breaklines=true,
    language=c++,
}

\lstset{style=mystyle}

\title{Algorithms Lab}
\author{Cristian Perez Jensen}

\begin{document}

\pagenumbering{roman}
\maketitle
\newpage
\microtypesetup{protrusion=false}
\tableofcontents
\microtypesetup{protrusion=true}
\newpage
\section*{List of symbols}

\vspace{10mm}

\begin{abbrv}
    \item[$\doteq$] Equality by definition
    \item[$\R$] Set of real numbers
    \item[$f: A \to B$] Function $f$ that maps elements of set $A$ to elements of set $B$

    \item

    \item[$\vec{v} \in \R^n$] $n$-dimensional vector
    \item[$\mat{M} \in \R^{m\times n}$] $m \times n$ matrix
    \item[$\transpose{\mat{M}}$] Transpose of matrix $\mat{M}$
    \item[$\inv{\mat{M}}$] Inverse of matrix $\mat{M}$
    \item[$\det{\mat{M}}$] Determinant of $\mat{M}$

    \item

    \item[$\odv*{f(x)}{x}$] Ordinary derivative of $f(x)$ \wrt $x$ at point $x\in\R$
    \item[$\pdv*{f(\vec{x})}{x}$] Partial derivative of $f(\vec{x})$ \wrt $x$ at point $\vec{x}\in\R^n$
    \item[$\grad{f(\vec{x})}{} \in \R^n$] Gradient of $f: \R^n \to \R$ at point $\vec{x}\in\R^n$
    \item[$\jacob{f(\vec{x})}{} \in \R^{n\times m}$] Jacobian of $f: \R^n \to \R^m$ at point $\vec{x}\in\R^n$
    \item[$\hess{f(\vec{x})}{} \in \R^{n\times n}$] Hessian of $f: \R^n \to \R$ at point $\vec{x}\in\R^n$
\end{abbrv}

\newpage\cleardoublepage\pagenumbering{arabic}

\section{Overview}

In the Algorithms Lab course, tasks entail the following,
\begin{enumerate}
    \item Find an appropriate model for the problem;
    \item Design a suitable algorithm to solve it \textit{efficiently};
    \item Implement (in C++) and test the algorithm on the given data.
\end{enumerate}
In general, problems should take 2 hours to solve.

Problems are formatted such that they consist of a story, a precise definition of the input and
output, and then a point distribution. Make sure to read the point distribution, because they often
contain simplified versions of the task, which serve as a \textit{roadmap} toward an efficient
algorithm to solve the problem.

\section{Binary search}

\marginnote{Before implementing binary search, first see whether a simple
    for-loop would suffice. This saves time.}

Whenever there is a monotonic relationship between values, \ie if $x_1 \geq x_2$, then always
$f(x_1) \geq f(x_2)$ or always $f(x_2) \leq f(x_1)$, binary search can find a given element in
$\bigo{\log n}$.\sidenote{Instead of $\bigo{n}$ if you would iterate over all possible inputs.}

\begin{algorithm}
    \caption{Binary search algorithm for finding index containing value $T$ in
        a sorted array $\vec{a}$ of length $n$.}
    \begin{algorithmic}[1]
        \Function{BinarySearch}{$\vec{a},n,T$}
        \State{$\ell \gets 0$}
        \State{$r \gets n-1$}
        \While{$\ell \leq r$}
        \State{$m \gets \floor*{\frac{\ell+r}{2}}$}
        \If{$a_m < T$}
        \State{$\ell \gets m+1$} \Comment{Search right side of $m$}
        \ElsIf{$a_m > T$}
        \State{$r \gets m-1$} \Comment{Search left side of $m$}
        \Else
        \State{\Return $m$} \Comment{$a_m=T$}
        \EndIf
        \EndWhile
        \State{\Return $-1$}
        \EndFunction
    \end{algorithmic}
\end{algorithm}

\begin{algorithm}
    \caption{Binary search can also be used for finding the minimum input value
        for which a constraint is still satisfied.}
    \begin{algorithmic}[1]
        \Function{BinarySearchLeftMost}{$\vec{a},n,T$}
        \State{$\ell \gets 0$}
        \State{$r \gets n$}
        \While{$\ell < r$}
        \State{$m \gets \floor*{\frac{\ell+r}{2}}$}
        \If{$a_m < T$}
        \State{$\ell \gets m+1$} \Comment{Search right side of $m$}
        \Else
        \State{$r \gets m$} \Comment{Search left side of $m$}
        \EndIf
        \EndWhile
        \State{\Return $\ell$}
        \EndFunction
    \end{algorithmic}
\end{algorithm}

\section{Dynamic programming}

\marginnote{At worst, the table must be filled out fully, thus the worst-time
    complexity of a dynamic programming algorithm is $\bigo{st}$, where $s$ is the
    size of the table, and $t$ the time complexity of computing a single
    subsolution.}

Dynamic programming solves a problem by reducing it to smaller subproblems of the same type. These
subproblems are described by a \textit{state} $s \in S$. To get an answer to the problem, the same
subproblems may need to be solved many times. Thus, in dynamic programming, the solutions to all
subproblems are stored, in a table that maps state to solution value, to be used
later.\sidenote{Dynamic programming is essentially recursion with memoization.}

To apply dynamic programming, we need to find three things,
\begin{itemize}
    \item A state space $S$ that describes subproblems as succinctly as possible;
    \item A recurrence relationship $r(s) = f(s, r(s_1), \ldots, r(s_n))$, where $\{ s_1, \ldots, s_n \}
              \subseteq S$ are the necessary subproblem states to compute $s$;
    \item Base cases $B \subseteq S$ with constant values that do not need to be computed, \ie $r(b)=c(b)$
          for all $b\in B$, where $c(b)$ maps base cases to their solution value.
\end{itemize}
Once we have those, we can easily construct a polynomial algorithm.

\marginnote{If possible, always use \texttt{std::vector} if the state space can
    be described by integers, since it has constant insert/access time complexity.
    Remember that \texttt{std::map} has $\bigo{\log n}$ insert/find/access time
    complexity.}

Dynamic programming differentiates between a top-down and a bottom-up approach. In the top-down
approach, the values are computed recursively, each time checking the table for a precomputed value
and base case condition before computing the state solution. In bottom-up, the smaller problems are
explicitly computed before computing the next state. In general, the top-down approach is easier to
implement. However, the bottom-up approach is more memory efficient, because it does not need to
keep track of function calls on the stack. Also, using the bottom-up approach, it is easier to
reason about the time complexity.

\section{Boost Graph Library}

The Boost Graph Library (BGL) is a library of common graph algorithms. In BGL, graphs are
represented as adjacency lists, \ie a vector of vectors.

\begin{listing}
    \caption{Basic type for a weighted graph.}
    \begin{lstlisting}
#include <boost/graph/adjacency_list.hpp>

typedef boost::adjacency_list<
  boost::vecS,
  boost::vecS,
  boost::undirectedS,  // Or boost::directedS
  boost::no_property,  // Vertex property
  boost::property<boost::edge_weight_t, int>
> graph;
  \end{lstlisting}
\end{listing}

A list of BGL functions can be found in \Cref{tab:graph-algos}. See the tutorial slides for how to
use them. Also, the BGL documentation contains detailed descriptions of how these algorithms work.

In the exam-like problems, Dijkstra's is the only function actually used commonly. Dijkstra's
algorithm computes the distance, in a weighted graph, from a source node to all other nodes. A
modified Kruskal's algorithm is often implemented with proximity structures to find whether two
vertices are reachable under some constraints.

\Cref{tab:graph-algos} can also be found in the tutorial slides.\sidenote{Also,
    the table contains links to the documentation of the functions, which contains
    good descriptions of the algorithms.} So, in the exam, if there is no way to
solve a graph problem with the other methods, then take a look at this table and
figure out how to potentially apply these algorithms.

\begin{table}[h]
    \centering
    \caption{Common graph algorithms that appear throughout the course.}
    \label{tab:graph-algos}
    \begin{tabular}{ll} \toprule
        Algorithm                                               & Runtime                     \\
        \midrule
        \texttt{boost::breadth\_first\_search}                  & $\bigo{n+m}$                \\
        \texttt{boost::depth\_first\_search}                    & $\bigo{n+m}$                \\
        \texttt{boost::dijkstra\_shortest\_path}                & $\bigo{n\log n + m}$        \\
        \texttt{boost::kruskal\_minimum\_spanning\_tree}        & $\bigo{m\log m}$            \\
        \texttt{boost::edmonds\_maximum\_cardinality\_matching} & $\bigo{mn\cdot\alpha(m,n)}$ \\
        \texttt{boost::strong\_components}                      & $\bigo{n+m}$                \\
        \texttt{boost::connected\_components}                   & $\bigo{n+m}$                \\
        \texttt{boost::biconnected\_components}                 & $\bigo{n+m}$                \\
        \texttt{boost::articulation\_points}                    & $\bigo{n+m}$                \\
        \texttt{boost::is\_bipartite}                           & $\bigo{n+m}$                \\
        \bottomrule                                                                           \\
    \end{tabular}
\end{table}

\section{Computational Geometry Algorithms Library}

\paragraph{Infinite precision and range.}

CGAL is a library that provides infinite precision and range types.\sidenote{As opposed to the
    default C++ types in \Cref{tab:types}.} However, large numbers also have a higher computational
cost in CGAL, so only use exactly as much algebra as needed in problems that require infinite
precision.

\begin{table}[h]
    \centering
    \caption{C++ types with their representational range.}
    \label{tab:types}
    \begin{tabular}{lccc} \toprule
        Type            & Bits (specific to this course) & Minimum   & Maximum    \\
        \midrule
        \texttt{int}    & $32$                           & $-2^{31}$ & $2^{31}-1$ \\
        \texttt{long}   & $64$                           & $-2^{63}$ & $2^{63}-1$ \\
        \texttt{double} & $64$                           & $-2^{53}$ & $2^{53}-1$ \\
        \bottomrule                                                               \\
    \end{tabular}
\end{table}

In general, the following guidelines should be followed to avoid unnecessary computational cost,
\begin{enumerate}
    \item Avoid square roots where possible, which is often possible, because the function is monotonic,
          \ie,\sidenote{This is especially useful when working with distances, \[ d((x_1,y_1), (x_2,y_2)) = \sqrt{(x_1 - x_2)^2 + (y_1 -
                      y_2)^2} ,\] because we only need squared distances if we only need them for comparison.} \[
              \forall x,y \geq 0 : \sqrt{x} < \sqrt{y} \iff x < y
              ;\]
    \item Avoid divisions, which is often possible in comparisons, \ie,\sidenote[][1em]{To keep the code
              clean, you can also use the \texttt{CGAL::Gmpq} type, which represents divisions by the numerator
              and denominator, and does this under the hood.} \[
              \forall x,y > 0 : \frac{a}{b} < \frac{c}{d} \iff ad < bc
              ;\]
    \item Estimate to check if loss of precision may occur. \Ie, first check whether the values will fit
          within one of the default C++ types (\Cref{tab:types}). If we need to multiply two values with $a$
          and $b$ bits, respectively, we will need a type with $a+b$ bits. If we need to add two values, we
          need $\max\{a,b\}+1$ bits.
\end{enumerate}

\paragraph{Geometretic computing.}

CGAL also provides predicates and constructions for geometry. The library provides three kernels,
shown in \Cref{tab:cgal-kernels}.

\begin{table*}[h]
    \centering
    \caption{CGAL kernels, ordered by increasing computational cost.}
    \label{tab:cgal-kernels}
    \begin{tabular}{ll} \toprule
        Kernel                                                                       & Feature                                         \\
        \midrule
        \texttt{CGAL::Exact\_predicates\_inexact\_constructions\_kernel}             & \makecell[l]{Constructions use \texttt{double}, \\so not exact} \\
        \texttt{CGAL::Exact\_predicates\_exact\_constructions\_kernel}               & \makecell[l]{Exact constructions,               \\supporting $+,\times,\div$.} \\
        \texttt{CGAL::Exact\_predicates\_inexact\_constructions\_kernel\_with\_sqrt} & \makecell[l]{Exact constructions,               \\supporting $+,\times,\div,\sqrt{\cdot}$.} \\
        \bottomrule                                                                                                                    \\
    \end{tabular}
\end{table*}

CGAL has many different geometries that it can represent, \eg, \texttt{K::Point\_2},
\texttt{K::Line\_2}, \texttt{K::Ray\_2}, \texttt{K::Segment\_2}. However, they are only necessary
if constructions are absolutely necessary.\sidenote[][2em]{An edge case is \texttt{K::Point\_2},
    which does not require constructions and are often useful.} Usually, there is a more efficient
method that requires no constructions. See the CGAL documentation for provided predicates and
constructions.

If you need to take the floor of an infinite precision type, such as \texttt{K::FT}, use the
function in \Cref{lst:floor}.

\begin{listing}
    \caption{Floor of infinite precision type. If the ceiling must be computed,
        the following identity can be used, \[ \lceil x \rceil = -\lfloor -x \rfloor .\]}
    \label{lst:floor}
    \begin{lstlisting}
typedef CGAL::Exact_predicates_inexact_constructions_kernel K;

double floor_to_double(const K::FT& x)
{
  double a = std::floor(CGAL::to_double(x));
  while (a > x) a -= 1;
  while (a+1 <= x) a += 1;
  return a;
}
  \end{lstlisting}
\end{listing}

\section{Greedy algorithms}

\marginnote{The greedy approach rarely yields optimal solutions, but it is easy
    to convince yourself that the greedy approach ``works``. This is why the proof
    step is important, but, in this course, there is no time to construct a proof.
    Thus, first exhaust all other options before resorting to a greedy approach.}

A greedy algorithm can be applied if \textit{locally optimal choices} result in a \textit{globally
    optimal solution}. Usually, these are tasks that require the construction of a set that is in some
sense globally optimal. In general, a greedy approach has the following steps,
\begin{enumerate}
    \item \textit{Greedy choice}: Given already chosen elements $c_1,\ldots,
              c_{k-1}$, decide how to choose $c_k$, based on some local optimality
          criterion;
    \item \textit{Proof}: Prove that the elements obtained in this way result in
          a globally optimal set;\sidenote{We can prove that a greedy solution works
              using an \textit{exchange argument} or a \textit{staying ahead} argument.
              We can disprove one via a counterexample.}
    \item \textit{Implementation}: Implement the greedy choice to be as efficient
          as possible.
\end{enumerate}
The hard part lies in the first step, where we need to figure out how to pick
the next element of the set, given already chosen elements.

\section{Split and list}

\marginnote{Only use split and list of $n\leq 50$, where $n$ is the amount of
    elements in the set. It is often necessary for $\mathcal{NP}$-hard problems to
    get full points.}

For some problems, we need to consider every possible ``configuration`` to solve it, resulting in
$\bigo{c \cdot 2^n}$ time complexity, which is okay for $n \approx 25$ in this course, where
$\bigo{c}$ is the runtime of checking whether a configuration satisfies some condition. In some
cases, using split and list, we can get it down to $\mathcal{O}(c \cdot 2^{\nicefrac{n}{2}})$,
which is okay for $n \approx 50$ in this course.

Split and list can be used if the elements $S$ can be split into $S_1$ and $S_2$ such that the
results of the configurations of $S_1$ and $S_2$ make up the result of a full configuration of $S$.

We iterate over all configurations of $S_1$ and $S_2$ and compute their results, stored in $L_1$
and $L_2$, respectively. Sort $L_2$. Then, for each $k_1\in L_1$, check if there is a $k_2 \in L_2$
(using binary search) such that their combination make up the target.

\section{Maximum flow}

\marginnote{\textit{Tip.} The max flow from $u$ to $v$ is the same as from $v$
    to $u$ in the reversed graph.}

In maximum flow problems, we have a graph where the edges are given flow capacity, which is how
much can flow through an edge. Then, the question becomes how much flow can go from a source vertex
to a sink vertex in such a graph. Using BGL, we compute the maximum flow of a graph with the
push-relabel algorithm ($\mathcal{O}(V^3)$).\sidenote{You also need to add residual connections to
    all edges, but this is done with the \texttt{edge\_adder} struct that is given in the maximum flow
    example of the course documentation.}

Common techniques that are very useful in such problems are the following,
\begin{itemize}
    \item \textit{Multiple sources/sinks}: If you need multiple sources (or
          sinks), you can simply add a supernode that has infinite capacity to all
          the sources (or sinks);
    \item \textit{Vertex capacities}: If vertices should have a certain capacity
          that is allowed to flow through it, use two vertices to represent it. The
          input vertex should take all inputs of the vertex and the output vertex
          should take all outputs. Then, add an edge from input to output with the
          vertex capacity;
    \item \textit{Undirected edges}: If you need an undirected edge between $a$
          and $b$ with capacity $c$, just add directed edges from $a$ to $b$ and $b$
          to $a$, both with capacity $c$;\sidenote{This works because if flow goes
              both ways, they might as well both stay on their original side (which the
              algorithm will do). So, the maximum flow through this undirected edge is
              achieved if $c$ goes from $a$ to $b$ and $0$ goes from $b$ to $a$ (or vice
              versa).}
    \item \textit{Minimum edge constraint}: If we need the following constraint
          on an edge $e=(u,v)$, \[
              c_{\min}(e) \leq f(e) \leq c_{\max}(e),
          \]
          where $f(e)$ is the flow through edge $e$, we need to adjust the demand, supply, and capacity as
          follows,
          \begin{align*}
              d_u  & \gets d_u + c_{\min}       \\
              s_v  & \gets s_v + c_{\min}       \\
              c(e) & \gets c_{\max} - c_{\min},
          \end{align*}
          where $d_u$ is the demand of $u$, \ie, the amount of capacity to the
          target, $s_v$ is the supply of $v$, \ie, the amount of capacity from the
          source, and $c(e)$ is the capacity of $e$.
\end{itemize}

\subsection{Minimum cut}

The maximum flow between two vertices can also be seen as the bottleneck between them. Thus, we can
also see this as the minimum cut, where we need to cut/block the bottleneck to disconnect the two
vertices. The actual vertices that are on the two sides of the cut can then be found by
breadth-first search on the edges with non-zero residual capacity (see slides).

\subsection{Bipartite matching}

In a bipartite matching problem, we want to compute to take as many non-adjacent edges as possible,
\ie, make as many assignments as possible. This can also be computed by maximum flow. First, we
need to construct the bipartite graph. Assign the source to one side with 1 flow, and the target to
the other side with 1 flow. Then, connect all pairs with 1 flow. The maximum flow is then the
maximum amount of matchings that can be made.

\begin{theorem}[K\"onig]
    \label{thm:konig}

    In a bipartite graph, the number of edges in a maximum matching is equal to the number of vertices
    in a minimum vertex cover.
\end{theorem}

The maximum independent set $I\subseteq V$ is the largest set of vertices, such that none of them
are connected by an edge, \[
    \centernot \exists u,v\in I : (u,v) \in E.
\]
The minimum vertex cover $C \subseteq V$ is the smallest set of vertices, such that every edge is
connected to one of the vertices in this set, \[
    \forall (u,v) \in E : u \in C \lor v \in C.
\]
Using \Cref{thm:konig}, we can compute the size of the minimum vertex cover $|C|$ of a bipartite
graph by computing the maximum flow. Then, we can compute the size of the maximum independent set
by $|I| = |V| - |C|$.

\subsection{Minimum cost maximum flow}

\marginnote{Only use minimum cost maximum flow if number $n \leq 1000$, where
    $n$ is the amount of vertices. If there are negative costs, only use it if $n
        \leq 600$.}

We can also associate cost with flow on the edges. In minimum cost maximum flow problems, we then
want to first maximize the flow, and then, as a second priority, compute the minimum cost. \Ie, we
want to find the cheapest among all maximum matchings.

We could also maximize the cost by making them negative. However, the non-negative solver is much
faster than the one that allows negative costs. Thus, if we need a negative cost, we should compute
some upper-bound $U$, and give $B-c$ cost, such that the costs become positive. Then, afterward, we
need to remove the added $B$ in some way. This can usually be done as a function of the maximum
flow, since that is how many times $B$ was added to the cost.

\section{Proximity structures}

In this course, we often have geometry problems with many points on an $x,y$ coordinate system. In
these cases, we also want to be efficient. In a problem, we might have points with some radius,
where we need to find the minimum radius such that some condition is satisfied. Or, we might need
to find the maximum distance a point can remain from all points when moving out of the convex hull.
Or, we might need to find the closest point for $m$ points in a large list of $n$ points. In this
case, naively computing the closest point would result in $\bigo{mn}$ complexity, but we can do
better. If we precompute a triangulation in $\bigo{n \log n}$, we can find closest points in
$\bigo{\log n}$. This results in a $\bigo{(n+m) \log n}$ time complexity, which is much better than
the naive version.

\subsection{Delaunay triangulation}

\begin{listing}
    \caption{Delaunay triangulation in C++. The exact constructions kernel is
        necessary if access to the Voronoi diagram is needed.}
    \label{lst:dt}

    \begin{lstlisting}
typedef CGAL::Exact_predicates_inexact_constructions_kernel K;
typedef CGAL::Delaunay_triangulation_2<K> Triangulation;

std::vector<K::Point_2> points(n);
// Read in points...

Triangulation t;
t.insert(points.begin(), points.end());
  \end{lstlisting}
\end{listing}

Delaunay triangulation for a set of discrete points is a triangulation such that no point is inside
the circumcircle of any triangle in the triangulation. Let a disk be of maximal radius if it passes
through three points, its center is inside the convex hull of the points, and the disk does not
contain any other points. The maximal empty disks of the graph make up a Delaunay triangulation.

Delaunay triangulation has the following properties,
\begin{itemize}
    \item It contains the Euclidean minimum spanning tree;
    \item Each point has an edge to all closest other points;
    \item It can be constructed efficiently in $\bigo{n \log n}$.
\end{itemize}

Furthermore, Delaunay triangulation is the straight-line dual of the Voronoi diagram. The Voronoi
diagram for a set of points $\mathcal{P}$ partitions the plane into regions for which the closest
point from $\mathcal{P}$ is the same. If you move along these edges, you will always be as far away
as possible from the points $\mathcal{P}$.

The dual of a face, can be found with \verb|t.dual(f)|. However, this is inefficient, because it
uses constructions. Often, we do not need to explicitly compute the dual vertex. We could often
simply use \verb|t.nearest_vertex(p)|, which only uses predicates. In the exam-like problems, we
only needed the dual vertex for a problem where we needed to compute the maximal radius of a face.
We can compute this by computing the distance between the dual and any vertex of the face,
\verb|CGAL::squared_distance(t.dual(f), f.vertex(0))|.

\begin{listing}
    \caption{If we need the graph where the faces are vertices and the minimum
        radius that can go between faces as edge weights, we can use this snippet.
        This is needed when doing motion planning between points, where we want to
        know how to move a disk without colliding with any points.}
    \begin{lstlisting}
// Let 0 be the infinite vertex
auto f = t.incident_faces(t.infinite_vertex());
do {
  f->info() = 0;
} while (++f != t.incident_faces(t.infinite_vertex()));

// Give each face a unique index
size_t num_faces = 1;
for (auto f = t.finite_faces_begin(); f != t.finite_faces_end(); ++f)
  f->info() = num_faces++;

std::vector<std::vector<std::pair<size_t, K::FT>>> adj(num_faces);

// Iterate over faces
for (
  auto f = t.finite_faces_begin();
  f != t.finite_faces_end();
  ++f
)
{
  size_t index = f->info();

  for (int i = 0; i < 3; i++)
  {
    size_t n_index = f->neighbor(i)->info();
    K::FT length = t.segment(f, i).squared_length();

    // Construct graph outside the DT
    adj[index].push_back({ n_index, length });

    // Infinite vertices
    if (n_index == 0)
      adj[0].push_back({ index, length });
  }
}

// Locate face in which point `p` is by `t.locate(p)`
  \end{lstlisting}
\end{listing}

\section{Linear programming}

\marginnote{Only use linear programming if $\min \{ n,m \} \leq 200$, where $n$
    is the amount of variables, and $m$ the amount of constraints.}

Linear programming is a linear optimization problem subject to $m$ linear constraints on $n$
variables. In general, a linear program looks like the following,
\begin{align*}
    \text{minimize} \;\;   & \transpose{\vec{c}} \vec{x} + c_0        \\
    \text{subject to} \;\; & \mat{a}_1 \vec{x} \leq b_1               \\
                           & \vdotswithin{\mat{a}_1 \vec{x} \leq b_1} \\
                           & \mat{a}_m \vec{x} \leq b_m.
\end{align*}
Note that we can also form constraints of the form $\mat{a}_j \vec{x} \geq b_j$
by adding the following constraint, \[
    -\mat{a}_j \vec{x} \leq -b_j.
\]
Furthermore, we can also specify a maximization objective by minimizing the following, \[
    -(\transpose{\vec{c}} \vec{x} + c_0),
\]
and then flipping the sign after solving the problem.

Geometrically, a linear program defines an $n$-dimensional convex polyhedron with $m$ faces. The
optimum objective value is found at one of the vertices of this polyhedron. There are an
exponential amount of vertices, thus the worst-case time complexity is exponential in $n$ and $m$,
but for small $\min \{ n,m \}$, the complexity is $\bigo{\max\{ n,m \}}$.

The following are tips when working with linear programming,
\begin{itemize}
    \item If you want to maximize a minimum value that is a function $f(x_i)$ of the linear programming
          variables $x_i$, you need to add a variable, $f_{\min}$, and the following constraints, \[
              \forall x_i : f_{\min} \leq f(x_i).
          \]
          Then maximize this variable. Ditto if you want to minimize the maximum value;

    \item If any of the constraints contain fractional coefficients, multiply the constraints such that all
          coefficients are whole numbers;

    \item The signed distance from a point $\vec{x}$ to a hyperplane $\transpose{\vec{a}}\vec{x} + b = 0$ is
          computed as follows, \[
              \frac{\transpose{\vec{a}}\vec{x} + b}{\lVert \vec{a} \rVert_2^2}.
          \]
          This is not specific to linear programming, but often appears in linear programming problems in
          this course.
\end{itemize}

\end{document}
