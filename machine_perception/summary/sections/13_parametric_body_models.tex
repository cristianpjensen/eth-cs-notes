\section{Parametric body models}

Being able to represent and track a body is interesting for many applications, such as virtual
reality and augmented reality. The easier problem is to estimate the projected pose from 2D images,
while the harder problem is to estimate 3D pose from images. For both problems, we need to model
the body and learn a feature representation for prediction.

\subsection{2D poses}

\paragraph{Body modeling.}

Body modeling entails finding a way to understand how the different parts of the body are linked.
The \textit{pictorial structure model} \citep{yang2011articulated} models the body as a graph,
where the joints are vertices and they are connected by edges. For example, in a simplified model,
the foot vertex might be connected to the knee vertex.

In a 2D image, we indicate the position of vertex $i$ by $\ell_i = [x_i, y_i]$. Then, given an
image $\tens{I}$ and a configuration $L = [\ell_1, \ell_k]$, we can define the score of that
configuration by \[
    S(\tens{I}, L) = \sum_{i \in V} \alpha_i \phi(\tens{I}, \ell_i) + \sum_{i,j \in E} \beta_{ij} \psi(\ell_i, \ell_j),
\]
where the first term measures the score of placing vertex $i$ at $\ell_i$ in image $\tens{I}$ and
the second term measures the deformation between connected vertices. The best configuration is the
one that minimizes the score.

We can further generalize this model to a mixture of non-oriented pictorial structures. Let $m_i$
be the mixture component type of vertex $i$. The mixture component expresses concepts as
orientations of a part, such as vertically vs horizontally oriented hand, front-view head vs
side-view head, or open versus closed hand. Formally, we compute the score by \[
    S(\tens{I}, L, M) = \sum_{i \in V} \alpha_i^{m_i} \phi(\tens{I}, \ell_i) + \sum_{i,j \in E} \beta_{ij}^{m_i,m_j} \psi(\ell_i, \ell_j)+ S(M),
\]
where $\alpha_i^{m_i}$ is the ``local appearance template'' for part $i$ with type $m_i$,
$\beta_{ij}^{m_i,m_j}$ expresses the likelihood of having template $m_i$ for pat $i$ and template
$m_j$ for part $j$ given the distance between $\ell_i$ and $\ell_j$, and $S(M)$ is the
co-occurrence bias defined by \[
    S(M) = \sum_{i,j\in E} b_{ij}^{m_i,m_j},
\]
where $b_{ij}^{m_i,m_j}$ is the pairwise co-occurrence prior. This allows us to model things such
that a vertical lower leg is more likely to be connected to a vertical upper leg.

\paragraph{Feature learning.}

One architecture that does feature representation learning by direct regression is
\textit{DeepPose} \citep{toshev2014deeppose}, which uses CNNs to directly compute $\ell_i$ for all
$i$. It does so in multiple stages, where in the first stage, it only gets the image as input. In
the further stages, it gets the image and the previous estimate and input and must refine the
estimate by taking only a patch around the previous estimate.

A different way of doing it is through heatmaps. \textit{Convolutional pose machines}
\citep{wei2016convolutional} predicts heatmaps for each vertex. Like DeepPose, it also uses a
refinement process with intermediate losses. The key is that the receptive field grows as stages
are applied, which allows the model to capture long-range dependencies.

\paragraph{Body modeling and feature learning.}

We can also combine the two methods by first obtaining the heatmaps and then refining the
predictions using body modeling.

\subsection{3D poses}

\paragraph{Linear blend skinning.}

Linear blend skinning (LBS) is the simplest method of transforming a rest pose into a specific
pose. It does so by transforming vertices as a weighted linear combination of global joint
transformations, \[
    \vec{t}_i' = \lft( \sum_{k} w_{ki} \mat{G}_k(\vec{\theta}, \mat{J}) \rgt) \vec{t}_i.
\]
Each bone-joint pair is given a weight $w_{k,i}$, which determines the influence of the
transformation of a bone on the joint. $\mat{G}_k(\vec{\theta})$ is the rigid bone transformation
for bone $k$, which transforms the original joint location $\vec{t}_i$, which is linearly weighted
by $w_{k,i}$. Here, $\vec{\theta}$ is the desired pose and $\mat{J}$ are the joint locations.

The problem with this approach is that it does not account for any variation in the body shape.
Also, often there are problems with unrealistic deformations, caused by the pose.

\paragraph{SMPL.}

The \textit{skinned multi-person linear} (SMPL) model \citep{loper2015smpl} solves the problems of
linear blend skinning by allowing the model to learn how to account for body shape variation and
deformations caused by the poses. First an overview of how these problems are solved is given, then
we will see how they are learned from data.

SMPL encodes human subjects by two parameters: a body shape parameter $\vec{\beta} \in \R^{10}$ and
a pose parameter $\vec{\theta} \in \R^{9K}$. The template mesh $\bar{\mat{T}}$ is the starting
pose, which transformations are relative to.\sidenote{Usually, the template mesh is a T-pose. The
    matrix $\bar{\mat{T}} \in \R^{N \times 3}$ contains the location of all vertices.} In the
following, we assume access to $\mat{B}_S(\vec{\beta})$ and $\mat{B}_P(\vec{\theta})$, which we
learn from data.

\begin{enumerate}
    \item First, translate the template mesh to the identity mesh for the specific body shape, parametrized
          by $\vec{\beta}$, \[
              \bar{\mat{T}} + \mat{B}_S(\vec{\beta}); \margintag{$\mat{B}_S(\vec{\beta})$ encodes the variation in the body shape from the mean shape.}
          \]
    \item Then, translate the identity mesh to correct for deformations caused by linear blend skinning,
          which is a function of the pose, \[
              \mat{T}_P(\vec{\beta}, \vec{\theta}) = \bar{\mat{T}} + \mat{B}_S(\vec{\beta}) + \mat{B}_P(\vec{\theta});
          \]
    \item Lastly, perform linear blend skinning on the resulting base mesh, \[
              \vec{t}_i' = \lft( \sum_{k} w_{ki} \mat{G}_k(\vec{\theta}, \mat{J}(\vec{\beta})) \rgt) (\vec{t}_i + \vec{s}_i(\vec{\beta}) + \vec{p}_i(\vec{\theta})).
          \]
\end{enumerate}

\paragraph{Learning shape variation.}

To learn the variations in body shape, we need a dataset of many body shapes. Each body shape is
represented by its joint locations. To be able to linearly represent any body shape with a small
amount of parameters, we perform PCA on this dataset, and only take the top principal components.
Each body shape can then be defined by a linear combination of these components, \[
    \mat{B}_S(\vec{\beta}; \mathcal{S}) = \sum_{n=1}^{|\vec{\beta}|} \beta_n \mat{S}_n.
\]
Specifically, the data must be relative to the template mesh joints, such that
$\mat{B}_S(\vec{\beta})$ is also relative to these joints. Then, we can use them as a measure of
variation from the template.

\paragraph{Learning pose-dependent deformations.}

Let $\mat{R}: \R^{|\vec{\theta}|} \to \R^{9K}$ be a function that maps the pose vector
$\vec{\theta}$ to a vector of concatenated relative rotation matrices.\sidenote{Each rotation
    matrix has dimensionality $3 \times 3$.} The used rig has $K=23$ joints, thus
$\mat{R}(\vec{\theta}) \in \R^{207}$. Let $\vec{\theta}^\star$ be the rest pose, then the vertex
deviations from the rest template are given by \[
    \mat{B}_P(\vec{\theta}; \mathcal{P}) = \sum_{n=1}^{9K} (\mat{R}_n(\vec{\theta}) - \mat{R}_n(\vec{\theta}^\star)) \mat{P}_n,
\]
where $\mat{P}_n \in \R^{3N}$ is a vector of vertex displacements.\sidenote{There are $N$ vertices,
    each having $x,y,z$ components.} Thus, the pose-dependent translation is a linear combination of
vertex displacements $\mathcal{P}$, dependent on the difference in pose. This matrix $\mathcal{P}$
is learned from a dataset of many poses.

\subsection{Learned gradient descent}

\begin{algorithm}[t]
    \begin{algorithmic}[1]
        \State Sample $\vec{\theta}, \vec{\beta}$ from data
        \State Sample $\mat{R}, \vec{t}, s$ uniformly from feasible range
        \State $\mat{X} \gets \mat{W} \mat{M}(\vec{\theta}, \vec{\beta})$
        \State $\vec{x} \gets s \Pi(\mat{R} \mat{X}) + \vec{t}$
        \State $\Theta_0 \gets \{ \vec{\theta}_0 = \vec{0}, \vec{\beta}_0 = \vec{0}, \mat{R}_0 = \mat{0}, \vec{t}_0 = \vec{0}, s_0 = 0 \}$
        \For {$n = 0, \ldots, N-1$}
        \State $\mat{X}_n \gets \mat{W} \mat{M}(\vec{\theta}_n, \vec{\beta}_n)$
        \State $\vec{x}_n \gets s_n \Pi(\mat{R}_n \mat{X}_n) + \vec{t}_n$
        \State $\mathcal{L}(\Theta_n) \gets L_{\mathrm{reproj}}(\vec{x}_n, \vec{x})$
        \State $\Delta \gets \mathcal{N}_{\vec{w}} \lft( \pdv{\mathcal{L}(\Theta_n)}{\Theta_n}, \Theta_n, \vec{x} \rgt)$
        \State $\Theta_{n+1} \gets \Theta_n + \Delta$
        \EndFor
    \end{algorithmic}
    \caption{Learned gradient descent training scheme. $\mathcal{N}_{\vec{w}}$ is the neural network that predicts the update rule. $\vec{x}$ is a 2D projection of the 3D pose $\mat{X}$ onto the image. $\mat{M}(\cdot, \cdot)$ computes the body mesh using SMPL. $\mat{W}$ is a matrix that maps vertices to $k$ joints of interest. The camera model is parametrized by the global rotation $\mat{R} \in \R^{3\times 3}$.}
    \label{alg:lgd}
\end{algorithm}

\textit{Learned gradient descent} (LGD) \citep{song2020human} is a method for fitting 3D human shapes
to images by combining gradient-based optimization and neural networks. It leverages a neural network
to predict the parameter update rule for each optimization iteration. This allows the algorithm to
converge in few steps. See \Cref{alg:lgd} for the full algorithm.
