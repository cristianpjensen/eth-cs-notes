\section{Optical flow}

\textit{Optical flow} is the pattern of apparent motion of objects, surfaces,
and edges in a visual scene caused by the relative motion between an observer
and the scene.\sidenote{Contrast this with \textit{motion field}, which is a
    2-dimensional image presenting the projection of the 3-dimensional motion of
    points in the scene onto the image plane.} In other words, optical flow does
not represent the actual motion of the objects in the scene (that would not be
computable from images), but rather the motion of the pixels in the image
plane. Optical flow is merely an approximation of the motion field. But, in
most cases, it is a pretty good approximation.

The problem statement of optical flow is to estimate the motion, \ie, velocity,
of each pixel, given two consecutive image frames. To solve this problem,
optical flow makes the following two assumptions,
\begin{itemize}
    \item \textit{Brightness constancy}: when a point $\vec{x}=[x, y]$ in the
          image at timepoint $t$ moves to point $\vec{x}+\delta_{\vec{x}}$ in the
          image at timepoint $t+\delta_t$, its luminance does not change, \[
              \tens{I}[x+\delta_x,y+\delta_y,t+\delta_t] = \tens{I}[x,y,t];
          \]
    \item \textit{Small motion}: the pixels do not move far between image frames.
          The implication of this is that we can use the Taylor expansion for a very
          good approximation around the point we care about.
\end{itemize}

Now, we can derive how to determine the velocity $\vec{v}=[u,v]$ at point
$\vec{x}=[x,y]$ as follows,
\begin{align*}
    \tens{I}[x,y,t] & = \tens{I}[x+\delta_x,y+\delta_y,t+\delta_t] \margintag{Brightness constancy}                                                  \\
                    & = \tens{I}[x,y,t] + \tens{I}_x\delta_x + \tens{I}_y\delta_y + \tens{I}_t\delta_t \margintag{First-order Taylor approximation},
\end{align*}
where $\tens{I}_x,\tens{I}_y,\tens{I}_t$ are the derivatives at the implied
location $[x,y,t]$. From this, we can derive the \textit{brightness constancy
    equation},
\begin{align*}
    \tens{I}[x,y,t] + \tens{I}_x\delta_x + \tens{I}_y\delta_y + \tens{I}_t\delta_t         & = \tens{I}[x,y,t] \\
    \tens{I}_x\delta_x + \tens{I}_y\delta_y + \tens{I}_t\delta_t                           & = 0               \\
    \tens{I}_x\frac{\delta_x}{\delta_t} + \tens{I}_y\frac{\delta_y}{\delta_t} + \tens{I}_t & = 0               \\
    \tens{I}_x u + \tens{I}_y v                                                            & = -\tens{I}_t.
\end{align*}
However, it cannot be solved, because we have two unknowns, $u$ and $v$, with
one equation per pixel. This is called the \textit{aperture problem}. The way
this is dealt with is the difference between the Lucas-Kanade and Horn-Schunck
algorithms, \ie, their difference is how they add more constraints to make the
problem tractable.

\subsection{Lucas-Kanade}

Lucas-Kanade applies a local method that assumes that pixels in the same
area, \ie, a patch around the pixel, have the same velocity. This gives us an
additional constraint for every pixel in the area, which we can solve by
least-squares estimation,
\begin{align*}
    \begin{bmatrix}
        \tens{I}_x[x_1,y_1,t] & \tens{I}_y[x_1,y_1,t] \\
        \vdots                & \vdots                \\
        \tens{I}_x[x_n,y_n,t] & \tens{I}_y[x_n,y_n,t]
    \end{bmatrix}
    \begin{bmatrix}
        u \\ v
    \end{bmatrix}
                                      & =
    -\begin{bmatrix}
         \tens{I}_t[x_1,y_1,t] \\
         \vdots                \\
         \tens{I}_t[x_n,y_n,t]
     \end{bmatrix}                                            \\
    \mat{A}\vec{v}                    & = -\vec{b}                    \\
    \intertext{This is equivalent to solving the following by least-squares
        estimation,}
    \transpose{\mat{A}}\mat{A}\vec{v} & = -\transpose{\mat{A}}\vec{b} \\
    \sum_{(x,y)\in P}
    \begin{bmatrix}
        \tens{I}_x^2          & \tens{I}_x \tens{I}_y \\
        \tens{I}_x \tens{I}_y & \tens{I}_y^2
    \end{bmatrix}
    \begin{bmatrix}
        u \\ v
    \end{bmatrix}
                                      & =
    - \sum_{(x,y)\in P}
    \begin{bmatrix}
        \tens{I}_x \tens{I}_t \\
        \tens{I}_y \tens{I}_t \\
    \end{bmatrix}
\end{align*}

This is the same matrix as the structure tensor we saw in the Harris corner
detector. Thus, corners are good places to compute flow. If there are no
corners, and \eg only an edge, then we have the aperture problem.

\subsection{Horn-Schunck}

Horn-Schunck applies a global method that assumes optical flow fields to be
smooth. It seeks to minimize the following energy function,\sidenote{In the
    discrete case that we care about, the integrals are replaced by sums.} \[
    E = \int\int \underbrace{(\tens{I}_x u_{xy} + \tens{I}_y v_{xy} + \tens{I}_t)^2}_{\text{brightness change penalty}} + \underbrace{\lambda (\lVert \bm{\nabla} u_{xy} \rVert^2 + \lVert \bm{\nabla} v_{xy} \rVert^2)}_{\text{flow change penalty}} dxdy,
\]
where $u_{xy}$ and $v_{xy}$ are the velocities at $[x,y]$ forming a flow field.
The first term minimizes the brightness change, which enforces the brightness
constancy assumption. The second term minimizes the change in flow, which
enforces the smooth flow field assumption. In the discrete case, this can be
seen as minimizing the difference in flow between neighbors, resulting in
smoother flow. A larger $\lambda$ leads to a smoother flow field.

The derivatives of this function are the following,
\begin{align*}
    \pdv{E}{u_{kl}} & = 2(u_{kl} - \bar{u}_{kl}) + 2\lambda (\tens{I}_x u_{kl} + \tens{I}_y u_{kl}+ \tens{I}_t) \tens{I}_x \margintag{$\bar{u}_{kl}$ and $\bar{v}_{kl}$ are the average flow of the 4 neighbors: $[k-1,l]$, $[k+1,l]$, $[k,l-1]$, $[k,l+1]$.} \\
    \pdv{E}{v_{kl}} & = 2(v_{kl} - \bar{v}_{kl}) + 2\lambda (\tens{I}_x u_{kl} + \tens{I}_y u_{kl}+ \tens{I}_t) \tens{I}_y.
\end{align*}
The extrema of $E$ can be found by setting the derivatives to zero and solving
for the unknowns $u$ and $v$. These form a linear system that can be solved
iteratively using update equations,
\begin{align*}
    \hat{u}_{kl} & = \bar{u}_{kl} - \frac{\tens{I}_x \bar{u}_{kl} + \tens{I}_y \bar{v}_{kl} + \tens{I}_t}{\lambda^{-1} + \tens{I}_x^2 + \tens{I}_y^2} \tens{I}_x  \\
    \hat{v}_{kl} & = \bar{v}_{kl} - \frac{\tens{I}_x \bar{u}_{kl} + \tens{I}_y \bar{v}_{kl} + \tens{I}_t}{\lambda^{-1} + \tens{I}_x^2 + \tens{I}_y^2} \tens{I}_y.
\end{align*}
These update equations are used to update the flow field iteratively until
convergence, starting from zero flow field.

This can also be used to compute flow field of frames with larger motions by
using coarse-to-fine flow estimation. This is done by first downscaling the
image until the displacement is 1 pixel. Then, compute flow of that and
upscale the image and flow field. Repeat this step until you are back at the
original image size.
