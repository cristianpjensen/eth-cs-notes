\section{The Frank-Wolfe algorithm}

\begin{marginfigure}
    \centering
    \incfig{frank-wolfe}
    \caption{Illustration of a Frank-Wolfe step.}
    \label{fig:frank-wolfe}
\end{marginfigure}

Projected gradient descent is the only algorithm we have seen that dealt with constrained
optimization problems. However, that algorithm came with the clear disadvantage that projections
can be very expensive, even when $\mathcal{X}$ is convex. The Frank-Wolfe algorithm solves
constrained optimization problems without projection steps. Instead, it makes use of a
\textit{linear minimization oracle} (LMO). For the feasible region $\mathcal{X} \subseteq
    \mathcal{R}^d$ and an arbitrary vector $\vec{g} \in R^d$,\sidenote{$\vec{g}$ can be thought of as
    an ``optimization direction''.} \[
    \mathrm{LMO}_{\mathcal{X}}(\vec{g}) \doteq \argmin_{\vec{z} \in \mathcal{X}} \transpose{\vec{g}} \vec{z}.
\]
Notice that this is the minimization of a linear function.

The Frank-Wolfe algorithm iteratively updates by calling the oracle in the direction of the
gradient,
\begin{align*}
    \vec{s}       & \doteq \mathrm{LMO}_{\mathcal{X}}(\grad{f(\vec{x}_t)}{})                              \\
    \vec{x}_{t+1} & \doteq (1- \gamma_t) \vec{x}_t + \gamma_t \vec{s}. \margintag{Move toward minimizer.}
\end{align*}

The algorithm reduces non-linear constrained optimization to linear optimization over the same set
$\mathcal{X}$; it is able to solve general non-linear constrained optimization problems by only
solving a simpler linear constrained optimization over the same set $\mathcal{X}$ in each
iteration, by calling the oracle. We solve this linear optimization problem in the direction of the
gradient, since that is the best linear approximation of $f$ at $\vec{x}_t$.

A nice property of the oracle is that if $\mathcal{X} = \mathrm{conv}(\mathcal{A})$, then
$\mathrm{LMO}_{\mathcal{X}}(\vec{x}) \in \mathcal{A}$. So, if we have a set $\mathcal{X}$ that is
the convex hull of a small number of points, such as the $\ell_1$-ball, we have an easy
optimization problem with runtime $\bigo{|\mathcal{A}|}.$

The advantages of this method are
\begin{itemize}
    \item Iterates are always feasible if $\mathcal{X}$ is convex;
    \item No projections, which are often harder to compute than linear optimization problems;
    \item Iterates always have a simple sparse representations: $\vec{x}_t$ is a convex combination of
          $\vec{x}_0$ and all $\vec{s}$ used so far.
\end{itemize}

\subsection{Linear minimization oracles}

\paragraph{LASSO.}

The LASSO problem in its standard form is given by \[
    \begin{aligned}
         & \text{minimize}   &  & \| \mat{A} \vec{x} - \vec{b} \|^2 \\
         & \text{subject to} &  & \| \vec{x} \|_1 \leq 1.
    \end{aligned}
\]
The constraint set $\mathcal{X} = \{ \vec{x} \in \R^d \mid \| \vec{x} \|_1 \leq 1 \}$ is the unit
$\ell_1$-ball. This is the convex hull of the unit basis vectors; $\mathcal{X} = \mathrm{conv}(\{
    \pm \vec{e}_1, \ldots, \pm \vec{e}_d \})$. The LMO for this set is easy to compute,
\begin{align*}
    \mathrm{LMO}_{\mathcal{X}}(\vec{g}) & = \argmin_{\vec{z}\in \mathcal{X}} \transpose{\vec{z}} \vec{g}                                 \\
                                        & = \argmin_{\vec{z} \in \{ \pm \vec{e}_1, \ldots, \pm \vec{e}_d \}} \transpose{\vec{z}} \vec{g} \\
                                        & = -\mathrm{sign}(g_i) \vec{e}_i, \quad i \doteq \argmax_{i \in [d]} |g_i|.
\end{align*}
So, we only have to identify $\vec{g}$'s largest coordinate, which is much more efficient than
projection onto an $\ell_1$-ball.

\subsection{Duality gap}

We define the duality gap of $\vec{x} \in \mathcal{X}$ as \[
    g(\vec{x}) \doteq \transpose{\grad{f(\vec{x})}{}} (\vec{x} - \vec{s}), \quad \vec{s} \doteq \mathrm{LMO}_{\mathcal{X}}(\grad{f(\vec{x})}{}).
\]
This can be interpreted as the optimality gap $\transpose{\grad{f(\vec{x})}{}} \vec{x} -
    \transpose{\grad{f(\vec{x})}{}} \vec{s}$ of the linear subproblem.

\begin{lemma}
    Suppose that the constrained minimization problem has a minimizer $\vec{x}^\star$. Let $\vec{x} \in \mathcal{X}$, then \[
        g(\vec{x}) \geq f(\vec{x}) - f(\vec{x}^\star),
    \]
    meaning that the duality gap is an upper bound for the optimality gap.
\end{lemma}

\begin{proof}
    \begin{align*}
        g(\vec{x}) & = \transpose{\grad{f(\vec{x})}{}} (\vec{x} - \vec{s})                                                                                                                                                    \\
                   & \geq \transpose{\grad{f(\vec{x})}{}} (\vec{x} - \vec{x}^\star) \margintag{$\vec{s}$ is the minimizer of $f(\vec{z}) \doteq \transpose{\grad{f(\vec{x})}{}} \vec{z}$.}                                    \\
                   & \geq f(\vec{x}) - f(\vec{x}^\star). \margintag{First-order characterization of convexity: $f(\vec{y}) \geq f(\vec{x}) + \transpose{\grad{f(\vec{x})}{}} (\vec{y} - \vec{x}), \forall \vec{x}, \vec{y}$.}
    \end{align*}
\end{proof}

Thus, we always have a computable upper bound $g(\vec{x}_t)$ on the unknown error $f(\vec{x}_t) -
    f(\vec{x}^\star)$, which is not the case in general for \textit{unconstrained} optimization.
Furthermore, at an optimal point $\vec{x}^\star$, $g(\vec{x}^\star) = 0$, which follows from the
optimality conditions for constrained convex
optimization.\sidenote{$\transpose{\grad{f(\vec{x})}{}} (\vec{x} - \vec{x}^\star) \geq 0, \forall
        \vec{x} \in \mathcal{X}$.}

\subsection{Convergence analysis}

To prove convergence, we need the following descent lemma,

\begin{lemma}
    \label{lem:fw-descent}

    For a step $\vec{x}_{t+1} \doteq \vec{x}_t + \gamma_t (\vec{s} - \vec{x}_t)$ with stepsize
    $\gamma_t \in [0,1]$, it holds that \[
        f(\vec{x}_{t+1}) \leq f(\vec{x}_t) - \gamma_t g(\vec{x}_t) + \gamma_t^2 \frac{L}{2} \| \vec{s} - \vec{x}_t \|^2,
    \]
    where $\vec{s} = \mathrm{LMO}_{\mathcal{X}}(\grad{f(\vec{x}_t)}{})$.
\end{lemma}

\begin{proof}
    \begin{align*}
        f(\vec{x}_{t+1}) & = f(\vec{x}_t + \gamma_t(\vec{s} - \vec{x}_t))                                                                                                                                                                                                                                               \\
                         & \leq f(\vec{x}_t) + \transpose{\grad{f(\vec{x}_t)}{}} \gamma_t (\vec{s} - \vec{x}_t) + \gamma_t^2 \frac{L}{2} \| \vec{s} - \vec{x}_t \|^2 \margintag{Smoothness: $f(\vec{y}) \leq f(\vec{x}) + \transpose{\grad{f(\vec{x})}{}}(\vec{y} - \vec{x}) + \frac{L}{2} \| \vec{x} - \vec{y} \|^2$.} \\
                         & = f(\vec{x}_t) - \gamma_t g(\vec{x}_t) + \gamma_t^2 \frac{L}{2} \| \vec{s} - \vec{x}_t \|^2. \margintag{Duality gap: $g(\vec{x}) \doteq \transpose{\grad{f(\vec{x})}{}} (\vec{x} - \vec{s})$.}
    \end{align*}
\end{proof}

Now, we can prove the main convergence theorem of the Frank-Wolfe algorithm.

\begin{theorem}
    Consider the constrained minimization problem where $f: \R^d \to \R$ is convex and smooth with
    parameter $L$, and $\mathcal{X}$ is convex, closed and bounded. (This means that the minimizer
    $\vec{x}^\star$ of $f$ over $\mathcal{X}$ exists and that all minimization oracles have
    minimizers.) With any $\vec{x}_0 \in \mathcal{X}$ and stepsizes \[
        \gamma_t \doteq \frac{2}{t+2},
    \]
    the Frank-Wolfe algorithm yields \[
        f(\vec{x}_T) - f(\vec{x}^\star) \leq \frac{2L}{T+1} \mathrm{diam}(\mathcal{X})^2,
    \]
    where $\mathrm{diam}(\vec{x}) \doteq \max_{\vec{x},\vec{y}\in \mathcal{X}} \| \vec{x} - \vec{y} \|$
    is the diameter of $\mathcal{X}$.
\end{theorem}

\begin{proof}
    Let $h(\vec{x}) \doteq f(\vec{x}) - f(\vec{x}^\star)$ and $C \doteq \frac{L}{2} \mathrm{diam}(\mathcal{X})^2$.
    \begin{align*}
        f(\vec{x}_{t+1}) - f(\vec{x}^\star) & \leq f(\vec{x}_t) - \gamma_t g(\vec{x}_t) + \gamma_t^2 \frac{L}{2} \|\vec{s} - \vec{x}_t\|^2 - f(\vec{x}^\star) \margintag{\Cref{lem:fw-descent} and subtract $f(\vec{x}^\star)$ from both sides.} \\
        h(\vec{x}_{t+1})                    & \leq h(\vec{x}_t) - \gamma_t h(\vec{x}_t) + \gamma_t^2 \frac{L}{2} \|\vec{s} - \vec{x}_t\|^2 \margintag{Duality gap, $g(\vec{x}) \geq h(\vec{x})$.}                                                \\
                                            & = (1- \gamma_t) h(\vec{x}_t) + \gamma_t^2 \frac{L}{2} \|\vec{s} - \vec{x}_t\|^2                                                                                                                    \\
                                            & \leq (1- \gamma_t) h(\vec{x}_t) + \gamma_t^2 C.
    \end{align*}

    Using our new definitions, we want to prove \[
        h(\vec{x}_T) \leq \frac{4C}{T+1}, \quad T \geq 1.
    \]
    We can prove this by induction. The base case is $T = 1$,
    \begin{align*}
        h(\vec{x}_1) & \leq (1- \gamma_0) h(\vec{x}_0) + \gamma_0^2 C \\
                     & = C \margintag{$\gamma_0 = 1$.}                \\
                     & \leq 2C.
    \end{align*}
    Suppose it holds for $T = k$, \[
        h(\vec{x}_k) \leq \frac{4C}{k+1}.
    \]
    then we need to show that it also holds for $T = k+1$,
    \begin{align*}
        h(\vec{x}_{k+1}) & \leq (1-\gamma_k) h(\vec{x}_k) + \gamma_k^2 C                                                                                                    \\
                         & \leq \lft( 1 - \frac{2}{k+2} \rgt) \frac{4C}{k+1} + \lft( \frac{2}{k+2} \rgt)^2 C \margintag{Induction step and $\gamma_k = \nicefrac{2}{k+2}$.} \\
                         & = \frac{k}{k+2} \frac{4C}{k+1} + \frac{4}{(k+2)^2} C                                                                                             \\
                         & = \frac{4C}{k+2} \lft( \frac{k}{k+1} + \frac{1}{k+2} \rgt)                                                                                       \\
                         & \leq \frac{4C}{k+2}.
    \end{align*}
    Thus, it holds for all $T \geq 1$.
\end{proof}

\paragraph{Affine invariance.}

Consider the problem of minimizing $f(x_1, x_2) \doteq x_1^2 + x_2^2$ over the unit square
$\mathcal{X} = \{ [x_1,x_2] \mid 0 \leq x_1 \leq 1, 0 \leq x2 \leq 1 \}$. This function is smooth
with $L=2$ and $\mathrm{diam}(\mathcal{X})^2 = 2$. This has the following error bound, \[
    f(\vec{x}_T) - f(\vec{x}^\star) \leq \frac{8}{T+1}.
\]
Next consider $f'(x_1, x_2) \doteq x_1^2 + (10x_2)^2$ over the rectangle $\mathcal{X}'' = \{ [x_1,
            x_2] \mid 0 \leq x_1 \leq 1, 0\leq x_2 \leq \nicefrac{1}{10} \}$. This function is smooth with $L'
    = 200$ and $\mathrm{diam}(\mathcal{X}')^2 = 1 + \nicefrac{1}{100}$. Thus, $f'$ has the error bound \[
    f'(\vec{x}_T) - f'(\vec{x}^\star) \leq \frac{404}{T+1}.
\]
Hence, according to our analysis, it seems that the error of the Frank-Wolfe algorithm on $f'$ over
$\mathcal{X}'$ is roughly 50 times larger than on $f$ over $\mathcal{X}$.

However, when we look more closely at the function, the two problems $(f,\mathcal{X})$ and
$(f',\mathcal{X}')$ are equivalent under a rescaling of $x_2$. The Frank-Wolfe algorithm is
invariant under affine transformations, thus there should be no difference in the analysis of the
two considered problems.

Formally, two problems $(f, \mathcal{X})$ and $(f', \mathcal{X}')$ are \textit{affinely equivalent}
if $f'(\vec{x}) = f(\mat{A} \vec{x} + \vec{b})$ and $\mathcal{X}' = \{ \inv{\mat{A}}(\vec{x} -
    \vec{b}) \mid \vec{x} \in \mathcal{X} \}$ for some invertible matrix $\mat{A}$ and some vector
$\vec{b}$. The consequence is that $f(\vec{x}) = f'(\vec{x}')$ if $\vec{x}' = \inv{\mat{A}}
    (\vec{x} - \vec{b})$.\sidenote{In our example problem, this means that we have \[
        \mat{A} = \begin{bmatrix} 1 & 0 \\ 0 & 10 \end{bmatrix}, \quad \vec{b} = \vec{0}.
    \]}

By the chain rule, we get \[
    \grad{f'(\vec{x}')}{} = \transpose{\mat{A}} \grad{f(\mat{A} \vec{x}' + \vec{b})}{} = \transpose{\mat{A}} f(\vec{x}).
\]

Consider the iterate $\vec{x}_k$ and its corresponding iterate $\vec{x}'_k = \inv{\mat{A}}
    (\vec{x}_k - \vec{b})$, in their respective problems. We can compute the their oracle calls by
\begin{align*}
    \mathrm{LMO}_{\mathcal{X}}(\grad{f(\vec{x}_k)}{})    & = \argmin_{\vec{z} \in \mathcal{X}} \transpose{\grad{f(\vec{x}_k)}{}} \vec{z}                                                                        \\
                                                         & \doteq \vec{s}                                                                                                                                       \\
    \mathrm{LMO}_{\mathcal{X}'}(\grad{f'(\vec{x}'_k)}{}) & = \argmin_{\vec{z}' \in \mathcal{X}'} \transpose{\grad{f'(\vec{x}'_k)}{}} \vec{z}'                                                                   \\
                                                         & = \argmin_{\inv{\mat{A}} (\vec{z} - \vec{b}) \in \mathcal{X}'} \transpose{\grad{f(\vec{x}_k)}{}} \mat{A} \inv{\mat{A}} (\vec{z} - \vec{b})           \\
                                                         & = \argmin_{\inv{\mat{A}} (\vec{z} - \vec{b}) \in \mathcal{X}'} \transpose{\grad{f(\vec{x}_k)}{}} \vec{z} - \transpose{\grad{f(\vec{x}_k)}{}} \vec{b} \\
                                                         & = \inv{\mat{A}} \lft( \lft(\argmin_{\vec{z} \in \mathcal{X}} \transpose{\grad{f(\vec{x}_k)}{}} \vec{z}\rgt) - \vec{b} \rgt)                          \\
                                                         & = \inv{\mat{A}} (\vec{s} - \vec{b}).
\end{align*}
Thus, the step directions $\vec{s}'$ and $\vec{s}$ also correspond to each other under the affine
transformation. As a consequence, the next iterates will also correspond to each other,
\begin{align*}
    \vec{x}_{k+1}  & = (1- \gamma) \vec{x}_k + \gamma \vec{s}                                                     \\
    \vec{x}'_{k+1} & = (1- \gamma) \vec{x}'_k + \gamma \vec{s}'                                                   \\
                   & = (1- \gamma) \inv{\mat{A}} (\vec{x}_k - \vec{b}) + \gamma \inv{\mat{A}} (\vec{s} - \vec{b}) \\
                   & = \inv{\mat{A}} ((1-\gamma)\vec{x}_k + \gamma \vec{s}) - \vec{b}                             \\
                   & = \inv{\mat{A}} \vec{x}_{k+1} - \vec{b}.
\end{align*}

In particular, after any number of steps, both problems will incur the same optimization error.
Thus, we need a better analysis that provides a bound that is invariant under affine
transformations. For this, we define a \textit{curvature constant}, \[
    C_{(f,\mathcal{X})} \doteq \sup_{\substack{\vec{x},\vec{s}\in \mathcal{X}, \gamma\in(0,1] \\ \vec{y} = (1-\gamma) \vec{x} + \gamma \vec{s})}} \frac{1}{\gamma^2} \lft( f(\vec{y}) - f(\vec{x}) - \transpose{\grad{f(\vec{x})}{}} (\vec{y} - \vec{x}) \rgt).
\]
This quantity serves as a notion of complexity of both the objective function $f$ and the
constraint set $\mathcal{X}$. It is essentially the supremum of the normalized pointwise vertical
distance between the graph of $f$, $f(\vec{y})$ and its linear approximation at $\vec{x}$,
$f(\vec{x}) + \transpose{\grad{f(\vec{x})}{}} (\vec{y} - \vec{x})$.

\begin{theorem}
    Consider the constrained minimization problem, where $f: \R^d \to \R$ is convex, and $\mathcal{X}$ is convex, closed and bounded. Let $C_{(f, \mathcal{X})}$ be the curvature constant of $f$ over $\mathcal{X}$. With any $\vec{x}_0 \in \mathcal{X}$ and with stepsizes \[
        \gamma_t = \frac{2}{t+2},
    \]
    the Frank-Wolfe algorithm yields \[
        f(\vec{x}_T) - f(\vec{x}^\star) \leq \frac{4 C_{(f, \mathcal{X})}}{T+1}.
    \]
\end{theorem}

\begin{proof}
    We can regain the descent lemma by rewriting the curvature constant. We know by the definition of the supremum, \[
        \frac{1}{\gamma^2} \lft( f(\vec{y}) - f(\vec{x}) - \transpose{\grad{f(\vec{x})}{}} (\vec{y} - \vec{x}) \rgt) \leq C_{(f,\mathcal{X})}. \margintag{$\forall \vec{x},\vec{s}\in \mathcal{X}, \gamma \in (0,1], \vec{y} = (1-\gamma) \vec{x} + \gamma \vec{s}.$}
    \]
    Setting the variables, \[
        \vec{x} \doteq \vec{x}_t, \quad \vec{y} \doteq \vec{x}_{t+1} = (1-\gamma_t) \vec{x}_t + \gamma_t \vec{s}, \quad \vec{y} - \vec{x} = -\gamma_t (\vec{x} - \vec{s}),
    \]
    we get
    \begin{align*}
        f(\vec{x}_{t+1}) & \leq f(\vec{x}_t) + \transpose{\grad{f(\vec{x}_t)}{}} (-\gamma_t (\vec{x} - \vec{s})) + \gamma_t^2 C_{(f, \mathcal{X})} \\
                         & = f(\vec{x}_t) - \gamma g(\vec{x}_t) + \gamma_t^2 C_{(f, \mathcal{X})}.
    \end{align*}
    The rest of the proof follows as in the previous analysis.
\end{proof}

You might suspect that this bound is worse than the best bound obtainable from the previous
analysis. However, one can show \[
    C_{(f, \mathcal{X})} \leq \frac{L}{2} \mathrm{diam}(\mathcal{X})^2.
\]

Furthermore, we can prove a convergence of the duality gap.

\begin{theorem}
    Let $f: \R^d \to \R$, $\mathcal{X}$ convex with $C_{(f, \mathcal{X})} < \infty$, $\vec{x}_0
        \in \mathcal{X}$, and $T \geq 2$. Then, choosing stepsize \[
        \gamma_t = \frac{2}{t+2},
    \]
    the Frank-Wolfe algorithm yields a $t$ with $1 \leq t \leq T$, such that \[
        g(\vec{x}_t) \leq \frac{\nicefrac{27}{2} \cdot C_{(f, \mathcal{X})}}{T+1}.
    \]

\end{theorem}

