\section{Projective geometry} \label{sec:projective-geometry}

\begin{notation}
  $\vec{x}$ & two-dimensional vector in real space $\R$ \\
  $\tilde{\vec{x}}$ & two-dimensional vector in projective space $\mathbb{P}$ \\
  $\overline{\vec{x}}$ & two-dimensional vector in projective space $\mathbb{P}$ with $w=1$ \\
  $\vec{X}$ & three-dimensional vector in real space $\R$ \\
  $\tilde{\vec{X}}$ & three-dimensional vector in projective space $\mathbb{P}$ \\
  $\overline{\vec{X}}$ & three-dimensional vector in projective space $\mathbb{P}$ with $w=1$ \\
\end{notation}

Using \textit{projective geometry}, we can model the imaging process of a
perspective camera,\sidenote{And various other transformations that Euclidean
space cannot, such as translation.} because it satisfies the following
constraints,
\begin{itemize}
  \item Straight lines must be mapped to straight lines;
  \item The size of an object must be inversely proportional to its distance
    from the camera;
  \item We must be able to represent points at infinity to model vanishing
    points.
\end{itemize}
In projective geometry, the homogeneous coordinate space is used, \[
  \tilde{\vec{x}} \propto \begin{bmatrix} \vec{x} \\ 1 \end{bmatrix} \in \mathbb{P}^2
,\]
which can be converted to Euclidean space, \[
  \tilde{\vec{x}} \mapsto \begin{bmatrix} \nicefrac{x}{w} \\ \nicefrac{y}{w} \end{bmatrix} \in \R^2
,\]
where $w$ is the last entry of $\tilde{\vec{x}}$. Homogeneous coordinates with
$w=0$ are called \textit{ideal points} and serve as a direction toward infinity,
rather than points.

In this space, lines can be represented as the following, \[
  \vec{\ell} = \transpose{\begin{bmatrix} a & b & c \end{bmatrix}}
,\]
where the points on this line satisfy $\bm{\ell}^\top\tilde{\bm{x}}=0$. Now,
if we want to find the point where two lines $\bm{\ell},\bm{\ell}'$
intersect, we need to find a point that satisfies
$\transpose{\vec{\ell}} \tilde{\vec{x}} = \transpose{\vec{\ell}'}\tilde{\vec{x}} = 0$. This is
exactly the cross product, \[
  \tilde{\vec{x}}_{\text{intersection}} = \vec{\ell} \times \vec{\ell}'
.\]
We can find the line between two points $\tilde{\vec{x}}_1,\tilde{\vec{x}}_2$
in the same way, \[
  \vec{\ell} = \tilde{\vec{x}}_1 \times \tilde{\vec{x}}_2
.\]

In three dimensions, we can also represent planes, \[
  \vec{\pi} = \transpose{\begin{bmatrix} a & b & c & d \end{bmatrix}}
.\]
A point lies on the plane $\vec{\pi}$ if and only if $\transpose{\vec{\pi}}
\vec{X}=0$.  If we want to find a plane from three points, we need to solve \[
  \transpose{\vec{\pi}} \begin{bmatrix} \vec{X}_1 & \vec{X}_2 & \vec{X}_3 \end{bmatrix} = \vec{0}
.\]
We can construct lines as the intersection of two planes, or construct
points as the intersection of three planes.

\subsection{Transformations}

Translations in Euclidean geometry are non-linear, but by using homogeneous
coordinates it becomes a linear operation, \[
  \begin{bmatrix} x \\ y \end{bmatrix} \mapsto \begin{bmatrix} 1 & 0 & t_x \\ 0 & 1 & t_y \\ 0 & 0 & 1 \end{bmatrix} \begin{bmatrix} x \\ y \\ w  \end{bmatrix} = \begin{bmatrix} x + t_x \\ y + t_y \\ w \end{bmatrix} 
.\]
Furthermore, we can make the following transformations,
\begin{itemize}
  \item Rigid transformations (rotation and translation) with the following
    matrix,\sidenote{This matrix has 3 degrees of freedom: $t_x,t_y,\theta$.}
    \[
      \begin{bmatrix} \cos\theta & -\sin\theta & t_x \\ \sin\theta & \cos\theta  & t_y \\ 0 & 0 & 1 \end{bmatrix}
    .\]
  \item Similarity transformations (rotation, translation, and scaling) with
    the following matrix,\sidenote{This matrix has 4 degrees of freedom:
    $t_x,t_y,\theta,s$.} \[
      \begin{bmatrix} s\cos\theta & -s\sin\theta & t_x \\ s\sin\theta & s\cos\theta & t_y \\ 0 & 0 & 1 \end{bmatrix}
    .\]
  \item Affine transformations (rotation, translation, scaling, and shearing)
    with the following matrix,\sidenote{This matrix has 6 degrees of freedom:
    $a_{1:6}$} \[
      \begin{bmatrix} a_1 & a_2 & a_3 \\ a_4 & a_5 & a_6 \\ 0 & 0 & 1 \end{bmatrix}
    .\]
  \item Projective transformations with the following matrix,\sidenote{This
    matrix has 8 degrees of freedom, because it is specified up to scale.} \[
      \begin{bmatrix} a & b & c \\ d & e & f \\ g & h & i \end{bmatrix}
    .\]
\end{itemize}

\subsection{Homography}

Suppose that we have point correspondences between two images. Now, we want to
find the projective transformation $\mat{H}$ that relates the two images. For
this, we use \textit{direct linear transformation}. We have the following
equations,\sidenote{The homography is constrained to map $\vec{x}_i$ to
$\vec{x}_i'$.}
\begin{align*}
  \lambda_i \bar{\vec{x}}'_i &= \mat{H} \bar{\vec{x}}_i \\
  \begin{bmatrix} \lambda_i x_i' \\ \lambda_i y_i' \\ \lambda_i \end{bmatrix}
  &=
  \begin{bmatrix}
    h_{11}x_i + h_{12}y_i + h_{13} \\
    h_{21}x_i + h_{22}y_i + h_{23} \\
    h_{31}x_i + h_{32}y_i + h_{33}
  \end{bmatrix}
.\end{align*}
\marginnote{\textit{Singular value decomposition} (SVD) decomposes a matrix
  $\mat{A} = \mat{U}\mat{\Sigma}\transpose{\mat{V}}$, where $\mat{U}$ and
  $\mat{V}$ are unitary matrices, and $\mat{\Sigma}$ is a diagonal scaling
  matrix. The SVD is very useful for solving (overdetermined) equations of the
  following form, \[
    \mat{A}\vec{x} = \vec{0}
  .\]
  Now, when solving such equations, we want to minimize $\| \mat{A}\vec{x} \|$
  subject to $\| \bm{x} \|=1$. We can solve for this by using SVD,
  \begin{align*}
    &\min \| \mat{A}\vec{x} \| & \| \vec{x} \| = 1 \\
    &\min \| \mat{U}\bm{\Sigma}\transpose{\mat{V}}\vec{x} \| & \| \vec{x} \| = 1 \\
    &\min \| \mat{\Sigma}\transpose{\mat{V}}\vec{x} \| & \| \vec{x} \| = 1 \\
    &\min \| \mat{\Sigma}\vec{b} \| & \| \mat{V}\vec{b} \| = 1
  .\end{align*}
  Then, $\| \mat{\Sigma}\vec{b} \|$ is minimized if \[
    \vec{b}=\transpose{\begin{bmatrix} 0 & \cdots & 0 & 1 \end{bmatrix}}
  .\]
  Thus, the solution is the following, \[
    \vec{x} = \mat{V} \begin{bmatrix} 0 & \cdots & 0 & 1 \end{bmatrix}^\top = \mat{V}_n \\
  ,\]
  where $\mat{V}_n$ is the last column vector of $\mat{V}$.}
Now, when we revert them back to the real coordinate system, we get the
following,
\begin{align*}
  \lambda_i &= h_{31}x_i + h_{32}y_i + h_{33} \\
  \lambda_i x_i' &= h_{11}x_i + h_{12}y_i + h_{13} \\
                 &\implies h_{11}x_i + h_{12}y_i + h_{13} - h_{31}x_i'x_i - h_{32}x_i'y_i - h_{33}x_i' = 0 \\
  \lambda_i y_i' &= h_{21}x_i + h_{22}y_i + h_{23} \\
                 &\implies h_{21}x_i + h_{22}y_i + h_{23} - h_{31}y_i'x_i - h_{32}y_i'y_i - h_{33}y_i' = 0
.\end{align*}
In matrix form, we can set this up as the following for $n$ points, \[
  \begin{bmatrix}
    x_1 & y_1 & 1 & 0 & 0 & 0 & -x_1'x_1 & -x_1'y_1 & -x_1' \\
    0 & 0 & 0 & x_1 & y_1 & 1 & -y_1'x_1 & -y_1'y_1 & -y_1' \\
    \vdots & \vdots & \vdots & \vdots & \vdots & \vdots & \vdots & \vdots & \vdots \\
    x_n & y_n & 1 & 0 & 0 & 0 & -x_n'x_n & -x_n'y_n & -x_n' \\
    0 & 0 & 0 & x_n & y_n & 1 & -y_n'x_n & -y_n'y_n & -y_n' \\
  \end{bmatrix}
  \begin{bmatrix}
    h_{11} \\
    h_{12} \\
    h_{13} \\
    h_{21} \\
    h_{22} \\
    h_{23} \\
    h_{31} \\
    h_{32} \\
    h_{33}
  \end{bmatrix}
  =
  \vec{0}
.\]

If we stack $n \geq 4$ points, we can use SVD to find the $\vec{h}$ vector that
minimizes the algebraic distance. However, this does not have a geometric
meaning. We would like a cost function that minimizes the distortion of the
points. An example of this is the symmetric transfer error, \[
  \hat{\mat{H}} = \argmin_{\mat{H}}
  \sum_{(\vec{x},\vec{x}')\in\mathcal{D}}
  d(\overline{\vec{x}},\mat{H}^{-1}\overline{\vec{x}}')^2 +
  d(\overline{\vec{x}}', \mat{H}\overline{\vec{x}})^2
,\]
where $\mathcal{D}$ is the collection of point correspondences, and $d(\cdot,
\cdot)$ is a distance function in real space. In practice, we use the algebraic
solution as a starting point and then refine it to minimize the geometric error.
