\section{Convex optimization}

In standard form, an optimization problem looks as follows,
\begin{align*}
    \text{minimize}   & \quad f(\vec{x})                                    \\
    \text{subject to} & \quad g_i(\vec{x}) = 0, \quad \forall i \in [n]     \\
                      & \quad h_j(\vec{x}) \leq 0, \quad \forall j \in [m],
\end{align*}
This is called a convex program if $f$ and all $h_j$ are convex and all $g_i$ are affine.
We will focus on convex programming. The constraints define a convex set
$\mathcal{C} \subseteq \mathcal{X}$, where $\mathcal{X}$ is the intersection of domains of $f$ and
all constraints. For our purposes, let $\mathcal{X} = \R^d$. The aim is to compute \[
    \argmin_{\vec{x} \in \mathcal{C}} f(\vec{x}).
\]

\begin{definition}[Lagrangian dual function]
    For an optimization problem, the Lagrangian $\mathcal{L}: \mathcal{X} \times \R^n \times \R^m \to \R$ is defined as \[
        \mathcal{L}(\vec{x}, \vec{\lambda}, \vec{\alpha}) \doteq f(\vec{x}) + \sum_{i=1}^{n} \lambda_i g_i(\vec{x}) + \sum_{j=1}^{m} \alpha_j h_j(\vec{x}).
    \]
    The scalars $\lambda_i$ and $\alpha_j$ are called Lagrange multipliers. The Lagrange dual function
    $\delta: \R^n \to \R^m \to \R \cup \{ -\infty \}$ is defined as \[
        \theta(\vec{\lambda}, \vec{\alpha}) \doteq \inf_{\vec{x} \in \mathcal{X}} \mathcal{L}(\vec{x}, \vec{\lambda}, \vec{\alpha}).
    \]
\end{definition}

\begin{lemma}[Weak duality] \label{lem:weak-duality}
    Let $\vec{x} \in \mathcal{C}$ (so it satisfies all constraints), $\vec{\lambda} \in \R^n$, $\vec{\alpha} \in \R^m$ with $\vec{\alpha} \geq \vec{0}$, and $\theta$ be the Lagrange dual function, then \[
        \theta(\vec{\lambda}, \vec{\alpha}) \leq f(\vec{x}).
    \]
    The dual Lagrange function lower bounds the primal solution.
\end{lemma}

\begin{proof}
    Define $\vec{x}$, $\vec{\lambda}$, and $\vec{\alpha}$ as in \Cref{lem:weak-duality}, then we have
    \begin{align*}
        \theta(\vec{\lambda}, \vec{\alpha}) & \doteq \inf_{\vec{x} \in \mathcal{X}} \mathcal{L}(\vec{x}, \vec{\lambda}, \vec{\alpha})     \\
                                            & \leq \mathcal{L}(\vec{x}, \vec{\lambda}, \vec{\alpha})                                      \\
                                            & = f(\vec{x}) + \sum_{i=1}^{n} \lambda_i g_i(\vec{x}) + \sum_{j=1}^{m} \alpha_j h_j(\vec{x}) \\
                                            & \leq f(\vec{x}). \margintag{We have $g_i(\vec{x}) = 0$ and $\alpha_j h_j(\vec{x}) \leq 0$.}
    \end{align*}
    This concludes the proof.
\end{proof}

As shown in \Cref{lem:weak-duality}, the Lagrange dual function $\theta$ lower bounds the primal
$f$ for all $\vec{x} \in \mathcal{C}$, $\vec{\lambda}$, and $\vec{\alpha} \geq \vec{0}$. We are
interested in the maximum lower bound, which results in the dual program,
\begin{align*}
    \text{maximize}   & \quad \theta(\vec{\lambda}, \vec{\alpha})       \\
    \text{subject to} & \quad \alpha_i \geq 0, \quad \forall i \in [n].
\end{align*}
This is a convex program, even if the primal is not convex.

\begin{corollary}
    \[
        \max_{\vec{\lambda}, \vec{\alpha} \geq \vec{0}} \theta(\vec{\lambda}, \vec{\alpha}) \leq \min_{\vec{x} \in \mathcal{C}} f(\vec{x}).
    \]
\end{corollary}

\begin{lemma}[Sufficient condition for strong duality]
    If Slater's condition holds (there exists an $\vec{x} \in \mathcal{C}$ such that $h_j(\vec{x}) < 0$
    for all $j \in [m]$), then we have strong duality, \[
        \max_{\vec{\lambda},\vec{\alpha} \geq \vec{0}} \theta(\vec{\lambda}, \vec{\alpha}) = \min_{\vec{x} \in \mathcal{X}} f(\vec{x}).
    \]
\end{lemma}

For most problems, a Slater point exists. So, most of the time, strong duality holds. If all $g_i$
and $h_j$ are differentiable, then the KKT (\textit{\textbf{K}arush-\textbf{K}uhn-\textbf{T}ucker})
conditions provide necessary conditions for strong duality.

\begin{lemma}[KKT necessary conditions for strong duality] \label{lem:kkt}
    Let $\vec{x}^\star$ and $(\vec{\lambda}^\star, \vec{\alpha}^\star)$ be feasible solutions to the
    primal and dual problems. Further, assume that strong duality holds. Then,
    \begin{align*}
        \alpha_j^\star h_j(\vec{x}^\star)                                                   & = 0, \quad \forall j \in [m] \margintag{Complementary slackness.}                                                                                                                                          \\
        \grad{\mathcal{L}(\vec{x}^\star, \vec{\lambda}^\star, \vec{\alpha}^\star)}{\vec{x}} & = \grad{f(\vec{x}^\star)}{} + \sum_{i=1}^{n} \lambda^\star_i \grad{g_i(\vec{x}^\star)}{} + \sum_{j=1}^{m} \alpha^\star_j \grad{h_j(\vec{x}^\star)}{} = \vec{0}. \margintag{Vanishing Lagrangian gradient.}
    \end{align*}
    The second condition can also be written as the following if $\mathcal{L}$ is finite, \[
        \vec{x}^\star \in \argmin_{\vec{x} \in \mathcal{X}} \mathcal{L}(\vec{x}, \vec{\lambda}^\star, \vec{\alpha}^\star).
    \]
\end{lemma}

\begin{proof}
    Define $\vec{x}^\star$, $\vec{\lambda}^\star$, and $\vec{\alpha}^\star$ as in \Cref{lem:kkt}, then we get the Master equation,
    \begin{align*}
        f(\vec{x}^\star) & = \theta(\vec{\lambda}^\star, \vec{\alpha}^\star) \margintag{Strong duality.}                                                  \\
                         & \doteq \inf_{\vec{x} \in \mathcal{X}} \mathcal{L}(\vec{x}, \vec{\lambda}^\star, \vec{\alpha}^\star)                            \\
                         & \leq \mathcal{L}(\vec{x}^\star, \vec{\lambda}^\star, \vec{\alpha}^\star)                                                       \\
                         & \doteq f(\vec{x}^\star) + \sum_{i=1}^{n} \lambda^\star_i g_i(\vec{x}^\star) + \sum_{j=1}^{m} \alpha^\star_j h_j(\vec{x}^\star) \\
                         & \leq f(\vec{x}^\star). \margintag{We have $g_i(\vec{x}) = 0$ and $\alpha_j h_j(\vec{x}) \leq 0$.}
    \end{align*}
    Thus, all inequalities become equalities. The first condition comes from the second-to-last and last line,
    which results in $\alpha_j^\star h_j(\vec{x}^\star) = 0$ for all $j \in [m]$. The second
    condition comes from the second and third line, where there must be vanishing gradient, because
    $\vec{x}^\star$ is the minimizer of $\mathcal{L}(\vec{x}, \vec{\lambda}^\star, \vec{\alpha}^\star)$.
\end{proof}

\begin{lemma}[KKT sufficient conditions for strong duality]
    Let the primal be a convex program. Further, let $\vec{x}^\star$ and $(\vec{\lambda}^\star,
        \vec{\alpha}^\star)$ be feasible solutions to the primal and dual, respectively. Lastly, let the KKT conditions hold,
    \begin{align*}
        \alpha_j^\star h_j(\vec{x}^\star)                                                   & = 0, \quad \forall j \in [m] \\
        \grad{\mathcal{L}(\vec{x}^\star, \vec{\lambda}^\star, \vec{\alpha}^\star)}{\vec{x}} & = \vec{0}.
    \end{align*}
    Then, strong duality holds.
\end{lemma}

\begin{proof}
    \begin{align*}
        f(\vec{x}^\star) & = f(\vec{x}^\star) + \sum_{i=1}^{n} \lambda_i g_i(\vec{x}^\star) + \sum_{j=1}^{m} \alpha_j h_j(\vec{x}^\star) \margintag{$\vec{x}^\star$ is feasible and we have complementary slackness.}                                                                                                                                                                                                                                \\
                         & = \mathcal{L}(\vec{x}^\star, \vec{\lambda}^\star, \vec{\alpha}^\star)                                                                                                                                                                                                                                                                                                                                                     \\
                         & = \inf_{\vec{x} \in \mathcal{X}} \mathcal{L}(\vec{x}, \vec{\lambda}^\star, \vec{\alpha}^\star) \margintag{$\mathcal{L}$ is convex, because convexity is preserved under summation and positive scaling. Also, affine functions remain convex under negative scaling. Furthermore, we have vanishing gradient, so $\vec{x}^\star$ is a global minimizer of $\mathcal{L}(\cdot, \vec{\lambda}^\star, \vec{\alpha}^\star)$.} \\
                         & \doteq \theta(\vec{\lambda}^\star, \vec{\alpha}^\star).
    \end{align*}
\end{proof}

\subsection{Support vector machine}

\begin{marginfigure}[7cm]
    \centering
    \incfig{svm}
    \caption{Support vector machines.}
    \label{fig:svm}
\end{marginfigure}

In SVMs (\textit{\textbf{S}upport \textbf{V}ector \textbf{M}achines}), we are given a dataset with
binary labels and we want to find the linear separator with the maximum margin. Formally, we are
given data $\{ (\vec{x}_1, y_1), \ldots, (\vec{x}_n, y_n) \} \subseteq \R^d \times \{ -1, +1 \}$
and we model the linear separator as \[
    f[\vec{w}, b](\vec{x}) \doteq
    \begin{cases}
        +1 & \transpose{\vec{w}} \vec{x} + b > 0  \\
        -1 & \transpose{\vec{w}} \vec{x} + b < 0.
    \end{cases}
\]
We want to model this as a convex program, so we need to get rid of the strict inequality and the
cases. We are given the dataset, so we can use information of $y$ to reverse the inequality if
necessary. Furthermore, we introduce a small scalar $\epsilon > 0$ to get a non-strict inequality, \[
    f[\vec{w}, b](\vec{x}, y) = y \lft( \transpose{\vec{w}} \vec{x} + b \rgt) \geq \epsilon > 0.
\]
Now, we have the following optimization problem,
\begin{align*}
    \text{maximize}   & \quad m(\vec{w}, b)                                                                             \\
    \text{subject to} & \quad y_i \lft( \transpose{\vec{w}} \vec{x}_i + b \rgt) \geq \epsilon, \quad \forall i \in [n],
\end{align*}
where $m: \R^d \times \R \to \R$ is the margin function. The margin function is the minimum distance
of any point to the linear separator. Let $\vec{x}^+$ and $\vec{x}^-$ be the positive and negative
support vectors, where a support vector is a point that touches the margin---see \Cref{fig:svm}. Then, we have the following definition of the margin,
\begin{align*}
    2 \cdot m(\vec{w}, b) & \doteq \| \mathrm{proj}_{\vec{w}}(\vec{x}^+) - \mathrm{proj}_{\vec{w}}(\vec{x}^-) \| \margintag{See \Cref{fig:svm}.}                                                          \\
                          & = \lft\| \frac{\transpose{\vec{w}} \vec{x}^+}{\| \vec{w} \|^2} \vec{w} - \frac{\transpose{\vec{w}} \vec{x}^-}{\| \vec{w} \|^2} \vec{w} \rgt\| \margintag{$\mathrm{proj}_{\vec{b}}(\vec{a}) = \frac{\transpose{\vec{a}} \vec{b}}{\| \vec{b} \|^2} \vec{a}$.} \\
                          & = \frac{\lft| \transpose{\vec{w}} \vec{x}^+ - \transpose{\vec{w}} \vec{x}^- \rgt|}{\| \vec{w} \|^2} \| \vec{w} \|                             \\
                          & = \frac{\lft| \transpose{\vec{w}} \vec{x}^+ - \transpose{\vec{w}} \vec{x}^- \rgt|}{\| \vec{w} \|}                                             \\
                          & = \lft| \transpose{\bar{\vec{w}}} (\vec{x}^+ - \vec{x}^-) \rgt|,
\end{align*}
where $\bar{\vec{w}}$ is a normalized $\vec{w}$. Note that the margin does not depend on the norm of
$\vec{w}$ or the intercept $b$---it only depends on the direction of $\vec{w}$. As a result, there are an
infinite number of solutions, which results in an ill-posed problem. However, only one solution
$(\vec{w},b)$ will satisfy \[
    \transpose{\vec{w}}\vec{x}^+ + b = 1, \quad \transpose{\vec{w}} \vec{x}^- + b = -1.
\]
Under this constraint, the margin can be computed as follows, \[
    2 \cdot m(\vec{w}, b) = \frac{2}{\| \vec{w} \|},
\]
because $| \transpose{\vec{w}} \vec{x}^+ - \transpose{\vec{w}} \vec{x}^- | = | (1 - b) - (-1 - b) | = 2$.
Thus, we get the following convex program,
\begin{align*}
    \text{minimize}   & \quad \frac{1}{2} \| \vec{w} \|^2 \margintag{Minimizing $\nicefrac{1}{f(\vec{x})}$ is equivalent to maximizing $f(\vec{x})$.} \\
    \text{subject to} & \quad 1 - y_i \lft( \transpose{\vec{w}} \vec{x}_i + b \rgt) \leq 0, \quad \forall i \in [n].
\end{align*}
It is easy to see that this convex program has a Slater point. Let $(\vec{w}, b)$ be a solution to the convex program, so \[
    y_i \lft( \transpose{\vec{w}} \vec{x}_i + b \rgt) \geq 1, \quad \forall i \in [n].
\]
Then, $(\gamma \vec{w}, \gamma b)$ for $\gamma > 1$ is a Slater point, because for all $i \in [n]$,
we have \[
    y_i \lft(\gamma \transpose{\vec{w}} \vec{x}_i + \gamma b \rgt) = \gamma y_i \lft( \transpose{\vec{w}} \vec{x}_i + b \rgt) > y_i \lft( \transpose{\vec{w}} \vec{x}_i + b \rgt) \geq 1.
\]
Hence, strong duality holds and we can use the KKT conditions to find the minimizer
$\vec{x}^\star$.

This convex program has the following Lagrangian,
\begin{align*}
    \mathcal{L}(\vec{w}, b, \vec{\alpha}) & = \frac{1}{2} \| \vec{w} \|^2 + \sum_{i=1}^{n} \alpha_i \lft( 1 - y_i \lft( \transpose{\vec{w}} \vec{x}_i + b \rgt) \rgt), \quad \alpha_i \geq 0     \\
                                          & = \frac{1}{2} \| \vec{w} \|^2 + \sum_{i=1}^{n} \alpha_i - \transpose{\vec{w}} \sum_{i=1}^{n} \alpha_i y_i \vec{x}_i - b \sum_{i=1}^{n} \alpha_i y_i.
\end{align*}
This function has the following gradients,
\begin{align*}
    \pdv{\mathcal{L}(\vec{w}, b, \vec{\alpha})}{\vec{w}} & = \vec{w} - \sum_{i=1}^{n} \alpha_i y_i \vec{x}_i \\
    \pdv{\mathcal{L}(\vec{w}, b, \vec{\alpha})}{b}       & = - \sum_{i=1}^{n} \alpha_i y_i.
\end{align*}
From the vanishing Lagrangian gradient KKT condition, we know \[
    \pdv{\mathcal{L}(\vec{w}^\star, b^\star, \vec{\alpha})}{\vec{w}} = \vec{0} \implies \vec{w}^\star = \sum_{i=1}^{n} \alpha_i y_i \vec{x}_i.
\]
Furthermore, we get the following constraint on $\alpha_i$, \[
    \pdv{\mathcal{L}(\vec{w}^\star, b^\star, \vec{\alpha})}{b} = 0 \implies \sum_{i=1}^{n} \alpha_i y_i = 0.
\]
Thus, we have the following dual program,
\begin{align*}
    \text{maximize}   & \quad \theta(\vec{\alpha}) \doteq \mathcal{L}(\vec{w}^\star, b^\star, \vec{\alpha}) \\
    \text{subject to} & \quad \alpha_i \geq 0, \quad \forall i \in [n]                                      \\
                      & \quad \sum_{i=1}^{n} \alpha_i y_i = 0.
\end{align*}
We can simplify the Lagrangian by making use of the found $\vec{w}^\star$,
\begin{align*}
    \theta(\vec{\alpha}) & \doteq \mathcal{L}(\vec{w}^\star, b^\star, \vec{\alpha})                                                                         \\
                         & = \frac{1}{2} \| \vec{w}^\star \|^2 + \sum_{i=1}^{n} \alpha_i - \| \vec{w}^\star \|^2                                            \\
                         & = \sum_{i=1}^{n} \alpha_i - \frac{1}{2} \| \vec{w}^\star \|^2                                                                    \\
                         & = \sum_{i=1}^{n} \alpha_i - \frac{1}{2} \sum_{i=1}^{n} \sum_{j=1}^{n} \alpha_i \alpha_j y_i y_j \transpose{\vec{x}_i} \vec{x}_j.
\end{align*}
Thus, the dual program is a quadratic optimization problem on a simplex, which is easy to optimize.
Once we have solved the dual program for $\vec{\alpha}^\star$, we can use this to compute $\vec{w}^\star$, \[
    \vec{w}^\star = \sum_{i=1}^{n} \alpha_i^\star y_i \vec{x}_i.
\]
Due to complementary slackness, the following holds, \[
    \alpha_i^\star \lft( 1 - y_i \lft( \transpose{\vec{w}_\star} \vec{x}_i + b^\star \rgt) \rgt) = 0.
\]
So, $\alpha_i^\star = 0$ often, which makes $\vec{\alpha}^\star$ sparse. In fact, $1 - y_i \lft(
    \transpose{\vec{w}_\star} \vec{x}_i + b^\star \rgt) = 0$ only if $\vec{x}_i$ is a support vector.
Thus, the solution $\vec{w}^\star$ is a sparse linear combination of support vectors.

We can compute the intercept by finding support vectors, \[
    \vec{x}^+ \in \argmin_{\vec{x}_i : y_i = +1} \transpose{\vec{w}} \vec{x}_i, \quad \vec{x}^- \in \argmax_{\vec{x}_i : y_i = -1} \transpose{\vec{w}} \vec{x}_i.
\]
Further, remember the constraints $\transpose{\vec{w}_\star} \vec{x}^+ + b = 1$ and
$\transpose{\vec{w}_\star} \vec{x}^- + b = -1$, from which we can derive the optimal intercept, \[
    b^\star = -\frac{1}{2} \lft( \transpose{\vec{w}_\star} \vec{x}^+ + \transpose{\vec{w}_\star} \vec{x}^- \rgt).
\]

\subsection{Soft-margin SVM}

The SVM only works if the data is linearly separable, but this does not hold in general. We can
work around this by introducing a slackness parameter to each data point, which gives the following
convex program,
\begin{align*}
    \text{minimize}   & \quad \frac{1}{2} \| \vec{w} \|^2 + C \sum_{i=1}^{n} \xi_i                              \\
    \text{subject to} & \quad y_i \lft( \transpose{\vec{w}} \vec{x}_i + b \rgt) \geq 1 - \xi_i, \quad i \in [n] \\
                      & \quad \xi_i \geq 0, \quad i \in [n].
\end{align*}
Here, we introduced slackness parameters that we wish to be as small as possible with some weight $C$.
This allows for incorrect classifications if it is justified. For example, the slack of an outlier
might be very high. The hyperparameter $C$ has an important role in balancing the size of the margin
with the number of neglected samples. As $C$ increases, the margin will get more narrow, but fewer points
will be neglected.

The optimal slackness parameters can be computed as follows, \[
    \xi_i^\star = \max \lft\{ 0, 1 - y_i \lft( \transpose{\vec{w}_\star} \vec{x}_i + b^\star \rgt) \rgt\}.
\]
And, $\alpha_i$ is further constrained to be less than $C$.

\subsection{Kernelization}

The data points might be separated by a non-linear separator. In this case, we can use a different
feature space by employing a feature function $\phi$. This results in the following solution
$\vec{w}^\star$, \[
    \vec{w}^\star = \sum_{i=1}^{n} \alpha_i^\star y_i \phi(\vec{x}_i).
\]
And, we get the following Lagrangian,
\begin{align*}
    \mathcal{L}(\vec{w}^\star, \vec{\alpha}) & = \sum_{i=1}^{n} \alpha_i - \frac{1}{2} \sum_{i=1}^{n} \sum_{j=1}^{n} \alpha_i \alpha_j y_i y_j \transpose{\phi(\vec{x}_i)} \phi(\vec{x}_j) \\
                                             & = \sum_{i=1}^{n} \alpha_i - \frac{1}{2} \sum_{i=1}^{n} \sum_{j=1}^{n} \alpha_i \alpha_j y_i y_j k(\vec{x}_i, \vec{x}_j).
\end{align*}
So, we do not even need the feature function to compute the Lagrangian---we only need a kernel.
Furthermore, we can use the kernel for classification (without loss of generalization, assume
$b^\star = 0$), \[
    \transpose{\vec{w}_\star} \phi(\vec{x}) = \sum_{i=1}^{n} \alpha_i^\star y_i k(\vec{x}_i, \vec{x}).
\]

\subsection{Multi-class SVM}

We can generalize the notion of a margin to multi-class problems by introducing weights $\vec{w}_z$
for each class $z \in [k]$. Then, we define the margin as the maximum $m \in \R$ such that \[
    m \leq \lft( \transpose{\vec{w}}_{z_i} \vec{y}_i + b_{z_i} \rgt) - \max_{z \neq z_i} \lft\{ \transpose{\vec{w}_z} \vec{y}_i + b_z \rgt\}, \quad \forall i \in [n].
\]
Intuitively, it is the closest distance of any point to a decision boundary. We then define the
following optimization problem, which we can solve
\begin{align*}
    \text{minimize}   & \quad \frac{1}{2} \| \vec{w} \|^2 = \frac{1}{2} \sum_{z=1}^{^k} \| \vec{w}_z \|^2                                                                                        \\
    \text{subject to} & \quad \lft( \transpose{\vec{w}_{z_i}} \vec{y}_i + b_{z_i} \rgt) - \max_{z \neq z_i} \lft\{ \transpose{\vec{w}_z} \vec{y}_i + b_z \rgt\} \geq 1, \quad \forall i \in [n].
\end{align*}
For data that is not linearly separable, we can again introduce slack,
\begin{align*}
    \text{minimize}   & \quad \frac{1}{2} \| \vec{w} \|^2 + C \sum_{i=1}^{n} \xi_i                                                                                                                      \\
    \text{subject to} & \quad \lft( \transpose{\vec{w}_{z_i}} \vec{y}_i + b_{z_i} \rgt) - \max_{z \neq z_i} \lft\{ \transpose{\vec{w}_z} \vec{y}_i + b_z \rgt\} \geq 1 - \xi_i, \quad \forall i \in [n] \\
                      & \quad \xi_i \geq 0, \quad \forall i \in [n].
\end{align*}

\subsection{Structural SVM}

SVMs can also be used to predict complex objects by making use of its structure. A naive way of
achieving this is to to give each unique structure its own class and doing multi-class SVM.
However, this would result in exponentially (or infinitely) many classes, which is intractable.
Thus, the structural SVM needs the following things,
\begin{enumerate}
    \item Compact representation of the output space;
    \item Algorithm that allows for efficient prediction---we cannot enumerate every possible structure;
    \item Notion of a prediction error---some structures are more similar than others;
    \item Efficient training algorithm that has a runtime complexity that is sublinear in the number of
          structures.
\end{enumerate}

The first can be solved by defining a joint feature map $\psi: \mathcal{X} \times \mathcal{Y} \to
    \R^d$ that combines properties of inputs and outputs.\sidenote{Notice that the dimensionality of
    this feature map is independent of the number of structures $|\mathcal{Y}|$.} Then, we can define
the following scoring function, \[
    f_{\vec{w}}(\vec{x}, \vec{y}) = \transpose{\vec{w}} \psi(\vec{x}, \vec{y}).
\]
We use this function to perform classification, \[
    c(\vec{x}) = \argmax_{\vec{y} \in \mathcal{Y}} f_{\vec{w}}(\vec{x}, \vec{y}).
\]
We would need to construct an algorithm to efficiently compute this argmax. In general, we cannot
enumerate over all structures, so we would need to make use of the structure of the data to somehow
compute this. For example, the output space might be decomposable into independent parts, such that
$\mathcal{Y} = \mathcal{Y}_1 \times \cdots \times \mathcal{Y}_m$.

Moreover, we define the margin as the maximum $m$ that satisfies \[
    \transpose{\vec{w}} \psi(\vec{x}_i, \vec{y}_i) - \max_{\vec{y} \neq \vec{y}_i} \transpose{\vec{w}} \psi(\vec{x}_i, \vec{y}) \geq m, \quad \forall i \in [n].
\]
Thus, we have the following optimization problem,
\begin{align*}
    \text{minimize}   & \quad \frac{1}{2} \| \vec{w} \|^2                                                                                                                                  \\
    \text{subject to} & \quad \transpose{\vec{w}} \psi(\vec{x}_i, \vec{y}_i) - \max_{\vec{y} \neq \vec{y}_i} \transpose{\vec{w}} \psi(\vec{x}_i, \vec{y}) \geq 1, \quad \forall i \in [n].
\end{align*}
Here, again the maximum would need a specialized algorithm.

To quantify errors, we introduce a loss function $\Delta: \mathcal{Y} \times \mathcal{Y} \to \R$,
where $\Delta(\vec{y}, \vec{y}')$ is the loss of predicting $\vec{y}'$ when the correct output is
$\vec{y}$. Reformulating the optimization problem with slack results in the following,
\begin{align*}
    \text{minimize}   & \quad \frac{1}{2} \| \vec{w} \|^2 + C \sum_{i=1}^{n} \xi_i                                                                                                                                  \\
    \text{subject to} & \quad \transpose{\vec{w}} \psi(\vec{x}_i, \vec{y}_i) - \max_{\vec{y} \neq \vec{y}_i} \transpose{\vec{w}} \psi(\vec{x}_i, \vec{y}) \geq \Delta(\vec{y}, \vec{y}_i) - \xi_i, \quad \forall i \in [n], \vec{y} \neq \vec{y}_i.
\end{align*}
Or, equivalently with (significantly) less constraints,
\begin{align*}
    \text{minimize}   & \quad \frac{1}{2} \| \vec{w} \|^2 + C \sum_{i=1}^{n} \xi_i                                                                                                                                  \\
    \text{subject to} & \quad \transpose{\vec{w}} \psi(\vec{x}_i, \vec{y}_i) - \max_{\vec{y} \neq \vec{y}_i} \lft\{ \transpose{\vec{w}} \psi(\vec{x}_i, \vec{y}) + \Delta(\vec{y}, \vec{y}_i) \rgt\} \geq - \xi_i, \quad \forall i \in [n].
\end{align*}

In conclusion, we need to design the following four elements,
\begin{enumerate}
    \item Joint feature map $\psi: \mathcal{X} \times \mathcal{Y} \to \R^d$;
    \item Loss function $\Delta: \mathcal{Y} \times \mathcal{Y} \to \R$;
    \item Algorithm during training to compute \[
            \tilde{\vec{y}} \in \argmax_{\vec{y} \neq \vec{y}_i} \lft\{ \transpose{\vec{w}} \psi(\vec{x}_i, \vec{y}) + \Delta(\vec{y}, \vec{y}_i) \rgt\};
        \]
    \item Inference algorithm to compute \[
            \hat{\vec{y}} \in \argmax_{\vec{y} \in \mathcal{Y}} \transpose{\vec{w}} \psi(\vec{x}, \vec{y}).
    \]
\end{enumerate}
