\section{Autoregressive models}

We saw that VAEs are approximate models, since they cannot exactly compute the likelihood
$p(\vec{x})$ to maximize it. \textit{Autoregressive models} solve this by computing the likelihood
with the chain rule, \[
    p(\vec{x}) = \prod_{i=1}^{n} p(x_i\mid \vec{x}_{1:i-1}).
\]

Autoregressive models use data from the same input variable at previous timesteps. This is where
they get their name from; they perform regression of themselves. In particular, autoregressive
models generate one element of the sequence at a time, conditioning on previously generated
elements. Thus, it takes $\vec{x}_{1:k}$ as input and outputs $x_{k+1}$. Because of this, they can
build a probability distribution over possible sequences, using the chain rule as above.

The hard part of this approach is that we must parametrize all possible conditional distributions
$p(x_{k+1} \mid \vec{x}_{1:k})$. Suppose we have a binary image consisting of $n$ pixels. Then, we
need \[
    \sum_{i=1}^{n} 2^{i-1} \in \bigo{2^n}
\]
parameters to parametrize this model. Thus, we need to make additional assumptions to find more
compact representations of the distribution.\sidenote{A naive solution would be to assume that all
    points are independent, which would result in \[
        p(\vec{x}) = \prod_{i=1}^{n} p(x_i).
    \]
    Then, we only require $n$ parameters. However, in practice, this would result in a random sampling
    of pixels, making the generations incoherent.} We will explore the idea of learning a function
$f_i: \{ 0,1 \}^{i-1} \to [0,1]$, parametrized by $\vec{\theta}_i$, which takes as input the
previous pixels and outputs the parameter for the Bernoulli distribution. The total number of
parameters is \[
    \sum_{i=1}^{n} |\vec{\theta}_i|.
\]

Furthermore, we need to think about in which order we generate the pixels of the image.
Intuitively, it would make sense to do left-right top-down or top-down left-right. However, in
practice, randomly ordering the pixels works just as well. One just needs to make sure that the
ordering remains the same for all data points.

\subsection{Fully visible sigmoid belief network}

In a \textit{fully visible sigmoid belief network} (FVSBN) \citep{frey1998graphical}, each timestep
has its own function $f_i$ that is modeled by logistic regression, \[
    f_i(\vec{x}_{1:i-1}) = \sigma\lft(\alpha_0^{(i)} + \alpha_1^{(i)} x_1 + \cdots + \alpha_{i-1}^{(i)} x_{i-1}\rgt).
\]

At the $i$-th timestep, we have $i$ parameters denoted by $\vec{\theta}_i = [\alpha_0, \ldots,
    \alpha_{i-1}]$. Thus, the total number of parameters is \[
    \sum_{i=1}^{n} |\bm{\theta}_i| = \sum_{i=1}^{n} i = \frac{n^2 + n}{2} \in \bigo{n^2},
\]
which is much better than the exponential number of parameters we had before.

\subsection{Neural autoregressive density estimator}

The problem with FVSBN is that they are likely not expressive enough for any meaningful tasks,
since they only consist of a single linear layer for each timestep. The \textit{neural
    autoregressive density estimator} (NADE) \citep{uria2016neural} offers an alternative
parametrization based on MLPs, where we have hidden layers, increasing expressivity, and the
weights are shared between timesteps, decreasing the number of parameters. Specifically, the hidden
layer activations can be computed by the following,
\begin{align*}
    \vec{h}_i & = \sigma(\mat{W}_{:,1:i-1} \vec{x}_{1:i-1} + \vec{b}) \\
    \hat{x}_i & = \sigma\lft(\mat{V}_{i,:} \vec{h}_i + c_i\rgt).
\end{align*}
The advantage of shared parameters is that the total number of parameters gets reduced from
$\bigo{n^2 d}$ to $\bigo{nd}$, and the hidden unit activations can be evaluated in $\bigo{nd}$
by using an alternative recursive definition,
\begin{align*}
    \vec{a}_0     & = \vec{0}                              \\
    \vec{a}_{i+1} & = \vec{a}_i + \mat{W}_{:,i} x_i        \\
    \vec{h}_i     & = \sigma\lft(\vec{a}_i + \vec{b}\rgt).
\end{align*}

Since NADE is a model for binary data, $\hat{x} \in [0,1]$ is the probability $p(x_i \mid
    \vec{x}_{1:i-1})$ at each timestep $i$. NADE is trained by maximizing the average log-likelihood,
\begin{align*}
    \frac{1}{T} \sum_{t=1}^{T} \log p\lft( \vec{x}^{(t)} \rgt) & = \frac{1}{T} \sum_{t=1}^{T} \log \prod_{i=1}^n p \lft( x_i^{(t)} \mid \vec{x}_{1:i-1}^{(t)} \rgt)                                                                                            \\
                                                               & = \frac{1}{T} \sum_{t=1}^{T} \sum_{i=1}^n \log p \lft( x_i^{(t)} \mid \vec{x}_{1:i-1}^{(t)} \rgt). \margintag{Thus, we optimize the log-likelihood exactly, which was not possible for VAEs.}
\end{align*}

During training, a \textit{teacher forcing} approach is used, where the ground truth values of
pixels are used for conditioning, rather than the predicted ones. This leads to more stable
training. However, at inference time we use the predicted value, since we do not have access to the
ground truth.

\subsection{Masked autoencoder distribution estimation}

The idea behind \textit{masked autoencoder distribution estimation} (MADE) \citep{germain2015made}
is to construct an autoencoder which fulfills the autoregressive property, such that its outputs
can be used as conditionals. In order to achieve this, no computational path between output unit
$\hat{x}_{k+1}$ and any of the input units $x_{k+1}, \ldots, x_n$ may exist. This has the result
that $\hat{x}_{k+1}$ is only conditioned on $\vec{x}_{1:k}$, relative to the ordering.

The way to do this is to first assign a number from $1$ to $n$ to each input unit $x_i$, which we
call the ordering. Then, for each hidden unit, uniformly sample an integer $m$, between $1$ and
$n-1$. In the hidden layers, we then only allow units in layer $\ell$ to propagate to units in
layer $\ell+1$ with higher or equal value. Finally, we allow connections between the last hidden
layer and the output only to units with value that is strictly greater. Let $m^{(\ell)}(k)$ be the
value assigned to the $k$-th element of layer $\ell$, then these constraints can be encoded in mask
matrices,
\begin{align*}
    \mat{M}^{\mat{W^{(\ell)}}}_{ij} & = \mathbb{1}\lft\{ m^{(\ell-1)}(j) \leq m^{(\ell)}(i) \rgt\} \\
    \mat{M}^{\mat{V}}_{ij}          & = \mathbb{1}\lft\{ m^{(\ell-1)}(j) < m^{(\ell)}(i) \rgt\}.
\end{align*}
Then, we alter the weight matrices by
\begin{align*}
    \bar{\mat{W}}^{(\ell)} & = \mat{W}^{(\ell)} \odot \mat{M}^{\mat{W^{(\ell)}}} \\
    \bar{\mat{V}}          & = \mat{V} \odot \mat{M}^{\mat{V}},
\end{align*}
and use those instead.

\begin{marginfigure}[-7cm]
    \centering
    \incfig{made-masking}
    \caption{MADE masking with $n=3$.}
    \label{fig:made-masking}
\end{marginfigure}

However, a problem with this approach is that it requires very large hidden layers to retain
expressivity. And, while it is possible to compute $p(\vec{x})$ in a single pass, sampling still
requires $n$ passes.

\subsection{Generating images}

\paragraph{Pixel-RNN.}

\begin{marginfigure}
    \centering
    \incfig{pixel-rnn}
    \caption{Pixel-RNN generation process.}
    \label{fig:pixel-rnn}
\end{marginfigure}

The idea behind \textit{Pixel-RNN} \citep{van2016pixel} is to generate image pixels starting from
the corner and modeling the dependency on previous pixels using an RNN. In particular, a pixel
value is dependent on its top-left neighboring pixels and the RNNs hidden state; see
\Cref{fig:pixel-rnn}. However, the problem with this approach is that the generation of new pixels
depends on the hidden state, making the generation process sequential, and thus slow.

\paragraph{Pixel-CNN.}

\begin{marginfigure}
    \centering
    \incfig{pixel-cnn}
    \caption{Pixel-CNN generation process. The black pixel depends explicitly on the yellow pixels,
        where the thick-lined pixels denote the masked convolutional layer. The black pixel should
        also depend on the gray pixels, but it does not due to the way the stacked masked
        convolutions work; a blind spot.}
    \label{fig:pixel-cnn}
\end{marginfigure}

We can solve the problem of Pixel-RNN by assuming that pixel values only depend on a context region
around the pixel. This is exactly what the \textit{Pixel-CNN} does \citep{van2016pixel}. This
allows for parallelization during training, because the context region values are known during
training.\sidenote{However, we still have to sequentially predict every token during inference, but
    this is an issue with all autoregressive models.} Just as Pixel-RNN, it starts from the top-left
corner, but it models the dependencies with a CNN. During training, we need to make sure that only
the previously generated pixels are used for prediction of the next, thus we need a mask that masks
out all unknown pixel values. However, stacking layers of masked convolutions creates a blind spot
in the convolution; see \Cref{fig:pixel-cnn}. The solution to this is to combine horizontal and
vertical stacks of convolutions \citep{van2016conditional}. The former conditions on the row so
far, while the latter conditions on all rows above. The final output is obtained by summing the two
outputs.

To enforce the autoregressive property, the model also needs to go over the color channels in an
autoregressive manner. Thus the conditional probability is expressed as \[
    p(\vec{x}_i \mid \vec{x}_{1:i-1}) = p(x_{i,\mathrm{R}} \mid \vec{x}_{1:i-1}) p(x_{i,\mathrm{G}} \mid \vec{x}_{1:i-1}, x_{i,\mathrm{R}}) p(x_{i,\mathrm{B}} \mid \vec{x}_{1:i-1}, x_{i,\mathrm{R}}, x_{i,\mathrm{G}}).
\]

\subsection{Generating audio}

\begin{marginfigure}
    \centering
    \incfig{dilated-convolutions}
    \caption{Stacked dilated convolutional layers in WaveNet.}
    \label{fig:dilated-convolutions}
\end{marginfigure}

\textit{WaveNet} \citep{oord2016wavenet} adapts the Pixel-CNN framework for audio data. However,
audio typically has much longer time horizons, since they consist of 16000 samples per second. To be
able to capture these long-term dependencies efficiently, WaveNet incorporates \textit{dilated
    convolutions} \citep{yu2016multiscale}. This allows for an exponential increase in the receptive
field. In dilated convolutions, the filter is applied with gaps between the filter elements. WaveNet
increases the dilation factor as we go up in the layers to attain an exponential receptive field; see
\Cref{fig:dilated-convolutions}.

\subsection{Variational RNN}

RNNs can also be used to generate sequences by sampling $\vec{h}^{(0)}$ and then sequentially
predicting the next element, and updating the hidden state. However, the generation of new
sequences is very slow, because of the sequential nature of the generation process. Another
limitation of vanilla RNNs is that their structure is entirely deterministic and thus limited in
expressive power.

The \textit{variational RNN} \citep{chung2015recurrent} (VRNN) is a recurrent version of the VAE. A
different VAE is used at each timestep. However, the VAEs are not the same as we have seen. In
particular, the prior distribution is no longer a standard Gaussian, but follows the distribution
$\mathcal{N}\lft(\vec{\mu}_0^{(t)}, \mathrm{diag}\lft(\vec{\sigma}_0^{(t)}\rgt)\rgt)$, where
$\lft(\vec{\mu}_0^{(t)}, \vec{\sigma}_0^{(t)}\rgt)$ is predicted by the network
$\phi_{\mathrm{prior}}\lft(\vec{h}^{(t)}\rgt)$ based on the hidden state of the RNN at each
timestep. This introduces randomness to the generation, whereas RNNs are typically deterministic.

The \textit{conditional VRNN} (C-VRNN) is an extension to the VRNN, where we can condition on \eg
style or, in the case of MNIST, which digit should be generated. During training, the model
predicts the conditional variable, called the posterior, and hidden state from the input
$\vec{x}^{(t)}$. From the hidden state it also predicts the conditional variable, called the prior.
It then minimizes the reconstruction error from the style variable, and the KL divergence between
prior and posterior. During inference, it only uses the priors for next token prediction, predicted
from the hidden state, since it does not have access to $\vec{x}^{(t)}$. Then, given the predicted
token, it predicts the posterior conditional variables, which are used to predict the next hidden
state $\vec{h}^{(t+1)}$.

\subsection{Transformers}

Transformers \citep{vaswani2017attention} are used in nearly all the state-of-the-art models at the
moment. At the basis of transformers lies the self-attention mechanism, where we extract key,
value, and query representations from the input. These are then used in a soft-lookup mechanism,
where the query and key values decide how much attention is paid to the values.\sidenote{Note the
    parallel with dictionaries in programming languages, which use a hard-lookup.}

It computes the keys, values, and queries by
\begin{align*}
    \mat{K} & = \mat{X} \mat{W}_K  \\
    \mat{V} & = \mat{X} \mat{W}_V  \\
    \mat{Q} & = \mat{X} \mat{W}_Q,
\end{align*}
where $\mat{X} \in \R^{n\times d}$ and $\mat{W}_K,\mat{W}_V,\mat{W}_Q \in \R^{d\times d}$. Now, we have
key, value, and query representations of the input that are all in $\R^{n \times d}$. Then, we can
compute the output by \[
    \mat{Y} = \mathrm{softmax}\lft( \frac{\mat{Q} \transpose{\mat{K}}}{\sqrt{d}} \rgt) \mat{V}.
\]
Intuitively, the softmax computes how much ``attention'' should be given to each of the previous
timesteps by outputting a probability distribution over timesteps.

Since $\mat{Y} \in \R^{n \times d}$ is of the same dimensionality as $\mat{X}$, we can stack
self-attention layers. This is the basis of the transformer architecture. However, this does not
respect the autoregressive property, thus we need to use a mask $\mat{M}$ to prevent the model from
accessing future timesteps, where \[
    \mat{M} = \begin{bmatrix}
        -\infty & -\infty & \cdots & -\infty \\
        0       & -\infty & \cdots & -\infty \\
        \vdots  & \vdots  & \ddots & \vdots  \\
        0       & 0       & \cdots & -\infty
    \end{bmatrix}.
\]
Then, the attention computation becomes \[
    \mat{Y} = \mathrm{softmax}\lft( \frac{\mat{Q} \transpose{\mat{K}}}{\sqrt{d}} + \mat{M} \rgt) \mat{V}.
\]

The computational complexity of self-attention is $\bigo{n^2 d}$, which is quite high for large
inputs, such as audio and images. But, we gain a maximum path length of $\bigo{1}$ to any of the
previous timesteps, making sure that no information gets ``forgotten'', and it allows for easy
error propagation during training, in contrast to RNNs.
