\section{Connectionism}

\subsection{McCulloch-Pitts neuron}

One of the first approaches to modeling functions of nervous functions with an abstract
mathematical model is the McCulloch-Pitts neuron \citep{mcculloch1943logical}. It treats neurons as
linear threshold elements, which receive and integrate a large number of inputs and produce a
Boolean output. More specifically, it receives $\vec{x} \in \{ 0,1 \}^n$ as input and has
$\vec{\sigma} \in \{ -1, 1 \}^n, \theta \in \R$ as parameters. Its transfer function is formalized
as \[
    f[\vec{\sigma}, \theta](\vec{x}) = \mathbb{1}\{ \transpose{\vec{\sigma}} \vec{x} \geq \theta \}.
\]
The synapses $\vec{\sigma}$ are inhibitory if $-1$ and excitatory if $+1$. However, the problem
with this model is that it does not specify how to set or adjust its parameters.

\subsection{Perceptron}

The perceptron \citep{rosenblatt1958perceptron} is the first model to perform supervised learning,
where patterns are represented as feature vectors $\vec{x} \in \R^d$ and have binary class
memberships $y \in \{ -1, +1 \}$. \citet*{rosenblatt1958perceptron} proposed to use a linear
threshold unit with synaptic weights $\vec{w} \in \R^d$ and threshold $b \in \R$, \[
    f[\vec{w}, b](\vec{x}) = \mathrm{sgn}\lft( \transpose{\vec{w}}\vec{x} + b \rgt),
\]
where \[
    \mathrm{sgn}(z) \doteq
    \begin{cases}
        +1 & z > 0  \\
        0  & z = 0  \\
        -1 & z < 0.
    \end{cases}
\]

This model implicitly induces a decision boundary, where \[
    \transpose{\vec{w}} \vec{x} + b \condeq 0 \iff \frac{\transpose{\vec{w}}\vec{x}}{\| \vec{w} \|} + \frac{b}{\| \vec{w} \|} \condeq 0.
\]
The perceptron thus models the decision boundary as a hyperplane in $\R^n$ with normal vector
$\nicefrac{\vec{w}}{\| \vec{w} \|}$ and $\nicefrac{-b}{\| \vec{w} \|}$ is the signed distance of
the hyperplane to the origin.\sidenote{In Hesse normal form, a hyperplane is formulated by \[
        \transpose{\vec{n}} \vec{x} - d = 0,
    \]
    where $\vec{n}$ is a unit vector and $d$ is the shortest distance of the hyperplane to the origin.
    The distance of any point $\vec{x}$ to the hyperplane is given by $|\transpose{\vec{n}} \vec{x} -
        d|$.} Furthermore, we can formalize how bad/good the model is for a data point by the signed
distance function, \[
    \gamma[\vec{w}, b](\vec{x}, y) = \frac{y \lft( \transpose{\vec{w}} \vec{x} + b \rgt)}{\| \vec{w} \|}.
\]
The sign of $\gamma(\cdot, \cdot)$ encodes the correctness of the classification. The following is
a short proof of this fact,
\begin{align*}
    f[\vec{w}, b](\vec{x}) = y & \iff \mathrm{sgn}\lft( \transpose{\vec{w}} \vec{x} + b \rgt) = y               \\
                               & \iff \mathrm{sgn}\lft( y \lft( \transpose{\vec{w}} \vec{x} + b \rgt) \rgt) = 1 \\
                               & \iff \mathrm{sgn}\lft( \gamma[\vec{w}, b](\vec{x}, y) \rgt) = 1                \\
                               & \iff \gamma[\vec{w}, b](\vec{x}, y) > 0.
\end{align*}
We define the margin of a classifier on training data $\mathcal{S}$ as the minimum signed distance, \[
    \gamma[\vec{w}, b](\mathcal{S}) = \min_{(\vec{x}, y) \in \mathcal{S}} \gamma[\vec{w},b](\vec{x}, y).
\]
If $\gamma[\vec{w},b](\mathcal{S}) > 0$, then the dataset has been linearly separated by a
hyperplane, formed by the parameters, \ie, all classifications are correct. Moreover, we say that
$f[\vec{w}, b]$ $\gamma$-separates the dataset $\mathcal{S}$ if $\gamma[\vec{w}, b](\mathcal{S})
    \geq \gamma$, where $\gamma$ is also called the margin.\sidenote{Finding the maximum margin
    classifier is known as support vector machines.}

\begin{marginfigure}
    \centering
    \incfig{linear-separability}
    \caption{Linear separability of negative and positive data points.}
    \label{fig:linear-separability}
\end{marginfigure}

The version space---see \Cref{fig:version-space}---is defined as the set of all model
parametrizations that correctly classify the data, \[
    \mathcal{V}(\mathcal{S}) \doteq \{ (\vec{w}, b) \mid \gamma[\vec{w}, b](\mathcal{S}) > 0 \} \subseteq \R^{d+1}.
\]
Geometrically, the version space of a linear threshold classifier is the intersection of $n$
half-spaces in $\R^{d+1}$ as each data point imposes a linear constraint on the parameters. Hence,
$\mathcal{S}$ is linearly separable if and only if $\mathcal{V}(\mathcal{S}) \neq \emptyset$.
Adding data points to the dataset can only shrink the version space.

\begin{marginfigure}
    \centering
    \incfig{version-space}
    \caption{In this version space, every line represents a data point's halfspace in which it is correctly classified. As can be seen, adding data points can only shrink the version space.}
    \label{fig:version-space}
\end{marginfigure}

\paragraph{The perceptron algorithm.} The groundbreaking aspect of \citep{rosenblatt1958perceptron} is that it specified how to
iteratively adjust the weights to provably find a solution for a linearly separable
dataset.\sidenote{A solution is defined as any parameters that correctly classify all data points.}
(However, it is not guaranteed that it will find a classifier with a small error if the data is not
linearly separable.) Given a dataset $\mathcal{S} = \{ (\vec{x}_i,y_i) \}_{i=1}^n$, the perceptron
algorithm aims to find some solution $(\vec{w},b) \in \mathcal{V}(\mathcal{S})$. Note that this
means that it does not aim to find classifiers with small error if $\mathcal{V}(\mathcal{S}) =
    \emptyset$.

The perceptron algorithm is a mistake-driven algorithm, meaning that it will only consider data
points that are misclassified by the current parameters. Given a misclassified data point
$(\vec{x}, y) \in \mathcal{S}$, \ie, $f[\vec{w},b](\vec{x}) \neq y$, it has the following update
rule,
\begin{align*}
    \vec{w} & \gets \vec{w} + y \vec{x} \\
    b       & \gets b + y.
\end{align*}
We keep going through the dataset, presenting data points in an arbitrary order, until every data point is correctly classified---see
\Cref{alg:perc-alg}. Note that this algorithm will never converge if $\mathcal{S}$ is not linearly
separable. If we start from $\vec{w}_0 \in \mathrm{span}(\vec{x}_1, \ldots, \vec{x}_n)$, then all weight vectors will remain in the span of the data, \[
    \vec{w}_t \in \mathrm{span}(\vec{x}_1, \ldots, \vec{x}_n), \quad \forall t.
\]

\begin{algorithm}[t]
    \begin{algorithmic}
        \State $\vec{w} \gets \vec{0}$
        \State $b \gets 0$
        \State $\mathrm{mistake} \gets \mathtt{true}$
        \While{$\mathrm{mistake} = \mathtt{true}$}
        \State $\mathrm{mistake} \gets \mathtt{false}$
        \For{$(\vec{x}, y) \in \mathcal{S}$}
        \If{$f[\vec{w},b](\vec{x}) \neq y$}
        \State $\vec{w} \gets \vec{w} + y \vec{x}$
        \State $b \gets b + y$
        \State $\mathrm{mistake} \gets \mathtt{true}$
        \EndIf
        \EndFor
        \EndWhile
        \State \Return $\lft( \vec{w}, b \rgt)$
    \end{algorithmic}
    \caption{The perceptron algorithm.}
    \label{alg:perc-alg}
\end{algorithm}

\begin{marginfigure}[-7cm]
    \centering
    \incfig{perceptron-algorithm}
    \caption{One update of the perceptron algorithm without a bias.}
    \label{fig:perceptron-algorithm}
\end{marginfigure}

\paragraph{Proof of convergence.} In order to prove convergence of the perceptron algorithm for linearly separable data, we will
restrict ourselves to the case where no bias is necessary and $b=0$. We denote the weights after
$t$ updates of the perceptron algorithm (ignoring correctly classified samples) as $\vec{w}_t$ and
we initialize $\vec{w}_0 = \vec{0}$.

\begin{lemma}
    \label{lem:perc-1}

    Let $\vec{w} \in \R^d$ with $\| \vec{w} \| = 1$ and $\gamma \doteq \gamma[\vec{w}](\mathcal{S}) >
        0$. (\Ie, $\mathcal{S}$ is $\gamma$-separable with $\vec{w}$.) Then, \[
        \transpose{\vec{w}} \vec{w}_t \geq t \gamma, \forall t.
    \]
\end{lemma}

\begin{proof}
    This can easily be shown by a recursion,
    \begin{align*}
        \transpose{\vec{w}} \vec{w}_{t+1} & = \transpose{\vec{w}}(\vec{w}_t + y \vec{x}) \margintag{Perceptron update.}                                                                        \\
                                          & = \transpose{\vec{w}}\vec{w}_t + y \transpose{\vec{w}}\vec{x} \margintag{Linearity.}                                                               \\
                                          & = \transpose{\vec{w}}\vec{w}_t + \gamma[\vec{w}](\vec{x}, y) \margintag{$\| \vec{w} \| = 1$.}                                                      \\
                                          & \geq \transpose{\vec{w}} \vec{w}_t + \gamma. \margintag{$\gamma = \min_{\vec{x}, y} \gamma[\vec{w}](\vec{x}, y) \leq \gamma[\vec{w}](\vec{x},y)$.}
    \end{align*}
    Now, it is easy to show the result by induction, starting from $\vec{w}_0 = \vec{0}$.
\end{proof}

\begin{lemma}
    \label{lem:perc-2}

    Let $R \doteq \max_{\vec{x} \in \mathcal{S}} \| \vec{x} \|$, then \[
        \| \vec{w}_t \| \leq R \sqrt{t}.
    \]
\end{lemma}

\begin{proof}
    This can easily be shown by a recursion,
    \begin{align*}
        \| \vec{w}_{t+1} \|^2 & = \| \vec{w}_t + y \vec{x} \|^2 \margintag{Perceptron update.}                                                                  \\
                              & = \| \vec{w}_t \|^2 + \| y \vec{x} \|^2 + 2 y \transpose{\vec{w}_t} \vec{x} \margintag{Cosine theorem.}                         \\
                              & = \| \vec{w}_t \|^2 + \| \vec{x} \|^2 + 2 \| \vec{w}_t \| \gamma[\vec{w}_t](\vec{x}, y)                                         \\
                              & \leq \| \vec{w}_t \|^2 + \| \vec{x} \|^2 \margintag{The perceptron update condition is $\gamma[\vec{w}_t](\vec{x}, y) \leq 0$.} \\
                              & \leq \| \vec{w}_t \|^2 + R^2.
    \end{align*}
    The claim follows by induction, starting from $\vec{w}_0 = \vec{0}$, and taking the square root.
\end{proof}

\begin{theorem}[\citep{novikoff1962convergence}]
    Let the dataset $\mathcal{S}$ be $\gamma$-separable and $R \doteq \max_{\vec{x} \in \mathcal{S}} \| \vec{x} \|$,
    then the perceptron algorithm converges in less than $\floor{\nicefrac{R^2}{\gamma^2}}$ steps.
\end{theorem}

\begin{proof}
    By \Cref{lem:perc-1,lem:perc-2}, we have the following inequality, \[
        1 \geq \cos \angle{(\vec{w}, \vec{w}_t)} = \frac{\transpose{\vec{w}} \vec{w}_t}{\| \vec{w} \| \| \vec{w}_t \|} = \frac{\transpose{\vec{w}}\vec{w}_t}{\| \vec{w}_t \|} \geq \frac{t \gamma}{R \sqrt{t}} = \frac{\sqrt{t} \gamma}{R}.
    \]
    When $\nicefrac{\sqrt{t} \gamma}{R} = 1$, then $\cos \angle{(\vec{w}, \vec{w}_t)} = 1$, which means
    that $\vec{w}_t$ has converged, since $\vec{w} \in \mathcal{V}(\mathcal{S})$. Hence, we have the
    following sufficient condition for convergence, \[
        t = \frac{R^2}{\gamma^2}.
    \]
    It might converge earlier, so we have an upper bound on the number of timesteps until convergence.
    Since $t$ is integer, this bound is $\floor{\nicefrac{R^2}{\gamma^2}}$.
\end{proof}

This theorem does not only guarantee convergence of the perceptron algorithm, but also relates the
separation margin $\gamma$ to the number of steps necessary for convergence. If $\gamma$ is large,
it should be easier to find parameters that classify all data points correctly than if $\gamma$ is
small, because then you have to be very precise---see \Cref{fig:linear-separability}.

However, the problem with this approach is that it requires linear separability of the data, which
is not fulfilled for simple problems like the XOR \citep{minsky1969introduction}, \[
    \mathcal{S} = \lft\{ \lft( \begin{bmatrix} 0 \\ 0 \end{bmatrix},1 \rgt), \lft( \begin{bmatrix} 1 \\ 1 \end{bmatrix},1 \rgt), \lft( \begin{bmatrix} 0 \\ 1 \end{bmatrix},-1 \rgt), \lft( \begin{bmatrix} 1 \\ 0 \end{bmatrix},-1 \rgt) \rgt\}.
\]

\paragraph{Number of unique linear classifications.}

Assume that we are given a dataset $\mathcal{S} \subset \R^d$ of $n$ points, then we define the set
of possible linear classifications of this dataset as, \[
    \mathcal{C}(\mathcal{S}, d) \doteq \lft| \lft\{ \vec{y} \in \{ -1, +1 \}^n \;\middle|\; \exists \vec{w} \in \R^d : \forall i \in [n] : y_i \lft( \transpose{\vec{w}} \vec{x}_i \rgt) > 0 \rgt\} \rgt|.
\]
We assume points to be in general position, which means that any subset $\Xi \subseteq \mathcal{S}$
with $|\Xi| \leq d$ is linearly independent.\sidenote{This is a very weak condition.}

\begin{theorem}[\citep{cover1965geometrical}]
    Given $n$ $d$-dimensional points in general position, \[
        \mathcal{C}(n+1, d) = 2 \sum_{i=0}^{d-1} \begin{pmatrix} n \\ i \end{pmatrix}.
    \]
\end{theorem}

\begin{proof}
    It is easy to show that the initial values are \[
        \mathcal{C}(1,d) = 2, \quad \mathcal{C}(n, 1) = 2, \quad \forall n, d \in \N. \margintag{An important insight for the second case is that $\vec{w}$ cannot move from the origin, because there is no bias.}
    \]
    Consider a realizable classification of $n$ points. \Ie, any classification of all $\vec{x} \in
        \mathcal{S}$ that is linearly separable. This classification has a non-empty version space
    $\mathcal{V}$. Let $\vec{x}_{n+1}$ be a pattern that we add to $\mathcal{S}$. This gives us two new
    version spaces, \[
        \mathcal{V}^+ \doteq \mathcal{V} \cap \lft\{ \vec{w} \;\middle|\; \transpose{\vec{w}} \vec{x}_{n+1} > 0 \rgt\}, \quad \mathcal{V}^- \doteq \mathcal{V} \cap \lft\{ \vec{w} \;\middle|\; \transpose{\vec{w}} \vec{x}_{n+1} < 0 \rgt\},
    \]
    Per existing classification, there are two possible situations,
    \begin{enumerate}
        \item \textit{$\mathcal{V}^+$ and $\mathcal{V}^-$ are non-empty.} Hence, $\vec{x}_{n+1}$ can be
              classified as either $+1$ or $-1$. This is the case if and only if there is a
              $\vec{w} \in \mathcal{V}$ such that $\transpose{\vec{w}} \vec{x}_{n+1} = 0$.\sidenote{Because
                  then we would be able to shift the hyperplane, formed by $\vec{w}$, infinitesimally to
                  allow arbitrary classification of $\vec{x}_{s+1}$ while keeping all other
                  classifications the same.} Recall that we want to know the number of classifications of
              this new dataset $\mathcal{S} \cup \{ \vec{x}_{n+1} \}$. For any classification of
              $\mathcal{S}$ that is in this situation, we can make two new classifications:
              $\vec{x}_{n+1}$ is classified $+1$ or $-1$. There are $\mathcal{C}(n,d-1)$ such
              classifications, because the constraint on $\vec{w}$ makes the problem effectively
              $(d-1)$-dimensional with $n$ data points. Hence, we gain $2 \cdot \mathcal{C}(n,d-1)$ classifications;

        \item \textit{$\mathcal{V}^+$ is non-empty and $\mathcal{V}^-$ is empty or $\mathcal{V}^+$ is empty
                  and $\mathcal{V}^-$ is non-empty.} In this case, we would only be able to create one new
              classification for each existing classification, and there are $\mathcal{C}(n,d) -
                  \mathcal{C}(n,d-1)$ such original classifications. Hence, we gain $\mathcal{C}(n,d) - \mathcal{C}(n,d-1)$ classifications.
    \end{enumerate}
    In conclusion, in total we can create
    \begin{align*}
        \mathcal{C}(n+1,d) & = \mathcal{C}(n,d) - \mathcal{C}(n,d-1) + 2 \cdot \mathcal{C}(n,d-1) \\
                           & = \mathcal{C}(n,d) + \mathcal{C}(n,d-1)
    \end{align*}
    classifications of $n+1$ data points. The claim follows by induction using Pascal's identity.
\end{proof}

\begin{corollary}
    For $n \leq d$, we have $\mathcal{C}(n, d) = 2^n$.
\end{corollary}

It turns out that after $n=2d$, there is a steep decrease in number of linear classifications,
quickly moving toward 0.

\subsection{Parallel distributed processing}

The philosophy behind modern machine learning comes from PDP (\textit{\textbf{P}arallel
    \textbf{D}istributed \textbf{P}rocessing}) \citep{rumelhart1986general,mcclelland1987parallel}. The
elements of PDP are the following,
\begin{enumerate}
    \item \textit{A set of processing units with states of activation}. The basic building blocks of models;
    \item \textit{Output functions for each unit}. They define how the output of the units are computed;
    \item \textit{A pattern of connectivity between units}. They define how the units interact with each other;
    \item \textit{Propagation rules for propagating patterns of activity}. Makes the dependence of the units
          explicit;
    \item \textit{Activation functions for units}. These make the model expressive beyond linearity;
    \item \textit{A learning rule to modify connectivity based on experience}. This is what the training data is used for;
    \item \textit{An environment within which the system must operate}. Formalized by a loss function.
\end{enumerate}
One can change and experiment with each of these design elements. The fact that we
still use this wording says much about the impact that PDP has had on modern deep learning.

\subsection{Hopfield networks}

The Hopfield network \citep{hopfield1982neural} models an associative memory, which aims to
reconstruct a ``memory'' from an input that has been subjected to noise. It does so by defining a
parameterized energy function via second-order interactions between $n$ binary neurons, \[
    H(\vec{X}) \doteq - \frac{1}{2} \sum_{i=1}^{d} \sum_{j=1}^{d} w_{ij} X_i X_j + \sum_{i=1}^{n} b_i X_i, \quad \vec{X} \in \{ -1, +1 \}^d.
\]
The couplings $w_{ij}$ quantify the interaction strength between neurons and the biases $b_i$ act
as thresholds. We constrain the weights such that \[
    w_{ii} = 0, w_{ij} = w_{ji}, \quad \forall i,j \in [d].
\]
Hopfield networks follow a simple dynamic starting from some input state. It iteratively updates
the neurons as follows, \[
    X_i \gets \begin{cases}
        +1 & H([\ldots, X_{i-1}, +1, X_{i+1}, \ldots]) \leq H([\ldots, X_{i-1}, -1, X_{i+1}, \ldots]) \\
        -1 & \text{otherwise}.
    \end{cases}
\]
Hence, $X_i$ becomes the value that minimizes the energy function, given the rest of the state. In
practice, we do not need to evaluate the full energy function for the update---we only need the
effective field per neuron, \[
    H_i \doteq \sum_{j=1}^{n} w_{ij} X_j - b_i.
\]
Then, updates can equivalently be expressed as \[
    X_i \gets \mathrm{sgn}(H_i), \quad \mathrm{sgn}(z) = \begin{cases}
        +1 & z \geq 0 \\
        -1 & z < 0.
    \end{cases}
\]

\paragraph{Hebbian learning.}

The goal of Hopfield networks is to use the update dynamics to evolve noisy stimulus toward a
target pattern. For example, we might want noisy greyscale images to converge to images of numbers
$0$--$9$. Given a set of patterns that we wish to memorize, $\mathcal{S} \subseteq \{ -1, +1 \}^d$,
Hebbian learning involves setting the weights as outer products, \[
    w_{ij} = \frac{1}{d} \sum_{t=1}^{n} x_{t,i} x_{t,j} \implies \mat{W} = \frac{1}{d} \sum_{t=1}^{n} \vec{x}_t \transpose{\vec{x}_t}.
\]
Intuitively, neurons that are frequently in the same state reinforce each other (positive
coupling), whereas neurons that are frequently in opposite states repel each other (negative
coupling).

The minimal requirement of considering a pattern $\vec{x}_t$ as memorized is that it is
meta-stable, \ie, when in the state of a pattern, the update rule will not make any updates,
\begin{align*}
    x_{t,i} & \condeq \mathrm{sgn} \lft( \sum_{j=1}^{d} w_{ij}x_{t,j} \rgt), \quad \forall i \in [d]. \margintag{We ignore biases from now on.}        \\
            & = \mathrm{sgn} \lft( \sum_{j=1}^{d} \frac{1}{d} \sum_{r=1}^{n} x_{r,i} x_{r,j} x_{t,j} \rgt) \margintag{Hebbian learning.}               \\
            & = \mathrm{sgn} \lft( x_{ti} + \underbrace{\frac{1}{d} \sum_{j=1}^{d} \sum_{r\neq t}^{n} x_{r,i} x_{r,j} x_{t,j}}_{\doteq C_{t,i}} \rgt).
\end{align*}
Here, $C_{t,i}$ is the cross-talk between patterns, and ideally $|C_{t,i}| < 1$, for all patterns
$t \in [n]$ and indices $i \in [d]$, because then the minimal requirement is fulfilled.

If we assume that the patterns have \iid random signs and we look at the limit $d \to \infty$, then
we have \[
    C_{t,i} \sim \mathcal{N}\lft( 0, \frac{n}{d} \rgt).
\]
The probability of a single sign being flipped is then \[
    P[-x_{t,i} C_{t,i} \geq 1] \approx \int_1^{\infty} \exp \lft( - \frac{dz^2}{2n} \rgt) \mathrm{d}z = \frac{1}{2} \lft( 1 - \mathrm{erf}\lft( \sqrt{\nicefrac{d}{2n}} \rgt) \rgt).
\]
Hence, the ratio $\nicefrac{n}{d}$ controls the asymptotic error rate. At $\nicefrac{d}{n} \approx
    0.138$, a phase transition occurs, beyond which an avalanche of errors occur. Requiring a pattern
to be retrieved with high probability, one gets a sublinear capacity bound of \[
    s \leq \frac{d}{2 \log_2 (d)}.
\]

Recently, research has been done on increasing the capacity of Hopfield networks by making use of
higher-order energy functions \citep{krotov2016dense,demircigil2017model}. The increased capacity
is the consequence of increased number of local minima in complex cost functions. Furthermore,
\citet{ramsauer2020hopfield} have investigated a connection between Hopfield networks and
transformers.
