\section{Deep neural networks}

Generally, deep learning models consist of a function $H: \R^n \to \R^p$ that extracts
$p$-dimensional features from $n$-dimensional data and a linear map $g$ that makes the final
prediction. The final learned function is then formalized by $\psi = g \circ H$. Machine learning
philosophies differ in the way that they extract features from the data. There are three main
philosophies,
\begin{itemize}
    \item Feature engineering: $H$ is hand-crafted to extract intuitive features that have good predictive
          power of the label. The problem with this approach is that it requires domain expertise;
    \item Feature expansion: $H$ maps to a high-dimensional feature space using kernels and implicit models;
    \item Compositionality: The feature extraction function $H$ is learned through the composition of $L$
          simple building blocks, \[
              H = H_L \circ H_{L-1} \circ \cdots \circ H_2 \circ H_1, \quad H_l : \R^{n_{l-1}} \to \R^{n_l}.
          \]
\end{itemize}

In compositional models, the partial maps $H_{1:l} \doteq H_l \circ \cdots \circ H_1$ produce
intermediate representations. These satisfy the Markov property and, as such, need to preserve
task-relevant information. Once information is lost, it cannot be recovered. The idea of the layers
is to make relevant information more accessible and explicit with increasing depth, such that $g$
can easily make a prediction.\sidenote{For example, the early layers of CNN typically learn
    low-level features, whereas the later layers will learn higher level features that make use of the
    low-level features.}

The key question is how to define the basic building blocks such that their composition is more
powerful than any one block can be. Consider two linear layers, \[
    F(\vec{x}; \mat{W}_2) = \mat{W}_2 \vec{x}, \quad G(\vec{x}; \mat{W}_1) = \mat{W}_1 \vec{x}.
\]
Their composition is again a linear layer, \[
    (F \circ G)(\vec{x}) = \mat{W}_2 \mat{W}_1 \vec{x} = \mat{W} \vec{x}, \quad \mat{W} = \mat{W}_2 \mat{W}_1.
\]
So, linear layers are not appropriate building blocks.

The key idea is to combine a linear map with a non-linearity, \[
    H(\vec{x}; \mat{W}) = \Phi(\mat{W} \vec{x}),
\]
where $\Phi$ is a non-linear element-wise activation function, \[
    \Phi(\vec{z}) = [\phi(z_1), \ldots, \phi(z_m)], \quad \phi: \R\to\R.
\]
Theoretically, a neural network with one hidden layer and a non-polynomial activation function is a
universal function approximator. This means that any function can be represented using a single
hidden layer and a non-linear activation function.\sidenote[][-3cm]{The following are commonly used
    activation functions with their derivatives,
    \begin{itemize}
        \item Sigmoid, $\sigma: \R \to (0,1)$,
              \begin{align*}
                  \sigma(z)  & \doteq \frac{1}{1 + e^{-z}} \\
                  \sigma'(z) & = \sigma(z) (1 - \sigma(z)) \\
                             & = \sigma(z) \sigma(-z);
              \end{align*}
        \item Hyperbolic tangent, $\tanh: \R \to (-1, 1)$,
              \begin{align*}
                  \tanh(z)  & \doteq \frac{e^z - e^{-z}}{e^z + e^{-z}} \\
                  \tanh'(z) & = 1 - \tanh^2(z);
              \end{align*}
        \item Rectified linear unit, $\mathrm{ReLU}: \R \to \R_{\geq 0}$,
              \begin{align*}
                  \mathrm{ReLU}(z)  & \doteq \max \{ 0, z \}   \\
                  \mathrm{ReLU}'(z) & = \mathbb{1}\{ z > 0 \}.
              \end{align*}
    \end{itemize}} However, in practice, this single hidden layer may need to be infinitely large. A
single hidden layer multi-layer perceptron (MLP) is formalized by the following function, \[
    \psi(\vec{x}; \vec{\beta}, \mat{W}) = \transpose{\vec{\beta}} \sigma(\mat{W} \vec{x}) = \sum_{j=1}^{m} \beta_j \sigma\lft(\transpose{\vec{w}_j} \vec{x}\rgt) = \sum_{j=1}^{m} \frac{\beta_j}{1 + \exp\lft( -\transpose{\vec{w}_j} \vec{x} \rgt)}. \margintag{In this case, $g(\vec{y}; \vec{\beta}) = \transpose{\vec{\beta}} \vec{y}$ and $H(\vec{x}; \mat{W}) = \mat{W} \vec{x}$.}
\]
In order to tune this model, we need to be able to compute its gradients, which tell us how to
locally optimize a loss function $\ell$. In this case, we choose a squared loss, $\ell(\hat{y}, y)
    = \frac{1}{2} (\hat{y} - y)^2$. The gradients are computed by
\begin{align*}
    \pdv*{\frac{1}{2} (\psi(\vec{x}; \vec{\beta}, \mat{W}) - y)^2}{\beta_j} & = (\psi(\vec{x}; \vec{\beta}, \mat{W}) - y) \pdv*{\psi(\vec{x}; \vec{\beta}, \mat{W})}{\beta_j}                                                                                                                                                                                   \\
                                                                            & = \frac{\psi(\vec{x}; \vec{\beta}, \mat{W}) - y}{1 + \exp \lft( -\transpose{\vec{w}_j} \vec{x} \rgt)}                                                                                                                                                                             \\
    \pdv*{\frac{1}{2} (\psi(\vec{x}; \vec{\beta}, \mat{W}) - y)^2}{w_{ji}}  & = (\psi(\vec{x}; \vec{\beta}, \mat{W}) - y) \pdv*{\psi(\vec{x}; \vec{\beta}, \mat{W})}{w_{ji}}                                                                                                                                                                                    \\
                                                                            & = (\psi(\vec{x}; \vec{\beta}, \mat{W}) - y) \beta_j \pdv*{\sigma\lft(\transpose{\vec{w}_j} \vec{x}\rgt)}{w_{ji}}                                                                                                                                                                  \\
                                                                            & = (\psi(\vec{x}; \vec{\beta}, \mat{W}) - y) \beta_j \sigma \lft( \transpose{\vec{w}_j} \vec{x} \rgt) \sigma \lft( - \transpose{\vec{w}_j} \vec{x} \rgt) \pdv*{\transpose{\vec{w}_j} \vec{x}}{w_{ji}} \margintag{$\sigma'(z) = \sigma(z) (1 - \sigma(z)) = \sigma(z) \sigma(-z)$.} \\
                                                                            & = \frac{\psi(\vec{x}; \vec{\beta}, \mat{W}) - y}{1 + \exp \lft( - \transpose{\vec{w}_j} \vec{x} \rgt)} \frac{\beta_j x_i}{1 + \exp \lft( \transpose{\vec{w}_j} \vec{x} \rgt)}.
\end{align*}
Learning is typically carried out through stochastic gradient descent, which iteratively selects a random
mini-batch $\mathcal{S}_t \subseteq \mathcal{S} = \{ (\vec{x}_i, y_i) \}_{i=1}^n$. The update rule is
then \[
    \vec{\theta}_{t+1} = \vec{\theta}_t - \eta \sum_{(\vec{x}, y) \in \mathcal{S}_t} \grad{\ell(\psi(\vec{x}; \vec{\theta}), y)}{\vec{\theta}}.
\]

\subsection{Backpropagation}

Calculating the gradient by hand for every model is very tedious and time consuming.
Backpropagation is an algorithm that can compute the gradient of any function, which consists of
building blocks with known gradients, in linear time. This algorithm makes use of dynamic
programming, which means that it breaks the problem down into smaller subproblems and re-uses
solutions to previously seen subproblems. In this case, the solution to the subproblems are the
gradients, which can be re-used in later gradients by the chain rule and sum rule. In short,
backpropagation exploits compositionality to efficiently compute the gradient.

Let $H_k: \R^n \to \R^m$ be an intermediate layer of a compositional model. It's Jacobi map is
defined as \[
    [\mat{J}_k]_{ij} \doteq \pdv{h_{ki}}{z_j}, \margintag{$h_{ki}$ is the $i$-th output of layer $k$ and $z_j$ is its $j$-th input.}
\]
which is an implicit function of the input $\vec{z} \in \R^n$ to $\vec{h}_k$. Furthermore, define
the error signal as \[
    \vec{\delta}_k \doteq \transpose{\lft[ \pdv{\ell}{\vec{h}_k} \rgt]},
\]
which quantifies how a change in loss is induced by a change in unit $k$'s activation. Using the
chain rule, we can find a recurrence relation between error signals, \[
    \vec{\delta}_k \doteq \transpose{\lft[ \pdv{\ell}{\vec{h}_k} \rgt]} = \transpose{\lft[ \pdv{\ell}{\vec{h}_{k+1}} \pdv{\vec{h}_{k+1}}{\vec{h}_k} \rgt]} \doteq \transpose{\mat{J}_{k+1}} \vec{\delta}_{k+1}.
\]
Thus, we can use dynamic programming to compute all error signals efficiently. Lastly, in order to
compute the gradient \wrt the parameters, we use the chain rule again,
\begin{align*}
    \pdv{\ell}{[\mat{W}_k]_{ij}} & = \pdv{\ell}{\vec{h}_k} \pdv{\vec{h}_k}{[\mat{W}_k]_{ij}}                                                                                                         \\
                                 & = \pdv{\ell}{\vec{h}_k} \lft[ \lft. \pdv{\phi(\vec{y})}{\vec{y}} \rgt|_{\vec{y} = \mat{W}_k \vec{z}_{k-1}} \rgt] \pdv*{\mat{W}_k \vec{z}_{k-1}}{[\mat{W}_k]_{ij}} \\
                                 & = \transpose{\vec{\delta}_k} \mathrm{diag} \lft( \phi'(\mat{W}_k \vec{z}_{k-1}) \rgt) \mathrm{vec}_i([\vec{z}_{k-1}]_j)                                           \\
                                 & = [\vec{\delta}_k]_i \cdot \phi'\lft(\transpose{[\mat{W}_k]_i} \vec{z}_{k-1}\rgt) \cdot [\vec{z}_{k-1}]_j.
\end{align*}
Intuitively, the local parameter gradient $\pdv{\ell}{\mat{W}_k}$ is the product of an upstream
vector $\vec{z}_{k-1}$, a downstream error signal $\vec{\delta}_k$, and the local sensitivity of
the unit $\phi'(\mat{W}_k \vec{z}_{k-1})$. Note that we first need to perform a forward pass in
order to compute these gradients in the backward pass.

Moreover, we have the following Jacobi maps for different activation functions,
\begin{itemize}
    \item Linear activation, $\mat{J}_k = \mat{W}$;
    \item ReLU layer, $\mat{J}_k = \mathrm{diag}(\mathbb{1}\{ \mat{W}\vec{z} > 0 \}) \mat{W}$;
    \item Sigmoid layer, $\mat{J}_k = \mathrm{diag}(\sigma(\mat{W} \vec{z}) \odot \sigma(-\mat{W} \vec{z}))
              \mat{W}$.
\end{itemize}

\subsection{Gradient methods}

In gradient descent, we iteratively update the parameters by \[
    \vec{\theta}_{k+1} = \vec{\theta}_k - \eta \grad{\ell(\vec{\theta}_k)}{}, \quad \eta > 0.
\]
A fundamental question is whether gradient descent will converge to an optimal solution. The key
intuition is that gradient descent can only work if the gradient does not change too much relative
to the step size. The gradient information must remain informative in a neighborhood around a
point.

\begin{definition}[Smoothness]
    A differentiable function $\ell: \R^d \to \R$ is $L$-smooth for some $L > 0$ if \[
        \| \grad{\ell(\vec{\theta})}{} - \grad{\ell(\vec{\theta}')}{} \| \leq L \| \vec{\theta} - \vec{\theta}' \|, \quad \forall \vec{\theta}, \vec{\theta}'.
    \]
    This is equivalent to the gradient being $L$-Lipschitz continuous.
\end{definition}

\begin{marginfigure}
    \centering
    \incfig{smoothness}
    \caption{Graph of the gradient of an $L$-smooth 1-dimensional function. Due to smoothness, the
        gradient may only stay within the grey area for every point. Higher $L$ makes the area larger, which means that
        the current gradient is less informative about its surroundings. Thus, we have to be more
        careful when applying gradient descent.}
    \label{fig:smoothness}
\end{marginfigure}

Smoothness implies a bound on the Hessian, \[
    \| \hess{\ell(\vec{\theta})}{} \| \leq L, \quad \forall \vec{\theta}.
\]
As a consequence, this means that smoothness bounds the largest eigenvalue of the Hessian. A large
$L$ means that the gradient can change quickly, making it more unstable, thus we need to lower the
stepsize. This is intuitively why the stepsize $\eta = \nicefrac{1}{L}$ works well.

\begin{definition}[$\epsilon$-critical point]
    Let $\ell$ be differentiable at $\vec{\theta}$, then $\vec{\theta}$ is an $\epsilon$-critical point if \[
        \| \grad{\ell(\vec{\theta})}{} \| \leq \epsilon.
    \]
\end{definition}

\begin{lemma}
    Gradient descent on an $L$-smooth, differentiable function $\ell: \R^d \to \R$ with step size $\eta = \nicefrac{1}{L}$ finds an $\epsilon$-critical point in at most \[
        k = \frac{2L\lft(\ell\lft(\vec{\theta}_0\rgt) - \ell^\star\rgt)}{\epsilon^2}
    \]
    steps.
\end{lemma}

Thus, smoothness is sufficient to find local minima. The question is what properties of $\ell$ will
ensure convergence to a global minimum.

\begin{definition}[Polyak-Lojasiewicz condition]
    A differentiable function $\ell: \R^d \to \R$ satisfies the Polyak-Lojasiewicz (PL) condition with parameter $\mu > 0$ if \[
        \frac{1}{2} \| \grad{\ell(\vec{\theta})}{} \|^2 \geq \mu (\ell(\vec{\theta}) - \ell^\star), \quad \forall \vec{\theta}.
    \]
\end{definition}

\begin{lemma}
    Let $\ell$ be differentiable, $L$-smooth, and $\mu$-PL. Then, gradient descent with stepsize $\eta = \nicefrac{1}{L}$ converges at a geometric rate, \[
        \ell\lft( \vec{\theta}_k \rgt) - \ell^\star \leq \lft( 1 - \frac{\mu}{L} \rgt)^k \lft( \ell(\vec{\theta}_0) - \ell^\star \rgt).
    \]
\end{lemma}

\begin{definition}[Strong convexity]
    A differentiable function $\ell: \R^d \to \R$ is $\mu$-strongly convex for some $\mu > 0$ if \[
        \ell(\vec{\theta}') \geq \ell(\vec{\theta}) + \langle \grad{\ell(\vec{\theta})}{}, \vec{\theta}' - \vec{\theta} \rangle + \frac{\mu}{2} \| \vec{\theta}' - \vec{\theta} \|^2, \quad \forall \vec{\theta}, \vec{\theta}'.
    \]
\end{definition}

Intuitively, strong convexity bounds the smallest eigenvalue of a locally quadratic approximation
of $\ell$.

\begin{lemma}[Strong convexity $\Rightarrow$ PL]
    Let $\ell$ be $\mu$-strongly convex, then it fulfills the PL condition with the same $\mu$.
\end{lemma}

Thus, strong convexity ensures convergence to a global optimum.

\paragraph{Momentum.}

The heavy ball method is an optimization algorithm with momentum, \[
    \vec{\theta}_{k+1} = \vec{\theta}_k - \eta \grad{\ell(\vec{\theta}_k)}{} + \beta (\vec{\theta}_k - \vec{\theta}_{k-1}), \quad \beta \in (0,1).
\]
Intuitively, we are acting as if the iterates have mass. As a consequence, we will pass small
gradient areas faster, and thus overcomes converging to local minima.

\paragraph{Adaptivity.}

AdaGrad uses the gradient history to adapt the stepsize per dimension, \[
    [\vec{\theta}_{k+1}]_i = [\vec{\theta}_k]_i - \eta_i^k \grad{\ell(\vec{\theta}_k)}{i}, \quad \eta_i^k \doteq \frac{\eta}{\sqrt{\gamma_i^k} + \delta},
\]
where \[
    \gamma_i^k = \gamma_i^{k-1} + \lft( \grad{\ell(\vec{\theta}_k)}{i} \rgt)^2.
\]

\paragraph{Acceleration.}

Nesterov's accelerated gradient descent has the following update rule,
\begin{align*}
    \vec{\theta}_{k+1}' & = \vec{\theta}_k + \beta (\vec{\theta}_k - \vec{\theta}_{k-1})   \\
    \vec{\theta}_{k+1}  & = \vec{\theta}_{k+1}' - \eta \grad{\ell(\vec{\theta}_{k+1}')}{}.
\end{align*}
While it is not intuitive why this works, it provides a faster convergence rate than vanilla gradient
descent.

\paragraph{Adam.}

Adam combines momentum and adaptivity to increase convergence speed. It defines the following
variables,
\begin{align*}
    \vec{g}_k & = \beta \vec{g}_{k-1} + (1- \beta) \grad{\ell(\vec{\theta}_k)}{}, \quad \beta \in [0,1]              \\
    \vec{h}_k & = \alpha \vec{h}_{k-1} + (1-\alpha) \grad{\ell(\vec{\theta}_k)}{}^{\odot 2}, \quad \alpha \in [0,1].
\end{align*}
The update rule is then \[
    \vec{\theta}_{k+1} = \vec{\theta}_k - \vec{\eta}_k \odot \vec{g}_k, \quad \vec{\eta}_k = \eta \oslash \lft( \sqrt{\vec{h}_k} + \delta \rgt).
\]

\subsection{Convolutional neural networks}

Images and audio have an extremely high dimensionality, which means that a fully connected layer
would have an extremely high parameter count. However, we can exploit the locality, scale, and
shift invariance of these types of data using a new operator; the convolution.

\begin{definition}[Convolution]
    Given two functions $f,h$, their convolution is defined as \[
        (f*h)(u) \doteq \int_{-\infty}^{\infty} h(u-t)f(t) \mathrm{d}t = \int_{-\infty}^{\infty} f(u-t)h(t) \mathrm{d}t.
    \]
\end{definition}

The convolution operator is shift invariant, meaning that if we shift and then apply the operator,
we get the same result as if we were to first apply the operator and then shift. Formally, \[
    f_{\Delta} * h = (f * h)_{\Delta}
\]
The converse is also true: any linear shift-invariant transformation can be written as a
convolution.

In practice, we have discrete data, which means that we have to define a discrete convolution
operator, \[
    (f*h)[u] \doteq \sum_{t=-\infty}^{\infty} f[t] h[u-t].
\]
This easily extends to two dimensions, \[
    (f*h)[x,y] = \sum_{u=-\infty}^{\infty} \sum_{v=-\infty}^{\infty} f[u,v] h[x-u, y-v].
\]
Typically, $f$ and $h$ are defined over a finite domain.

The cross-correlation operator is equal to a convolution with a flipped kernel, \[
    (f\star h)[u] \doteq \sum_{t=-\infty}^{\infty} h[t] f[u+t].
\]

Convolutional neural networks (CNN) learn the kernel of convolutional layers and stacks them in a
compositional way to extract features from images. In this way, these layers exploit the shift
invariance, locality, and scale of the data. Furthermore, it increases the statistical efficiency
\wrt MLPs, because of the shared parameters.

As MLPs, CNNs alternate between (linear) convolutional layers and non-linear element-wise functions
to increase model capacity. Moreover, it employs max-pooling layers to decrease the dimensionality
of the input. It does so by taking only the maximum of every $k \times k$ patch. This has the
effect that the input is subsampled $k$ times. After many max-pooling layers, the data will no
longer be location dependent, which allows us to throw away spatial information by flattening the
data. From there, linear layers can be used to make the final prediction.
