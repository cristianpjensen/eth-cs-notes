\section{Generative models}

In \textit{discriminative models}, the goal is to learn a function that maps inputs $\vec{x}$ to
their correct output $y$. On the other hand, \textit{generative models} aim to learn the underlying
hidden structure of a dataset by modeling the distribution $p_{\mathrm{model}}$ to generate new
samples that resemble the distribution $p_{\mathrm{data}}$.

\begin{figure}
    \centering
    \incfig{taxonomy}
    \caption{Taxonomy of generative models.}
    \label{fig:taxonomy}
\end{figure}

Generative models can be classified into two main categories:
\begin{itemize}
    \item \textit{Explicit models} explicitly define the probability distribution
          $p_{\mathrm{model}}$ and then sample from it;
    \item \textit{Implicit models} define a model from which we can sample. By being able to sample
          from the model, it implicitly induces a probability distribution $p_{\mathrm{model}}$.
\end{itemize}
See \Cref{fig:taxonomy} for further classifications of different models.
