\section{Normalizing flows}

As we have seen, VAEs learn meaningful representation through latent variables, but they suffer
from intractable marginal likelihoods. On the other hand, autoregressive models have a tractable
likelihood, but lack a latent space and direct feature learning mechanism. \textit{Normalizing
    flows} \citep{rezende2015variational} try to have the best of both worlds; meaningful latent space
and a tractable likelihood. They achieve this by leveraging the change of variable technique of
integration.

\subsection{Change of variables}

In the one-dimensional, integration by substitution works as follows. Let $g: [a,b] \to I$ be a
differentiable function with a continuous derivative and $f: I \to \R$ be a continuous function,
where $I \subset \R$ is an interval. Then, \[
    \int_{g(a)}^{g(b)} f(x)\mathrm{d}x = \int_a^b f(g(u)) g'(u) \mathrm{d}u.
\]

Similarly, we can make the same change of variables transformation to probability distributions.
Let $z \sim p_Z$, $x = f(z)$, where $f(\cdot)$ is a monotonic and differentiable function with an
inverse $z = \inv{f}(x)$. Then, the probability density function of $x$ is \[
    p_X(x) = p_Z\lft(\inv{f}(x)\rgt) \lft|\odv{\inv{f}(x)}{x}\rgt| = p_Z(z) \lft|\odv{\inv{f}(x)}{x}\rgt|.
\]

We can generalize this to any dimensionality by the following, \[
    p_X(\vec{x}) = p_Z \lft( \inv{f}(\vec{x}) \rgt) \lft| \det{\jacob{\inv{f}(\vec{x})}{\vec{x}}} \rgt| = p_Z(\vec{z}) \inv{\lft| \det{\jacob{f(\vec{z})}{\vec{z}}} \rgt|}, \margintag{We need to normalize by the determinant of $f$, because the total probability must be $1$. Recall that the determinant quantifies the volume change of an operation.}
\]
where the last equality is a property of Jacobians of invertible functions. We can view the
absolute determinant as computing the volume of change of $f$. Normalizing flow models parametrize
$f_{\vec{\theta}}$.

Thus, this gives us a way to compute the probability of $\vec{x}$ in an exact and tractable manner.
The downside of this approach is that $f$ must be invertible, thus we must carefully parametrize
the model, which means that we cannot use any model we might want to use. Furthermore, this has the
consequence that it must preserve dimensionality, thus the latent space must be as large as the
data space.

From a computational perspective, we require the Jacobian of the transformation to be computed
efficiently. In general, computing the Jacobian takes $\bigo{d^3}$ to compute for a $d \times d$
matrix. However, this is not fast enough. A way to achieve linear complexity is to design $f$ such
that its Jacobian is a triangular matrix, which takes $\bigo{d}$ to compute.\sidenote{The
    determinant of a triangular matrix is the product of its diagonal entries.} This requirement
further reduces the number of modeling decisions we can make.

\subsection{Coupling layers}

\begin{marginfigure}
    \centering
    \incfig{coupling-layer}
    \caption{Diagram of a coupling layer. $h$ is an invertible element-wise operation and $\beta$ is
        can be arbitrarily complex and does not need to be invertible.}
    \label{fig:coupling-layer}
\end{marginfigure}

A \textit{coupling layer} \citep{dinh2014nice} is a type of network layer that effectively meets
the above requirements of a normalizing flow function. It is invertible and offers efficient
computation of the determinant. It consists of two functions; $\beta$ and $h$. $\beta$ can be any
neural network and does not necessarily need to be invertible.\sidenote{This is very important,
    because requiring a neural network to be invertible would significantly reduce the number of
    available modeling decisions.} $h$ is an element-wise operation that is invertible \wrt its first
argument, given the second. $h(\vec{x}_A, \beta(\vec{x}_B))$ produces the first half of the input
and $\vec{x}_B$ produces the second half.

This gives the following function, \[
    f: \begin{bmatrix} \vec{x}_A \\ \vec{x}_B \end{bmatrix} \mapsto \begin{bmatrix} h(\vec{x}_A, \beta(\vec{x}_B)) \\ \vec{x}_B \end{bmatrix}.
\]
The inverse of this function is given by \[
    \inv{f}: \begin{bmatrix} \vec{y}_A \\ \vec{y}_B \end{bmatrix} \mapsto \begin{bmatrix} \inv{h}(\vec{y}_A, \beta(\vec{y}_B)) \\ \vec{y}_B \end{bmatrix}.
\]
The Jacobian matrix can be efficiently computed by
\begin{align*}
    \jacob{f(\vec{x})}{} & = \begin{bmatrix}
                                 \pdv{\vec{y}_A}{\vec{x}_A} & \pdv{\vec{y}_A}{\vec{x}_B} \\
                                 \pdv{\vec{y}_B}{\vec{x}_A} & \pdv{\vec{y}_B}{\vec{x}_B}
                             \end{bmatrix}                                                                        \\
                         & = \begin{bmatrix}
                                 h'(\vec{x}_A, \beta(\vec{x}_B)) \pdv{\vec{x}_A}{\vec{x}_A} & h'(\vec{x}_A, \beta(\vec{x}_b)) \pdv{\beta(\vec{x}_B)}{\vec{x}_B} \\
                                 \pdv{\vec{x}_B}{\vec{x}_A}                                 & \pdv{\vec{x}_B}{\vec{x}_B}
                             \end{bmatrix} \\
                         & = \begin{bmatrix}
                                 h'(\vec{x}_A, \beta(\vec{x}_B)) & h'(\vec{x}_A, \beta(\vec{x}_B)) \beta'(\vec{x}_B) \\
                                 \mat{0}                         & \mat{I}
                             \end{bmatrix}.
\end{align*}
When implementing this, we notice that this layer leaves part of its input unchanged. The role of
the two subsets in the partition thus gets exchanged in alternating layers, so that both subsets
get updates. In practice, we often randomly choose the splits to ensure proper mixing.

\subsection{Composing transformations}

Often, a single non-linear transformation is insufficient to capture complex patterns. Especially,
because a single layer leaves part of the input unchanged. Thus, to achieve more complex
transformations, we can compose multiple transformations together. We then get the following
function \[
    \vec{x} = f(\vec{z}) = (f_m \circ \cdots \circ f_1)(\vec{z}).
\]
Again using the change of variables, we can then compute the likelihood by \[
    p_X(\vec{x}) = p_Z \lft( \inv{f}(\vec{x}) \rgt) \prod_{k=1}^{m} \inv{\lft| \det{\jacob{f_k(\vec{x})}{}} \rgt|}. \margintag{$\det{\mat{A} \mat{B}} = \det{\mat{A}} \det{\mat{B}}$.}
\]

\subsection{Training and inference}

At training time, we can learn the model by maximizing the exact log likelihood over the dataset, \[
    \log p_X(\mathcal{D}) = \sum_{i=1}^n \log p_Z \lft( \inv{f}(\vec{x}_i) \rgt) + \sum_{k=1}^{m} \log \inv{\lft| \det{\jacob{f_k(\vec{x}_i)}{}} \rgt|}.
\]

At inference time, we generate a sample $\vec{x}$ by drawing a random $\vec{z} \sim p_Z$ and
transform it via $f$, obtaining $\vec{x} = f(\vec{z})$. To evaluate the probability of an
observation, we use the inverse transform to get its latent variable $\vec{z} = \inv{f}(\vec{x})$,
and compute its probability at $p_Z(\vec{z})$. Generally, $p_Z$ is chosen to be a simple
distribution, such as $\mathcal{N}(\vec{0}, \mat{I})$.

\subsection{Architectures}

The main difference between flow architectures is the choice of $h$ and the way it splits the input
into $\vec{x}_A$ and $\vec{x}_B$.

\paragraph{NICE.} NICE \citep{dinh2014nice} splits the data by dividing the input into two parts $\vec{x}_A =
    \vec{x}_{1:\nicefrac{d}{2}}$ and $\vec{x}_B = \vec{x}_{\nicefrac{d}{2}+1:d}$, swapping the order
randomly. In the coupling layer, it uses an additive coupling law and the output is computed as \[
    \begin{bmatrix} \vec{y}_A \\ \vec{y}_B \end{bmatrix} = \begin{bmatrix} \vec{x}_A + \beta(\vec{x}_B) \\ \vec{x}_B \end{bmatrix}.
\]

\begin{marginfigure}
    \centering
    \incfig{nvp-masking}
    \caption{Masking used by RealNVP \citep{dinh2016density}.}
    \label{fig:nvp-masking}
\end{marginfigure}

\paragraph{RealNVP.} RealNVP \citep{dinh2016density} employs a multi-scale architecture that uses two types of
partitioning: checkerboard and channel-wise masking; see \Cref{fig:nvp-masking}. At each scale, the
model alternates between checkerboard patterns. Then, it does a squeezing operation to go from $C
    \times H \times W$ dimensionality to $4C \times \nicefrac{H}{2} \times \nicefrac{W}{2}$. Lastly, it
uses channel-wise masking. This sequence of steps ensures that all data can interact with each
other. Instead of a simple additive $h$ as in NICE, RealNVP implements it as an affine mapping, \[
    \begin{bmatrix} \vec{y}_A \\ \vec{y}_B \end{bmatrix} = \begin{bmatrix} \vec{x}_A \odot \exp(s(\vec{x}_B)) + t(\vec{x}_B) \\ \vec{x}_B \end{bmatrix}.
\]
Here, $s$ and $t$ can be arbitrarily complex.

\paragraph{GLOW.} GLOW \citep{kingma2018glow} is an architecture that utilizes invertible $1 \times 1$ convolutions,
affine coupling layers, and multi-scale architecture. It consists of $L$ levels, each of which is
composed of $K$ steps of flow. The $L$ levels allow for effective processing of all parts of the
input, while the $K$ steps are used to increase the flexibility of the transformation $f$.

A step of flow consists of applying activation norm, an invertible $1 \times 1$ convolution, and a
coupling layer, in order. The activation norm is similar to batch norm, but it normalizes each
input channel. As we saw, a permutation of the input is needed in order to be able to process the
entire input. The $1 \times 1$ convolution with $C$ filters is a generalization of a permutation in
the channel dimension. This allows us to learn the required permutation. The coupling layer is
implemented as in RealNVP, while the split is performed along the channel dimension only, because
the convolution is $1 \times 1$.
