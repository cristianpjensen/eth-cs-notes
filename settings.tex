\setcounter{secnumdepth}{2}
\setcounter{tocdepth}{2}

% Packages
\usepackage[utf8]{inputenc}
\usepackage[T1]{fontenc}
\usepackage[activate={true,nocompatibility},final,tracking=true,kerning=true,spacing=true,factor=1100,stretch=10,shrink=10]{microtype}
\usepackage{tikz}
\usepackage{pgfplots}
\pgfplotsset{
    colormap={diverging}{
            rgb255=(215,25,28)
            rgb255=(253,174,97)
            rgb255=(255,255,191)
            rgb255=(171,221,164)
            rgb255=(43,131,186)
        },
    colormap={sequential}{
            rgb255=(255,255,204)
            rgb255=(161,218,180)
            rgb255=(65,182,196)
            rgb255=(44,127,184)
            rgb255=(37,52,148)
        },
    compat=1.16,
}

\usetikzlibrary{calc,arrows,shapes,positioning}
\usepgfplotslibrary{fillbetween}

\usepackage{xparse}
\usepackage{xcolor}
\usepackage{xspace}
\usepackage{amsmath,amsfonts,amsthm,amssymb,mathrsfs,bbm,mathtools,nicefrac,bm,centernot}
\usepackage{enumitem}
\usepackage{hyperref}
\usepackage[font+=footnotesize,justification=justified,singlelinecheck=false,labelfont=bf,labelsep=period]{caption}
\usepackage[justification=c]{subcaption}
\usepackage{float}
\usepackage{longtable}
\usepackage{derivative}
\usepackage[most]{tcolorbox}
\usepackage[parfill]{parskip}
\usepackage{graphicx}
\usepackage{cleveref}
\usepackage{booktabs}
\usepackage{algpseudocode}
\usepackage{makecell}
\usepackage[framemethod=TikZ]{mdframed}
\usepackage{thmtools}
\usepackage{natbib}

% Figure support as explained in Gilles Castel's blog post
\usepackage{import}
\usepackage{xifthen}
\usepackage{pdfpages}
\usepackage{transparent}
\newcommand{\incfig}[1]{%
    \def\svgwidth{\columnwidth}
    \import{./figures/}{#1.pdf_tex}
}

\geometry{
    paperwidth=210mm,
    paperheight=297mm,
    left=42pt,
    top=72pt,
    textwidth=320pt,
    marginparsep=20pt,
    marginparwidth=180pt,
    textheight=650pt,
    footskip=40pt,
}

% Fix some stuff
% http://tex.stackexchange.com/questions/76273/multiple-pdfs-with-page-group-included-in-a-single-page-warning
\pdfsuppresswarningpagegroup=1

% Theorem environments
\mdfsetup{skipabove=1em,skipbelow=0em, innertopmargin=10pt, innerbottommargin=10pt}

\declaretheoremstyle[headfont=\bfseries, bodyfont=\normalfont, mdframed={ nobreak }]{boxstyle}
\declaretheoremstyle[headfont=\bfseries, bodyfont=\normalfont, numbered=no]{noboxstyle}

\declaretheorem[style=boxstyle, name=Theorem]{theorem}
\declaretheorem[sibling=theorem, style=boxstyle, name=Example]{example}
\declaretheorem[sibling=theorem, style=boxstyle, name=Definition]{definition}
\declaretheorem[sibling=theorem, style=boxstyle, name=Lemma]{lemma}

\declaretheorem[style=noboxstyle, name=Remark]{remark}
\declaretheorem[style=noboxstyle, name=Properties]{properties}
\declaretheorem[style=noboxstyle, name=Observation]{observation}

% Listing environment
\newcounter{listing}
\newcommand\listingname{Listing}
\makeatletter
\newcommand\listoflistings{%
    \section*{List of listings}%
    %  \begin{fullwidth}%
    \@starttoc{loe}%
    %  \end{fullwidth}%
}
\renewcommand\thelisting
{\@arabic\c@listing}
\def\fps@listing{tbp}
\def\ftype@listing{1}
\def\ext@listing{loe}
\def\fnum@listing{\listingname\nobreakspace\thelisting}
\newenvironment{listing}[1][htbp]
{\begin{@tufte@float}[#1]{listing}{}
        \begin{mdframed}[backgroundcolor=black!5]}
            {\end{mdframed}
    \end{@tufte@float}}
\let\l@listing\l@figure
\makeatother

% Algorithm environment
\newcounter{algorithm}
\newcommand\algorithmname{Algorithm}
\makeatletter
\newcommand\listofalgorithms{%
    \section*{List of algorithms}%
    \@starttoc{loe}%
}
\renewcommand\thealgorithm
{\@arabic\c@algorithm}
\def\fps@algorithm{tbp}
\def\ftype@algorithm{1}
\def\ext@algorithm{loe}
\def\fnum@algorithm{\algorithmname\nobreakspace\thealgorithm}
\newenvironment{algorithm}[1][htbp]
{\begin{@tufte@float}[#1]{algorithm}{}
        \begin{mdframed}[backgroundcolor=white]}
            {\end{mdframed}
    \end{@tufte@float}}
\let\l@algorithm\l@figure
\makeatother

\algdef{SE}[REPEATN]{RepeatN}{End}[1]{\algorithmicrepeat\ #1 \textbf{times}}{\algorithmicend}
