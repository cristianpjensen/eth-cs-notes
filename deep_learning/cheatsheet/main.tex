\documentclass[10pt]{article}

\usepackage[a4paper, margin=0.5cm]{geometry}
\usepackage[utf8]{inputenc}
\usepackage{multicol}
\usepackage{amsmath,amsfonts,amsthm,amssymb,mathrsfs,bbm,mathtools,nicefrac,bm,centernot,colonequals,dsfont}
\usepackage{derivative}
\usepackage[skip=.5\baselineskip-2pt]{parskip}
\usepackage[mathspacing=normal]{savetrees}
\usepackage[document]{ragged2e}
\usepackage{enumitem}
\usepackage{microtype}

\usepackage{mathpazo}
\usepackage[scaled=0.90]{helvet}
\usepackage[scaled=0.85]{beramono}
\usepackage[T1]{fontenc}
\usepackage{textcomp}

\usepackage{color,soul}
\usepackage{xcolor}

% Obscure delimiters
\DeclarePairedDelimiter{\floor}{\lfloor}{\rfloor}
\DeclarePairedDelimiter{\ceil}{\lceil}{\rceil}
\DeclarePairedDelimiter{\ip}{\langle}{\rangle}
\DeclarePairedDelimiter{\abs}{\lvert}{\rvert}
\DeclarePairedDelimiter{\norm}{\lVert}{\rVert}

% Better left and right
\newcommand{\lft}{\mathopen{}\left}
\newcommand{\rgt}{\aftergroup\mathclose\aftergroup{\aftergroup}\right}

% Common sets
\newcommand{\R}{\mathbb{R}}
\newcommand{\Q}{\mathbb{Q}}
\newcommand{\N}{\mathbb{N}}
\newcommand{\Z}{\mathbb{Z}}

% Expected value and variance
\newcommand{\E}{\mathbb{E}}
\newcommand{\Var}{\mathrm{Var}}

% Input and output space
\newcommand{\X}{\mathcal{X}}
\newcommand{\Y}{\mathcal{Y}}

% Vector, matrix, and tensor
\renewcommand{\vec}[1]{\bm{#1}}
\newcommand{\mat}[1]{\bm{#1}}
\newcommand{\tens}[1]{\bm{\mathsf{#1}}}

% Big-O notation
\newcommand{\bigo}[1]{\mathcal{O}\lft(#1\rgt)}

% Common functions
\newcommand{\transpose}[1]{#1^\top}
\newcommand{\inv}[1]{#1^{-1}}
\renewcommand{\det}[1]{\mathrm{det}\lft(#1\rgt)}
\newcommand{\tr}[1]{\mathrm{tr}\lft(#1\rgt)}
\newcommand{\diag}[1]{\mathrm{diag}\lft(#1\rgt)}

% Distributions
\newcommand{\sample}{\sim}
\newcommand{\sampleiid}{\overset{\mathrm{iid}}{\sim}}
\NewDocumentCommand{\Gauss}{O{0}O{1}o}{\mathcal{N}\lft(#1,#2 \IfValueT{#3}{\mid #3}\rgt)}
\newcommand{\MultiGauss}{\Gauss[\vec{0}][\mat{I}]}
\NewDocumentCommand{\Unif}{m}{\mathrm{Unif}\parentheses*{#1}}

% Math comment in smaller font size
\newcommand{\mathcomment}[1]{\mathrm{\footnotesize #1}}

% Gradient and Hessian operator
\newcommand{\grad}[2]{\vec{\nabla}_{\! #2} #1}
\newcommand{\hess}[2]{\vec{\nabla}^2_{\! #2} #1}
\newcommand{\jacob}[2]{\vec{D}_{#2} #1}

% List of symbols
\newlist{abbrv}{itemize}{1}
\setlist[abbrv,1]{label=,labelwidth=1in,align=parleft,itemsep=0.1\baselineskip,leftmargin=!}

% Add a period to the end of an abbreviation unless there's one
% already, then \xspace
\makeatletter
\DeclareRobustCommand\onedot{\futurelet\@let@token\@onedot}
\def\@onedot{\ifx\@let@token.\else.\null\fi\xspace}

\def\eg{\emph{e.g}\onedot} \def\Eg{\emph{E.g}\onedot}
\def\ie{\emph{i.e}\onedot} \def\Ie{\emph{I.e}\onedot}
\def\cf{\emph{c.f}\onedot} \def\Cf{\emph{C.f}\onedot}
\def\etc{\emph{etc}\onedot} \def\vs{\emph{vs}\onedot}
\def\wrt{w.r.t\onedot} \def\dof{d.o.f\onedot}
\makeatother

% Argument optima
\DeclareMathOperator*{\argmax}{argmax} % no space, limits underneath in displays
\DeclareMathOperator*{\argmin}{argmin} % no space, limits underneath in displays

% Margin tags
% \usepackage{marginfix}
\let\marginnotetemp\marginnote
\let\marginnote\relax
\usepackage{marginnote}
\let\marginnotepkg\marginnote

\NewDocumentCommand{\margintag}{O{0\baselineskip}m}{%
    \checkoddpage%
    \ifoddpage%
        {\marginnotepkg{\footnotesize #2}[#1]}%
    \else%
        {\reversemarginpar\marginnotepkg{\footnotesize #2}[#1]}%
    \fi}%
% \NewDocumentCommand{\margintagt}{O{0\baselineskip}m}{\marginpar{\vspace{#1}\footnotesize #2}}
% \NewDocumentCommand{\safefootnote}{om}{\footnotemark\margintag[#1]{\textsuperscript{\tiny\arabic{footnote}} \normalfont\justifying #2}}

\renewcommand\marginnote\marginnotetemp



\setlength{\columnseprule}{0.4pt}

\title{Deep Learning Cheatsheet}

\newenvironment{topic}[1]
{\textbf{\sffamily \footnotesize \colorbox{black}{\rlap{\textbf{\textcolor{white}{#1}}}\hspace{\linewidth}\hspace{-2\fboxsep}}}}
{}

\newenvironment{subtopic}[1]
{\begin{center}\textbf{\footnotesize \sffamily #1}\end{center}}
{}

\renewcommand{\det}[1]{|#1|}
\renewcommand{\mat}[1]{\ensuremath{\mathbf{#1}}}
\renewcommand{\vec}[1]{\ensuremath{\mathbold{#1}}}
\renewcommand{\Var}{\mathbb{V}}

\newcommand{\Break}{\vfill\null\columnbreak}

\begin{document}

\setlength{\columnsep}{0.2cm}

\begin{multicols*}{3}

    \scriptsize

    \begin{subtopic}{General}
        \begin{itemize}[left=0pt]
            \item \hl{Cosine theorem: $\| \vec{x}-\vec{y} \|^2 = \| \vec{x} \|^2 + \| \vec{y} \|^2 - 2\langle \vec{x},\vec{y} \rangle$}.
            \item $\frac{1}{\sqrt{d}} \| \vec{x} \|_2 \leq \| \vec{x} \|_{\infty} \leq \| \vec{x} \|_2 \leq \| \vec{x} \|_1 \leq \sqrt{d} \| \vec{x} \|_2$.
            \item Cosine angle: $\cos \angle{(\vec{x},\vec{y})} = \frac{\transpose{\vec{x}} \vec{y}}{\| \vec{x} \| \|
                        \vec{y} \|}$.
            \item Cauchy-Schwarz: $|\langle \vec{x}, \vec{y} \rangle| \leq \| \vec{x} \| \| \vec{y} \|$.
            \item Completing the square: If $\mat{M}$ is symmetric and invertible:
                \begin{align*}
                    \transpose{\vec{x}} \mat{M} \vec{x} - 2 \transpose{\vec{b}} \vec{x} & = \transpose{(\vec{x} - \inv{\mat{M}} \vec{b})} \mat{M} (\vec{x} - \inv{\mat{M}} \vec{b}) \\
                                                                                        & \quad \quad - \transpose{\vec{b}} \inv{\mat{M}} \vec{b}.
                \end{align*}
            \item Woodbury's identity: $\inv{(\mat{I} + \mat{U} \mat{V})} = \mat{I} - \mat{U} \inv{(\mat{I} + \mat{V} \mat{U})} \mat{V}$.
            \item Points of $\mathcal{S}$ are in general position if any subset $\Xi \subseteq \mathcal{S}$ with
                $|\Xi| \leq \delta$ is linearly independent.
            \item Sign function: $\mathrm{sgn}(z) = \begin{cases} +1 & z \geq 0 \\ -1 & z < 0 \end{cases}$.
            \item $\mat{M}$ is positive (semi-)definite if $\transpose{\vec{x}} \mat{M} \vec{x} \geq 0$ (PSD) or $\transpose{\vec{x}} \mat{M} \vec{x} > 0$ (PD) for all $\vec{x}$.
            \item $\mathrm{rank}(\mat{A}\mat{B}) \leq \min \{ \mathrm{rank}(\mat{A}), \mathrm{rank}(\mat{B}) \}$. \\
                If $\mat{A} \in \R^{m \times n}$, then $\mathrm{rank}(\mat{A}) \leq \min \{ m,n \}$.
            \item $\tanh(z) = \nicefrac{\exp(x) - \exp(-x)}{\exp(x) + \exp(-x)}$, $\sigma(z) = \nicefrac{1}{1+\exp(-z)}$, $\mathrm{ReLU}(z) = \begin{cases} z & z \geq 0 \\ 0 & z < 0 \end{cases}$.
            \item $\tanh(z) = 2 \sigma(2z) - 1$.
            \item $\mathrm{ReLU}(z) = \frac{|z| + z}{2}$, $|z| = \mathrm{ReLU}(z) + \mathrm{ReLU}(-z)$.
            \item $\mathrm{Cov}(\vec{x}) = \E[(\vec{x}-\E[\vec{x}])\transpose{(\vec{x} - \E[\vec{x}])}] = \E[\vec{x}\transpose{\vec{x}}] - \E[\vec{x}] \transpose{\E[\vec{x}]}$.
            \item Integration by part: $\int u\mathrm{d}v = uv - \int v\mathrm{d}u$.
            \item Geometric series: $\sum_{k=0}^{\infty} ar^k = \frac{a}{1-r}$ if $|r| < 1$.
            \item $\log(1+x) \approx x$ for small $x$.
            \item $1-x \leq \exp(-x) \implies (1-\epsilon)^n \leq \exp(-n \epsilon)$.
            \item For increasing $f$, $\argmax_{x} x = \argmax_{x} f(x)$.
            \item Group $(\mathcal{G}, \circ)$ satisfies (1) \textit{Associativity}: $(g \circ h) \circ k = f
                    \circ (h \circ k)$ for all $g,h,k \in \mathcal{G}$; (2) \textit{Identity}: There exists a
                unique $e \in \mathcal{G}$ satisfying $g \circ e = e \circ g = g$ for all $g \in
                    \mathcal{G}$; (3) \textit{Inverse}: For each $g \in \mathcal{G}$ there is a unique inverse
                $\inv{g} \in \mathcal{G}$ such that $g \circ \inv{g}  = \inv{g} \circ g = e$; (4) \textit{Closure}:
                For every $g, h \in \mathcal{G}$, we have $g \circ h \in \mathcal{G}$.
        \end{itemize}
    \end{subtopic}

    \begin{subtopic}{Gradients}
        \hl{\textit{Make sure to always type-check gradients!}}
        \begin{align*}
            \vec{z} = \transpose{\vec{a}} \vec{x} & \implies \pdv{L}{\vec{x}} = \pdv{L}{\vec{z}} \vec{a} \\
            \vec{z} = \mat{W} \vec{x} & \implies \pdv{L}{\mat{W}} = \pdv{L}{\vec{z}} \transpose{\vec{x}} \\
            \vec{z} = f(\vec{x}) \text{ element-wise} & \implies \pdv{L}{\vec{x}} = \pdv{L}{\vec{z}} \odot f'(\vec{x})   \\
            \vec{z} = \transpose{\vec{x}}\mat{A}\vec{x} & \implies \pdv{L}{\vec{x}} = \pdv{L}{\vec{z}} (\mat{A}+\transpose{\mat{A}})\vec{x} \\
            \vec{z} = \frac{1}{2} \| \mat{A} \vec{x} \|^2 & \implies \pdv{L}{\vec{x}} = \pdv{L}{\vec{z}} \transpose{\mat{A}} \mat{A} \vec{x} \\
            \vec{z} = \mathrm{softmax}(\vec{x}) & \implies \pdv{\vec{z}}{\vec{x}} = \mathrm{diag}(\vec{z}) - \vec{z}\transpose{\vec{z}} \\
            \sigma'(z) & = \sigma(z) (1-\sigma(z)) \\
            \tanh'(z) & = 1 - z^2 \\
            \cos'(z) & = -\sin(z) \\
            \sin'(z) & = \cos(z) \\
            \mathrm{ReLU}'(z) & = \mathbb{1}\{ z \geq 0 \} \\
            |x|' & = \frac{x}{|x|} \\
            \lft( \frac{f}{g} \rgt)'(x) & = \frac{f'(x) g(x) - f(x) g'(x)}{g^2(x)} \\
            \pdv{L}{x_i} & = \sum_{j=1}^{m} \pdv{L}{y_j} \pdv{y_j}{x_i}.
        \end{align*}
    \end{subtopic}

    \begin{subtopic}{Information theory}
        \begin{align*}
            H(p) & \doteq \E_{X \sim p}[-\log p(X)] \\
            D_{\mathrm{KL}}(p\;\|\;q) & \doteq \E_{X \sim p}\lft[ -\log \lft( \frac{q(X)}{p(X)} \rgt) \rgt] \\
            H(p, q) & \doteq \E_{X \sim p}[-\log q(X)] \\
            H(p, q) & = H(p) + D_{\mathrm{KL}}(p\;\|\;q) \\
            \mathrm{I}(p; q) & = H(p) - H(p \mid q) \\
        \end{align*}
    \end{subtopic}

    \begin{subtopic}{Probability}
        \begin{align*}
            p(X \mid Y) & = \frac{p(Y \mid X) p(X)}{p(Y)} \\
            p(X_{1:n}) & = p(X_1) \prod_{i=2}^n p(X_i \mid X_{1:i-1}) \\
            p(X_{1:i-1},X_{i+1:n}) & = \sum_{x_i} p(X_{1:i-1}, X_i=x_i, X_{i+1:n}).
        \end{align*}
    \end{subtopic}

    \Break

    \begin{subtopic}{Gaussian}
        Definition: \[
            \mathcal{N}(\vec{x}; \vec{\mu}, \mat{\Sigma}) \doteq \frac{1}{\sqrt{(2\pi)^{d} \det{\mat{\Sigma}}}} \exp \lft( -\frac{1}{2} \transpose{(\vec{x}-\vec{\mu})} \inv{\mat{\Sigma}} (\vec{x} - \vec{\mu}) \rgt).
        \]

        Arithmetic:
        \begin{align*}
            \vec{x} + \vec{y} & \sim \mathcal{N}(\vec{\mu}_{\vec{x}} + \vec{\mu}_{\vec{y}}, \mat{\Sigma}_{\vec{x}} + \mat{\Sigma}_{\vec{y}})      \\
            \mat{M} \vec{x} & \sim \mathcal{N}(\mat{M} \vec{\mu}, \mat{M}\mat{\Sigma}\transpose{\mat{M}}).
        \end{align*}

        Let $p_1 = \mathcal{N}(\vec{\mu}_1, \mat{\Sigma}_1)$ and $p_2 = \mathcal{N}(\vec{\mu}_2, \mat{\Sigma}_2)$, then
        \begin{align*}
            H & = \frac{d}{2} \log(2\pi e) + \frac{1}{2} \log \det{\mat{\Sigma}_1} \\
            D_{\mathrm{KL}} & = \frac{1}{2} \biggl[ \log \frac{\det{\mat{\Sigma}_2}}{\det{\mat{\Sigma}_1}} - d + \tr{\inv{\mat{\Sigma}_2} \mat{\Sigma}_1} + \transpose{(\vec{\mu}_2 - \vec{\mu}_1)} \inv{\mat{\Sigma}_2} (\vec{\mu}_2 - \vec{\mu}_1) \biggr].
        \end{align*}
    \end{subtopic}

    \begin{subtopic}{Convexity}
        \begin{gather*}
            f(\lambda \vec{x} + (1-\lambda) \vec{y}) \leq \lambda f(\vec{x}) + (1-\lambda) f(\vec{y}) \\
            f(\vec{y}) \geq f(\vec{x}) + \langle \grad{f(\vec{x})}{}, \vec{y} - \vec{x} \rangle       \\
            \langle \grad{f(\vec{x})}{} - \grad{f(\vec{y})}{}, \vec{x} - \vec{y} \rangle \geq 0       \\
            \colorbox{yellow}{$\displaystyle \| \hess{f(\vec{x})}{} \|_2 \geq 0$}.
        \end{gather*}
    \end{subtopic}

    \begin{subtopic}{Strong convexity}
        \begin{gather*}
            f(\vec{y}) \geq f(\vec{x}) + \langle \grad{f(\vec{x})}{}, \vec{y} - \vec{x} \rangle + \frac{\mu}{2} \| \vec{x} - \vec{y} \|^2 \\
            \colorbox{yellow}{$\displaystyle \| \hess{f(\vec{x})}{} \|_2 \geq \mu$}.
        \end{gather*}
    \end{subtopic}

    \begin{subtopic}{Smoothness}
        \begin{gather*}
            \| \grad{f(\vec{x})}{} - \grad{f(\vec{y})}{} \| \leq L \| \vec{x} - \vec{y} \|                                                 \\
            f(\vec{x}) \leq f(\vec{y}) + \langle \grad{f(\vec{y})}{}, \vec{x} - \vec{y} \rangle + \frac{L}{2} \| \vec{x} - \vec{y} \|^2 \\
            \colorbox{yellow}{$\displaystyle \| \hess{f(\vec{x})}{} \|_2 \leq L$}.
        \end{gather*}
    \end{subtopic}

    \begin{topic}{Connectionism}

        \begin{subtopic}{McCulloch-Pitts neuron}
            \[
                f[\vec{\sigma},\theta](\vec{x}) \doteq \mathbb{1}\{ \transpose{\vec{\sigma}}\vec{x} \geq \theta \}, \quad \vec{\sigma} \in \{ -1,1 \}^n, \vec{x} \in \{ 0,1 \}^n, \theta \in \R.
            \]
        \end{subtopic}

        \begin{subtopic}{Perceptron}
            \begin{align*}
                f[\vec{w},b](\vec{x})          & \doteq \mathrm{sgn}(\transpose{\vec{w}}\vec{x} + b)                    \\
                \gamma[\vec{w},b](\vec{x},y)   & \doteq \frac{y (\transpose{\vec{w}}\vec{x} + b)}{\| \vec{w} \|}        \\
                \gamma[\vec{w},b](\mathcal{S}) & \doteq \min_{(\vec{x},y) \in \mathcal{S}} \gamma[\vec{w},b](\vec{x},y) \\
                \mathcal{V}(\mathcal{S})       & \doteq \{ (\vec{w},b) \mid \gamma[\vec{w},b](\mathcal{S}) > 0 \}.
            \end{align*}
            Decision boundary hyperplane: $\nicefrac{\transpose{\vec{w}} \vec{x}}{\| \vec{w} \|} + \nicefrac{b}{\| \vec{w} \|}
                \overset{!}{=} 0$. If $\gamma[\vec{w},b](\mathcal{S}) > 0$, then $\mathcal{S}$ is
            linearly separated. $\mathcal{V}(\mathcal{S}) \neq \emptyset$ iff $\mathcal{S}$ is
            linearly separable. Adding points to $\mathcal{S}$ makes $\mathcal{V}(\mathcal{S})$ smaller.

            The perceptron algorithm tries to find any $(\vec{w},b) \in \mathcal{V}(\mathcal{S})$. It does not
            aim to find solution with smaller error if $\mathcal{V}(\mathcal{S}) = \emptyset$. Iterative
            mistake-driven algorithm: \[
                f[\vec{w},b](\vec{x}) \neq y \implies \vec{w} \gets \vec{w} + y \vec{x}, \quad b \gets b + y.
            \]
            For all iterates $\vec{w}_t \in \mathrm{span}(\vec{x}_1, \ldots, \vec{x}_n)$.

            Convergence can be proven by using $\transpose{\vec{w}} \vec{w}_t \geq t \gamma$ and $\| \vec{w}_t
                \| \leq R \sqrt{t}$, where $\| \vec{w} \| = 1$, $\gamma[\vec{w}](\mathcal{S}) > 0$, and $R =
                \max_{\vec{x} \in \mathcal{S}} \| \vec{x} \|$. Then, bound $1 \geq \cos\angle{(\vec{w},\vec{w}_t)}
                = \nicefrac{\transpose{\vec{w}} \vec{w}_t}{\| \vec{w} \| \| \vec{w}_t \|}$. \hl{It converges within
                $\lfloor \nicefrac{R^2}{\gamma^2} \rfloor$ iterations.}

            Number of unique classifications: \[
                \mathcal{C}(\mathcal{S},d) = \lft| \lft\{ \vec{y} \in \{ -1,+1 \}^n \;\middle|\; \exists \vec{w} : \forall i : y_i (\transpose{\vec{w}}\vec{x}_i) > 0 \rgt\} \rgt|
            \]
            Assume that points are in general position. Cover's theorem: \[
                \mathcal{C}(n+1,d) = 2 \sum_{i=0}^{d-1} \begin{pmatrix} n \\ i \end{pmatrix}.
            \]
            Proof: Base cases are both 2 and adding a point has two cases $\Rightarrow$ Recurrence: \[
                \mathcal{C}(n+1,d) = \mathcal{C}(n,d) + \mathcal{C}(n,d-1).
            \]
            For $n \leq d$, we have $\mathcal{C}(n,d) = 2^n$. After $n=2d$, there is a steep decrease in number
            of linear classification, quickly moving toward $0$.
        \end{subtopic}

        \begin{subtopic}{Parallel distributed processing}
            (1) A set of processing units with states of activation; (2) Output functions for each
            unit; (3) A pattern of connectivity between units; (4) Propagation rules for propagating
            patterns of activity; (5) Activation functions for units; (6) A learning rule to modify
            connectivity based on experience; (7) An environment within which the system must operate.
        \end{subtopic}

        \begin{subtopic}{Hopfield networks}
            Models an associative memory, which aims to reconstruct a memory from an input that has been subjected to noise. Energy function via second-order interactions between $n$ binary neurons: \[
                H(\vec{x}) \doteq \frac{1}{2} \sum_{i=1}^{d} \sum_{j=1}^{d} w_{ij}x_ix_j + \sum_{i=1}^{n} b_ix_i, \quad \vec{x} \in \{ -1, 1 \}^d.
            \]
            We have $w_{ii} = 0$ and $w_{ij} = w_{ji}$. Simple dynamics: \[
                x_i \gets \begin{cases}
                    +1 & H([., x_{i-1}, +1, x_{i+1}, .]) \leq H([., x_{i-1}, -1, x_{i+1}, .]) \\
                    -1 & \mathrm{otherwise}.
                \end{cases}
            \]
            \hl{Or: $x_i \gets \mathrm{sgn}(H_i)$, where $H_i \doteq \sum_{j=1}^{d} w_{ij} x_j - b_i$.}

            Hebbian learning (neurons frequently in the same state reinforce): \[
                \mat{W} = \frac{1}{d} \lft( \sum_{t=1}^{n} \vec{x}_t \transpose{\vec{x}_t} - \mat{I}_d \rgt).
            \]
            Pattern $\vec{x}_t$ is memorized if meta-stable: update rule makes no updates to it: \[
                x_{ti} = \mathrm{sgn}(x_{ti} + C_{ti}), \quad C_{ti} \doteq \frac{1}{d} \sum_{j=1}^{d} \sum_{r\neq t}^{n} x_{ri}x_{rj}x_{tj}.
            \]
            If cross-talk $|C_{ti}| < 1$ for all $i$, then $\vec{x}_t$ is meta-stable.

            If i.i.d.\ random signs, then $C_{ti} \sim \mathcal{N}(0, \nicefrac{n}{d})$. At
            $\nicefrac{d}{n} \approx 0.138$, ``avalanche of errors'' occurs.
        \end{subtopic}

    \end{topic}

    \begin{topic}{Feedforward networks}

        \begin{subtopic}{Regression models}
            Mean-squared error: $\ell[\vec{\theta}](\mathcal{S}) = \frac{1}{2} \sum_{i=1}^{n} \lft( f[\vec{\theta}](\vec{x}_i) - y_i \rgt)^2$.

            Linear model: $\ell[\vec{w}](\mathcal{S}) = \frac{1}{2} \| \transpose{\mat{X}} \vec{w} - \vec{y}
                \|^2$ $\Rightarrow$ Closed form solution: $\vec{w}^\star = \inv{\lft( \transpose{\mat{X}}\mat{X}
                    \rgt)}\transpose{\mat{X}} \vec{y}$.

            Logistic regression (binary outputs) use sigmoid: $\sigma(z) \doteq \nicefrac{1}{1 + \exp(-z)}$.
            Binary cross-entropy loss: \[
                \ell[\vec{\theta}](\mathcal{S}) = -\frac{1}{n} \sum_{i=1}^{n} y_i \log \sigma(f[\vec{\theta}](\vec{x}_i)) + (1-y_i)\log(1-f[\vec{\theta}](\vec{x}_i)).
            \] (Multi-class) Cross-entropy loss: \[
                \ell[\vec{\theta}](\vec{x},y) = -\log \mathrm{softmax}_y(f[\vec{\theta}](\vec{x})).
            \]
        \end{subtopic}

        \begin{subtopic}{Layers and units}
            Mapping: $f[\mat{W},\vec{b}](\vec{x}) = \phi(\mat{W}\vec{x}+\vec{b})$, where $\phi$ is a pointwise activation function. DNNs compose: $G = F_L[\vec{\theta}_L] \circ \cdots \circ F_1[\vec{\theta}_1]$.

            Intermediate layers are permutation symmetric: $\mat{F}[\mat{W},\vec{b}](\vec{x}) = \transpose{\mat{P}}
                \phi(\mat{P}\mat{W}\vec{x} + \mat{P}\vec{b}) = \transpose{\mat{P}} F[\mat{P} \mat{W},
                        \mat{P}\vec{b}](\vec{x})$ $\Rightarrow$ Parameters are not unique.
        \end{subtopic}

        \begin{subtopic}{Linear networks}
            Linear layers are closed under composition $\Rightarrow$ We do not gain representational power from composing them.
        \end{subtopic}

        \begin{subtopic}{Residual networks}
            Residual layers: $F[\mat{W},\vec{b}](\vec{x}) = \vec{x} + \phi(\mat{W}\vec{x} + \vec{b}) - \phi(\vec{0})$.

            Then: $F[\vec{0}, \vec{0}] = \mathrm{Id}$. This makes it such that the model learns how to
            incrementally learn a better representation, rather than having to learn it at every layer. If
            input and output have different dimensionality, linearly project $\vec{x}$. Using this
            architecture, the depth can be increased significantly, because gradients propagate better.
        \end{subtopic}

        \begin{subtopic}{Sigmoid networks}
            MLP with sigmoid activation: \[
                g[\vec{v}, \mat{W}, \vec{b}](\vec{x}) \doteq \transpose{\vec{v}} \sigma(\mat{W}\vec{x}+\vec{b}), \quad \vec{v},\vec{b}\in\R^m, \mat{W}\in\R^{m\times n}.
            \]
            Function class: $\mathcal{G}_n \doteq \bigcup_{m=1}^{\infty} \mathcal{G}_{n,m}$, where \[
                \mathcal{G}_{n,m} \doteq \{ g[\vec{v},\mat{W},\vec{b}] \mid \vec{b}\in\R^m, \mat{W}\in\R^{m\times n} \}.
            \]
            Or, as a linear span of units, \[
                \mathcal{G}_{n} = \mathrm{span} \{ \sigma(\transpose{\vec{w}}\vec{x} + b) \mid \vec{w} \in \R^n, b \in \R \}.
            \]
            This set universally approximates $\mathcal{C}(\R^n)$. But, it does not provide insight into how
            depth affects performance $\Rightarrow$ Barron: Let $f: \R^n \to \R$ with finite $\mathcal{C}_f$ and any $r
                > 0$, there is a sequence of one hidden layer MLPs $(g_m)_{m \in \N}$ such that \[
                \int_{r \mathbb{B}} (f(\vec{x}) - g_m(\vec{x}))^2 \mu(\mathrm{d}\vec{x}) \leq \bigo{\nicefrac{1}{m}},
            \]
            where $r\mathbb{B}$ is a $r$-radius ball and $\mu$ is any probability measure. Relaxing the notion
            of approximation to squared error over ball with radius $r$ gives a decay of $\nicefrac{1}{m}$ on
            the approximation error.
        \end{subtopic}

        \begin{subtopic}{ReLU networks}
            ReLU: $(z)_+ \doteq \max_{0,z}$. Consider a layer of $m$ ReLU units, then each unit is
            active or inactive: $\mathbb{1}\{ \mat{W}\vec{x} + \vec{b} > 0 \} \in \{ 0,1 \}^m$. As
            such, we can partition the input space into cells that have the same activation pattern: \[
                \mathcal{X}_{\vec{\kappa}} \doteq \{ \vec{x} \mid \mathbb{1}\{ \mat{W}\vec{x} + \vec{b} > 0 \} = \vec{\kappa} \}.
            \]
            The number of cells is a proxy for the complexity of a network. Consider a ReLU network with $L$
            layers of $m > n$ width. The number of linear regions is lower bounded by \[
                \colorbox{yellow}{$\displaystyle R(m,L) \geq R(m) \lfloor \nicefrac{m}{n} \rfloor^{n(L-1)}$}.
            \]
            Thus, complexity is related to depth.

            Piecewise linear functions are approximators of $\mathcal{C}(\R)$ and a piecewise linear function
            $g$ with $m$ pieces can be written as \[
                g(x) = ax + b + \sum_{i=1}^{m-1} c_i(x - x_i)_+.
            \]
            Using the lifting lemma, ReLU networks are universal approximators of $\mathcal{C}(\R)$.
        \end{subtopic}

    \end{topic}

    \begin{topic}{Gradient-based learning}

        \begin{subtopic}{Backpropagation}
            Backpropagation computes gradient in linear time if we know the gradient of all basic
            blocks in the function. Assume we have the following function, \[
                F[\vec{\theta}](\vec{x}) \doteq (F_L \circ \cdots \circ F_1)(\vec{x}), \quad \vec{h}_{\ell} \doteq F_{\ell}[\vec{\theta}_{\ell}](\vec{h}_{\ell-1}), \quad \vec{h}_0 = \vec{x}.
            \]
            We need the following, \[
                \vec{\delta}_{\ell} = \pdv{h(\vec{\theta})}{\vec{h}_{\ell}}.
            \]
            We have a recurrence, \[
                \colorbox{yellow}{$\displaystyle \vec{\delta}_{\ell} = \vec{\delta}_{\ell+1} \pdv{\vec{h}_{\ell+1}}{\vec{h}_{\ell}}$}, \quad \vec{\delta}_L = \pdv{h(\vec{\theta})}{\vec{h}_L}.
            \]
            Then, to compute parameter gradients, \[
                \colorbox{yellow}{$\displaystyle \pdv{h(\vec{\theta})}{\vec{\theta}_{\ell}} = \pdv{h(\vec{\theta})}{\vec{h}_{\ell}} \pdv{\vec{h}_{\ell}}{\vec{\theta}_{\ell}} = \vec{\delta}_{\ell} \pdv{\vec{h}_{\ell}}{\vec{\theta}_{\ell}}$}.
            \]
            Forward propagation: $\bigo{\text{\# parameters}}$ passes. Backpropagation: $\bigo{\text{\# outputs}}$ passes.
        \end{subtopic}

        \begin{subtopic}{Gradient descent}
            Update rule: \[
                \colorbox{yellow}{$\displaystyle \vec{\theta}^{t+1} = \vec{\theta}^t - \eta \grad{h(\vec{\theta}^t)}{\vec{\theta}}$}, \quad \eta > 0, \quad h(\vec{\theta}) \doteq \ell \circ F[\vec{\theta}].
            \]
            Discretization of ODE: \[
                \mathrm{d}\vec{\theta} = -\grad{h(\vec{\theta})}{} \mathrm{d}t.
            \]
            Trajectory outcome (point where $\grad{h(\vec{\theta})}{} = \vec{0}$) depends on initial
            conditions.

            Gradient descent can only be successful if gradients change slowly $\Rightarrow$ Smoothness: $h$ is
            $L$-smooth if \[
                \| \grad{h(\vec{\theta})}{} - \grad{h(\vec{\theta})'}{} \| \leq L \| \vec{\theta} - \vec{\theta}' \|, \quad \forall \vec{\theta},\vec{\theta}'.
            \]
            Or: $\| \hess{h(\vec{\theta})}{} \|_2 \leq L$ for all $\vec{\theta}$. If $\eta = \nicefrac{1}{L}$,
            then we have sufficient decrease, \[
                \colorbox{yellow}{$\displaystyle h(\vec{\theta}_{t+1}) \leq h(\vec{\theta}_t) - \frac{1}{2L} \| \grad{h(\vec{\theta}_t)}{} \|^2$}, \quad \forall t.
            \]
            Let $h$ be $L$-smooth and $\eta=\nicefrac{1}{L}$, then an $\epsilon$-critical point will be found
            in at most \[
                T = \frac{2L}{\epsilon^2} (h(\vec{\theta}^0) - h(\vec{\theta}^\star)).
            \]
            Proof: Sufficient decrease $\Rightarrow$ Telescoping sum and $\min \leq \sum$.

            $h$ satisfies PL-inequality with $\mu > 0$ if \[
                \colorbox{yellow}{$\displaystyle \frac{1}{2} \| \grad{h(\vec{\theta})}{} \|^2 \geq \mu (h(\vec{\theta}) - h(\vec{\theta}^\star))$}, \quad \forall \vec{\theta}.
            \]
            Intuition: If $\vec{\theta}$ has small gradient, then it is near-optimal. Let $h$ be $L$-smooth and
            $\mu$-PL and $\eta = \nicefrac{1}{L}$, then \[
                h(\vec{\theta}_T) - h(\vec{\theta}^\star) \leq \lft( 1 - \frac{\mu}{L} \rgt)^T (h(\vec{\theta}_0) - h(\vec{\theta}^\star)).
            \]
            Proof: Sufficient decrease $\Rightarrow$ PL $\Rightarrow$ Subtract $h(\vec{\theta}^\star)$ both sides.

            Newton's method: $\vec{\theta}^{t+1} = \vec{\theta}^t - \inv{\hess{h(\vec{\theta}^t)}{}} \grad{h(\vec{\theta}^t)}{}$.
            
        \end{subtopic}

        \begin{subtopic}{Acceleration, adaptivity, and momentum}
            Nesterov acceleration achieves better theoretical guarantees than GD: \[
                \vec{\chi}^{t+1} = \vec{\theta}^t + \gamma(\vec{\theta}^t - \vec{\theta}^{t-1}), \quad \vec{\theta}^{t+1} = \vec{\chi}^{t+1} - \eta \grad{h(\vec{\chi}^{t+1})}{}.
            \]

            Momentum intuition: If gradient is stable, we can make bolder steps. Heavy Ball: \[
                \colorbox{yellow}{$\displaystyle \vec{\theta}^{t+1} = \vec{\theta}^t - \eta \grad{h(\vec{\theta}^t)}{} + \beta (\vec{\theta}^t - \vec{\theta}^{t-1})$}, \quad \beta \in [0,1].
            \]
            Assuming constant gradient $\vec{\delta}$, we have \[
                \vec{\theta}^{t+1} = \vec{\theta}^t - \eta \lft( \sum_{\tau=1}^{t-1} \beta^{\tau} \rgt) \vec{\delta}.
            \]
            Thus, learning rate increases in case of a stable gradient.

            Adaptivity intuition: Different parameters behave differently $\Rightarrow$ Parameter-specific learning
            rates:
            \begin{align*}
                \theta_i^{t+1} & = \theta_i^t - \eta_i^t \partial_i h(\vec{\theta}^t)                                                                       \\
                \eta_i^t       & \doteq \frac{\eta}{\sqrt{\gamma_i^t} + \delta}, \quad \gamma_i^t \doteq \gamma_i^{t-1} + [\partial_i h(\vec{\theta}^t)]^2.
            \end{align*}
            Parameters with small gradient magnitude will have a larger step size.

            Adam combines adaptivity and momentum:
            \begin{align*}
                \vec{g}_t          & = \beta \vec{g}_{t-1} + (1-\beta) \grad{h(\vec{\theta}_t)}{}, \quad \beta \in [0,1]                               \\
                \vec{\gamma}_t     & = \alpha\vec{\gamma}_{t-1} + (1-\alpha) \grad{h(\vec{\theta}_t)}{}^{\odot 2}, \quad \alpha \in [0,1]              \\
                \vec{\theta}_{t+1} & = \vec{\theta}_t - \vec{\eta}_t \odot \vec{g}_t, \quad \vec{\eta}_t = 1 \oslash (\sqrt{\vec{\gamma}_t} + \delta).
            \end{align*}
            $\vec{g}_t$ is a smooth gradient estimator and $\vec{\gamma}_t$ measures the stability of the optimization landscape.
        \end{subtopic}

        \begin{subtopic}{Stochastic gradient descent}
            When the dataset is too large, computing the full gradient is infeasible $\Rightarrow$ Estimate
            gradient with a mini-batch (SGD). SGD outperforms GD in practice, because it has a lower
            chance of getting stuck in a local optimum due to variance in the gradient estimator.
        \end{subtopic}

    \end{topic}

    \begin{topic}{Convolutional networks}

        \begin{subtopic}{Convolution}
            Integral operator: $(Tf)(u) \doteq \int_{t_1}^{t_2} H(u,t) f(t) \mathrm{d}t$.

            Fourier transform: $(\mathcal{F}f)(u) \doteq \int_{-\infty}^\infty e^{-2\pi itu} f(t) \mathrm{d}t$.

            Convolution: $(f * h)(u) \doteq \int_{-\infty}^{\infty} h(u-t) f(t) \mathrm{d}t$.

            Commutative: $f * h = h * f$, Shift-equivariant: $f_{\Delta} * h = (f * h)_{\Delta}$, Convolution
            as Fourier: $f * h = \mathcal{F}^{-1}(\mathcal{F}f \cdot \mathcal{F}h)$ $\Rightarrow$ Can
            do convolution in $\bigo{n \log n}$ with Fast Fourier Transform. All proofs are done by
            defining new variables dependent on existing ones. Linear shift-equivariant operator $\iff$
            Convolution.

            Discrete convolution: \hl{$(f * h)[u] = \sum_{t=-\infty}^{\infty} f[t]h[u-t]$}.

            Cross-correlation: $(f \star h)[u] = \sum_{t=-\infty}^{\infty} f[t] h[u+t]$.

            Toeplitz matrix $\mat{H}_n^h \in \R^{(n+m-1)\times n}$: \[
                \mat{H}_n^h \doteq
                \begin{bmatrix}
                    h_1    & 0      & \cdots & 0      & 0       \\
                    h_2    & h_1    & \cdots & 0      & 0       \\
                    \vdots & \vdots & \ddots & \vdots & \vdots  \\
                    0      & 0      & \cdots & h_m    & h_{m-1} \\
                    0      & 0      & \cdots & 0      & h_m
                \end{bmatrix}.
            \]
            This can then be applied to a vectorized $\vec{f} \in \R^n$. Effectively a proof that convolution
            is linear with increased statistical efficiency.
        \end{subtopic}

        \begin{subtopic}{Convolutional networks}
            Images are 2D, so use 2D definition of convolution: \[
                (\mat{X} * \mat{W})[i,j] = \sum_{k=-\infty}^{\infty} \sum_{l=-\infty}^{\infty} x_{i-k,j-l} w_{k,l}.
            \]
            Let $\mat{X}^{\ell}$ be output of the $\ell$-th convolutional layer and $\mat{\Delta}^{\ell} \doteq
                \pdv{h}{\mat{X}^{\ell}}$, then \[
                \mat{\Delta}^{\ell-1} = \mat{\Delta}^{\ell} \star \mat{W}^{\ell}, \quad \pdv{h}{\mat{W}^{\ell}} = \mat{\Delta}^{\ell} * \mat{X}^{\ell-1}.
            \]
            Max-pooling layer:
            \begin{align*}
                x^{\ell}_{ij}                        & = \max \{ x^{\ell-1}_{i+k,j+l} \mid k, l \in [0,r) \} \\
                \pdv{x_{ij}^{\ell}}{x^{\ell-1}_{mn}} & = \mathbb{1}\{ (m,n) = (i^\star,j^\star) \},
            \end{align*}
            where $(i^\star, j^\star)$ are the indices of maximum value in the forward pass.

            Data generally has multiple channels: \[
                (\tens{X} * \tens{W})[c,i,j] = \sum_{r=1}^{C_{in}} \sum_{k=-K}^{K} \sum_{l=-K}^{K} w_{c,r,k,l} x_{r,i-k,j-l}.
            \]
            Fully connected in channels and local in spatial dimensions.

            Output size ($x$: input, $p$: padding, $k$: kernel, $s$: stride): \[
                \colorbox{yellow}{$\displaystyle \lft\lfloor \frac{x+2p-k}{s} + 1 \rgt\rfloor$}.
            \]

            Inception: (1) Use $1 \times 1$ convolutions to reduce channels; (2) Use multiple filter
            sizes and concatenate solving trade-off between channels and kernel size; (3) Softmax at
            intermediate layers for better error propagation.
            
        \end{subtopic}

    \end{topic}

    \begin{topic}{Recurrent neural networks}
        Activations are computed recursively to handle variable-length data: $\vec{z}_t = F[\vec{\theta}](\vec{z}_{t-1},\vec{x}_t)$.
        Dependent on application, compute output variables: $\vec{y}_t = G[\vec{\varphi}](\vec{z}_t)$.

        Elman RNN: $F[\mat{U},\mat{V}](\vec{z},\vec{x}) = \phi(\mat{U}\vec{z} + \mat{V}\vec{x})$, $G[\mat{W}](\vec{z}) = \psi(\mat{W}\vec{z})$.

        Bidirectional RNNs do both ways and concatenate. Stacked RNNs connect layers horizontally: \[
            \vec{z}_{t,l} = \phi(\mat{U}_l \vec{z}_{t-1,l} + \mat{V}_l \vec{z}_{t,l-1}), \quad \vec{z}_{t,0} = \vec{x}_t.
        \]

        Let $L = \sum_{t=1}^{T} \ell(\hat{\vec{y}}_t, \vec{y}_t)$, then we have gradients: \[
            \pdv{L}{\mat{U}} = \sum_{t=1}^{T} \pdv{L}{\vec{z}_t} \pdv{\vec{z}_t}{\mat{U}}, \quad \pdv{L}{\mat{V}} = \sum_{t=1}^{T} \pdv{L}{\vec{z}_t} \pdv{\vec{z}_t}{\mat{V}},
        \]
        where \[
            \colorbox{yellow}{$\displaystyle \pdv{L}{\vec{z}_t} = \sum_{i=t}^{T} \pdv{\ell(\hat{\vec{y}}_i, \vec{y}_i)}{\hat{\vec{y}}_i} \pdv{\hat{\vec{y}}_i}{\vec{z}_i} \prod_{j=t+1}^i \dot{\mat{\Phi}}_j \mat{U}$},
        \]
        where $\dot{\mat{\Phi}}_j = \mathrm{diag}(\phi'(\mat{U}\vec{z}_{j-1} + \mat{V}\vec{x}_j))$. This is only
        stable if $\| \dot{\mat{\Phi}}_j \mat{U} \|_2 = 1$. \hl{For most activation functions: $\| \dot{\mat{\Phi}} \|_2 \leq
            \alpha$. So, $\| \mat{U} \|_2 > \nicefrac{1}{\alpha}$ $\Rightarrow$ Exploding gradient,
        $\| \mat{U} \|_2 < \nicefrac{1}{\alpha}$ $\Rightarrow$ Vanishing gradient.}

        \begin{subtopic}{Gated memory}
            Solve vanishing gradient by gating.

            LSTM ($\vec{z}_t$: cell state, $\vec{\zeta}_t$: hidden state):
            \begin{align*}
                \vec{z}_t         & = \underbrace{\sigma(\mat{F}\tilde{\vec{x}}_t) \odot \vec{z}_{t-1}}_{\text{forget gate}} + \underbrace{\sigma(\mat{G}\tilde{\vec{x}}_t) \odot \tanh(\mat{V}\tilde{\vec{x}}_t)}_{\text{storage gate}}, \quad \tilde{\vec{x}}_t = [\vec{\zeta}_{t-1}, \vec{x}_t]                                                                                                \\
                \vec{\zeta}_t     & = \underbrace{\sigma(\mat{H}\tilde{\vec{x}}_t) \odot \tanh(\mat{U}\vec{z}_t)}_{\text{output gate}}.
            \end{align*}
            $\vec{z}_t$ is only modified by $\odot$ and $+$, which allows for easy error propagation.

            GRU simplifies the LSTM to only 3 weight matrices:
            \begin{align*}
                \vec{z}_t         & = \vec{\sigma} \odot \vec{z}_{t-1} + (1-\vec{\sigma}) \odot \tilde{\vec{z}}_t, \quad \vec{\sigma} = \sigma(\mat{G}\tilde{\vec{x}}_t) \\
                \tilde{\vec{x}}_t & = [\vec{z}_{t-1}, \vec{x}_t]                                                                                                         \\
                \tilde{\vec{z}}_t & = \tanh(\mat{V}[\vec{\zeta}_t \odot \vec{z}_{t-1}, \vec{x}_t]), \quad \vec{\zeta}_t = \sigma(\mat{H}[\vec{z}_{t-1} \vec{x}_t]).
            \end{align*}

            Gradient clipping solves exploding gradient
        \end{subtopic}

        \begin{subtopic}{Linear recurrent model}
            Simplify GRU to be linear such that it can exploit prefix scan parallelism which has $\bigo{\log T}$ runtime: \[
                \vec{z}_t = \vec{\sigma} \odot \vec{z}_{t-1} + (1-\vec{\sigma}) \odot \tilde{\vec{z}}_t, \quad \vec{\sigma} = \sigma(\mat{G}\vec{x}_t), \quad \tilde{\vec{z}}_t = \mat{V}\vec{x}_t.
            \]

            We can ensure that the gradients do not explode by parametrizing the GRU smartly. The LRU has
            hidden state evolution in a discrete time linear system: $\vec{z}_{t+1} = \mat{A} \vec{z}_t + \mat{B} \vec{x}_t$.

            Diagonalize $\mat{A} = \mat{P}\mat{\Lambda}\inv{\mat{P}}$, where $\mat{\Lambda} =
                \mathrm{diag}(\lambda_1, \ldots, \lambda_m), \lambda_i \in \mathbb{C}$. This linear system is
            stable if the modulus of the eigenvalues is bounded by $1$. Thus, we parameterize them such that
            the moduli can only be in $(0,1)$ in the following way: $\lambda_i = \exp(-\exp(\nu_i))(\cos(\phi_i) + \sin(\phi_i)i)$.
            This uses $\exp(\theta i) = \cos(\theta) + \sin(\theta)i$ and that we can represent complex numbers
            in polar coordinate form via modulus $r$ and phase $\phi$: \[
                z = r(\cos(\phi) \sin(\phi) i), \quad r = |z|.
            \]
            Thus, \hl{$r_i = \exp(-\exp(\nu_i)) \in (0,1)$.} Thus, at initialization we sample \[
                \phi_i \sim \mathrm{Unif}([0,2\pi]), \quad r_i \sim \mathrm{Unif}(I), \quad I \subseteq [0,1].
            \]
            And compute $\nu_i = \log(-\log r_i)$. The modulus always remains upper bounded by $1$. This is
            equivalent to Glorot initialization.

            \hl{We do not lose any representational power,} because we can put all representational power into the
            output map: \[
                \vec{y}_t = \mathrm{MLP}(\mathrm{Re}(\mat{G}\vec{\zeta}_t)), \quad \mat{G} \in \mathbb{C}^{k \times m}.
            \]
        \end{subtopic}

        \Break

        \begin{subtopic}{Sequence learning}
            Generate sequence step-by-step, given another sequence: \[
                p(\vec{y}_{1:n} \mid \vec{x}_{1:m}) = \prod_{i=1}^n p(y_i \mid \vec{y}_{1:i-1}, \vec{x}_{1:m}).
            \]
            This is done by mapping input sequence to latent representation (encoder RNN): \[
                x_1, \ldots, x_m \mapsto \vec{\zeta}.
            \]
            Decoder RNN uses this along with history to predict next token: \[
                \vec{\zeta}, y_1, \ldots, y_{t-1} \mapsto \vec{z}_{t-1}.
            \]
            This is mapped to a distribution over next tokens, \[
                \vec{z}_{t-1} \mapsto \vec{\mu}_t, \quad y_t \sim p(\vec{\mu}_t).
            \]
            Problem: Lossy compression of input sequence. Solution: Use attention such that decoder can look at
            full input sequence at every step: \[
                a_{ij} = \mathrm{softmax}_j(\mathrm{MLP}([\vec{z}_{i-1}, \vec{\zeta}_j])), \quad \vec{c}_t = \sum_{i=1}^{m} a_{ti} \vec{\zeta}_i.
            \]
            This allows for alignment between input and output sequence.

            Problem with teacher forcing: Mismatch between test and train.
        \end{subtopic}

    \end{topic}

    \begin{topic}{Transformers}

        \begin{subtopic}{Self-attention}
            Let $\mat{X} \in \R^{T \times d}$ denote a sequence of input embeddings. Problem: They are non-contextual. Self-attention: \[
                \mat{Q} = \mat{X} \mat{W}_Q, \quad \mat{K} = \mat{X} \mat{W}_K, \quad \mat{V} = \mat{X}\mat{W}_V,
            \]
            where $\mat{W}_Q, \mat{W}_K \in \R^{d\times d_k}$ and $\mat{W}_V \in \R^{d \times d_v}$. Attention: \[
                \colorbox{yellow}{$\displaystyle \mat{\Xi} = \mathrm{softmax} \lft( \frac{\mat{Q} \transpose{\mat{K}}}{\sqrt{d_k}} \rgt) \mat{V}$}.
            \]
            Multi-headed SA performs attention $h$ times in parallel.
        \end{subtopic}

        \begin{subtopic}{Cross-attention}
            Two sequences $\mat{A} \in \R^{T_a \times d_a}$, $\mat{B} \in \R^{T_b \times d_b}$ and we want to give information of $\mat{B}$ to $\mat{A}$: \[
                \mat{Q} = \mat{A}\mat{W}_Q, \quad \mat{K} = \mat{B} \mat{W}_K, \quad \mat{V} = \mat{B}\mat{W}_V,
            \]
            where $\mat{W}_Q \in \R^{d_a \times d_k}$, $\mat{W}_K \in \R^{d_b\times d_k}$, and $\mat{W}_V \in
                \R^{d_b \times d_v}$.

        \end{subtopic}

        \begin{subtopic}{Positional encoding}
            Attention is permutation equivariant $\Rightarrow$ Add positional encoding matrix $\mat{P} \in \R^{T \times d}$ with \[
                p_{tk} = \begin{cases}
                    \sin(t \omega_k) & k \mod 2 = 0  \\
                    \cos(t \omega_k) & k \mod 1 = 0,
                \end{cases}
                \quad \omega_k \doteq C^{\nicefrac{k}{d}}.
            \]
        \end{subtopic}

        \begin{subtopic}{Machine translation}
            Encoder-decoder architecture where the encoder applies MHSA to input sequence and decoder
            applies masked MHSA to history and then cross-attention with contextualized input
            sequence. Furthermore, MLP, Layer normalization, and residual layers are used.
        \end{subtopic}

        \begin{subtopic}{BERT}
            BERT is a transformer-based pretrained language model that is used for finetuning on
            downstream NLP tasks. BERT tokenizes uses WordPiece tokenization and prepends a
            \texttt{[CLS]} token. \hl{When the weights of the encoders are pretrained, we can place
            additional layers on top that operate on the contextualized BERT tokens.}

            Two pre-training stages:
            \begin{enumerate}[left=0pt]
                \item Predicting masked out tokens using its left and right context as input (Cloze test; trains BERT's
                      understanding of language);
                \item Binary next sentence classification, where the model must classify two sentences as being
                      consecutive or not (trains BERT to infer relationships between sentences).
            \end{enumerate}
            Sentences are separated by \texttt{[SEP]}.
        \end{subtopic}

        \begin{subtopic}{Vision transformer}
            ViT adapts the transformer to images by treating projected $16 \times 16$ image patches
            as tokens. \hl{A possible reason for this model's effectiveness is that this architecture
            carries less inductive bias than CNN-based models.} In general, this seems to be beneficial
            for very large datasets.
        \end{subtopic}

    \end{topic}

    \begin{topic}{Geometric deep learning}
        GDL models neural networks that satisfy invariances by design.

        \begin{subtopic}{Invariance and equivariance}
            $f$ (arbitrary number of inputs) is order-invariant iff \[
                f(\mat{X}) = f(\mat{P}\mat{X}), \quad \mat{X} \in \R^{M \times d},
            \]
            where $\mat{P}$ is a permutation matrix.

            $f$ (arbitrary number of inputs and same number of outputs) is equivariant iff \[
                f(\mat{X}) = \transpose{\mat{P}} f(\mat{P} \mat{X}).
            \]

            We want models that have these properties by design.
        \end{subtopic}

        \begin{subtopic}{Deep sets}
            Let $\phi: R \to \R^d$ be a pointwise feature extractor network. Deep Sets obtains an
            order-invariant representation of the input set by summing their features up. This
            representation can be given to any network $\rho: \R^d \to \mathcal{Y}$: \[
                f(\vec{x}_1, \ldots, \vec{x}_M) = \rho \lft( \sum_{m=1}^{M} \phi(\vec{x}_m) \rgt).
            \]
            We can easily turn this into an equivariant map by providing $\vec{x}_m$ to $\rho: R \times \R^d
                \to \mathcal{Y}$: \[
                f(\vec{x}_1, \ldots, \vec{x}_M)_i = \rho \lft( \vec{x}_i, \sum_{m=1}^{M} \phi(\vec{x}_m) \rgt).
            \]
            This architecture is universal for a fixed $d$, but it requires mapping that are highly
            discontinuous for $M \to \infty$.
        \end{subtopic}

        \begin{subtopic}{PointNet}
            PointNet is a Deep Sets architecture on three-dimensional point clouds. The model employs
            T-net blocks, which apply rigid transformations to the point cloud, which is permutation
            invariant. These are applied twice alternatingly with MLPs to form $\phi$. This gives a
            64-dim intermediate feature vector and 1024-dim final feature vector. The features are
            aggregated by a max-pool operator.

            Object classification: $\rho$ is an MLP with a softmax head that takes the global feature vector as
            input.

            Object segmentation: $\rho$ concatenates intermediate local feature and final global feature, which
            is given to MLP with softmax head.
        \end{subtopic}

        \begin{subtopic}{Graph neural networks}
            Let $\mat{A}$ be the adjacency matrix of an undirected graph. $f$ is order-invariant on a graph if \[
                \colorbox{yellow}{$\displaystyle f(\mat{X},\mat{A}) = f(\mat{P}\mat{X}, \mat{P}\mat{A}\transpose{\mat{P}})$}.
            \]
            And equivariant if \[
                \colorbox{yellow}{$\displaystyle f(\mat{X}, \mat{A}) = \transpose{\mat{P}} f(\mat{P}\mat{X},\mat{P}\mat{A}\transpose{\mat{P}})$}.
            \]
            Let $\mat{X}_m = \{ \{ \vec{x}_n \mid a_{nm} = 1 \} \}$ (multiset of neighbors' features). $\phi$
            takes $\vec{x}_m$ and $\mat{X}_m$ as input (any pair of isomorphic graphs result in same feature
            representations): \[
                \phi(\vec{x}_m, \mat{X}_m) = \phi \lft( \vec{x}_m \bigoplus_{\vec{x} \in \mat{X}_m} \psi(\vec{x}) \rgt).
            \]
            This is a message-passing scheme.

            Graph convolutional networks (GCNs) aggregates local neighborhoods with a fixed set of weights
            (coupling matrix), \[
                \colorbox{yellow}{$\displaystyle \bar{\mat{A}} \doteq \mat{D}^{-\nicefrac{1}{2}} (\mat{A} + \mat{I}) \mat{D}^{-\nicefrac{1}{2}}$}, \quad \mat{D} = \mathrm{diag}(\vec{d}), \quad d_m = 1 + \sum_{n=1}^{M} a_{nm}.
            \]
            Element-wise: \[
                \bar{a}_{ij} = \frac{a_{ij} + \delta_{ij}}{\sqrt{d_i d_j}}, \quad \delta_{ij} = \mathbb{1}\{ i = j \}.
            \]
            Now, $\bar{\mat{A}}\mat{X}$ computes the average feature over neighbors and the node itself.
            GCNs introduce learnable parameters $\mat{W}$: \[
                \colorbox{yellow}{$\displaystyle f[\mat{W}](\mat{X}, \mat{A}) = \sigma(\bar{\mat{A}} \mat{X} \mat{W})$}.
            \]
            This is an equivariant function, which can be stacked. Then, we can use the final representations
            to do graph classification or node classification. \hl{Note that $\| \bar{\mat{A}} \|_2 \leq
            1$, so the values will not explode.}

            Limitations: Requires a depth equal to diameter of graph to exchange information between all nodes;
            In very deep GCNs, node features become indistinguishable due to smoothing of $\bar{\mat{A}}$;
            Bottleneck effect of how much information can be stored in fixed-size representations. There are no
            canonical solutions to these problems.

            GATs introduce attention (which is equivariant) in the neighborhood function and replaces
            $\bar{\mat{A}}$. It does so by parametrizing the coupling matrix $\mat{Q}$,
            \begin{align*}
                q_{mn}                       & = \mathrm{softmax}_n ( \rho(\transpose{\vec{u}} [\mat{V}\vec{x}_m, \mat{V}\vec{x}_n, \vec{x}_{mn}]) ) \\
                \sum_{n=1}^{M} a_{mn} q_{mn} & = 1, \quad \forall m \in [M].
            \end{align*}
            Here, $\vec{x}_{ij}$ is an edge feature.

            Despite better adaptivity, GATs are still message-passing algorithms. Such algorithms have inherent
            limitations in the type of graphs they can distinguish. The Weisfeiler-Lehman (WL) isomorphism
            test checks whether two graphs are \emph{potentially} isomorphic. It does so in linear time. \hl{It can be shown that GCNs and
            GATs cannot distinguish graphs beyond the WL-test.}

            1-WL: (1) Every node starts with the same color; (2) If a vertex $v$ has neighbors with
            colors $\{ c_1, \ldots, c_k \}$, then $c_v$ gets a new color encoding $[c_v, \{\{ c_1,
            \ldots, c_k \}\}]$. If the color partitions of the two graphs are different, the graphs
            cannot be isomorphic. If the color partitions are equivalent, the graphs are potentially
            isomorphic.
        \end{subtopic}

        \begin{subtopic}{Spectral graph theory}
            Laplacian operator measures local deviation from the mean in vanishingly small neighborhoods: \[
                \Delta f = \sum_{i=1}^{d} \pdv[order=2]{f}{x_i}.
            \]
            The Fourier basis can be defined as the eigenfunction of the Laplacian.

            Graph Laplacian: \hl{$\mat{L} = \mat{D} - \mat{A}$}. Degree-normalized Laplacian: $\tilde{\mat{L}} =
                \mat{D}^{-\nicefrac{1}{2}} (\mat{D} - \mat{A}) \mat{D}^{-\nicefrac{1}{2}}$. We can generalize
            Fourier transform to graphs: Diagonalize $\mat{L} = \mat{U} \mat{\Lambda} \transpose{\mat{U}}$
            (exists because symmetric and PSD) and $\mat{U}$ can be seen as graph Fourier basis and
            $\mat{\Lambda}$ as frequencies. Graph convolution can be computed as pointwise multiplication in
            Fourier domain: \[
                \mat{X} * \mat{Y} = \mat{U}((\transpose{\mat{U}}\mat{X}) \odot (\transpose{\mat{U}} \mat{Y})).
            \]
            This can be learned by \[
                G[\vec{\theta}](\mat{L}) \mat{X} = \mat{U} G[\vec{\theta}](\mat{\Lambda}) \transpose{\mat{U}} \mat{X}.
            \]
            Problem: Eigendecomposition of $\mat{L}$ takes $\bigo{M^3}$. Solution: Use polynomial kernels: \[
                \colorbox{yellow}{$\displaystyle \mat{U} \lft( \sum_{k=0}^{K} \alpha_k \mat{\Lambda} \rgt) \transpose{\mat{U}} \mat{X} = \sum_{k=1}^{K} \alpha_k \mat{L}^k \mat{X}$}.
            \]
            Here, the polynomial order $K$ defines the kernel size (or neighborhood size). $\vec{\alpha} \in
                \R^K$ are parameters.
        \end{subtopic}

    \end{topic}

    \begin{topic}{Tricks of the trade}

        \begin{subtopic}{Initialization}
            Parameters are typically chosen with a fixed variance by sampling from \[
                \vec{\theta} \sim \mathcal{N}(0, \sigma^2), \quad \vec{\theta} \sim \mathrm{Unif}([-\sqrt{3}\sigma,\sqrt{3}\sigma]).
            \]
            LeCun init, \hl{$\sigma = \nicefrac{1}{\sqrt{n}}$}, preserves input variance. Glorot init, \hl{$\sigma =
            \sqrt{\nicefrac{2}{n+m}}$}, normalizes magnitude of gradient (intuition: backpropagation combines
            $n$-dim input and $m$-dim output). Kaiming init, \hl{$\sigma = \sqrt{\nicefrac{2}{n}}$}, is designed to
            be used with ReLU by observing that only half of the units are active in expectation. Orthogonal
            init considers the weights holistically by initializing as orthogonal matrix (benefit: eigenvalues
            are $\pm 1$).
        \end{subtopic}

        \begin{subtopic}{Weight decay}
            \[
                \colorbox{yellow}{$\displaystyle \vec{\theta}_{t+1} = \vec{\theta}_t - \eta (\grad{h(\vec{\theta}_t)}{} - \mu \vec{\theta}_t) = (1-\eta \mu) \vec{\theta}_t - \grad{h(\vec{\theta}_t)}{}$}.
            \]
            Equivalent to (1) gradient descent with $\ell_2$ regularization, (2) Bayesian prior that weights
            are sampled from normal distribution, (3) Lagrangian with constraint $\| \vec{\theta} \| \leq \mu$.

            Under the basis of the Hessian's eigenvectors, the optimum of $\ell_2$-regularized $\ell_{\mu}$ is \[
                \vec{\theta}_\mu^\star = \mathrm{diag}\lft( \frac{\lambda_i}{\lambda_i + \mu} \rgt) \vec{\theta}^\star.
            \]
            Each axis is scaled based on sensitivity. If $\lambda_i \gg \mu$, then
            $\nicefrac{\lambda_i}{\lambda_i + \mu} \approx 1$, so the solution does not change much.
        \end{subtopic}

        \begin{subtopic}{Early stopping}
            Stop if validation performance does not improve for past $p$ epochs. Crudely
            equivalent to weight decay with $\mu$ if stopped at $T \approx \nicefrac{1}{\eta \mu}$.
        \end{subtopic}

        \begin{subtopic}{Dropout}
            Randomly disable subset of parameters during training $\Rightarrow$ Units become less dependent
            on one another $\Rightarrow$ Instead of units being specialized, they become generally useful.

            Two view: (1) regularization and (2) ensemble of networks: \[
                p[\vec{\theta}](\vec{y} \mid \vec{x}) = \sum_{\vec{b} \in \{ 0,1 \}^P} p(\vec{b}) p[\vec{\theta} \odot \vec{b}](\vec{y} \mid \vec{x}).
            \]
            This can be approximated by scaling weights by their probability of being active.
        \end{subtopic}

        \begin{subtopic}{Normalization}
            Goal: Make all units more similar. Norm.\ is beneficial because the gradient no longer
            depends on scale of data $\Rightarrow$ High LR without risk of divergence. A unit $f: \R^d \to \R$ can be normalized by \[
                \bar{f} = \frac{f - \E[f(\vec{x})]}{\sqrt{\Var[f(\vec{x})]}}.
            \]
            This removes 2 DOF (bias and variance) $\Rightarrow$ Explicitly parameterize: \hl{$\tilde{f}[\mu,\gamma](\vec{x}) = \mu + \gamma \bar{f}(\vec{x})$.}

            BatchNorm approximates $\E[f]$ and $\Var[f]$ over a \hl{mini-batch}: \[
                \E[f] \approx \frac{1}{|\mathcal{B}|} \sum_{\vec{x} \in \mathcal{B}} f(\vec{x}), \quad \Var[f] \approx \frac{1}{|\mathcal{B}|} \sum_{\vec{x} \in \mathcal{B}} (f(\vec{x}) - \E[f])^2.
            \]

            Normalization is very effective and sometimes essential. We used to believe that it was because it
            helped combat covariance shift. However, a modern motivation shows that normalization is the same
            as weight normalization and scaling by $\nicefrac{\| \vec{w} \|_{\mat{I}}}{\| \vec{w}
                \|_{\mat{\Sigma}}}$, where $\mat{\Sigma} = \E[\vec{x}\transpose{\vec{x}}]$.

            LayerNorm normalizes over the \hl{feature dimension} instead of batch dimension: \[
                \E[f] = \frac{1}{d} \sum_{i=1}^{d} f_i(\vec{x}), \quad \Var[f] = \frac{1}{d} \sum_{i=1}^{d} (f_i(\vec{x}) - \E[f])^2.
            \]
            And, the data is normalized by \[
                \bar{f}_i = \frac{f_i - \E[f]}{\sqrt{\Var[f]}}.
            \]
            Layers need to have a sufficient width $d$ to get stable statistics, but it is not batch
            dependent anymore.
        \end{subtopic}

        \begin{subtopic}{Weight normalization}
            Normalize weights before applying them: \[
                \colorbox{yellow}{$\displaystyle f[\vec{v},\gamma](\vec{x}) = \phi(\transpose{\vec{w}}\vec{x}), \quad \vec{w} = \frac{\gamma}{\| \vec{v} \|_2} \vec{v}$}.
            \]
            Gradients: \[
                \pdv{h}{\gamma} = \pdv{h}{\vec{w}} \pdv{\vec{w}}{\gamma}, \quad \pdv{h}{\vec{v}} = \frac{\gamma}{\| \vec{v} \|} \pdv{h}{\vec{w}} \lft( \mat{I} - \frac{\vec{w}\transpose{\vec{w}}}{\| \vec{w} \|^2} \rgt).
            \]
            Here $\mat{I} - \nicefrac{\vec{w}\transpose{\vec{w}}}{\| \vec{w} \|^2}$ is the projection matrix
            onto the complement of $\vec{w}$ $\Rightarrow$ The direction of $\vec{w}$ is projected out.
        \end{subtopic}

        \begin{subtopic}{Data augmentation}
            Transform input data to train invariances.

            Label smoothing: Replace labels by noisy probability distributions, because classifiers are not
            good at dealing with mislabeled data.
        \end{subtopic}

        \begin{subtopic}{Distillation}
            Let $F$ be the teacher and $G$ its student and we want the student to match its teacher's
            logits. Then, we have tempered cross-entropy loss:
            \begin{align*}
                \ell(\vec{x}) & = \sum_{y \in \mathcal{Y}} \frac{\exp(\nicefrac{F_y(\vec{x})}{T})}{\sum_{y' \in \mathcal{Y}} \exp(\nicefrac{F_{y'}(\vec{x})}{T})} \cdot \\
                              & \quad \quad \quad \quad \lft( \frac{1}{T} G_y(\vec{x}) - \log \sum_{y' \in \mathcal{Y}} \exp(\nicefrac{G_{y'}(\vec{x})}{T}) \rgt).
            \end{align*}
            Gradient: \[
                \pdv{h}{G_y} = \frac{1}{T} \lft( \frac{\exp(\nicefrac{F_y(\vec{x})}{T})}{\sum_{y' \in \mathcal{Y}} \exp(\nicefrac{F_{y'}(\vec{x})}{T})} - \frac{\exp(\nicefrac{G_y(\vec{x})}{T})}{\sum_{y' \in \mathcal{Y}} \exp(\nicefrac{G_{y'}(\vec{x})}{T})} \rgt).
            \]
        \end{subtopic}

    \end{topic}

    \begin{topic}{Neural tangent kernel}

        \begin{subtopic}{Linearized model}
            \[
                f[\vec{\theta}] \approx f[\vec{\theta}_0] + \langle \grad{f[\vec{\theta}_0]}{}, \vec{\theta} - \vec{\theta}_0 \rangle.
            \]
            $f$ is non-linear w.r.t.\ $\vec{x}$, but linear w.r.t.\ $\vec{\theta}$ $\rightarrow$ Define linear model with gradient feature map: \[
                h[\vec{\beta}](\vec{x}) = f[\vec{\theta}_0](\vec{x}) + \transpose{\vec{\beta}} \grad{f[\vec{\theta}_0](\vec{x})}{}.
            \]
            Kernel method with $k(\vec{x}, \vec{x}') = \langle \grad{f[\vec{\theta}_0](\vec{x})}{},
                \grad{f[\vec{\theta}_0](\vec{x}')}{} \rangle$ and MSE loss, \[
                \vec{\beta}^\star = \transpose{\mat{\Phi}} \inv{\mat{K}} (\vec{y}-\vec{f}), \quad \mat{K} = \mat{\Phi} \transpose{\mat{\Phi}}, \quad \vec{\phi}_i = \grad{f[\vec{\theta}_0](\vec{x}_i)}{}.
            \]
            Predictions: \[
                h^\star(\vec{x}) = \transpose{\vec{k}(\vec{x})} \inv{\mat{K}} (\vec{y} - \vec{f}).
            \]
            Linearized models are non-competitive with full networks. Also they may be intractable due to
            number of samples and parameters. Benefit: We can look at DNNs through the lens of kernel methods
            if the parameters do not evolve far away from $\vec{\theta}_0$.
        \end{subtopic}

        \begin{subtopic}{Training dynamics}
            Gradient flow ODE with MSE loss: \[
                \dot{\vec{\theta}}_t = \sum_{i=1}^{n} (y_i - f[\vec{\theta}_t](\vec{x}_i)) \grad{f[\vec{\theta}_t](\vec{x})}{}.
            \]
            Functional gradient flow: \[
                \dot{f}_j[\vec{\theta}_t] = \transpose{\dot{\vec{\theta}}_t} \grad{f[\vec{\theta}_t](\vec{x}_j)}{} = \sum_{i=1}^{n} (y_i - f[\vec{\theta}_t](\vec{x}_i)) k[\vec{\theta}_t](\vec{x}_i, \vec{x}_j).
            \]
            In matrix form: \[
                \text{\hl{$\dot{\vec{f}}[\vec{\theta}_t] = \mat{K}[\vec{\theta}_t](\vec{y} - \vec{f}[\vec{\theta}_t])$}}, \quad \dot{\vec{f}}[\vec{\theta}_t] = -\mat{K}[\vec{\theta}_t] \grad{\ell(\vec{\theta}_t)}{\vec{f}[\vec{\theta}]}.
            \]
            NTK $\mat{K}[\vec{\theta}_t]$ governs the evolution of the joint sample predictions. Problem: NTK
            has a dependence on the parameters.
        \end{subtopic}

        \begin{subtopic}{Infinite width}
            In practice it has been found that as the width $m$ of a model is scaled, the parameters stay
            more closely to their initialization during gradient descent. \hl{It can be shown (under
            basic assumptions) that the NTK converges in probability to a deterministic limit as the
            model is scaled to infinite width:} $k[\vec{\theta}] \xrightarrow{m \to \infty} k_{\infty}$.

            The deterministic limit depends only on the law of initialization. Under these training dynamics,
            minimizing MSE equates to solving a kernel regression problem with $k_{\infty}$. This provides
            insight into why overparameterization works so well in practice, despite having the ability to
            overfit.
        \end{subtopic}

        \begin{subtopic}{NTK constancy}
            The NTK remains constant under gradient flow: $\pdv{\mat{K}[\vec{\theta}_t]}{t} = \vec{0}$.
        \end{subtopic}

    \end{topic}

    \begin{topic}{Bayesian learning}
        Goal is to compute the Bayesian predictive posterior: \[
            f(\vec{x}) = \int p(\vec{\theta} \mid \mathcal{S}) f[\vec{\theta}](\vec{x}) \mathrm{d}\vec{\theta},
        \]
        where \[
            p(\vec{\theta} \mid \mathcal{S}) = \frac{p(\mathcal{S} \mid \vec{\theta})}{p(\mathcal{S})}, \quad p(\mathcal{S}) = \int p(\vec{\theta}) p(\mathcal{S} \mid \vec{\theta}) \mathrm{d}\vec{\theta}.
        \]
        $p(\mathcal{S})$ is intractable but often not necessary. An isotropic Gaussian prior leads to MAP optimizing $\ell_2$ regularization.

        \begin{subtopic}{Markov chain Monte Carlo}
            However, we do not want MAP only, we want the full distribution. MCMC methods sample from
            the posterior by constructing a Markov chain, where the stationary distribution is the
            posterior.

            Detailed Balance Equation: \[
                q(\vec{\theta}) \Pi(\vec{\theta}' \mid \vec{\theta}) = q(\vec{\theta}') \Pi(\vec{\theta} \mid \vec{\theta}'), \quad \forall \vec{\theta}, \vec{\theta}'.
            \]
            Then, the Markov chain is time reversible and has the unique stationary distribution $q$.

            \textit{Metropolis-Hastings} samples from an arbitrary Markov chain with kernel $\tilde{\Pi}$ and adjusts it
            such that DBE is satisfied for the posterior. It does so by constructing a new kernel:
            \begin{align*}
                \Pi(\vec{\theta}' \mid \vec{\theta})    & = \tilde{\Pi}(\vec{\theta}' \mid \vec{\theta}) \alpha(\vec{\theta}' \mid \vec{\theta})                                                                                                        \\
                \alpha(\vec{\theta}' \mid \vec{\theta}) & = \min \lft\{ 1, \frac{p(\vec{\theta} \mid \mathcal{S}) \tilde{\Pi}(\vec{\theta}' \mid \vec{\theta})}{p(\vec{\theta}' \mid \mathcal{S}) \tilde{\Pi}(\vec{\theta} \mid \vec{\theta}')} \rgt\}.
            \end{align*}
            This is the unique choice of acceptance function $\alpha$ that has a one-sided structure. If $\tilde{\Pi}$ is
            symmetric, we only need the ratio of posteriors, not $p(\mathcal{S})$. Problems: Burn-in
            period can be arbitrarily long due to poor $\tilde{\Pi}$ leading to high rejection
            probabilities.

            \textit{Hamiltonian Monte Carlo} obtains posterior averages. Energy function: \[
                E(\vec{\theta}) = - \sum_{\vec{x},y} \log p[\vec{\theta}](y \mid \vec{x}) - \log p(\vec{\theta}).
            \]
            The Hamiltonian augments with momentum vector: \[
                H(\vec{\theta}, \vec{v}) = E(\vec{\theta}) + \frac{1}{2} \transpose{\vec{v}} \inv{\mat{M}} \vec{v}.
            \]
            Hamiltonian dynamics: \[
                \dot{\vec{v}} = -\grad{E(\vec{\theta})}{}, \quad \dot{\vec{\theta}} = \vec{v}.
            \]
            HMC discretizes with stepsize $\eta$: \[
                \vec{\theta}_{t+1} = \vec{\theta}_t + \eta \vec{v}_t, \quad \vec{v}_{t+1} = \vec{v}_t - \eta \grad{E(\vec{\theta}_t)}{}.
            \]
            We sample the posterior by following momentum-based gradient descent dynamics. (We can also view
            this as momentum-based gradient descent leading to a single sample approximation of the predictive
            distribution.) Problem: We need to compute the full gradient.

            \textit{Langevin dynamics} extends HMC by friction: \[
                \dot{\vec{\theta}} = \vec{v}, \quad \mathrm{d}\vec{v} = -\grad{E(\vec{\theta})}{} \mathrm{d}t - \mat{B} \vec{v}\mathrm{d}t + \mathcal{N}(\vec{0}, 2 \mat{B} \mathrm{d}t).
            \]
            Friction reduces momentum and dissipates kinetic energy, while the Wiener noise injects
            stochasticity. Discretize:
            \begin{align*}
                \vec{\theta}_{t+1} & = \vec{\theta}_t + \eta \vec{v}_t                                                                                           \\
                \vec{v}_{t+1}      & = (1-\eta \gamma) \vec{v}_t - \eta s \grad{\tilde{E}(\vec{\theta})}{} + \sqrt{2 \gamma \eta} \mathcal{N}(\vec{0}, \mat{I}).
            \end{align*}
            Here, $\tilde{E}$ is a stochastic potential function which is the empirical loss over a random
            mini-batch of data.
        \end{subtopic}

        \begin{subtopic}{Gaussian process}
            GPs is a fully tractable Bayesian method. $f$ is a GP if for every finite subset $\{ \vec{x}_1, \ldots, \vec{x}_n \} \subseteq \mathcal{X}$, the resulting finite marginal is jointly normally distributed, \[
                [f(\vec{x}_1), \ldots, f(\vec{x}_n)] \sim \mathcal{N}(\vec{\mu}, \mat{\Sigma}).
            \]
            In GPs, the mean can be computed by deterministic regression and the covariance matrix is evaluated
            by a kernel function: \[
                \sigma_{ij} = k(\vec{x}_i, \vec{x}_j).
            \]
            The kernel function can be seen as a prior over function space that describes how related the
            output values corresponding to the two input value should be. RBF kernel encodes that close input
            value should have close output values: \[
                k(\vec{x}, \vec{x}') = \exp \lft( -\gamma \| \vec{x} - \vec{x}' \|^2 \rgt).
            \]
            Linear networks assume a random Gaussian weight vector: \[
                \vec{w} \sim \mathcal{N}\lft(\vec{0}, \frac{\sigma^2}{d} \mat{I}_d\rgt).
            \]
            Outputs are computed by $y_i = \transpose{\vec{w}}\vec{x}_i$. Vectorized: $\vec{y} = \mat{X} \vec{w}$.
            This is a Gaussian vector: \[
                \vec{y} \sim \mathcal{N}\lft(\vec{0}, \frac{\sigma^2}{d} \transpose{\mat{X}} \mat{X} \rgt).
            \]
            In other words, it is a GP with the following kernel: \[
                k(\vec{x}, \vec{x}') = \frac{\sigma^2}{d} \transpose{\vec{x}}\vec{x}'.
            \]
            We can do this for multiple units because the preactivations of units in the same layer are
            independent, conditioned on the input. In general, we do not get the same effect if we increase the
            depth, because there is randomness not only in the weights but also the preactivations. However, a
            deep preactivation process is ``near normal'' for high-dimensional inputs.

            We can extend this to non-linear networks. But, the activations are no longer Gaussian due to the
            non-linearity. \hl{However, due to CLT, they are effectively shaped back into Gaussians when they
            propagate to the next layer.} The mean function and kernels are computed by \[
                \mu(\vec{x}^{\ell}) = \E[\phi(\mat{W}^{\ell-1} \vec{x}^{\ell-1})], \quad k^{\ell}(\vec{x}_i, \vec{x}_j) = \E[\phi(\vec{x}^{\ell-1}_i) \phi(\vec{x}^{\ell-1}_j)].
            \]
            We can now use kernel regression: \[
                f^\star(\vec{x}) = \transpose{\vec{k}(\vec{x})} \mat{K}^{-1}\vec{y}, \quad k = k^L.
            \]
            In conclusion, DNNs in the infinite-width limit can be thought of as GPs (because then all
            preactivations can be viewed as Gaussians). Benefit: Uncertainty quantification, No training,
            Leverage tricks form kernel machines. Problems: Computing $f^\star$ and storing $\mat{K}^{\ell}$
            are not feasible, It is much less efficient than optimizing weights with gradient descent.

        \end{subtopic}

    \end{topic}

    \begin{topic}{Statistical learning theory}

        \begin{subtopic}{VC theory}
            Let $\mathcal{F}$ be a class of binary classifiers. Then the following is the set of possible classification outcomes over a dataset $\mathcal{S}$: \[
                \mathcal{F}(\mathcal{S}) = \{ [f(\vec{x}_1), \ldots, f(\vec{x}_n)] \in \{ 0,1 \}^n \mid f \in \mathcal{F} \}.
            \]
            Further define maximum number: \[
                \mathcal{F}(n) = \sup_{|\mathcal{S}| = n} |\mathcal{F}(\mathcal{S})|.
            \]
            $\mathcal{F}$ shatters $\mathcal{S}$ if $|\mathcal{F}(\mathcal{S})| = 2^n$ (every
            possible labeling is realized by some function $f \in \mathcal{F}$). VC dimensionality is defined as \[
                \mathrm{VC}(\mathcal{F}) = \argmax_{n \in \mathbb{N}} \sup_{|\mathcal{S}|=n} \mathbb{1}\{ \mathcal{F}(n) = 2^n \}.
            \]
            Under uniform convergence, VC inequality holds: \[
                \mathbb{P}\lft( \sup_{f \in \mathcal{F}} |\hat{\ell}(f) - \ell(f)| > \epsilon \rgt) \leq 8 |\mathcal{F}(n)| \exp \lft( -\frac{n \epsilon^2}{32} \rgt),
            \]
            where $\ell$ is the expected loss and $\hat{\ell}$ is the empirical loss. Intuition: No
            generalization guarantees can be given if $\mathcal{F}$ can be fit to any labeling.

            Randomization experiments with CIFAR-10 observations:
            \begin{enumerate}[left=0pt]
                \item DNNs can perfectly fit the training data;
                \item When randomly replacing training labels, the models can still perfectly fit the data
                      (memorization);
                \item The training time does not increase much when labels are randomized;
                \item When randomly shuffling pixels, the models can also perfectly fit the data $\Rightarrow$ Inductive bias of
                      CNNs does not provide much benefit in this regard.
            \end{enumerate}
            These findings are unexplainable by the above theory.

            Overparameterization can lead to double descent phenomenon, where large models will eventually
            start generalizing better after overfitting.

            The flatness of local minima are linked to their generalization ability, because at flat minima,
            small perturbations in the parameters will only have a small effect on performance. These can be
            found by small-batch SGD, weight averaging, or entropy SGD.
        \end{subtopic}

        \begin{subtopic}{PAC-Bayesian}
            For any $p \gg q$ and $p$-measurable $X$: $\E_q[X] \leq D_{\mathrm{KL}}(q \;\|\; p) + \log \E_p[\exp(X)]$.

            For a fixed $p$, any $q$, and $\epsilon \in (0,1)$, we have the following with probability greater
            than or equal to $\epsilon$, \[
                \E_q[\ell(f)] - \E_q[\hat{\ell}(f)] \leq \sqrt{\frac{2}{n} \lft( D_{\mathrm{KL}}(q \;\|\; p) + \log \frac{2 \sqrt{n}}{\epsilon} \rgt)}.
            \]
            Here, $p$ is a prior over parameters, $q$ the posterior, and we bound the expected generalization
            gap over stochastic classifiers $\Rightarrow$ $\bigo{\nicefrac{1}{\sqrt{n}}}$ bound on generalization
            error. However, it only applies to stochastic classifiers, not single classifiers.

            This motivated the PAC Bayes loss function \[
                \ell_{\mathrm{PAC}}(q) \doteq \E_q[\hat{\ell}] + \sqrt{\frac{2}{n} \lft( D_{\mathrm{KL}}(q \;\|\; p) + \log \frac{2\sqrt{n}}{\epsilon} \rgt)}.
            \]
            Effectively just a regularization term. The KL term can be computed in closed form if prior and
            posterior are Gaussian.
        \end{subtopic}

    \end{topic}

    \begin{topic}{Generative models}

        \begin{subtopic}{Autoencoders}
            Encoder $\bm{\mathcal{E}}$ maps data to latents and decoder $\bm{\mathcal{D}}$ maps
            latents to data. We want $\bm{\mathcal{D}}(\bm{\mathcal{E}}(\vec{x})) = \vec{x}$. Linear
            autoencoder with MSE loss is PCA of covariance matrix $\frac{1}{n} \transpose{\mat{X}}
                \mat{X}$ and taking $m$ principal eigenvectors. Intuition: Retain as much variance as possible.

            Linear factor analysis: Model latents with $p(\vec{x}) = \int p(\vec{z})
                p(\vec{x}\mid \vec{z}) \mathrm{d}\vec{z}$. $\vec{z} \sim \mathcal{N}(\vec{0},
                \mat{I})$. $\vec{x} = \vec{\mu} + \mat{W} \vec{z} + \vec{\eta}$, $\vec{\eta} \sim
                \mathcal{N}(\vec{0}, \mat{\Sigma})$. $\vec{\mu}^\star = \frac{1}{2} \sum_{i=1}^{n}
                \vec{x}_i$. $\vec{x} \sim \mathcal{N}(\vec{\mu}^\star, \mat{W} \transpose{\mat{W}}
                + \mat{\Sigma})$ (identifiable up to orthogonal transformation). $\vec{z} \mid \vec{x}$
                can be found and approaches PCA when $\sigma \to 0$ in $\mat{\Sigma}$.

            VAE optimizes log-likelihood of data $\Rightarrow$ Intractable $\Rightarrow$ ELBO:
            \begin{align*}
                \log p[\vec{\theta}](\vec{x}) & \geq \E_{p[\vec{\vartheta}](\vec{z} \mid \vec{x})} [\log p[\vec{\theta}](\vec{x} \mid \vec{z})] \\
                                              & \quad \quad \quad - D_{\mathrm{KL}}(p[\vec{\vartheta}](\vec{z} \mid \vec{x}) \;\|\; p(\vec{z})).
            \end{align*}
            $p[\vec{\vartheta}]$ is the encoder distribution and $p[\vec{\theta}]$ is the decoder
            distribution. This is effectively a reconstruction loss with a regularization term, where
            the regularization term ensures a well-behaved latent space.

            In general, the posterior $p[\vec{\vartheta}](\vec{z} \mid \vec{x})$ is intractable, so we restrict
            it to Gaussians, \[
                \vec{z} \mid \vec{x}, \vec{\vartheta} \sim \mathcal{N}(\vec{\mu}[\vec{\vartheta}](\vec{x}), \mat{\Sigma}[\vec{\vartheta}](\vec{x})).
            \]
            And the prior is $\mathcal{N}(\vec{0}, \mat{I})$. Then, the KL divergence can be computed in a
            closed form: \[
                \frac{1}{2} \lft( \| \vec{\mu}[\vec{\vartheta}](\vec{x}) \|^2 + \tr{\mat{\Sigma}[\vec{\vartheta}](\vec{x})} - \log \det{\mat{\Sigma}[\vec{\vartheta}](\vec{x})} - m \rgt).
            \]
            This can be optimized using the reparameterization tick.
        \end{subtopic}

        \begin{subtopic}{Generative adversarial networks}
            Log-likelihood is not the only way to optimize a model. GANs provide a training signal by
            introducing a binary classifier that distinguishes between samples from ``nature'' (1) and the
            generator (0): discriminator.

            Augmented distribution over samples: \[
                \tilde{p}(\vec{x},y) = \frac{1}{2} (yp(\vec{x}) + (1-y)p[\vec{\theta}](\vec{x})).
            \]
            Bayes-optimal classifier: \[
                \mathbb{P}(y=1 \mid \vec{x}) = \frac{p(\vec{x})}{p(\vec{x}) + p[\vec{\theta}](\vec{x})}.
            \]
            Minimizing the logistic log-likelihood w.r.t.\ this discriminator gives the following loss for the
            generator, \[
                \ell^\star(\vec{\theta}) = D_{\mathrm{JS}}(p \;\|\; p[\vec{\theta}]) - \log 2,
            \]
            where
            \begin{align*}
                D_{\mathrm{JS}}(p \;\|\; q) & \doteq H\lft(\frac{p + q}{2} \rgt) - \frac{H(p) + H(q)}{2} \\
                D_{\mathrm{JS}}(p \;\|\; q) & \doteq \frac{1}{2} D_{\mathrm{KL}}\lft( p \;\middle\|\; \frac{p+q}{2} \rgt) + \frac{1}{2} D_{\mathrm{KL}}\lft( q \;\middle\|\; \frac{p+q}{2} \rgt).
            \end{align*}

            But, the optimal classifier is intractable, so we train a parametrized one
            $q[\vec{\varphi}]$, \[
                \vec{\theta}^\star, \vec{\varphi}^\star \in \argmin_{\vec{\theta}} \argmax_{\vec{\varphi}} \ell(\vec{\theta}, \vec{\varphi}),
            \]
            where \[
                \ell(\vec{\theta}, \vec{\varphi}) = \E_{\tilde{p}[\vec{\theta}]}[y \log q[\vec{\varphi}](\vec{x}) + (1-y) \log (1-q[\vec{\varphi}](\vec{x}))].
            \]
            We have the following bound: \[
                \ell^\star(\vec{\theta}) \geq \sup_{\vec{\varphi}} \ell(\vec{\theta}, \vec{\varphi}).
            \]
            Problem: Gradient descent-ascent is not guaranteed to converge. Solution: Extragradient
            optimization algorithm:
            \begin{align*}
                \vec{\theta}_{t+1}  & = \vec{\theta}_t - \eta \grad{\ell(\vec{\theta}_{t+\nicefrac{1}{2}}, \vec{\varphi}_t)}{\vec{\theta}}, \quad \vec{\theta}_{t+\nicefrac{1}{2}} = \vec{\theta}_t - \eta \grad{\ell(\vec{\theta}_t, \vec{\varphi}_t)}{\vec{\theta}}       \\
                \vec{\varphi}_{t+1} & = \vec{\varphi}_t + \eta \grad{\ell(\vec{\theta}_t, \vec{\varphi}_{t+\nicefrac{1}{2}})}{\vec{\varphi}}, \quad \vec{\varphi}_{t+\nicefrac{1}{2}} = \vec{\varphi}_t + \eta \grad{\ell(\vec{\theta}_t, \vec{\varphi}_t)}{\vec{\varphi}}.
            \end{align*}
            In practice, we also need to use a different loss function for the generator: \[
                \ell(\vec{\theta} \mid \vec{\varphi}) = \E_{p[\vec{\theta}]}[-\log q[\vec{\varphi}](\vec{x})],
            \]
            because otherwise the gradient goes to infinity when $q[\vec{\varphi}](\vec{x}) = 1$, which makes
            the generator saturate.
        \end{subtopic}

        \begin{subtopic}{Diffusion models}
            Map a simple distribution to a complex one in many steps: \[
                \pi = \pi_T \mapsto \pi_{T-1} \mapsto \cdots \mapsto \pi_0 \approx p.
            \]
            \textit{SDE view}: \[
                \mathrm{d}\vec{x}_t = -\frac{1}{2} \beta_t \vec{x}_t \mathrm{d}t \sqrt{\beta_t} \mathrm{d}\omega_t.
            \]
            Time-reversed: \[
                \mathrm{d}\vec{x}_t = \lft( -\frac{1}{2} \beta_t \vec{x}_t - \beta_t \grad{\log q_t(\vec{x}_t)}{\vec{x}_t} \rgt) \mathrm{d}t + \sqrt{\beta_t} \mathrm{d}\tilde{\omega}_t.
            \]
            Denoising amounts to approximating a vector field over the gradient of the probability distribution
            moving towards areas with high probability density. Score models approximate $\grad{\log
                    q_t(\vec{x}_t)}{\vec{x}_t}$.

            \textit{ELBO view}: Forward process: \[
                \vec{x}_t = \sqrt{1-\beta_t} \vec{x}_{t-1} + \sqrt{\beta_t} \vec{\epsilon}_t, \quad \vec{\epsilon}_t \sim \mathcal{N}(\vec{0}, \mat{I}).
            \]
            Energy of the stochastic process evolves as $\E[\| \vec{x}_t \|^2 \mid \vec{x}_{t-1}] =
                (1-\beta_t)\| \vec{x}_{t-1} \|^2 + \beta_t \mathrm{tr}(\mat{I})$. If $\E[\| \vec{x}_0 \|^2] =
                \mathrm{tr}(\mat{I}) = \dim(\vec{x}_0)$, then energy is conserved throughout the process.

            Closed form ($\bar{\alpha}_t = \prod_{\tau=1}^t (1-\beta_{\tau})$): \[
                \colorbox{yellow}{$\displaystyle \vec{x}_t \sim \mathcal{N}(\sqrt{\bar{\alpha}_t}, (1-\bar{\alpha}_t) \mat{I})$}.
            \]
            Backward process: \[
                \vec{x}_{t-1} \sim \mathcal{N}(\vec{\mu}[\vec{\theta}](\vec{x}_t, t), \mat{\Sigma}[\vec{\theta}](\vec{x}_t, t)).
            \]
            ELBO: \[
                \log p[\vec{\theta}](\vec{x}_0) \geq \sum_{t=0}^{T} \ell_t,
            \]
            where \[
                \colorbox{yellow}{$\displaystyle \ell_t = \begin{cases}
                    \E_q[\log p[\vec{\theta}](\vec{x}_0 \mid \vec{x}_1)]                                                          & t = 0     \\
                    -D_{\mathrm{KL}}(q(\vec{x}_t \mid \vec{x}_{t-1}, \vec{x}_0) \;\|\; p[\vec{\theta}](\vec{x}_t \mid \vec{x}_{t+1})) & 0 < t < T \\
                    -D_{\mathrm{KL}}(q(\vec{x}_T \mid \vec{x}_0) \;\|\; \pi)                                                          & t = T.
                \end{cases}$}
            \]
            The KL divergences can be analytically computed because all $q_t$ are Gaussians an we parameterized
            the network as a Gaussian. The $q$ targets are derived as \[
                q(\vec{x}_{t-1} \mid \vec{x}_t, \vec{x}_0) = \mathcal{N}(\vec{\mu}(\vec{x}_t, \vec{x}_0, t), \tilde{\beta}_t),
            \]
            where
            \begin{align*}
                \vec{\mu}(\vec{x}_t, \vec{x}_0, t) & = \frac{\sqrt{\bar{\alpha}_{t-1}} \beta_t}{1-\bar{\alpha}_t} \vec{x}_0 + \frac{1-\bar{\alpha}_{t-1}}{1-\bar{\alpha}_t} \sqrt{1-\beta_t} \vec{x}_t \\
                \tilde{\beta}_t                    & = \frac{1-\bar{\alpha}_{t-1}}{1-\bar{\alpha}_t} \beta_t.
            \end{align*}
            Thus $\ell_t$ simplify to \[
                \ell_t = -\frac{1}{2 \sigma_t^2} \| \vec{\mu}(\vec{x}_t, \vec{x}_0, t) - \vec{\mu}[\vec{\theta}](\vec{x}_t,t) \|^2,
            \]
            where $\sigma_t^2 \in [\beta_t, \tilde{\beta}_t]$.

            By noting the closed form forward process, we have \[
                \vec{x}_0 = \frac{1}{\sqrt{\bar{\alpha}_t}} \vec{x}_t - \frac{\sqrt{1-\bar{\alpha}_t}}{\bar{\alpha}_t} \vec{\epsilon}.
            \]
            We can thus rewrite \[
                \vec{\mu}(\vec{x}_t, \vec{x}_0, t) = \frac{1}{\sqrt{\alpha_t}} \lft( \vec{x}_t - \frac{\beta_t}{\sqrt{1-\bar{\alpha}_t}} \vec{\epsilon} \rgt).
            \]
            Note that $\vec{\epsilon}$ fully determines $\vec{x}_t$ and $\vec{x}_0$ is constant. So, we only
            need to predict $\vec{\epsilon}$. Simplified loss:
            \begin{align*}
                \E_q[\ell_t \mid \vec{x}_0] & = \E_{\vec{\epsilon}}[\lambda(t) \| \vec{\epsilon} - \vec{\epsilon}[\vec{\theta}](\vec{x}_t, t) \|^2] \\
                \lambda(t)                  & = \frac{\beta_t^2}{2 \sigma_t^2 \alpha_t (1-\bar{\alpha}_t)}.
            \end{align*}
            In practice, the loss is approximated by \[
                \ell(\vec{\theta} \mid \vec{x}_0) = \frac{1}{T} \sum_{t=1}^{T} \lft[ \| \vec{\epsilon} - \vec{\epsilon}[\vec{\theta}](\vec{x}_t, t) \|^2 \;\middle|\; \vec{x}_0 \rgt].
            \]

            Entropy bound: $H(\vec{x}_{t-1} \mid \vec{x}_t) \leq H(\vec{x}_t \mid \vec{x}_{t-1})$.
            The entropy of the reverse process is bounded by the entropy of the forward process.
        \end{subtopic}

    \end{topic}

    \begin{topic}{Adversarial attacks}

        The attacker wants to make small changes to the input such that the model gives a different result.

        \begin{subtopic}{$p$-norm robustness}
            Consider a multi-class classifier $f: \R^d \to [m]$. The goal of an adversarial attack is to find a perturbation $\vec{\eta}$ such that \[
                f(\vec{x} + \vec{\eta}) \neq f(\vec{x}), \quad \| \vec{\eta} \|_p \leq \epsilon.
            \]

            Consider a binary affine classifier and $p=2$, \[
                f(\vec{x}) = \argmax \{ \transpose{\vec{w}_1} \vec{x} + b_1, \transpose{\vec{w}_2}\vec{x} + b_2 \}.
            \]
            Assume $\vec{x}$ is classified as $1$ and we want to find $\vec{\eta}$ such that $\vec{x} +
                \vec{\eta}$ is classified as $2$ such that $\| \vec{\eta} \|_2$ is minimized $\Rightarrow$ Convex program:
            \begin{align*}
                \text{minimize}   & \quad \frac{1}{2} \| \vec{\eta} \|_2^2                                               \\
                \text{subject to} & \quad \transpose{(\vec{w}_1 - \vec{w}_2)} (\vec{x} + \vec{\eta}) + b_1 - b_2 \leq 0.
            \end{align*}
            Set gradient of Lagrangian to zero $\Rightarrow$ $\vec{\eta}^\star = \lambda (\vec{w}_2 -
                \vec{w}_1)$. Then, find $\lambda$ that satisfies constraint $\Rightarrow$ $\lambda \geq
                \nicefrac{f_1(\vec{x}) - f_2(\vec{x})}{\| \vec{w}_1 - \vec{w}_2 \|_2^2}$. Thus: \[
                \vec{\eta}^\star = \frac{f_1(\vec{x}) - f_2(\vec{x})}{\| \vec{w}_2 - \vec{w}_1 \|_2^2} (\vec{w}_2 - \vec{w}_1).
            \]
            This can be generalized to any source $i$ and target $j$. In the general case, we can linearize the
            model and iteratively solve the above convex program.
        \end{subtopic}

        \begin{subtopic}{Robust training}
            Robust training systematically makes models robust to adversarial attacks by extending the loss function to neighborhoods of training points: \[
                \ell(\vec{x}) \mapsto \max_{\vec{\eta} : \| \vec{\eta} \|_p \leq \epsilon} \ell(\vec{x} + \vec{\eta}).
            \]
            This can be solved with projected gradient ascent: \[
                \vec{\eta}_{t+1} = \Pi(\vec{\eta}_t + \alpha \grad{\ell(\vec{x} + \vec{\eta}_t)}{\vec{\eta}}).
            \]
            Fast gradient sign method (FGSM) performs one iteration with $p=\infty$ resulting in \hl{$\vec{\eta} =
            \epsilon\cdot\mathrm{sgn}(\grad{\ell(\vec{x})}{\vec{x}})$}.

                Projection onto $\ell_1$-ball has no closed form, but doable in $\bigo{n\log n}$.

            Projection onto $\ell_2$-ball: $\Pi_{\| \cdot \|_2 \leq \epsilon}(\vec{x}) = \frac{\epsilon \vec{x}}{\max \{ \epsilon, \| \vec{x} \|_2 \}}$.

            Projection onto $\ell_{\infty}$-ball: $\Pi_{\| \cdot \|_{\infty} \leq \epsilon}(\vec{x}) = [\mathrm{clamp}(x_i; -\epsilon, \epsilon)]_{i \in [d]}$.

        \end{subtopic}

    \end{topic}

\end{multicols*}

\end{document}
