\section{Recurrent neural networks}

\textit{Recurrent neural networks} (RNN) are a type of neural network that processes sequential data,
such as text and video. Unlike traditional neural networks, which take fixed-length inputs, RNNs can
take inputs of variable length.\sidenote{This is useful for data structures such as text, where the
    number of words in a text is not fixed.} Generally, RNNs have the following form, \[
    \vec{h}^{(t)} = f_{\vec{\theta}}\lft(\vec{h}^{(t-1)}, \vec{x}^{(t)}\rgt).
\]
In this way, an RNN behaves much like a dynamical system. If we unroll the timesteps, it becomes
clear how $\vec{h}^{(t)}$ depends on $\vec{x}^{(1:t)}$, \[
    \vec{h}^{(t)} = f_{\vec{\theta}}\lft(f_{\vec{\theta}}\lft(\cdots \lft(f_{\vec{\theta}}\lft(\vec{h}^{(0)}, \vec{x}^{(1)}\rgt), \cdots\rgt), \vec{x}^{(t-1)}\rgt), \vec{x}^{(t)}\rgt).
\]
Because of this, we can see $\vec{h}^{(t)}$ as a representation of $\vec{x}^{(1:t)}$.

RNNs can have different applications, for example, \textit{one to one}, where at each time step we
have one input and one output,\sidenote{\Eg, part-of-speech tagging.}, \textit{one to many}, where
we have one input and we output a sequence of elements,\sidenote{\Eg, image captioning, where the
    image is the one input, and the caption is a sequence of words.} \textit{many to one}, where we
have a sequence of inputs and one output,\sidenote{\Eg, sentiment classification, where the input
    is a text and the output is a single output that determines how positive or negative the text is.}
and \textit{many to many}, where we map a sequence to another sequence of a different
length.\sidenote{\Eg, machine translation, where we map a sentence in one language to a sentence of
    another.}

\subsection{Elman RNN}

The Elman RNN \citep{elman1990finding} is characterized by a hidden vector $\vec{h}^{(t)}$, which
forms the state of the network at timestep $t$. The hidden state is updated at each timestep by
combining the previous hidden state with the input, \[
    f_{\vec{\theta}} \lft( \vec{h}^{(t-1)}, \vec{x}^{(t)} \rgt) = \tanh\lft(\mat{W}_h \vec{h}^{(t-1)} + \mat{W}_x \vec{x}^{(t)}\rgt). \margintag{We use the $\tanh$ activation function, because it is centered at 0.}
\]
Since $\vec{h}^{(t)}$ represents the subsequence $\vec{x}^{(1:t)}$, we can use it as input to an
MLP, \[
    \hat{\vec{y}} = \mat{W}_y \vec{h}^{(t)}.
\]
Then, we can compute the loss function as the sum of each individual loss function, \[
    \mathcal{L} \doteq \sum_{t=1}^{T} \ell^{(t)}.
\]

\begin{marginfigure}[2cm]
    \centering
    \incfig{rnn}
    \caption{The computational graph of an unrolled recurrent neural network. The inputs $\vec{x}_{1:T}$ and outputs $\vec{y}_{1:T}$ are omitted.}
    \label{fig:rnn}
\end{marginfigure}

We use \textit{backpropagation through time} (BPTT) to compute the gradient of an RNN. This
involves first unrolling the RNN; see \Cref{fig:rnn}. Then we can compute the gradient by
backpropagation on the resulting computational graph, \[
    \pdv{\ell^{(t)}}{\mat{W}} = \sum_{k=1}^{t} \pdv{\ell^{(t)}}{\hat{\vec{y}}^{(t)}} \pdv{\hat{\vec{y}}^{(t)}}{\vec{h}^{(t)}} \pdv{\vec{h}^{(t)}}{\vec{h}^{(k)}} \pdv{^+ \vec{h}^{(k)}}{\mat{W}},
\]
where $\pdv{^+ \vec{h}^{(k)}}{\mat{W}}$ is the immediate derivative that treats $\vec{h}^{(k-1)}$
as constant \wrt $\mat{W}$.

Let's only consider the following term of the product,
\begin{align*}
    \pdv{\vec{h}^{(t)}}{\vec{h}^{(k)}} & = \prod_{i=k+1}^t \pdv{\vec{h}^{(i)}}{\vec{h}^{(i-1)}}                                                                                     \\
                                       & = \prod_{i=k+1}^t \pdv*{\sigma \lft( \mat{W}_h \vec{h}^{(i-1)} + \mat{W}_x \vec{x}^{(i)} \rgt)}{\vec{h}^{(i-1)}}                           \\
                                       & = \prod_{i=k+1}^t \transpose{\mat{W}_h} \mathrm{diag} \lft( \sigma' \lft( \mat{W}_h \vec{h}^{(i-1)} + \mat{W}_x \vec{x}^{(i)} \rgt) \rgt).
\end{align*}
Assuming that the norm of the gradient of $\sigma$ is upper bounded by some $\gamma \in
    \R$,\sidenote{For example, the gradient of $\tanh$ is bounded by $1$.} \ie, \[
    \lft\| \mathrm{diag} \lft(\sigma' \lft( \mat{W}_h \vec{h}^{(t-1)} + \mat{W}_x \vec{x}^{(t)} \rgt) \rgt) \rgt\| < \gamma.
\]
Let $\lambda_1$ be the largest eigenvalue of $\mat{W}_h$, then we have two cases,
\begin{enumerate}
    \item $\lambda_1 < \frac{1}{\gamma}$. Then we have the following, \[
              \lft\| \pdv{\vec{h}^{(i)}}{\vec{h}^{(i-1)}} \rgt\| \leq \|\mat{W}_h\| \lft\|\mathrm{diag}\lft( \sigma' \lft( \mat{W}_h \vec{h}^{(t-1)} + \mat{W}_x \vec{x}^{(t)} \rgt) \rgt)\rgt\| < \frac{1}{\gamma} \gamma = 1. \margintag{Triangle inequality and $\| \mat{W}_h \| = \lambda_1$.}
          \]
          Let $\eta < 1$ be the upper bound of all gradients between $\vec{h}^{(i)}$ and $\vec{h}^{(i-1)}$,
          then by induction, we have \[
              \lft\| \pdv{\vec{h}^{(t)}}{\vec{h}^{(k)}} \rgt\| = \lft\| \prod_{i=k+1}^t \pdv{\vec{h}^{(i)}}{\vec{h}^{(i-1)}} \rgt\| < \eta^{t-k}.
          \]
          This converges to zero, as $t \to \infty$. Thus, we have a \textit{vanishing gradient};
    \item $\lambda_1 > \frac{1}{\gamma}$. Using the same logic as in the other case, this yields \[
              \lft\| \pdv{\vec{h}^{(t)}}{\vec{h}^{(k)}} \rgt\| = \lft\| \prod_{i=k+1}^t \pdv{\vec{h}^{(i)}}{\vec{h}^{(i-1)}} \rgt\| > \eta^{t-k}.
          \]
          for an upper bound $\eta > 1$. Thus, this diverges to $\infty$ as $t \to \infty$. Thus, we have a
          \textit{exploding gradient}.
\end{enumerate}
Thus, we will always have vanishing or exploding gradients when using an Elman RNN. This makes it
hard for the architecture to capture long-term dependencies. A naive solution is to truncate BPTT to
only go back $m$ steps. However, then we still lose long-term signals, which we wanted to preserve.

\subsection{Long-short term memory}

We need to make sure there is constant error flow. For this, we need a connection between timesteps
that avoids exploding and vanishing gradients. A simple solution is the leaky unit,
\begin{align*}
    \hat{\vec{h}}^{(t)} & = f_{\vec{\theta}}\lft(\vec{h}^{(t-1)}, \vec{x}^{(t)}\rgt) \\
    \vec{h}^{(t)}       & = \alpha \vec{h}^{(t-1)} + (1- \alpha) \hat{\vec{h}}^{t}.
\end{align*}

The \textit{long-short term memory} (LSTM) architecture \citep{hochreiter1997long} takes this idea
further by keeping a separate memory cell. Access to this cell is protected through gates to make
sure that there is always a path between units, such that errors can propagate; see
\Cref{fig:lstm}. Furthermore, in contrast to the leaky unit, the LSTM Learns how to ``remember''
and ``forget'' information.

\begin{figure}
    \centering
    \incfig{lstm}
    \caption{LSTM architecture. The yellow squares are neural networks, and the white squares are point-wise operators. As can be seen, there is an ''information highway`` that can easily propagate errors at the top, because of the minimal modifications made to it.}
    \label{fig:lstm}
\end{figure}

The cell of an LSTM consists of 4 gates. All gates get $\vec{h}^{(t-1)}$ and $\vec{x}^{(t)}$ as
input. In particular, these gates have the following instructions,
\begin{itemize}
    \item $f: \R^d \times \R^d \to [0,1]^d$ is the \textit{forget gate} and has the role of scaling
          the old cell state $\vec{h}^{(t-1)}$. It ``decides which information should be forgotten''
          from the previous cell state. It does so by outputting a vector in $[0,1]^d$ where $0$ means
          forgetting completely and $1$ means remembering everything;

    \item $g: \R^d \times \R^d \to [-1, 1]^d$ is the \textit{gate} that decides what to write in the
          cell state $\vec{c}^{(t)}$;

    \item $i: \R^d \times \R^d \to [0,1]^d$ is the \textit{input gate} and has the role of ``deciding
          which values of the cell state $\vec{c}^{(t)}$ should be updated'' at the current time step.
          Again, it does so by outputting a vector in $[0,1]^d$ to decide what of the output of the $g$
          gate should be written to the cell state $\vec{c}^{(t)}$;

    \item $o: \R^d \times \R^d \to [0,1]^d$ is the \textit{output gate} and has the role of
          ``deciding which values of the current cell state $\vec{c}^{(t)}$ should be put in the output
          of the cell $\vec{h}^{(t)}$''.
\end{itemize}

We first compute all the gates,
\begin{align*}
    \vec{f}^{(t)} & = \sigma \lft( \mat{W}_{hf} \vec{h}^{(t-1)} + \mat{W}_{xf} \vec{x}^{(t)} \rgt) \margintag{You can also chain multiple LSTM units one after another, which results in this computation being performed multiple times per layer. If this is the case, then we replace $\vec{x}^{(t)}$ by the hidden vector of the previous unit for all units after the first.} \\
    \vec{i}^{(t)} & = \sigma \lft( \mat{W}_{hi} \vec{h}^{(t-1)} + \mat{W}_{xi} \vec{x}^{(t)} \rgt)                                                                                                                                                                                                                                                                                 \\
    \vec{o}^{(t)} & = \sigma \lft( \mat{W}_{ho} \vec{h}^{(t-1)} + \mat{W}_{xo} \vec{x}^{(t)} \rgt)                                                                                                                                                                                                                                                                                 \\
    \vec{g}^{(t)} & = \tanh \lft( \mat{W}_{hg} \vec{h}^{(t-1)} + \mat{W}_{xg} \vec{x}^{(t)} \rgt).
\end{align*}
Then, we compute the outputs that are propagated to the next layer,
\begin{align*}
    \vec{c}^{(t)} & = \vec{f}^{(t)} \odot \vec{c}^{(t-1)} + \vec{i}^{(t)} \odot \vec{g}^{(t)} \margintag{$\vec{f}^{(t)} \odot \vec{c}^{(t-1)}$ computes what to keep/forget and $\vec{i}^{(t)} \odot \vec{g}^{(t)}$ computes what to add.} \\
    \vec{h}^{(t)} & = \vec{o}^{(t)} \odot \tanh \lft( \vec{c}^{(t)} \rgt).
\end{align*}

The addition in the computation of $\vec{c}^{(t)}$ allows for gradients to directly propagate
through $\vec{c}^{(t-1)} \odot \vec{f}$. Also, it allows the model to ``select'' what information
should be retained. For example, at a high level in text, it might be helpful to store information
such as gender and countries of origin. See \citep{olah2015understanding} for more information
about the gates.

\subsection{Gradient clipping}

While LSTMs are a great solution to the vanishing gradient problem, we still have the possibility
of exploding gradients. This is what \textit{gradient clipping} solves. The idea is to limit the
maximum value of the gradient if it surpasses a predetermined threshold. In practice, the gradient
descent step gets transformed into the following update rule, \[
    \vec{\theta} \gets \begin{cases}
        \vec{\theta} - \gamma \vec{g}                         & \| \vec{g} \| \leq k \\
        \vec{\theta} - \gamma \frac{k}{\| \vec{g} \|} \vec{g} & \text{otherwise},
    \end{cases}
\]
where $\vec{g}$ is the gradient, $\gamma$ is the learning rate, and $k$ is the gradient threshold.
