\section{Tracking} \label{sec:tracking}

Tracking is the following of the movement of an object (point, region, or
template), \ie, we want to find the position of the object in the consecutive
frames. Thus, the problem statement is that we want to find the position of an
object in frame $t+1$, given that we know its position in frame $t$.

To make sure that the object detection location does not ``teleport``, we can
give the models the prior that objects have a constant velocity.\sidenote{This
    is especially useful when the camera is stationary.} We can do this by
penalizing points that are not in the direction of the current velocity of the
object.

\subsection{Point}

If we want to track a point between frames, we can make the assumption that the
color of the point will remain the same, and optimize the following error
function (one-dimensional case),
\begin{align*}
    E(h)           & = ( \tens{I}[x,t]) - \tens{I}[x+h,t+1] )^2                                                                   \\
                   & \approx ( \tens{I}[x,t] - \tens{I}[x,t+1] - h\tens{I}_x[x,t])^2 \margintag{First-order Taylor approximation} \\
    \pdv*{E(h)}{h} & = -2 \tens{I}_x[x,t] ( \tens{I}[x,t] - \tens{I}[x,t+1] - h\tens{I}_x[x,t]).                                  \\
    \intertext{At the minimum, where the derivative is 0, we get the following,}
    h              & = \frac{\tens{I}[x,t+1] - \tens{I}[x,t]}{\tens{I}_x[x,t]}.
\end{align*}

However, if the gradient is 0, \ie, when the image region is flat,
$\tens{I}_x[x,t]$ will be 0 and thus we have no information about which
direction the point moved in. This is very similar to the problem in optical
flow, called the aperture problem.

Furthermore, we assume to always move toward the closest minimum. But, the
image is non-convex in general, which has multiple solutions. This can be
solved by a high framerate, because then the movement between frames is small.

Like in optical flow, in two dimensions, we need more constraints to solve for
the displacement. And, like in optical flow, we can solve this by either
solving it globally (Horn-Schunck) or locally (Lucas-Kanade).

\subsection{Template}

In template tracking, we want to track an object between frames, represented by
a bounding box, which we call its template. The most naive solution is to
simply compare pixel intensities and minimizing the following error function, \[
    E(u,v) = \sum_{x,y} (\tens{I}[x+u,y+v] - \tens{T}[x,y])^2,
\]
where $\tens{T}$ is the template. However, the object in the template might
make a transformation such as a rotation or scaling. We could also parametrize
those transformations, instead of only a translation, by generalizing the
previous error function to any transformation family $W$,
\begin{align*}
    E(\vec{p})                          & = \sum_{\vec{x}} \lft( \tens{I}[W(\vec{x}; \vec{p})] - \tens{T}[\vec{x}] \rgt)^2,                                                                                                                                                             \\
    \intertext{where $\vec{p}$ parametrizes a ``warp`` $W$. However, we cannot
        directly solve for $\vec{p}$, thus we iteratively improve it by updating it
        with some $\delta_{\vec{p}}$,}
                                        & = \sum_{\vec{x}} \lft( \tens{I}[W(\vec{x};\vec{p} + \delta_{\vec{p}})] - \tens{T}[\vec{x}] \rgt)^2                                                                                                                                            \\
                                        & \approx \sum_{\vec{x}} \lft( \tens{I}[W(\vec{x};\vec{p})] + \delta_{\vec{p}} \grad{\tens{I}[W(\vec{x};\vec{p})]}{\vec{p}} - \tens{T}[\vec{x}] \rgt)^2 \margintag{First-order Taylor approximation}                                            \\
                                        & = \sum_{\vec{x}} ( \tens{I}[W(\vec{x};\vec{p})] + \delta_{\vec{p}} \grad{\tens{I}[W(\vec{x};\vec{p})]}{W(\vec{x};\vec{p})} \grad{W(\vec{x};\vec{p})}{\vec{p}} - \tens{T}[\vec{x}] )^2 \margintag{Chain rule}                                  \\
    \pdv*{E(\vec{p})}{\delta_{\vec{p}}} & = 2 \sum_{\vec{x}} \begin{aligned}[t] & \transpose{\lft( \grad{\tens{I}[W(\vec{x};\vec{p})]}{W(\vec{x};\vec{p})} \grad{W(\vec{x};\vec{p})}{\vec{p}} \rgt)}                                                          \\
                   & \lft( \tens{I}[W(\vec{x};\vec{p})] + \delta_{\vec{p}} \grad{\tens{I}[W(\vec{x};\vec{p})]}{W(\vec{x};\vec{p})} \grad{W(\vec{x};\vec{p})}{\vec{p}} - \tens{T}[\vec{x}] \rgt).
                                                             \end{aligned} \margintag{Chain rule}                                                                        \\
    \intertext{We can then solve for $\delta_{\vec{p}}$ by setting the gradient to 0,}
    \delta_{\vec{p}}                    & = \inv{\mat{H}} \sum_{\vec{x}} \transpose{ \lft( \grad{\tens{I}[W(\vec{x};\vec{p})]}{W(\vec{x};\vec{p})} \grad{W(\vec{x};\vec{p})}{\vec{p}} \rgt)} \lft( \tens{T}[\vec{x}] - \tens{I}[W(\vec{x};\vec{p})] \rgt),                              \\
    \intertext{with Hessian matrix $\mat{H}$ defined as follows,}
    \mat{H}                             & = \sum_{\vec{x}} \transpose{ \lft( \grad{\tens{I}[W(\vec{x};\vec{p})]}{W(\vec{x};\vec{p})} \grad{W(\vec{x};\vec{p})}{\vec{p}} \rgt)}  \lft( \grad{\tens{I}[W(\vec{x};\vec{p})]}{W(\vec{x};\vec{p})} \grad{W(\vec{x};\vec{p})}{\vec{p}} \rgt).
\end{align*}
We iteratively update the warp by $\delta_{\vec{p}}$ until convergence.

However, this method is not robust to image noise, since it assumes that pixel
intensities remain the same between frames. Furthermore, some three-dimensional
transformations are impossible to parametrize as a two-dimensional warp of an
image. For example, a person turning around \wrt the camera cannot be
parametrized as a two-dimensional warp.

\subsection{Tracking by detection}

We can also track by detecting keypoints in each frame and minimizing the
distance to the feature descriptor of the point we are tracking. This does not
require the assumption that the image intensities remain the same, and can
track large displacements.

We can also use this to track a region by matching keypoint descriptors of
the template with the next frame. Then, removing outliers with \eg RANSAC.
The bounding box in the next frame is then the box that encapsulates all
points that are left.

However, in many cases, we do not have an exact template, but we want to
track any object of a specific type. This can be done by detecting the objects
independently in each frame and then associating the detections over time
with a bipartite matching algorithm. This is especially important
when there are multiple objects in the scene.

\subsection{Online learning}

Often, the template changes throughout frames, but, until now we have assumed
that the template is constant. However, it is more realistic to assume that the
appearance of the tracked object changes. Online learning accounts for this by
collecting all detected features of the tracked object throughout the frames
and the background. Then, it uses these to train the detector every frame.
Thus, the detector improves every frame, and updates it representation of the
object.

The advantage of this is that it is robust to changes to the environment, \eg,
if we go from a dark to a bright environment. However, the disadvantage is that
the representation of the object can gradually drift to something else. But,
this can be avoided by not allowing the model to drift too far from our
initial template with additional constraints.
