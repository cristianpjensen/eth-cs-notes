\section{Statistical learning theory}

\subsection{PAC learning}

PAC (\textit{\textbf{P}robably \textbf{A}pproximately \textbf{C}orrect}) learning is a subfield of
machine learning that concerns itself with the definition of ``learnable'' and if something is
``learnable'', whether it can be learned by empirically minimizing a loss function. We will be
using the following fundamental definitions and notations,
\begin{itemize}
    \item $\mathcal{X}$: Data space over which there is some distribution $p$;
    \item $\mathcal{C}$: Concept class, whose elements decide the labels of $\mathcal{X}$. We can
          think of it as a set of functions $c: \mathcal{X} \to \mathcal{Y}$;
    \item $\mathcal{H}$: Hypothesis class, which is a space of solutions that the model can consider,
          \eg, the weight space of a linear model. Generally, $|\mathcal{H}| \ll |\mathcal{C}|$.
\end{itemize}
No additional prior knowledge on $p$ is available, which makes this setting differ from Bayesian
approaches, which require a prior on $\mathcal{X}$.

The setting is that nature decides the distribution over $\mathcal{X}$ and its concept class
$\mathcal{C}$. We draw samples from $\mathcal{X}$ according to $p$ and observe its labels from
$\mathcal{C}$---this forms the dataset $\mathcal{Z} \subseteq \mathcal{X} \times \mathcal{Y}$. The
learning algorithm receives $\mathcal{Z}$ and $\mathcal{H}$ as input and selects a hypothesis
$\hat{c} \in \mathcal{H}$. Lastly, the hypothesis $\hat{c}$ is evaluated on its generalization
error, \[
    \mathcal{R}(\hat{c}) \doteq \mathbb{P}(\hat{c}(X) \neq c(X)) = \E[\mathbb{1}\{ \hat{c}(X)\neq c(X) \}].
\]
In practice, the expected risk is approximated using held-out validation data, since we do not have
access to the distribution $p$ over $X$.

\begin{definition}[Learning algorithm]
    Let $\mathcal{H}$ and $c \in \mathcal{C}$ be a hypothesis class and a concept. A learning algorithm $\mathcal{A}$
    is an algorithm that receives as input a labeled dataset $\mathcal{Z} = \{ (\vec{x}_1, y_1),
        \ldots, (\vec{x}_n,y_n) \}$ with $y_i = c(\vec{x}_i)$ for all $i \in [n]$ and outputs a
    hypothesis $\hat{c} \in \mathcal{H}$.
\end{definition}

The objective of the learning algorithm is to minimize the generalization error. However, this is
intractable, because we do not have access to $p$, only its samples. Hence, we approximate it by
the empirical error, \[
    \hat{c} \in \argmin_{c \in \mathcal{H}} \frac{1}{n} \sum_{i=1}^{n} \mathbb{1}\{ \hat{c}(\vec{x}_i) \neq c(\vec{x}_i) \}.
\]

\begin{definition}[Learnable] \label{def:learnable}
    A learning algorithm $\mathcal{A}$ can learn a concept $c \in \mathcal{C}$ if there exists a
    polynomial function $\mathrm{poly}(\cdot, \cdot, \cdot)$ such that for any distribution $p$ on
    $\mathcal{X}$ and $\epsilon, \delta \in (0, \nicefrac{1}{2})$, if $\mathcal{A}$ receives as input
    a sample $\mathcal{Z}$ of size $n \geq \mathrm{poly}(\nicefrac{1}{\epsilon}, \nicefrac{1}{\delta},
        \mathrm{size}(c))$, then $\mathcal{A}$ outputs $\hat{c}$ such that \[
        \mathbb{P}(\mathcal{R}(\hat{c}) \leq \epsilon) \geq 1-\delta.
    \]
    Note that this probability is taken over the randomness of $\mathcal{Z}$ and the potential
    randomness of $\mathcal{A}$. The value $\mathrm{size}(c)$ indicates the size of the representation
    of concept $c$.
\end{definition}

Here, $\epsilon$ is an error parameter and $\delta$ denotes a confidence value.

\begin{definition}[PAC learnable]
    A concept class $\mathcal{C}$ is PAC learnable from a hypothesis class $\mathcal{H}$ if there is
    an algorithm that can learn any concept in $\mathcal{C}$.
\end{definition}

\begin{definition}[Efficient PAC learning]
    If $\mathcal{A}$ runs in time polynomial in only $\nicefrac{1}{\epsilon}$ and
    $\nicefrac{1}{\delta}$, $\mathcal{C}$ is efficiently PAC learnable.
\end{definition}

\subsection{Axis-aligned rectangles}

\begin{marginfigure}[-10cm]
    \centering
    \incfig{axis-aligned-rectangles}
    \caption{The dots, along with their color, represent the dataset $\mathcal{Z} \sim p$. The concept class $\mathcal{C}$ and
        hypothesis class $\mathcal{H}$ is the set of all rectangles in $\R^2$.}
    \label{fig:axis-aligned-rectangles}
\end{marginfigure}

We will now consider the problem of learning an axis-aligned rectangle---see
\Cref{fig:axis-aligned-rectangles}---and we will prove that this problem is PAC learnable. For
this, we need to show that there exists a $\mathrm{poly}(\cdot, \cdot, \cdot)$ such that for any $R
    \in \mathcal{C}$, $p$ on $\R^2$, and $\epsilon, \delta > 0$, the condition in \Cref{def:learnable}
holds. In this problem, $\mathrm{size}(R) = 4$. The algorithm we use is taking the tightest
rectangle around the points.

\begin{marginfigure}[-1cm]
    \centering
    \incfig{rig}
    \caption{The event $\hat{R}IG$ is the event that $\hat{R}$ intersects all four strips.}
    \label{fig:rig}
\end{marginfigure}

Let $\epsilon, \delta > 0$ and $R \in \mathcal{C}$. We now define an event $\hat{R}IG$
(\textit{$\bm{\hat{R}}$ \textbf{I}s \textbf{G}ood Enough}). Let $T^{\epsilon}_{\uparrow}$ be the
upper strip of $R$ such that $\mathbb{P}(T^{\epsilon}_{\uparrow}) = \nicefrac{\epsilon}{4}$---see
\Cref{fig:rig}.\sidenote{$\mathbb{P}(A)$, where $A$ is a set, is the probability of a data point
    $\vec{x} \sim p$ being in the set $\vec{x} \in A$.} Analogously define $T^\epsilon_\downarrow$,
$T^\epsilon_\leftarrow$, $T^\epsilon_\rightarrow$.

The $\hat{R}IG$ event occurs when a hypothesis $\hat{R}$ intersects all four $T^\epsilon_*$. This
only occurs if there is a sample from $p$ within every $T_*^\epsilon$. We have constructed these
spaces, such that the probability of a single sample being within one is $\nicefrac{\epsilon}{4}$.
Hence, the probability that a single sample from $p$ misses is $1-\nicefrac{\epsilon}{4}$, so the
probability that $n$ samples miss is $\lft( 1 - \nicefrac{\epsilon}{4} \rgt)^n$. The probability of
the $\hat{R}IG$ event is thus \[
    \mathbb{P}(\hat{R}IG) \geq 1 - 4 \cdot \lft( 1 - \frac{\epsilon}{4} \rgt)^n \geq 1 - 4 \exp \lft( -\frac{n \epsilon}{4} \rgt). \margintag{$1-x \leq \exp(-x)$.}
\]

The risk is the density of the difference between the sets $R$ and $\hat{R}$, \[
    \mathcal{R}(\hat{R}) = \mathbb{P}(\hat{R}(X) \neq R(X)) = \mathbb{P}((R \backslash \hat{R}) \cup (\hat{R} \backslash R)) = \mathbb{P}(R \backslash \hat{R}).
\]
Since $\mathbb{P}(R \backslash \hat{R}) \leq \epsilon$ is guaranteed in---but not limited to---the
event of $\hat{R}IG$, we have the following bound on the generalization error, \[
    \mathbb{P}(\mathcal{R}(\hat{R}) \leq \epsilon) = \mathbb{P}(\mathbb{P}(R \backslash \hat{R}) \leq \epsilon) \geq \mathbb{P}(\hat{R}IG).
\]
As a result, \[
    \mathbb{P}\lft( \mathcal{R}(\hat{R}) \leq \epsilon \rgt) \geq 1 - 4 \exp \lft( -\frac{n \epsilon}{4} \rgt).
\]
For $n \geq \frac{4}{\epsilon} \log \lft( \frac{4}{\delta} \rgt)$, we then have \[
    \mathbb{P}\lft( \mathcal{R}(\hat{R}) \leq \epsilon \rgt) \geq 1-\delta.
\]
This concludes the proof.

\subsection{Stochastic setting}

In general, the label of an instance $\vec{x} \in \mathcal{X}$ is not deterministically determined
by the underlying concept---it is modeled by a distribution on $\mathcal{X} \times \mathcal{Y}$.
This reflects that two instances with identical features may have different labels. Thus, the
definition of a dataset is altered, where it is now $\mathcal{Z} = \{ (\vec{x}_1, y_1), \ldots,
    (\vec{x}_n, y_n) \}$, where $(\vec{x}_i, y_i) \sim p$ for all $i \in [n]$. In this setting, the
goal is to find a hypothesis $\hat{c} \in \mathcal{H}$ with small generalization error, \[
    \mathcal{R}(\hat{c}) = \mathbb{P}_{\vec{x},y \sim p} (\hat{c}(\vec{x}) \neq y) = \E_{\vec{x},y\sim p} \lft[ \mathbb{1}\{ \hat{c}(\vec{x}) \neq y \} \rgt].
\]
If the Bayes-optimal classifier is not an element of the concept class $\mathcal{C}$, then it is
impossible to construct an algorithm that finds a $\hat{c} \in \mathcal{H}$ such that
$\mathcal{R}(\hat{c}) \leq \epsilon$ for all $\epsilon \in (0,\nicefrac{1}{2})$ for a polynomial
number of data points. Thus, instead we aim to minimize the generalization error to the best
concept, \[
    \mathcal{R}(\hat{c}) - \inf_{c \in \mathcal{C}} \mathcal{R}(c) \leq \epsilon.
\]

\begin{definition}[Learnable in the stochastic setting]
    A learning algorithm $\mathcal{A}$ can learn a concept $c \in \mathcal{C}$ from $\mathcal{H}$ if
    there is a $\mathrm{poly}(\cdot, \cdot, \cdot)$ such that for any distribution $p$ on
    $\mathcal{X} \times \mathcal{Y}$ and for any $\epsilon,\delta\in (0,\nicefrac{1}{2})$, if
    $\mathcal{A}$ receives as input a sample $\mathcal{Z}$ of size $n \geq
        \mathrm{poly}(\nicefrac{1}{\epsilon},\nicefrac{1}{\delta},\mathrm{size}(c))$, then $\mathcal{A}$ outputs $\hat{c} \in \mathcal{H}$ such that \[
        \mathbb{P}_{\mathcal{Z} \sim p} \lft( \mathcal{R}(\hat{c}) - \inf_{c \in \mathcal{C}} \mathcal{R}(c) \leq \epsilon \rgt) \geq 1-\delta.
    \]
\end{definition}

\subsection{Vapnik-Chervonenkis}

The VC (\textit{\textbf{V}apnik \textbf{C}hervonenkis}) dimension is defined as the cardinality of
the largest set of points that a concept class $\mathcal{C}$ can shatter, which means that there
exists a concept in $\mathcal{C}$ for every labeling of the points. Intuitively, this is a measure
of the complexity of the concept class.

% Let $\epsilon > 0$, $\mathcal{Z} = \{ (\vec{x}_1, y_1), \ldots, (\vec{x}_n, y_n) \}$, and
% $\hat{c}^\star_n$ be the hypothesis obtained by empirical risk minimization with
% $\mathcal{H}=\mathcal{C}$, \[
%     \hat{c}^\star_n \in \argmin_{c \in \mathcal{C}} \frac{1}{2} \lft| \{ (\vec{x}_i, y_i) \mid c(\vec{x}_i \neq y_i) \} \rgt|.
% \]
% If $\mathcal{C}$ is finite, then \[
%     \mathbb{P}\lft( \mathcal{R}(\hat{c}^\star_n) - \inf_{c \in \mathcal{C}} \mathcal{R}(c) > \epsilon \rgt) \leq 2 |\mathcal{C}| \exp \lft( -2n \epsilon^2 \rgt).
% \]
% For an uncountably large hypothesis class, we need the VC-dimension $\mathcal{VC}_{\mathcal{C}}$.
% The VC-dimension is a complexity measure of $\mathcal{C}$. If $\mathcal{VC}_{\mathcal{C}} > 2$,
% then \[
%     \mathbb{P}\lft( \mathcal{R}(\hat{c}^\star_n) - \inf_{c \in \mathcal{C}} \mathcal{R}(c) > \epsilon \rgt) \leq 9n^{\mathcal{VC}_{\mathcal{C}}} \exp \lft( -\frac{n \epsilon^2}{32} \rgt).
% \]

\begin{theorem}[\citep{vapnik1974theory}] \label{thm:vapnik1974}
    Assume a finite concept class such that $|\mathcal{C}| < \infty$ and $\mathcal{R}(c^\star) = 0$
    for the optimal $c^\star \in \mathcal{C}$. Further, define the empirical risk minimizer
    $c_n^\star$ as  \[
        c^\star_n \in \{ c \in \mathcal{C} \mid \hat{\mathcal{R}}_n(c) = 0 \}.
    \]
    Then, for every $n \in \N$ and $\epsilon > 0$, \[
        \mathbb{P}(\mathcal{R}(\hat{c}^\star_n) > \epsilon) \leq |\mathcal{C}| \exp(-n \epsilon).
    \]
    And, \[
        \E[\mathcal{R}(\hat{c}^\star_n)] \leq \frac{1 + \log |\mathcal{C}|}{n}.
    \]
\end{theorem}

\begin{proof}
    We will only show the first statement,
    \begin{align*}
        \mathbb{P}(\mathcal{R}(c^\star_n) > \epsilon) & \leq \mathbb{P} \lft( \max_{c \in \mathcal{C} : \hat{\mathcal{R}}_n(c) = 0} \mathcal{R}(c) > \epsilon \rgt)                              \\
                                                      & = \mathbb{P} \lft( \lft( \max_{c \in \mathcal{C}} \mathbb{1}\{ \hat{\mathcal{R}}_n(c) = 0 \} \cdot \mathcal{R}(c) \rgt) > \epsilon \rgt) \\
                                                      & = \E \lft[ \max_{c \in \mathcal{C}} \mathbb{1}\{ \hat{\mathcal{R}}_n(c) = 0 \} \cdot \mathbb{1}\{ \mathcal{R}(c) > \epsilon \} \rgt]     \\
                                                      & \leq \sum_{c \in \mathcal{C} : \mathcal{R}(c) > \epsilon}  \E \lft[ \mathbb{1} \{ \hat{\mathcal{R}}_n(c) = 0 \} \rgt]                    \\
                                                      & = \sum_{c \in \mathcal{C} : \mathcal{R}(c) > \epsilon}  \mathbb{P}(\hat{\mathcal{R}}_n(c) = 0)                                           \\
                                                      & \leq |\mathcal{C}| (1-\epsilon)^n                                                                                                        \\
                                                      & \leq |\mathcal{C}| \exp(-n \epsilon).
    \end{align*}
    This concludes the proof.
\end{proof}

\begin{theorem}[VC inequality]
    \[
        \mathbb{P} \lft( \mathcal{R}(\hat{c}^\star_n) - \inf_{c \in \mathcal{C}} \mathcal{R}(c) > \epsilon \rgt) \leq \mathbb{P} \lft( \sup_{c \in \mathcal{C}} |\hat{\mathcal{R}}_n(c) - \mathcal{R}(c)| > \frac{\epsilon}{2} \rgt).
    \]
\end{theorem}

\begin{proof}
    \begin{align*}
        \mathcal{R}(\hat{c}^\star_n) - \inf_{c \in \mathcal{C}} \mathcal{R}(c) & = \mathcal{R}(\hat{c}^\star_n) - \hat{\mathcal{R}}_n(\hat{c}^\star_n) + \hat{\mathcal{R}}_n(\hat{c}^\star_n) - \inf_{c \in \mathcal{C}} \mathcal{R}(c)                                                                                               \\
                                                                               & \leq \mathcal{R}(\hat{c}^\star_n) - \hat{\mathcal{R}}_n(\hat{c}^\star_n) + \hat{\mathcal{R}}_n(c^\star) - \mathcal{R}(c^\star) \margintag{$\hat{\mathcal{R}}_n(c_n^\star) \leq \hat{\mathcal{R}}_n(\tilde{c})$ for all $\tilde{c} \in \mathcal{C}$.} \\
                                                                               & \leq \sup_{c \in \mathcal{C}} | \hat{\mathcal{R}}_n(c) - \mathcal{R}(c) | + \sup_{c \in \mathcal{C}} | \hat{\mathcal{R}}_n(c) - \mathcal{R}(c) |                                                                                                     \\
                                                                               & \leq 2 \sup_{c \in \mathcal{C}} | \hat{\mathcal{R}}_n(c) - \mathcal{R}(c) |.
    \end{align*}
    The result follows.
\end{proof}

Note that this bound requires uniform convergence, since we minimize $\hat{c}_n^\star$ \wrt
empirical risk. So, the classifier is selected conditional on the data, in which case the weak law
of large numbers does not apply.

% \begin{lemma}[Markov inequality]
%     Let $X$ be a non-negative random variable, then \[
%         \mathbb{P}(X \geq \epsilon) \leq \frac{\E[X]}{\epsilon}.
%     \]
% \end{lemma}
%
% \begin{lemma}[Hoeffding's inequality]
%     Let $X$ be a random variable with $\E[X] = 0$ and $X \in [a,b]$, then for $s > 0$, \[
%         \E[\exp(sX)] \leq \exp \lft( \frac{s^2 (b-a)^2}{8} \rgt).
%     \]
% \end{lemma}

% This is not a very informative result, because the bound does not depend on the distribution, only
% its support. However, it is good enough if we know nothing of the distribution---we can find
% tighter bounds if we have information of it.

\begin{theorem}[\citep{hoeffding1994probability}] \label{thm:hoeffding}
    Let $X_1, \ldots, X_n$ be independent random variables such that $X_i \in [a_i, b_i]$ and $S_n = \sum_{i=1}^{n} X_i$, then for any $t > 0$,
    \begin{align*}
        \mathbb{P}(S_n - \E[S_n] \geq t)  & \leq \exp \lft( -\frac{2t^2}{\sum_{i=1}^{n} (b_i - a_i)^2} \rgt)  \\
        \mathbb{P}(S_n - \E[S_n] \leq -t) & \leq \exp \lft( -\frac{2t^2}{\sum_{i=1}^{n} (b_i - a_i)^2} \rgt).
    \end{align*}
\end{theorem}

In the case of normalized sums $\tilde{S}_n = \nicefrac{S_n}{n}$ and $t=n\epsilon$, we have
\begin{align*}
    \mathbb{P}\lft( \tilde{S}_n - \E \lft[ \tilde{S}_n \rgt] \geq \epsilon \rgt) & \leq \exp \lft( -\frac{2n^2 \epsilon^2}{\sum_{i=1}^{n} (b_i - a_i)^2} \rgt)          \\
                                                                                 & = \exp \lft( -\frac{2n \epsilon^2}{\sum_{i=1}^{n} \nicefrac{(b_i - a_i)^2}{n}} \rgt) \\
                                                                                 & \xrightarrow{n\to \infty} 0.
\end{align*}
We can also bound the absolute deviation,
\begin{align*}
    \mathbb{P}\lft( \lft| \tilde{S}_n - \E \lft[ \tilde{S}_n \rgt] \rgt| \geq \epsilon \rgt) & = \mathbb{P}(\tilde{S}_n - \E[\tilde{S}_n] \geq \epsilon) + \mathbb{P}(\tilde{S}_n - \E[\tilde{S}_n] \leq -\epsilon) \margintag{The two cases are disjoint.} \\
                                                                                             & \leq 2 \exp \lft( -\frac{2n \epsilon^2}{\sum_{i=1}^{n} \nicefrac{(b_i - a_i)^2}{n}} \rgt).
\end{align*}

\begin{theorem}
    Assume $|\mathcal{C}| \leq N$, then for all $\epsilon > 0$, \[
        \mathbb{P}\lft( \sup_{c \in \mathcal{C}} | \hat{\mathcal{R}}_n(c) - \mathcal{R}(c) | > \epsilon \rgt) \leq 2 N \exp \lft( -2n \epsilon^2 \rgt).
    \]
\end{theorem}

\begin{proof}
    This is a trivial application of the union bound and \Cref{thm:hoeffding},
    \begin{align*}
        \mathbb{P}\lft( \sup_{c \in \mathcal{C}} | \hat{\mathcal{R}}_n(c) - \mathcal{R}(c) | > \epsilon \rgt) & = \mathbb{P}\lft( \bigvee_{c \in \mathcal{C}} |\hat{\mathcal{R}}_n(c) - \mathcal{R}(c)| > \epsilon \rgt)                                           \\
                                                                                                              & \leq \sum_{c\in \mathcal{C}} \mathbb{P}\lft( |\hat{\mathcal{R}}_n(c) - \mathcal{R}(c)| > \epsilon \rgt)                                            \\
                                                                                                              & \leq \sum_{c\in \mathcal{C}} 2 \exp \lft( - 2n \epsilon^2 \rgt) \margintag{\Cref{thm:hoeffding} and $\hat{\mathcal{R}}_n: \mathcal{C} \to [0,1]$.} \\
                                                                                                              & \leq 2N \exp \lft( -2n \epsilon^2 \rgt).
    \end{align*}
    This concludes the proof.
\end{proof}

If we define the confidence $\delta \doteq 2N \exp \lft( -2n \epsilon^2 \rgt)$, then the precision
is \[
    \epsilon = \sqrt{\frac{\log N - \log \lft( \nicefrac{\delta}{2} \rgt)}{2n}}.
\]
Thus, we have a $\bigo{\log |\mathcal{C}|}$ dependence for the precision, which is not bad, unless
$|\mathcal{C}| \to \infty$. In conclusion, we have $\mathcal{R}(c) - \hat{\mathcal{R}}_n(c) \leq
    \epsilon$ with high probability uniformly for all concepts $c \in \mathcal{C}$, \[
    \mathcal{R}(c) \leq \hat{\mathcal{R}}_n(c) + \sqrt{\frac{\log N - \log \lft( \nicefrac{\delta}{2} \rgt)}{2n}}.
\]
This is the key difference between \Cref{thm:vapnik1974} and
\Cref{thm:hoeffding}---\Cref{thm:vapnik1974} only applies to the empirical risk minimizer.

However, the cardinality of $\mathcal{C}$ might be infinite. We can deal with this in two ways,
\begin{enumerate}
    \item Represent hypotheses by the classifications that they yield;
    \item Measure the $\mathcal{VC}$ dimension of $\mathcal{C}$ and select a hypothesis class with finite
          $\mathcal{VC}$-dimensionality.
\end{enumerate}

For example, we can apply the first solution to a linear classification problem. There are an
infinite number of hyperplane separators, but there are a finite number of ways that the datapoints
can be labeled. Instead of treating each hyperplane independently, we bundle all hyperplanes that
lead to the same classifications together.

