\section{Coordinate descent}

A problem with gradient descent in large-scale learning is that we need to compute the full
gradient in every iteration. This can be problematic for functions $f: \R^d \to \R$ with large $d$.
The idea of \textit{coordinate descent} is to update only one coordinate of the iterate at a time.
For this, we only need to compute one coordinate of $\grad{f(\vec{x}_t)}{}$ at a time, which we
assume to be a factor of $d$ faster to compute.

However, we also expect to pay a price for this in terms of the number of iterations required. It
turns out that, in the worst case, the number of iterations will increase by a factor of $d$, so we
only stand to gain. Under additional assumptions about $f$, coordinate descent can lead to provable
speedups.

\begin{definition}[Polyak-Łojasiewicz (PL) inequality]
    Let $f: \R^d \to \R$ be a differentiable function with a global minimum $\vec{x}^\star$. We say
    that $f$ satisfies the PL inequality if the following holds for some $\mu > 0$, \[
        \frac{1}{2} \| \grad{f(\vec{x})}{} \|^2 \geq \mu (f(\vec{x}) - f(\vec{x}^\star)), \quad \forall \vec{x} \in \R^d.
    \]
\end{definition}

\begin{corollary}
    Let $f: \R^d \to \R$ satisfy the PL-inequality for any $\mu > 0$, then every critical point
    ($\grad{f(\vec{x})}{} = \vec{0}$) is a global minimum of $f$.
\end{corollary}

There are non-convex functions that satisfy the PL inequality, as long as the above corollary is
satisfied.

\begin{lemma}
    Let $f: \R^d \to \R$ be differentiable and strongly convex with parameter $\mu > 0$. Then $f$
    satisfies the PL inequality for the same $\mu$.
\end{lemma}

\begin{proof}
    We start from the definition of strong convexity,
    \begin{align*}
        f(\vec{x}^\star) & \geq f(\vec{x}) + \transpose{\grad{f(\vec{x})}{}}(\vec{x}^\star - \vec{x}) + \frac{\mu}{2} \| \vec{x}^\star - \vec{x} \|^2              \\
                         & \geq f(\vec{x}) + \min_{\vec{y}} \lft( \transpose{\grad{f(\vec{x})}{}}(\vec{y}-\vec{x}) + \frac{\mu}{2} \| \vec{y} - \vec{x} \|^2 \rgt) \\
                         & = f(\vec{x}) - \frac{1}{2 \mu} \| \grad{f(\vec{x})}{} \|^2,
    \end{align*}
    where the last equality is found by first-order optimality of $g_{\vec{x}}(\vec{y}) \doteq
        \transpose{\grad{f(\vec{x})}{}}(\vec{y} - \vec{x}) + \frac{\mu}{2} \| \vec{y} - \vec{x} \|^2$,
    which is convex and hence any critical point is a global minimizer.
\end{proof}

To analyze coordinate descent, we need the notion of \textit{coordinate-wise smoothness}.

\begin{definition}[Coordinate-wise smoothness]
    Let $f: \R^d \to \R$ be differentiable, and $\mathcal{L} = [L_1, \ldots, L_d] \in \R^d_+$.
    Function $f$ is called coordinate-wise smooth (with parameter $\mathcal{L}$) if for every
    coordinate $i=1,\ldots,d$, the following holds, \[
        f(\vec{x} + \lambda \vec{e}_i) \leq f(\vec{x}) + \lambda \grad{f(\vec{x})}{i} + \frac{L_i}{2} \lambda^2, \quad \forall \vec{x}\in \R^d, \lambda \in \R.
    \]
\end{definition}

Compare this to the definition of smoothness (\Cref{def:smoothness}). In our new coordinate-wise
definition, we define $\vec{y}$ as $\vec{x} + \lambda \vec{e}_i$, since we only want to change
coordinate $i$. Hence, $\vec{y}-\vec{x}$ becomes $\lambda \vec{e}_i$. From there, it is easy to
that $\transpose{\grad{f(\vec{x})}{}}(\vec{y} - \vec{x})$ becomes $\lambda \grad{f(\vec{x})}{i}$,
and $\| \vec{x}-\vec{y} \|$ becomes $\lambda$. Thus, smoothness with parameter $L$ implies
coordinate-wise smoothness with parameter $\mathcal{L} = [L, \ldots, L]$.

\begin{example}
    $f(x_1, x_2) = x_1^2 + 10x_2^2$ is smooth with parameter $L = 20$, but $f$ is coordinate-wise
    smooth with parameter $\mathcal{L} = [2, 20]$. Such differences will become important later when
    showing faster convergence of coordinate descent.
\end{example}

In general, coordinate (gradient) descent algorithms perform the following actions,
\begin{align*}
     & \text{choose $i \in [d]$}                                              \\
     & \vec{x}_{t+1} = \vec{x}_t - \gamma_i \grad{f(\vec{x}_t)}{i} \vec{e}_i.
\end{align*}

\begin{lemma}[Coordinate-wise sufficient decrease]
    Let $f: \R^d \to \R$ be differentiable and coordinate-wise smooth with parameter $\mathcal{L} =
        [L_1, \ldots, L_d]$. With active coordinate $i$ in iteration $t$ and stepsize \[
        \gamma_i \doteq \frac{1}{L_i},
    \]
    coordinate descent satisfies \[
        f(\vec{x}_{t+1}) \leq f(\vec{x}_t) - \frac{1}{2 L_i} | \grad{f(\vec{x}_t)}{i} |^2.
    \]
\end{lemma}

\begin{proof}
    Let $\lambda = \nicefrac{-\grad{f(\vec{x}_t)}{i}}{L_i}$, then $\vec{x}_{t+1} = \vec{x}_t +
        \lambda \vec{e}_i$. Then, we can apply coordinate-wise smoothness,
    \begin{align*}
        f(\vec{x}_{t+1}) & \leq f(\vec{x}_t) + \lambda \grad{f(\vec{x}_t)}{i} + \frac{L_i}{2} \lambda^2 \margintag{Coordinate-wise smoothness.} \\
                         & = f(\vec{x}_t) - \frac{1}{L_i} | \grad{f(\vec{x}_t)}{i} |^2 + \frac{1}{2L_i} | \grad{f(\vec{x}_t)}{i} |^2            \\
                         & =  f(\vec{x}_t) - \frac{1}{2 L_i} | \grad{f(\vec{x}_t)}{i} |^2.
    \end{align*}
\end{proof}

\subsection{Randomized coordinate descent}

In \textit{randomized gradient descent}, the active coordinate is chosen uniformly at random,
\begin{align*}
    i             & \sim \mathrm{Unif}([d])                                  \\
    \vec{x}_{t+1} & = \vec{x}_t - \gamma_i \grad{f(\vec{x}_t)}{i} \vec{e}_i.
\end{align*}

Randomized coordinate descent is at least as fast as gradient descent on smooth functions if we
assume that it is $d$ times cheaper to update one coordinate than the full iterate
\citep{nesterov2012efficiency}. If we additionally assume the PL inequality, we can obtain
geometric convergence as in the strongly convex case of gradient descent.

\begin{theorem}[Uniform sampling convergence]
    Let $f: \R^d \to \R$ be differentiable with global minimum $\vec{x}^\star$. Suppose that $f$ is coordinate-wise smooth with parameter $L$ and satisfies the PL inequality with parameter $\mu > 0$. Choosing stepsize \[
        \gamma_i \doteq \frac{1}{L},
    \]
    randomized gradient descent with arbitrary $\vec{x}_0$ yields \[
        \E[f(\vec{x}_T) - f(\vec{x}^\star)] \leq \lft( 1 - \frac{\mu}{dL} \rgt)^T (f(\vec{x}_0) - f(\vec{x}^\star)).
    \]
\end{theorem}

\begin{proof}
    Coordinate-wise sufficient decrease yields \[
        f(\vec{x}_{t+1}) \leq f(\vec{x}_t) - \frac{1}{2L} | \grad{f(\vec{x}_t)}{i} |^2.
    \]
    By taking the expectation of both sides with respect to $i$ and conditioned on $\vec{x}_t$, we
    obtain
    \begin{align*}
        \E_i [f(\vec{x}_{t+1}) \mid \vec{x}_t] & \leq \E_i \lft[ f(\vec{x}_t) - \frac{1}{2L} | \grad{f(\vec{x}_t)}{i} |^2 \;\middle|\; \vec{x}_t \rgt] \\
                                               & = f(\vec{x}_t) - \frac{1}{2L} \sum_{i=1}^{d} \frac{1}{d} | \grad{f(\vec{x}_t)}{i} |^2                 \\
                                               & = f(\vec{x}_t) - \frac{1}{2dL} \| \grad{f(\vec{x}_t)}{} \|^2                                          \\
                                               & \leq f(\vec{x}_t) - \frac{\mu}{dL} (f(\vec{x}_t) - f(\vec{x}^\star)). \margintag{PL inequality.}
    \end{align*}
    Subtracting $f(\vec{x}^\star)$ from both sides and taking the expectation over $\vec{x}_t$, we obtain \[
        \E[f(\vec{x}_{t+1}) - f(\vec{x}^\star)] \leq \lft( 1 - \frac{\mu}{dL} \rgt) \E[f(\vec{x}_t) - f(\vec{x}^\star)].
    \]
    The statement follows.
\end{proof}

\subsection{Importance sampling}

As seen, uniformly random selection of the active coordinate does not yield a better bound than
gradient descent. However, we have not made use of the fact that the coordinate-wise smoothness
parameters can differ. Intuitively, we would want to sample coordinates with high smoothness more
frequently than coordinates with low smoothness, since they change more rapidly, leading to faster
convergence to the optimum. This leads us to \textit{importance sampling}
\citep{nesterov2012efficiency},
\begin{align*}
    i             & \sim \mathrm{Categorical}\lft( \frac{L_1}{\sum_{j=1}^{d} L_j}, \ldots, \frac{L_d}{\sum_{j=1}^{d} L_j} \rgt) \\
    \vec{x}_{t+1} & = \vec{x}_t - \frac{1}{L_i} \grad{f(\vec{x}_t)}{i} \vec{e}_i.
\end{align*}

\begin{theorem}[Importance sampling convergence]
    Let $f: \R^d \to \R$ be differentiable with a global minimum $\vec{x}^\star$. Suppose that $f$ is coordinate-wise smooth with parameter $\mathcal{L}=[L_1, \ldots, L_d]$ and satisfies the PL inequality with parameter $\mu > 0$. Let \[
        \bar{L} = \frac{1}{d} \sum_{i=1}^{d} L_i
    \]
    be the average of coordinate-wise smoothness constants. Then, coordinate descent with importance
    sampling and arbitrary $\vec{x}_0$ yields \[
        \E[f(\vec{x}_T - f(\vec{x}^\star)] \leq \lft( 1 - \frac{\mu}{d\bar{L}} \rgt)^T (f(\vec{x}_0) - f(\vec{x}^\star)).
    \]
\end{theorem}

\begin{proof}
    The proof is nearly identical to the proof of randomized coordinate descent. The difference lies
    in the expectation over $i$. Importance sampling yields
    \begin{align*}
        \E_i \lft[ f(\vec{x}_t) - \frac{1}{2L_i} | \grad{f(\vec{x}_t)}{i} |^2 \;\middle|\; \vec{x}_t \rgt] & = f(\vec{x}_t) - \sum_{i=1}^{d} \frac{L_i}{\sum_{j=1}^{d} L_j} \frac{1}{2L_i} | \grad{f(\vec{x}_t)}{i} |^2                       \\
                                                                                                           & =  f(\vec{x}_t) - \frac{1}{2d\bar{L}} \sum_{i=1}^{d} | \grad{f(\vec{x}_t)}{i} |^2. \margintag{$d \bar{L} = \sum_{i=1}^{d} L_i$.}
    \end{align*}
    The rest follows identically.
\end{proof}

Note that $\bar{L}$ can be much smaller than $L = \max_{i=1}^d L_i$, so coordinate descent with
important sampling is potentially faster than randomized gradient descent. In the worst-case, both
algorithms are the same.

\subsection{Steepest coordinate descent}

In contrast to random coordinate descent, \textit{steepest coordinate descent} chooses the active
coordinate according to the coordinate with the largest gradient (\textit{Gauss-Southwell} rule),
\begin{align*}
    i             & \in \argmax_{j \in [d]} | \grad{f(\vec{x}_t)}{j} |       \\
    \vec{x}_{t+1} & = \vec{x}_t - \gamma_i \grad{f(\vec{x}_t)}{i} \vec{e}_i.
\end{align*}
The main difference from the previous algorithms is that this algorithm is deterministic, thus we do
not need to take the expectation.

\begin{theorem}[Steepest coordinate descent convergence]
    Let $f: \R^d \to \R$ be differentiable with a global minimum $\vec{x}^\star$. Suppose that $f$ is coordinate-wise smooth with parameter $L$ and satisfies the PL inequality with parameter $\mu > 0$. Choosing stepsize \[
        \gamma_i \doteq \frac{1}{L},
    \]
    steepest coordinate descent with arbitrary $\vec{x}_0$ yields \[
        f(\vec{x}_T) - f(\vec{x}^\star) \leq \lft( 1 - \frac{\mu}{dL} \rgt)^T (f(\vec{x}_0) - f(\vec{x}^\star)).
    \]
\end{theorem}

This is not good. It needs the same amount of iterations as randomized coordinate descent, but each
iteration takes as long as in gradient descent.\sidenote{Note that a function may be
    coordinate-wise smooth with an $L$ for all coordinates that is smaller than smoothness, so it is
    not completely fair to compare this to gradient descent.} However, this algorithm allows for a
speedup in certain cases. Furthermore, it may be possible to efficiently maintain the maximum
absolute gradient value throughout the iterations, so that the full evaluation of the gradient can
be avoided.

\cite{nutini2015coordinate} showed that a better convergence result can be obtained for strongly
convex functions, when strong convexity is measured \wrt the $\ell_1$-norm instead of the
$\ell_2$-norm. \Ie, \[
    f(\vec{y}) \geq f(\vec{x}) + \transpose{\grad{f(\vec{x})}{}} (\vec{y}-\vec{x}) + \frac{\mu_1}{2} \|\vec{y} - \vec{x}\|_1^2.
\]
Due to $\| \vec{y} - \vec{x} \|_1 \geq \| \vec{y} - \vec{x} \|_2$, $f$ is then also strongly convex
with $\mu = \mu_1$. On the other hand, if $f$ is $\mu$-strongly convex \wrt the $\ell_2$-norm, then
$f$ is $\nicefrac{\mu}{d}$-strongly convex \wrt the $\ell_1$-norm, due to $\|\vec{y}-\vec{x}\|_2
    \geq \nicefrac{\|\vec{y}-\vec{x}\|_1}{\sqrt{d}}$.

\begin{lemma}
    Let $f: \R^d \to \R$ be differentiable and strongly convex with parameter $\mu_1 > 0$ \wrt the
    $\ell_1$-norm. Then, $f$ is $\mu_1$-strongly convex \wrt the Euclidean norm, so a global minimum
    $\vec{x}^\star$ exists. Furthermore, $f$ satisfies the PL inequality \wrt the $\ell_{\infty}$-norm with the same $\mu_1$, \[
        \frac{1}{2} \| \grad{f(\vec{x})}{} \|_{\infty}^2 \geq \mu_1 (f(\vec{x}) - f(\vec{x}^\star)).
    \]
\end{lemma}

\begin{theorem}
    Let $f: \R^d \to \R$ be differentiable with a global minimum $\vec{x}^\star$. Suppose that $f$ is
    coordinate-wise smooth with parameter $L$ and satisfies the PL inequality \wrt $\ell_{\infty}$-norm
    with parameter $\mu_1 > 0$. Choosing stepsize \[
        \gamma_i \doteq \frac{1}{L},
    \]
    steepest coordinate descent with arbitrary $\vec{x}_0$ yields \[
        f(\vec{x}_T) - f(\vec{x}^\star) \leq \lft( 1 - \frac{\mu_1}{L} \rgt)^T (f(\vec{x}_0) - f(\vec{x}^\star)).
    \]
\end{theorem}

\begin{proof}
    \begin{align*}
        f(\vec{x}_{t+1}) & \leq f(\vec{x}_t) - \frac{1}{2L} | \grad{f(\vec{x}_t)}{i} |^2 \margintag{Sufficient decrease.}                                                             \\
                         & = f(\vec{x}_t) - \frac{1}{2L} \| \grad{f(\vec{x}_t)}{} \|_{\infty}^2 \margintag{Active coordinate of steepest coordinate descent is the maximum gradient.} \\
                         & \leq f(\vec{x}_t) - \frac{\mu_1}{L} (f(\vec{x}_t) - f(\vec{x}^\star)). \margintag{PL inequality.}
    \end{align*}
    Subtracting $f(\vec{x}^\star)$ from both sides yields \[
        f(\vec{x}_{t+1}) - f(\vec{x}^\star) \leq \lft( 1 - \frac{\mu_1}{L} \rgt) (f(\vec{x}_t) - f(\vec{x}^\star)).
    \]
    The statement follows.
\end{proof}

We see that if $\mu_1 = \nicefrac{\mu}{d}$, we do not gain anything. However, this is not the case
in general. If, for the worst case $\vec{x},\vec{y}$, which satisfy $\| \vec{y} - \vec{x} \| =
    \nicefrac{\| \vec{y} - \vec{x} \|_1}{\sqrt{d}}$, strong convexity holds with $\mu' > \mu$, then we
can achieve $\mu_1 = \nicefrac{\mu'}{d} > \nicefrac{\mu}{d}$, resulting in better convergence.

\subsection{Greedy coordinate descent}

\begin{marginfigure}
    \centering
    \begin{tikzpicture}
        \begin{axis}[
                width=\textwidth,
                height=\textwidth,
                domain=-1:1,
                y domain=-1:1,
                zmin=0,
                zmax=4,
                view={0}{90},
                samples=150,
                colormap={black}{gray(0cm)=(0),gray(1cm)=(0)},
            ] \addplot3[contour gnuplot={number=10, label distance=10000pt}]{x^2 + y^2 + abs(x - y)};
        \end{axis}
    \end{tikzpicture}
    \caption{Level set plot of $f(\vec{x}) = \| \vec{x} \|^2 + \abs{x_1 - x_2}$. The global minimum is $[0,0]$, but greedy coordinate descent cannot escape any point $[x, x]$, s.t. $|x| \leq \nicefrac{1}{2}$.}
    \label{fig:greedy-fail}
\end{marginfigure}

\textit{Greedy coordinate descent} is a variant that does not even require $f$ to be differentiable.
In each iteration, we make the step that maximizes the progress in the chosen coordinate. This
requires performing a \textit{line search} by solving a one-dimensional optimization problem,
\begin{align*}
     & \text{choose $i \in [d]$}                                                    \\
     & \vec{x}_{t+1} \in \argmin_{\lambda \in \R} f(\vec{x}_t + \lambda \vec{e}_i).
\end{align*}

However, greedy coordinate descent can get stuck in non-optimal points---see
\Cref{fig:greedy-fail}. Thus, we need some additional conditions to make sure this does not occur.

\begin{theorem} \label{thm:separable}
    Let $f: \R^d \to \R$ be of the form \[
        f(\vec{x}) \doteq g(\vec{x}) + h(\vec{x}), \quad h(\vec{x}) = \sum_{i=1}^{d} h_i(x_i), \quad \vec{x} \in \R^d,
    \]
    with $g$ convex and differentiable, and the $h_i$ convex.

    Let $\vec{x}$ be a point such that greedy coordinate descent cannot make progress in any
    coordinate. Then $\vec{x}$ is a global minimum of $f$.
\end{theorem}

In the context of machine learning, an important class of functions that satisfies the conditions
of \Cref{thm:separable} is the following form, \[
    f(\vec{x}) + \lambda \| \vec{x} \|_1,
\]
where $\lambda \| \vec{x} \|_1$ is a separable $\ell_1$-regularization term used in for example
LASSO \citep{tibshirani1996regression}.
