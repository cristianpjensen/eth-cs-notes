\section{Stereo matching}

In humans and machines, depth perception is very important to interact with the
environment. In humans, this is done by two eyes that are slightly different.
From this difference, depth can be estimated. We can do the same with cameras
with potentially even more views, which could result in better depth
estimation.

\marginnote{If the cameras are not perfectly on the same $x$-axis or do not
have the same rotation, we can solve it by rectifying them using a homography
\citep{loop1999computing}. Geometrically, this means first projecting the two
images onto a common plane that is parallel to the baseline. It can also be
seen as rectifying the epipolar lines.}

The standard setup in stereo geometry is two identical cameras that are only
displaced on the $x$-axis. In other words, they have the same rotation and
height. Then, it is easy to compute the fundamental matrix, \[
  \mat{F} = \begin{bmatrix} 0 & 0 & 0 \\ 0 & 0 & 1 \\ 0 & -1 & 0 \end{bmatrix}
.\]
Then, we have the following epipolar lines,
\begin{align*}
  \vec{\ell}_1 &= \transpose{\vec{x}_2} \mat{F} = \transpose{\begin{bmatrix} 0 & 1 & -y_2 \end{bmatrix}} \\
  \vec{\ell}_2 &= \mat{F} \vec{x}_1 = \transpose{\begin{bmatrix} 0 & 1 & -y_1 \end{bmatrix}}
,\end{align*}
which tells us that all points will have the same $y$-axis, and vary on the
$x$-axis.

If there is a three-dimensional point at $\vec{X} = [X,Y,Z]$, then we have the
following relationships with the two-dimensional points on the image planes,
\begin{align*}
  x_1 &= f \frac{X}{Z} \margintag{Under the assumption that camera 1 is at the origin \wrt camera 2.} \\
  y_1 &= f \frac{Y}{Z} \\
  x_2 &= f \frac{X+b_x}{Z} \margintag{Translating camera to the left is the same as translating the world to the right. $b_x$ is the baseline on the $x$-axis.} \\
  y_2 &= f \frac{Y}{Z} \\
\end{align*}
The \textit{stereo disparity} (which is the difference in pixel locations) $d=
x_1 - x_2$ can then be computed as follows,
\begin{align*}
  d &= x_1 - x_2 \\
  &= f \frac{X}{Z} - f \frac{X-b_x}{Z} \\
  &= f \frac{b_x}{Z} \\
  \intertext{We can then solve for the depth $Z$,}
  Z &= \frac{f b_x}{d} \\
  \intertext{Furthermore, we can compute how much the disparity changes as the
  depth changes,}
  \odv{d}{Z} &= \odv*{\frac{f b_x}{Z}}{Z} \\
  &= \frac{f b_x}{Z^2}
,\end{align*}
which tells us that the uncertainty in the pixel location grows quadratically
with depth, causing large depth error. But, this uncertainty is also inversely
proportional to the baseline, thus we can increase $b_x$ to see depth better at
deep locations. However, this also comes at the cost that the search problem of
matching the pixels between images is much harder, because there are more
occlusions. Furthermore, the disparity becomes smaller as the baseline
increases, which makes it harder to achieve fine-grained resolution, since
pixels are not infinitely small. See \Cref{fig:stereo-baseline} for an
illustration of this.

\begin{figure}[t]
    \centering
    \incfig{stereo-baseline}
    \caption{In stereopsis, a large baseline makes it easier to estimate the
    depth of further points, but it will miss close points.}
    \label{fig:stereo-baseline}
\end{figure}

\begin{marginfigure}
    \centering
    \incfig{matching-algorithm}
    \caption{Matching algorithm, where the points on the corresponding lines are
      being matched to each other. A line to the right means that there is an
      obstruction.}
    \label{fig:matching-algorithm}
\end{marginfigure}

Now, we need to figure out how to match the pixels in one image to another. We
know that they are on the same lines. Note that the search is limited to the
epipolar line, so we only need to match along the common epipolar line of the
two image planes. We also need to points to enforce spatial consistency, since
close points in the environment are also between the image planes. Also, we can
never turn back, \ie, if $x_1$ is matched to $x_2$, then $x_1' > x_1$ cannot
match to a point $x_2' < x_1$. We then associate with each matching of pixels a
cost that is computed from their intensities. We can then use a matching
algorithm to compute the optimal matching \citep{baker1981depth}.

\subsection{Multi-view stereo}

We can also add more cameras to the setup. An obvious advantage is that we can
set the cameras up such that we have multiple baselines
\citep{okutomi1993multiple}. This has the result that we no longer have the
trade-off between large and small baseline, because we can use both. This makes
it possible to compute the depth of far points accurately and deal with the
visibility problem of large baselines. An example of a multi-view setup being
more robust to occlusions is \Cref{fig:visibility-problem}. The left two
cameras would not be able to find the depth of the point on the cube, but the
introduction of the right camera made it possible.

\begin{figure}
    \centering
    \incfig{visibility-problem}
    \caption{Visibility problem that is resolved by adding a third camera.}
    \label{fig:visibility-problem}
\end{figure}

\subsection{Volumetric stereo}

In volumetric stereo, we discretize the three-dimensional space into
voxels.\sidenote{A voxel is a ``three-dimensional pixel``.} We then ``carve``
the geometry out of these voxels. This gives us a reconstruction of the scene,
from which we can create a depth map by taking the first occupied voxel for
each pixel. We assign a photoconsistency cost to each voxel, and then we make
the minimum cost cut in this graph.

\begin{marginfigure}
    \centering
    \incfig{voxel}
    \caption{Above view of voxel carving.}
    \label{fig:voxel}
\end{marginfigure}

We can also use \textit{visibility-first} methods. In these methods, we first
handle the visibility problem, and then we carve out the voxels. We do this by
first segmenting the foreground from the background for every camera. We then
backproject the foreground silhouette to the image plane. All these pixels are
visible from this camera. The visual hull is then the intersection of these
silhouettes. We only keep the voxels that are part of the visual hull, which
gives us a tight bound on the object.

\begin{marginfigure}[0.5cm]
    \centering
    \incfig{visual-hull}
    \caption{The visual hull of the object with three cameras. As we increase
      the amount of cameras, the intersection becomes smaller and thus the bound
      on the object tighter.}
    \label{fig:visual-hull}
\end{marginfigure}
