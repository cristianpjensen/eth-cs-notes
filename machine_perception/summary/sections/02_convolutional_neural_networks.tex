\section{Convolutional neural networks}

When dealing with high-dimensional data, such as images, it is not practical to work with MLPs,
because the amount of parameters would explode in size.\sidenote{Mapping a $256 \times 256 \times
        3$ input image to a $1$-dimensional output would already require nearly 2 million parameters.} By
making use of the locality of images, we can drastically decrease the number of parameters.

\subsection{Convolution}

\begin{marginfigure}
    \centering
    \incfig{correlation}
    \caption{Illustration of applying a correlation to a pixel.}
    \label{fig:correlation}
\end{marginfigure}

The \textit{correlation} operator takes a filter $\tens{K}$, moves it along the entire image, and
outputs the patch-wise multiplication, for each patch of the same size as the filter. It is defined
as follows, \[
    (\tens{K} \star \tens{I})[i,j] = \sum_{m=-k}^{k} \sum_{n=-k}^{k} \tens{K}[m, n] \tens{I}[i+m, j+n].
\]
The \textit{convolution} operator is very similar. The only difference is that the kernel is
mirrored in a convolution, \[
    (\tens{K} * \tens{I})[i,j] = \sum_{m=-k}^k \sum_{n=-k}^{k} \tens{K}[m,n] \tens{I}[i-m, j-n].
\]

Theoretically, the convolution operator is more useful, because it is
commutative.\sidenote{$\tens{I} * \tens{K} = \tens{K} * \tens{I}$, but $\tens{I} \star \tens{K}
        \neq \tens{K} \star \tens{I}$.} In practice with neural networks, it does not matter, since the
weights will just be updated to be the same, except that they are mirrored. Thus, we will only be
referring to the convolution from now on.

A convolution operator $C$ is a linear, shift-equivariant transformation, \ie,
\begin{align*}
    C(\alpha \vec{x} + \beta) & = \alpha C(\vec{x}) + \beta \margintag{Linearity.}              \\
    T_{\vec{t}}(C(\vec{x}))   & = C(T_{\vec{t}}(\vec{x})). \margintag{Translation equivariant.}
\end{align*}
Since convolutions are linear, discrete convolutions can be implemented using matrix multiplication, \[
    \tens{K} * \tens{I} = \begin{bmatrix}
        k_1    & 0      & 0      & \cdots & 0      \\
        k_2    & k_1    & 0      & \cdots & 0      \\
        k_3    & k_2    & k_1    & \cdots & 0      \\
        \vdots & \vdots & \vdots & \ddots & \vdots \\
        0      & 0      & 0      & \cdots & k_m
    \end{bmatrix}
    \begin{bmatrix}
        I_1 \\ I_2 \\ I_3 \\ \vdots \\ I_n
    \end{bmatrix}
    .
\]

\subsection{Convolutional neural network}

\begin{marginfigure}
    \centering
    \incfig{cnn}
    \caption{Example schematic of a CNN architecture.}
    \label{fig:cnn}
\end{marginfigure}

A \textit{convolutional neural network} (CNN) \citep{krizhevsky2012imagenet} is composed of a
sequence of \textit{convolutional layers} and \textit{pooling layers}, followed by a final
\textit{dense layer} (MLP).

This architecture is loosely inspired by the brain, which, at a high level, first extracts
high-level features and then more and more specific features. Furthermore, the HMAX model of the
brain distinguishes between simple and complex cells, which correspond to the linear layer and
max-pool layer, respectively.

\paragraph{Convolutional layer.}

\begin{marginfigure}
    \centering
    \incfig{convolutional-layer}
    \caption{Schematic of a convolutional layer. Each input-output channel pair has its own kernel,
        so $\vec{\theta}$ has $K\times K \times C_{\mathrm{in}} \times C_{\mathrm{out}}$ parameters.}
    \label{fig:convolutional-layer}
\end{marginfigure}

A convolutional layer applies many filters $\tens{W} \in \R^{\mathrm{C_{in}} \times
        C_{\mathrm{out}} \times k \times k}$ to an input image $\tens{Z} \in \R^{C_{\mathrm{in}} \times H
        \times W}$ by convolution, \[
    \mat{Z}^{(\ell)}_j = \sum_{k=1}^{C_{\mathrm{in}}} \mat{W}^{(\ell)}_{kj} * \mat{Z}^{(\ell - 1)}_k + b_j, \quad j \in [C_{\mathrm{out}}].
\]

This filter is learned. Hence, we need to derive its derivative. We will focus on the single filter
case ($C_{\mathrm{in}} = C_{\mathrm{out}} = 1$) for simplicity, whose forward pass is computed by \[
    z^{(\ell)}[i,j] = \sum_{m=-k}^{k} \sum_{n=-k}^{k} w^{(\ell)}[m,n] z^{(\ell-1)}[i-m,j-n] + b.
\]

We can express the derivative of the cost function $\mathcal{L}$ \wrt the output of the
$(\ell-1)$-th layer as the following,
\begin{align*}
    \delta^{(\ell-1)}[i,j] & = \pdv{\mathcal{L}}{z^{(\ell-1)}[i{,}j]}                                                                                                   \\
                           & = \sum_{i'} \sum_{j'} \pdv{\mathcal{L}}{z^{(\ell)}[i'{,}j']} \pdv{z^{(\ell)}[i'{,}j']}{z^{(\ell-1)}[i{,}j]}                                \\
                           & = \sum_{i'} \sum_{j'} \delta^{(\ell)}[i',j'] \pdv*{\sum_{m} \sum_{n} w^{(\ell)}[m{,}n] z^{(\ell-1)}[i'-m{,}j'-n] + b}{z^{(\ell-1)}[i{,}j]} \\
                           & = \sum_{i'} \sum_{j'} \delta^{(\ell)}[i',j'] w^{(\ell)}[i'-i,j'-j].
\end{align*}
From this, we can see that we can compute all values of $\delta^{(\ell-1)}$ by a single convolution, \[
    \delta^{(\ell-1)} = \delta^{(\ell)} * \mathrm{Flip}\lft( \mat{W}^{(\ell)} \rgt) = \delta^{(\ell)} \star \mat{W}^{(\ell)}.
\]

Using this value, we can compute the derivative \wrt the weights, which we need for the parameter
update in algorithms such as gradient descent,
\begin{align*}
    \pdv{\mathcal{L}}{w^{(\ell)}[m{,}n]} & = \sum_{i} \sum_{j} \pdv{\mathcal{L}}{z^{(\ell)}[i{,}j]} \pdv{z^{(\ell)}[i{,}j]}{w^{\ell}[m{,}n]}                                        \\
                                         & = \sum_{i} \sum_{j} \delta^{(\ell)}[i,j] \pdv*{\sum_{m'} \sum_{n'} w^{(\ell)}[m'{,}n'] z^{(\ell-1)}[i-m'{,}j-n'] + b}{w^{(\ell)}[m{,}n]} \\
                                         & = \sum_{i} \sum_{j} \delta^{(\ell)}[i,j] z^{(\ell-1)}[i-m,j-n].
\end{align*}
Again, this has the form of a convolution, thus we can compute all derivatives of $\mat{W}^{(\ell)}$ by convolution, \[
    \pdv{\mathcal{L}}{\mat{W}^{(\ell)}} = \delta^{(\ell)} * \mat{Z}^{(\ell-1)}.
\]

As shown, we can use convolutions for both computing the forward pass, as well as the backward
pass. Thus, we first do the forward pass, then compute all $\delta^{(\ell)}$ for all layers $\ell$
by convolution, and finally we can compute the derivative by convolution as well.

\paragraph{Pooling layers.}

Pooling layers makes the data more manageable. The most common pooling layer is \textit{max
    pooling}, which outputs the maximum value for each patch. It can be seen as a non-linear
convolutional filter, where it simply outputs the maximum value. Usually, pooling is done with a
stride, such that the output becomes exponentially smaller; see \Cref{fig:max-pooling}. Because of
this, the \textit{receptive field} becomes exponentially larger.

\begin{marginfigure}
    \centering
    \incfig{max-pooling}
    \caption{Toy example of max pooling.}
    \label{fig:max-pooling}
\end{marginfigure}

The forward pass is computed as follows,
\[
    z^{(\ell)}[i,j] = \max \lft\{ z^{(\ell-1)}[i',j'] \;\middle|\; i' \in [si:si+k], j' \in [sj:sj+k] \rgt\},
\]
where $s$ is the stride and $k$ is the kernel size. Let $[i^\star,j^\star]$ be the indices which
corresponded to the maximum value in the forward pass, then we can compute the error propagation in
the backward pass by \[
    \pdv{z^{(\ell)}[i{,}j]}{z^{(\ell-1)}[i'{,}j']} = \mathbb{1}\{ [i',j'] = [i^\star,j^\star] \}.
\]

Note that the max pooling layer has no learnable parameters. Hence, the backward pass is only a
propagation of the error, and not used for a weight update.

\paragraph{Dense layer.}

The dense layer is simply a linear layer that maps the final convolutional layer to the network's
output. All previous convolutional and pooling layers can be seen as extracting features from the
image, while the final dense layer makes the actual prediction.
