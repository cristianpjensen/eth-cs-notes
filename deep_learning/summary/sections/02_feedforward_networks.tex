\section{Feedforward networks}

\subsection{Regression models}

In least squares, we attempt to fit a linear model, \[
    f[\vec{w}](\vec{x}) = \transpose{\vec{w}}\vec{x},
\]
to data points with a mean squared error (MSE) loss, \[
    \ell[\vec{w}](\mathcal{S}) = \frac{1}{2} \sum_{i=1}^{s} \lft( \transpose{\vec{w}} \vec{x} - y_i \rgt)^2.
\]
Summarizing the patterns into a design matrix $\mat{X} \in \R^{d \times s}$ and output vector
$\vec{y} \in \R^s$, we get the following loss, \[
    \ell[\vec{w}](\mathcal{S}) = \frac{1}{2} \| \transpose{\mat{X}} \vec{w} - \vec{y} \|^2.
\]
This loss function is convex, so we can find the minimizer by setting the gradient to zero, \[
    \grad{\ell[\vec{w}](\mathcal{S})}{\vec{w}} = \transpose{\mat{X}} \mat{X} \vec{w} - \transpose{\mat{X}} \vec{y} \condeq \vec{0}.
\]
This gives the ordinary least squares estimator (OLSE), \[
    \vec{w}^\star = \inv{(\transpose{\mat{X}} \mat{X})} \transpose{\mat{X}} \vec{y}.
\]

In logistic regression, the outputs are binary. Hence, we make use of the sigmoid function $\sigma:
    \R \to (0,1)$, \[
    \sigma(z) \doteq \frac{1}{1 + \exp(-z)}.
\]
Hence, the model has the following form, \[
    f[\vec{w}](\vec{x}) = \sigma \lft( \transpose{\vec{w}}\vec{x} \rgt).
\]
This outputs the probability of the label of $\vec{x}$ being $1$. We train this model to optimize
the cross-entropy loss, \[
    \ell[\vec{w}](\mathcal{S}) = \frac{1}{s} \sum_{i=1}^{s} -\log \sigma \lft( (2y_i - 1) \transpose{\vec{w}} \vec{x}_i \rgt).
\]
This problem does not have a closed-form solution, but we can optimize the weights by stochastic
gradient descent (SGD) with the following gradient, \[
    \grad{\ell[\vec{w}](\langle \vec{x}_i, y_i \rangle)}{\vec{w}} = \lft( \sigma \lft( \transpose{\vec{w}} \vec{x}_i \rgt) - y_i \rgt) \vec{x}_i.
\]

\subsection{Layers and units}

A mapping is a function with vectors as input and output. The following function is an example of a
mapping, \[
    f[\mat{W}, \vec{b}](\vec{x}) = \phi(\mat{W} \vec{x} + \vec{b}), \quad \mat{W} \in \R^{m \times n}, \vec{b} \in \R^n,
\]
where $\phi$ is a pointwise activation function and $m$ is the width of the layer.

Deep neural networks compose maps in sequence, \[
    G = F^L\lft[ \vec{\theta}^L \rgt] \circ \cdots \circ F^1 \lft[ \vec{\theta}^1 \rgt],
\]
where $\vec{\theta}^{\ell}$ are the (adjustable) weights of layer $\ell$. Intuitively, models with
higher depth are able to extract features with increasing complexity. Such networks induce
intermediate results (or layer activations), \[
    \vec{x}^{\ell} \doteq \lft( F^{\ell} \circ \cdots \circ F^{1} \rgt)(\vec{x}) = F^{\ell}\lft( \vec{x}^{\ell-1} \rgt).
\]
The intermediate layers are permutation symmetric, meaning that the units within a hidden layer are
interchangeable if we change the order of the weights accordingly, \[
    F[\mat{W}, \vec{b}](\vec{x}) = \inv{\mat{P}} \phi(\mat{P}\mat{W}\vec{x} + \mat{P}\vec{b}) = \inv{\mat{P}} F[\mat{P}\mat{W},\mat{P}\vec{b}](\vec{x}),
\]
where $\mat{P}$ is a permutation matrix.\sidenote{A permutation matrix $\mat{P} \in \R^{n \times
            n}$ satisfies the following condition, \[
        \sum_{i=1}^{n} p_{ij} = \sum_{j=1}^{n} p_{ij} = 1, \quad \forall i, j \in [n].
    \]} Hence, the parameters are not unique in feedforward networks.

The layers---as presented---differ only in their choice of activation function,
\begin{itemize}
    \item Linear activation, \[
              \phi = \mathrm{Id};
          \]
    \item Sigmoid activation, \[
              \phi = \sigma;
          \]
    \item ReLU activation, \[
              \phi = (z)_+ = \max \{ 0, z \}.
          \]
\end{itemize}

An essential part of training neural networks is constructing the loss function. For a regression
problem, a simple---and popular---choice is the squared error loss, \[
    \ell[\vec{\theta}](\vec{x}, \vec{y}) = \frac{1}{2} \| \vec{y} - f[\vec{\theta}](\vec{x}) \|^2.
\]
For a multi-class classification problem, the final layer must be the softmax, which outputs a
categorical probability distribution over classes, \[
    \mathrm{softmax}_i(\vec{z}) = \frac{\exp(z_i)}{\sum_{j=1}^{n} \exp(z_j)}.
\]
Usually, this type of model optimizes the cross-entropy loss.

In a perfect world, we would want to minimize the expected risk, \[
    \E[\ell(y, f[\vec{\theta}](\vec{x}))].
\]
However, since we do not have access to the underlying probability distribution of the data, this
is intractable. Hence, we minimize the empirical risk, \[
    \frac{1}{s} \sum_{i=1}^{s} \ell(y_i, f[\vec{\theta}](\vec{x}_i)).
\]
In practice, we partition the dataset into training and validation sets. Then, we directly minimize
the empirical risk of the training set, and approximate the expected risk with the validation set.

\subsection{Linear and residual networks}

Linear layers are closed under composition, meaning that we do not gain any representational power
by increasing the depth. However, linear analysis are nice to work with for theoretical analysis.

Residual layers are formalized as follows, \[
    F[\mat{W}, \vec{b}](\vec{x}) = \vec{x} + (\phi(\mat{W} \vec{x} + \vec{b}) - \phi(\vec{0})).
\]
They have the following property, \[
    F[\mat{0}, \vec{0}] = \mathrm{Id}.
\]
In most architectures, learning the identity map is non-trivial. However, it is desirable to
incrementally learn a better representation, rather than having to learn it at every layer.
Intuitively, the residual layer learns how to change its input representation.

A problem with the above formalization is that the input and output must have the same
dimensionality. This is solved by a projection, \[
    F[\mat{V}, \mat{W}, \vec{b}](\vec{x}) = \mat{V} \vec{x} + (\phi(\mat{W} \vec{x} + \vec{b}) - \phi(\vec{0})), \quad \mat{V}, \mat{W} \in \R^{m\times n}.
\]

\citet{he2016deep} showed that increasing model depth with residual layers leads to better
performance than when using normal layers. This small change allows model depths of up to 100–-200
layers. DenseNet \citet{zhu2017densenet} makes use of a similar idea of shortcutting information by
feeding the output of all upstream layer activations to every layer, \[
    \vec{x}^{\ell+1} = F^{\ell+1}\lft( \vec{x}^\ell, \ldots, \vec{x}^1, \vec{x} \rgt).
\]

\subsection{Sigmoid networks}

\marginnote{The sigmoid function and hyperbolic tangent, \[
        \sigma(z) \doteq \frac{1}{1 + \exp(-z)}, \quad \tanh(z) \doteq \frac{\exp(z) - \exp(-z)}{\exp(z) + \exp(-z)},
    \]
    are representationally equivalent, because you can always obtain the one from the other by the
    following identity, \[
        \tanh(z) = 2 \sigma(2z) - 1.
    \]}

We will now look at which functions an MLP with sigmoid activation function, \[
    g[\vec{v},\mat{W},\vec{b}](\vec{x}) \doteq \transpose{\vec{v}} \sigma(\mat{W}\vec{x} + \vec{b}), \quad \vec{v}, \vec{b} \in \R^m, \mat{W} \in \R^{m\times n},
\]
are able to express. The function class of MLPs is formalized by
\begin{align*}
    \mathcal{G}_n     & \doteq \bigcup_{m=1}^\infty \mathcal{G}_{n,m}                                                                                                                 \\
    \mathcal{G}_{n,m} & \doteq \lft\{ g \;\middle|\; g(\vec{x}) = \transpose{\vec{v}} \sigma(\mat{W}\vec{x} + \vec{b}), \vec{v}, \vec{b} \in \R^m, \mat{W} \in \R^{m\times n} \rgt\}.
\end{align*}
An alternative way of expressing this is as a linear span of units, \[
    \mathcal{G}_n = \mathrm{span} \lft\{ \sigma \lft( \transpose{\vec{w}}\vec{x} + b \rgt) \;\middle|\; \vec{w} \in \R^n, b \in \R \rgt\}.
\]

\begin{definition}[Function distance metric]
    $d_{\mathcal{K}}$ is a distance metric over a compact set $\mathcal{K}$ induced by the uniform norm, \[
        d_{\mathcal{K}}(f,g) \doteq \| f - g \|_{\infty,K}, \quad \| f \|_{\infty,\mathcal{K}} \doteq \sup_{\vec{x} \in \mathcal{K}} |f(\vec{x})|.
    \]
\end{definition}

\begin{definition}[Function class distance metric]
    Let $f$ be a function and $\mathcal{G}$ a function class, then their distance is computed as \[
        d_{\mathcal{K}}(f,\mathcal{G}) \doteq \inf_{g \in \mathcal{G}} d_{\mathcal{K}}(f,g).
    \]
\end{definition}

\begin{definition}[Universal function approximator]
    A function class $\mathcal{F}$ is approximated by function class $\mathcal{G}$ on $\mathcal{K}$ if, and only if, \[
        d_{\mathcal{K}}(f, \mathcal{G}) = 0, \quad \forall f \in \mathcal{F}.
    \]
    If this holds for all compact sets $\mathcal{K}$, then $\mathcal{G}$ is a universal approximator of
    $\mathcal{F}$.
\end{definition}

\begin{theorem}[Weierstrass theorem]
    Polynomials are universal approximators of $\mathcal{C}(\R)$, where $\mathcal{C}(\R)$ is the set
    of all continuous functions over $\R$.
\end{theorem}

\begin{theorem}[\citep{leshno1993multilayer}]
    Let $\phi \in \mathcal{C}^{\infty}(\R)$, but not a polynomial, then \[
        \mathrm{span}(\{ \phi(ax + b) \mid a,b \in \R \})
    \]
    universally approximates $\mathcal{C}(\R)$.
\end{theorem}

Hence, an MLP with $1$-dimensional input and output is a universal function approximator, if the
activation function is not a polynomial.

\begin{lemma}[Lifting lemma \citep{pinkus1999approximation}]
    Let $\phi$ be such that \[
        \mathrm{span}(\{ \phi(ax + b) \mid a,b \in \R \})
    \]
    universally approximates $\mathcal{C}(\R)$, then \[
        \mathrm{span}\lft( \lft\{ \phi \lft( \transpose{\vec{w}}\vec{x} + b \rgt) \mid \vec{w} \in \R^n, b \in \R \rgt\} \rgt)
    \]
    universally approximates $\mathcal{C}(\R^n)$.
\end{lemma}

Thus, we can lift the previous result into $n$ dimensions, making MLPs universal approximators of
continuous functions of any dimensionality. Moreover, this does not only hold for the sigmoid
function, but for any smooth activation function that is not a polynomial.

However, this does not give us any insights into how depth affects performance, because the theorem
assumes a single hidden layer of arbitrary width. Also, it does not provide a bound on the width of
the hidden layer in order to achieve some desired error.

\begin{theorem}[\citep{barron1993universal}]
    For every $f: \R^n \to \R$ with finite $\mathcal{C}_f$ and any $r > 0$, there is a sequence of one hidden layer MLPs $(g_m)_{m \in \mathbb{N}}$ such that \[
        \int_{r \mathcal{B}} (f(\vec{x}) - g_m(\vec{x}))^2 \mu(d \vec{x}) \leq \bigo{\frac{1}{m}},
    \]
    where $r \mathbb{B} \doteq \{ \vec{x} \in \R^n \mid \| \vec{x} \| \leq r \}$ and $\mu$ is any
    probability measure on $r \mathbb{B}$.
\end{theorem}

Hence, if we relax the notion of approximation to squared error over a ball with radius $r$, we
gain a decay of $\nicefrac{1}{m}$ for the approximation error. Further, the approximation error
bound does not depend on the input dimensionality $n$.

\subsection{ReLU networks}

The ReLU activation function is defined as \[
    (z)_+ \doteq \max \{ 0,z \}.
\]
Consider a layer of $m$ ReLU units on a fixed input $\vec{x}$. In this situation, each unit is
either active or inactive, where active means that its input is positive, \[
    \mathbb{1}\{ \mat{W} \vec{x} + \vec{b} > 0 \} \in \{ 0,1 \}^m.
\]
In this way, we can partition the input space into cells that have the same activation pattern, \[
    \mathcal{X}_{\vec{\kappa}} \doteq \{ \vec{x} \mid \mathbb{1}\{ \mat{W} \vec{x} + \vec{b} > 0 \} = \vec{\kappa} \}.
\]
We can measure the complexity of a network as the amount of these cells it has. Firstly, we have
the trivial upper bound $|\{ \mathbb{1}\{ \mat{W} \vec{x} + \vec{b} > 0 \} \mid \vec{x} \in \R^n
    \}| \leq 2^m$. However, we would like to obtain a stricter bound. We can represent each hidden unit
as a hyperplane $\transpose{\vec{w}_i}\vec{x} + b_i$. On one side the unit would be active and
inactive on the other. Geometrically, we can thus think of it as a space, where each hidden unit
represents a hyperplane. The connected regions of these hyperplanes are the activation patterns.

\begin{marginfigure}
    \centering
    \incfig{connected-regions}
    \caption{Connected regions, partitioned according to activation pattern. Each hyperplane represents a hidden unit. This shows an MLP with 2-dimensional input and 3-dimensional hidden layer.}
    \label{fig:connected-regions}
\end{marginfigure}

\begin{theorem}[\citep{zaslavsky1975facing}]
    Let $\mathcal{H}$ be a set of $m$ hyperplanes in $\R^n$. Denote by $R(\mathcal{H})$ the number of connected regions of $\R^n \backslash \mathcal{H}$, then \[
        R(\mathcal{H}) \leq \sum_{i=0}^{\min \{ n,m \}} \begin{pmatrix} m \\ i \end{pmatrix} \doteq R(m).
    \]
    This upper bound is attained by hyperplanes in general position.
\end{theorem}

This gives us a tighter bound on the number of activation patterns.

\begin{theorem}[\citep{montufar2014number}]
    Consider a ReLU network with $L$ layers of width $m > n$. The number of linear regions is lower bounded by \[
        R(m, L) \geq R(m) \lft\lfloor\frac{m}{n}\rgt\rfloor^{n(L-1)}.
    \]
\end{theorem}

Finally we have a result that relates model complexity to layer depth. By letting the amount of
possible activation patterns represent complexity, this is a good argument for why deep networks
tend to perform well.

\begin{theorem}[\citep{shekhtman1982piecewise}]
    Piecewise linear functions are dense in $\mathcal{C}([0,1])$.
\end{theorem}

\begin{theorem}[Lebesgue]
    A piecewise linear function with $m$ pieces can be written as \[
        g(x) = ax + b + \sum_{i=1}^{m-1} c_i(x - x_i)_+.
    \]
\end{theorem}

\begin{marginfigure}
    \centering
    \incfig{piecewise-linear-approx}
    \caption{Piecewise linear approximation of a continuous function.}
    \label{fig:piecewise-linear-approx}
\end{marginfigure}

Hence, we can rewrite any piecewise linear function with $m$ pieces as a sum of $m-1$ ReLUs. In 1
dimension, we can approximate any function by uniformly spacing out points on the function and
connecting them as a piecewise linear function---see \Cref{fig:piecewise-linear-approx}. We can
lower approximation error by increasing the number of units, approaching $0$ as $m \to \infty$.
Using the lifting lemma, we get the following result.

\begin{theorem}[ReLU universality]
    Networks with one hidden layer of ReLU units are universal function approximators.
\end{theorem}

% \begin{theorem}
%     Every continuous piecewise linear function $g: \R^n \to \R$ can be written as a signed sum of
%     $k$-Hinges with $k \leq n+1$, where a $k$-Hinge is \[
%         g(\vec{x}) = \max_{j=1}^k \transpose{\vec{w}_j}\vec{x} + b_j.
%     \]
% \end{theorem}
%
% \begin{theorem}
%     Every continuous piecewise linear function $f$ can be written as a difference of two polyhedral functions.
% \end{theorem}
