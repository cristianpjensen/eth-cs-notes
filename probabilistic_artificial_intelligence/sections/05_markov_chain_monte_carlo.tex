\section{Markov chain Monte Carlo} \label{sec:mcmc}

\textit{Markov chain Monte Carlo} (MCMC) methods seek to approximate the
intractable distribution $p$ by drawing $m$ approximate samples from $p$,
\begin{align*}
    p(y^\star\mid\bm{x}^\star, \vec{y}) & = \int p(y^\star\mid\bm{x}^\star,\theta) p(\theta\mid \vec{y}) d\theta                                                 \\
                                        & = \E_{\theta\sim p(\cdot\mid \vec{y})} [p(y^\star\mid\bm{x}^\star,\theta)]                                             \\
                                        & \approx \frac{1}{m} \sum_{i=1}^m p \lft( y^\star\mid\vec{x}^\star,\theta^{(i)} \rgt). \margintag{Law of large numbers}
\end{align*}

Using \textit{Hoeffding's inequality}, we can compute a bound on the error, \[
    p\lft( \lft|\E_{\theta\sim p(\cdot\mid\vec{y})}[p(y^\star\mid\bm{x}^\star,\theta)] - \frac{1}{m}\sum_{i=1}^m p \lft( y^\star\mid\bm{x}^\star,\theta^{(i)} \rgt)\rgt| > \epsilon \rgt) \leq 2\exp \lft( -2m\epsilon^2 \rgt).
\]
Thus, the probability of error decreases exponentially in $m$. To get a
probability $\leq \delta$ of error $> \epsilon$, we need
\begin{align*}
    2\exp(-2m\epsilon^2) & \leq \delta                                   \\
    -2m\epsilon^2        & \leq \log \frac{\delta}{2}                    \\
    m                    & \geq \frac{\log 2 - \log \delta}{2\epsilon^2}
\end{align*}
samples. Intuitively, if we want a lower error or lower probability, we need
more samples.

However, we cannot sample directly from the posterior $p(\theta\mid\vec{y})$,
because it is intractable. The key idea of MCMC is to construct a Markov
chain with stationary distribution $p(\theta\mid\vec{y})$ and sample from
that.

\subsection{Markov chains}

\begin{definition}[Markov chain]
    A Markov chain is a sequence of random variables $(X_t)_{t\in\N_0}$ with
    prior $P(X_1)$ and transition probabilities $P(X_{t+1}\mid X_t)$ independent
    of $t$. The Markov assumption is thus the following, \[
        X_{t+1} \bot X_{1:t-1} \mid X_t.
    \]
    Intuitively, this states that future behavior is independent of past states
    given the present state. In other words, all past information is encapsulated
    by the current state.
\end{definition}

\begin{definition}[Stationary distribution]
    A distribution $\pi$ is stationary with respect to the transition function
    $P$ iff $P(X_n) = P(X_{n+1})$. In other words, the probability distribution
    over states remains the same between timesteps. The following must hold for
    all $x$, \[
        \pi(x) = \sum_{x'} P(x\mid x')\pi(x').
    \]
\end{definition}

\begin{definition}[Ergodicity]
    A Markov chain is ergodic iff there exists a $t\in\N_0$ such that for any
    $x,x'$, the following holds, \[
        P^{(t)} (x'\mid x) > 0,
    \]
    where $P^{(t)}(x'\mid x)$ is the probability to reach $x'$ from $x$ in
    \textit{exactly} $t$ steps. Intuitively, this means that any state is
    reachable from another within the same amount of steps.
\end{definition}

\begin{remark}
    An easy way of ensuring that a Markov chain is ergodic is to add
    ``self-loops`` to every vertex.
\end{remark}

\begin{theorem}[Fundamental theorem of ergodic Markov chains]
    An ergodic Markov chain has a unique and positive stationary distribution
    $\pi(X)>0$ such that for all $x$, the following holds, \[
        \lim_{n\to\infty} P(X_n = x) = \pi(x),
    \]
    independent of the initial distribution $P(X_1)$. \Ie, ergodic Markov chains
    always converge to a stationary distribution.
\end{theorem}

Thus, making use of the fundamental theorem of ergodic Markov chains, we can
construct an ergodic Markov chain such that its stationary distribution
coincides with the posterior distribution, $\pi(x) = p(\theta\mid\vec{y})$. If
we then sample ``sufficiently long`` from this Markov chain, $X_t$ is drawn
from a distribution that is ``very close`` to the stationary distribution
$\pi$, which is equal to the posterior distribution.

\begin{definition}[Detailed balance equation]
    A Markov chain satisfies the detailed balance equation for an unnormalized
    distribution $q$ iff the following holds for any $x,x'$, \[
        q(x)P(x'\mid x) = q(x')P(x\mid x').
    \]
\end{definition}

\begin{theorem}
    If a finite Markov chain satisfies the detailed balance equation with
    respect to $q$, then $\frac{1}{Z}q$ is a stationary distribution.
\end{theorem}

\begin{proof}
    Let $p_t = q$. Then for any $x$, the following holds,
    \begin{align*}
        p_{t+1}(x) & = \sum_{x'} P(x\mid x')p_t(x') \margintag{Markov assumption}         \\
                   & = \sum_{x'} P(x\mid x') q(x')                                        \\
                   & = \sum_{x'} P(x'\mid x) q(x)   \margintag{Detailed balance equation} \\
                   & = q(x) \sum_{x'} P(x' \mid x)                                        \\
                   & = q(x).
    \end{align*}
    Thus, $q$ is the stationary distribution.
\end{proof}

\subsection{Sampling}

If we can show that the detailed balance equation holds for the
\textit{unnormalized} posterior distribution, then we know that the posterior
distribution is the stationary distribution of the Markov chain.\sidenote{The
    only problem is that we do not know the rate of convergence to the stationary
    distribution.} Thus, we do not need to know the true posterior. It suffices
to know its unnormalized version.\sidenote{Recall that the normalizer was the
    intractable part of $p(\theta\mid\vec{y})$.}

The \textit{Metropolis-Hastings algorithm} constructs a Markov chain with the
posterior as stationary distribution. It uses an arbitrary proposal transition
distribution $R(x'\mid x)$ which, given we are in state $x$, proposes a new
state $x'$.\sidenote{The rate of convergence of this algorithm strongly depends
    on the choice of $R$.} Following the proposal with probability \[
    \alpha(x'\mid x) \doteq \min \lft\{  1, \frac{q(x')R(x\mid x')}{q(x)R(x'\mid x)} \rgt\},
\]
yields a Markov chain with the desired stationary distribution
$\frac{1}{Z}q(x)$, because it makes it satisfy the detailed balance equation.

\begin{algorithm}
    \caption{The Metropolis-Hastings algorithm. Each iteration, with a random
        probability, follow the proposal distribution.}

    \begin{algorithmic}[1]
        \Function {MetropolisHastings}{R}
        \State {initialize $x$}
        \For {$t=1,\ldots,T$}
        \State {$x' \sim R(x' \mid x)$}
        \State {$u \sim \mathrm{Unif}([0,1])$}
        \If {$u \leq \alpha(x' \mid x)$}
        \State {$x \gets x'$}
        \EndIf
        \EndFor
        \EndFunction
    \end{algorithmic}
\end{algorithm}

\subsection{Proposal distributions}

We want to converge to the stationary distribution as fast as possible. The
proposal distribution has a big influence on this.

\paragraph{Gibbs sampling.}

A popular example algorithm for specifying a proposal distribution $R$ is
Gibbs sampling. Gibbs sampling works by iteratively improving the variables.
It starts with an initial assignment $x$ to all variables, fixing the
observed variables to their observed value. Then, iteratively uniformly pick
a variable $X_i$ to update given the rest of the set values by sampling
$p(X_i \mid x_{1:i-1}, x_{i+1:n})$.\sidenote{Sampling from this distribution
    is typically efficient.}

\begin{algorithm}
    \caption{Gibbs sampling.}

    \begin{algorithmic}[1]
        \Function{GibbsSampling}{}
        \State {initialize $\vec{x} = [x_1,\ldots,x_n]\in\R^n$}
        \For {$t=1,\ldots,T$}
        \State {uniformly sample $i$ from $\{ 1,\ldots,n \}$}
        \State {$\vec{x}_{-i} \gets [x_1,\ldots,x_{i-1},x_{i+1},\ldots,x_n]$}
        \State {update $x_i$ by sampling from $p(x_i \mid \vec{x}_{-i})$}
        \EndFor
        \EndFunction
    \end{algorithmic}
\end{algorithm}

Then, Gibbs sampling is a Metropolis-Hastings algorithm with the following
proposal distribution, \[
    R(x'\mid x) \doteq
    \begin{cases}
        p(x'_i \mid x'_{1:i-1}, x'_{i+1:n}) & \text{\footnotesize $x'$ differs from $x$} \\
        0                                   & \text{\footnotesize otherwise}
    \end{cases}
\]
and acceptance distribution $\alpha(x'\mid x) = 1$ for all $x,x'$.

\paragraph{Gaussian.}

Generally, we focus on positive distributions written as the following, \[
    p(x) = \frac{1}{Z} \exp (-f(x)),
\]
where $f$ is called an energy function (high energy $\equiv$ low probability,
low energy $\equiv$ high probability). Then, the acceptance distribution
becomes the following, \[
    \alpha(x'\mid x) = \min \lft\{ 1, \frac{R(x\mid x')}{R(x'\mid x)}\exp(f(x) - f(x')) \rgt\}.
\]

One option for the proposal distribution is \[
    R(x'\mid x) = \mathcal{N}(x'; x,\tau\mat{I}).
\]
Since this $R$ is symmetric, \[
    \frac{R(x\mid x')}{R(x'\mid x)} = 1.
\]
Thus, if $R$ proposes to move to a region with lower energy, the acceptance
probability will always be 1. If $R$ proposes to move to a region with higher
energy, the probability moves toward 0 dependent on how much higher the
region is.

However, we want to move as quickly as possible through the function, going
through all high-density areas. But, this $R$ is ``uninformed`` and thus
proposes to go in any direction. We would like to propose areas with lower
energy, which are high-density areas for $p(x)$. Then, we will have less
iterations where the proposal simply gets rejected.

\paragraph{Metropolis adjusted Langevin algorithm.}

An improvement to the Gaussian proposal distribution is the Metropolis
adjusted Langevin algorithm (MALA), \[
    R(x'\mid x) = \mathcal{N}(x'; x-\tau\nabla f(x),2\tau\mat{I}).
\]
This proposes moving to high-density areas, thus it much more efficiently
converges to the stationary distribution.

The problem with this is that it requires access to the full energy function to
compute the gradient, which can be expensive for large datasets. This is solved
by stochastically estimating the gradient $\nabla f(x)$.
