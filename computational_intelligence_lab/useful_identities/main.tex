\documentclass{article}

% Typesetting
\usepackage[utf8]{inputenc}
\usepackage[T1]{fontenc}
\usepackage[activate={true,nocompatibility},final,tracking=true,kerning=true,spacing=true,factor=1100,stretch=10,shrink=10]{microtype}
\microtypecontext{spacing=nonfrench}

% Margins
\usepackage[a4paper, margin=3cm]{geometry}

% Math
\usepackage{amsmath,amsfonts,amsthm,amssymb,mathrsfs,bbm,mathtools,nicefrac,bm,centernot}
\usepackage{derivative}

% Font
\usepackage[osf,sc]{mathpazo}

% Tables
\usepackage{booktabs,array,tabularx}
\setlength\tabcolsep{0.5cm}
\def\arraystretch{2.0}
\newcolumntype{L}{>{$\displaystyle}l<{$}}
\newcolumntype{R}{>{$\displaystyle}r<{$}}

% Better left and right: https://tex.stackexchange.com/a/2610/217707
\newcommand{\lft}{\mathopen{}\mathclose\bgroup\left}
\newcommand{\rgt}{\aftergroup\egroup\right}

% Common sets
\newcommand{\R}{\mathbb{R}}
\newcommand{\Q}{\mathbb{Q}}
\newcommand{\N}{\mathbb{N}}
\newcommand{\Z}{\mathbb{Z}}

% Expected value and variance
\newcommand{\E}{\mathbb{E}}
\newcommand{\Var}{\mathrm{Var}}

% Vector, matrix, and tensor
\renewcommand{\vec}[1]{\bm{#1}}
\newcommand{\mat}[1]{\bm{#1}}
\newcommand{\tens}[1]{\bm{\mathsf{#1}}}

% Common functions
\newcommand{\transpose}[1]{#1^\top}
\renewcommand{\det}[1]{\mathrm{det}\lft(#1\rgt)}
\newcommand{\trace}[1]{\mathrm{tr}\lft(#1\rgt)}
\newcommand{\diag}[1]{\mathrm{diag}\lft(#1\rgt)}

\title{\textsf{\textbf{Useful Identities}}}
\author{\textsf{Cristian Perez Jensen}}
\date{}

\begin{document}

\maketitle

\section*{\textit{Linear algebra}}

\begin{tabularx}{\textwidth}{@{}lLXR@{}}
    \textbf{Subadditivity}   & \| \vec{x} + \vec{y} \| \leq \| \vec{x} \| + \| \vec{y} \|                                                                                                                 \\
    \textbf{Cauchy-Schwarz}  & |\langle \vec{x}, \vec{y} \rangle| \leq \| \vec{x} \| \| \vec{y} \|                                                                                                        \\
    \textbf{Cosine theorem}  & \| \vec{x} - \vec{y} \|^2 = \| \vec{x} \|^2 + \| \vec{y} \|^2 - 2 \langle \vec{x}, \vec{y} \rangle                                                                         \\
                             & \| \mat{A} \|_F^2 = \trace{\transpose{\mat{A}} \mat{A}}                                                                                                                    \\
                             & \| \mat{A} \|_F^2 = \sum_{i=1}^{\min \{ n,m \}} \sigma_i^2                                          &  & \mat{A} = \mat{U} \mathrm{diag}(\vec{\sigma}) \transpose{\mat{V}} \\
                             & \| \mat{A} \|_2 = \sigma_1                                                                          &  & \mat{A} = \mat{U} \mathrm{diag}(\vec{\sigma}) \transpose{\mat{V}} \\
    \textbf{Cyclic property} & \trace{\mat{A} \mat{B} \mat{C}} = \trace{\mat{C} \mat{A} \mat{B}} = \trace{\mat{B} \mat{C} \mat{A}}                                                                        \\
\end{tabularx}

\section*{\textit{Probability}}

\begin{tabularx}{\textwidth}{@{}lLXR@{}}
    \textbf{Sum rule}     & p(x) = \int p(x, y) \mathrm{d}y               \\
    \textbf{Product rule} & p(x, y) = p(x \mid y) p(y) = p(y \mid x) p(x) \\
    \textbf{Bayes rule}   & p(x \mid y) = \frac{p(y \mid x) p(x)}{p(y)}   \\
\end{tabularx}

\section*{\textit{Machine learning}}

\begin{tabularx}{\textwidth}{@{}lLXR@{}}
    \textbf{Sigmoid}                          & \sigma(z) \doteq \frac{1}{1 + \exp(-z)}                               \\
    \textbf{Sigmoid derivative}               & \odv{\sigma(z)}{z} = \sigma(z) (1 - \sigma(z)) = \sigma(z) \sigma(-z) \\
    \textbf{Hyperbolic tangent}               & \tanh(z) \doteq \frac{\exp(z) - \exp(-z)}{\exp(z) + \exp(-z)}         \\
    \textbf{Hyperbolic tangent derivative}    & \odv{\tanh(z)}{z} = 1 - \tanh^2(z)                                    \\
    \textbf{Rectified linear unit}            & \mathrm{ReLU}(z) \doteq \max \{ 0, z \}                               \\
    \textbf{Rectified linear unit derivative} & \odv{\mathrm{ReLU}(z)}{z} = \mathbb{1} \{ z > 0 \}                    \\
\end{tabularx}

\end{document}
