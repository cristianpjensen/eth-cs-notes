\section{Epipolar geometry}

\begin{figure}[h!]
    \centering
    \incfig{epipolar-geometry}
    \caption{Epipolar geometry.}
    \label{fig:epipolar-geometry}
\end{figure}

In epipolar (two-view) geometry, there are three main questions that we need
to answer to find the 3-dimensional position of points that two cameras
depict:
\begin{enumerate}
  \item Given a point $\bm{x}$ in image 1, how does this constrain the
    position of the corresponding point $\bm{x}'$ in image 2?
  \item Given a set of corresponding image points $\{ \bm{x}_i
    \leftrightarrow \bm{x}_i' \}_{i=1}^n$, what are the matrices $\mat{P}$ and
    $\mat{P'}$ for the two cameras?
  \item Given corresponding image points $\{ \bm{x}_i \leftrightarrow
    \bm{x}_i' \}_{i=1}^n$ and cameras $\mat{P},\mat{P}'$, what is the position
    of $\bm{X}$ in space.
\end{enumerate}

A point $\bm{X}$ is shown on an image by projecting that point onto the focal
center of a camera. Thus, a point the 3-dimensional point corresponding to
$\bm{x}$ in an image 1 has to be on the line that projects onto that point.
If we project this line onto image 2, we get all the possible points where
this point $\bm{x}$ could appear in image 2. However, since we know it did
not appear behind camera 1, we know that it cannot appear left of the
epipolar point $\bm{e}'$ in image 2. The epipolar point is the projection of
camera 1's position onto camera 2's image plane.

As can be seen in Figure ..., this constructs the epipolar plane. If we know
that a point is on an epipolar plane, we also know on which lines this point
will appear in both images. Thus, we have a bundle of planes that go through
the baseline that map lines to lines between the images.

In summary, an epipole is the projection of the projection center of the
other camera. \Ie, it is the vanishing point of the other camera. The
baseline is the line between the two epipolar poles, and the epipolar plane
is a plane containing the baseline. Lastly, an epipolar line is the
intersection of the epipolar plane with the image, which always comes in
corresponding pairs.

\subsection{Correspondence geometry}

\begin{figure}[ht]
    \centering
    \incfig{image-1-constraint}
    \caption{Knowing $\vec{x}_1$ constrains its corresponding point $\vec{x}_2$
    to be on the epipolar line $\vec{\ell}_2$, and not behind $\vec{e}_2$,
    because then $\vec{x}_1$ would be behind camera 1, which is not possible.}
    \label{fig:image-1-constraint}
\end{figure}

\begin{definition}[Cross-product matrix]
  The cross product matrix is the conversion of a vector to a matrix that
  would be equivalent to performing a cross product with that vector, \[
    [\bm{a}]_\times \doteq \begin{bmatrix}
      0 & a_3 & -a_2 \\
      -a_3 & 0 & a_1 \\
      a_2 & -a_1 & 0
    \end{bmatrix}
  .\]
\end{definition}

The fundamental matrix $\mat{F}$ is the matrix that maps points from image 1
to lines in image 2, and vice versa.\sidenote{Since the points are
2-dimensional and the lines are 1-dimensional, $\mat{F}$ must be rank 2.}
Assume we know the camera matrices $\mat{P},\mat{P}'$ of the two cameras. Then,
the epipolar point in image 2 is $\mat{P}'\bm{C}$. To get the epipolar line
$\bm{\ell}'$, we need some point on the line. \sidenote{Since we already know
one point on that line, which is the epipolar point, we can then compute the
line between them with the cross product.} We can get some point on the
projection line onto $\bm{x}$ by the pseudo-inverse of $\mat{P}$, denoted by
$\mat{P}^+$.\sidenote{The pseudo-inverse can be computed with SVD as
$\mat{P}^+=\mat{V}\frac{1}{\mat{\Sigma}}\mat{U}^\top$.} Thus, we can find a point
on $\bm{\ell}'$ by computing $\mat{P}'\mat{P}^+ \bm{x}$. In consusion, we can
compute the line by the following equation,\sidenote{Note that $\bm{C}$ is a
3-dimensional point (denoted as slanted), not a matrix.} \[
  \bm{\ell}' = \mat{P}'\bm{C}\times \mat{P}'\mat{P}^+\bm{x}
.\]
Thus, we define the fundamental matrix as \[
  \mat{F} = [\bm{e}']_\times \mat{P}'\mat{P}^+
.\]

The fundamental matrix has to satisfy the condition that for any pair of
corresponding points $\bm{x}\leftrightarrow\bm{x}'$, the following
holds,\sidenote{Recall that if $\bm{x}'$ is on line $\bm{\ell}'$, it should
satisfy $\bm{x}'^\top\bm{\ell}'=0$.} \[
  \bm{x}'^\top \mat{F}\bm{x} = 0
.\]

\paragraph{Computing the fundamental matrix.}

Given that we know 8 point correspondences,\sidenote{We assume that all our
point correspondences are correct, \ie they satisfy
$\bm{x}'^\top\mat{F}\bm{x}=0$.} we can compute $\mat{F}$ using direct
linear transformation, \[
  \begin{bmatrix}
    x_i'x_i & x'_iy_i & \bm{x}' & y_i'x_i & y_1'y_i & y_i' & x_1 & y_1 & 1
  \end{bmatrix}
  \begin{bmatrix}
    f_{11} \\
    f_{12} \\
    f_{13} \\
    f_{21} \\
    f_{22} \\
    f_{23} \\
    f_{31} \\
    f_{32} \\
    f_{33}
  \end{bmatrix}
  = \begin{bmatrix} 0 \end{bmatrix}
.\]
However, the problem with this is that the values of the matrix grow very
large, because of the multiplications. Using DLT, this will always result in
the best matrix being very similar to the following, \[
  \begin{bmatrix}
    0 & 0 & 0 \\
    0 & 0 & 0 \\
    0 & 0 & 1
  \end{bmatrix}
.\]
Thus, we need to first normalize the coordinates of the images to be in $[-1,
1]\times[-1, 1]$ before computing $\mat{F}$.

\paragraph{Singularity constraint.}

Enforce the rank 2 constraint of $\mat{F}$. ...

\subsection{Camera geometry}

\subsection{Scene geometry}
