\section{Epipolar geometry}

\begin{remark}
    In this section, all points are in projective space. They are not denoted by
    $\tilde{\vec{x}}$ and $\tilde{\vec{X}}$, but rather by $\vec{x}$ and
    $\vec{X}$.
\end{remark}

Epipolar geometry is the geometry of stereo vision, where two cameras view a
scene. When two cameras view a scene from distinct positions, there are
geometric relations between the 3-dimensional points and their 2-dimensional
projections on the image planes. These geometric relations lead to constraints
between the image points that we can make use of. These relations are derived
based on the assumption that the two cameras are approximated by the pinhole
camera model (\cref{sec:camera-model}).

\begin{figure}[h!]
    \centering
    \incfig{epipolar-geometry}
    \caption{Epipolar geometry.}
    \label{fig:epipolar-geometry}
\end{figure}

\Cref{fig:epipolar-geometry} shows the geometric relationships between the
3-dimensional points and its projections onto the images. $\vec{X}$ is the
3-dimensional point that is projected onto camera 1 as $\vec{x}_1$ and camera 2
as $\vec{x}_2$ (as seen in the pinhole camera model). Together, $\vec{C}_1$,
$\vec{C}_2$, and $\vec{X}$ form the \textit{epipolar plane} $\vec{\pi}$.
$\vec{x}_1$ and $\vec{x}_2$ must also be on this plane. More specifically, the
intersections between the epipolar plane and image planes are lines called the
\textit{epipolar lines} $\vec{\ell}_1$ and $\vec{\ell}_2$. $\vec{x}_1$ must be
on $\vec{\ell}_1$ and $\vec{x}_2$ must be on $\vec{\ell}_2$.

We can also see this from a different perspective. The projections of
$\vec{C}_2$ onto image 1, and $\vec{C}_1$ onto image 2, are called the
\textit{epipoles} $\vec{e}_1$ and $\vec{e}_2$. These are computed as follows,
\begin{align*}
    \vec{e}_1 & = \mat{P}_1 \vec{C}_2  \\
    \vec{e}_2 & = \mat{P}_2 \vec{C}_1,
\end{align*}
where $\mat{P}_1$ and $\mat{P}_2$ are the camera matrices of cameras 1 and 2,
respectively. The line formed between $\vec{x}_1$ and $\vec{e}_1$ is the
epipolar line (same for $\vec{x}_2$ and $\vec{e}_2$). This gives us the
following constraint,
\begin{align*}
    \vec{\ell}_1 & = \vec{e}_1 \times \vec{x}_1  \\
    \vec{\ell}_2 & = \vec{e}_2 \times \vec{x}_2.
\end{align*}
Intuitively, this is because both $\vec{C}_1$ and $\vec{C}_2$ are on the
epipolar plane, thus so must be $\vec{e}_1$ and $\vec{e}_2$. The epipolar lines
are the intersection of the epipolar plane with the image planes, and we know
two points that are on both, $\vec{x}$ and $\vec{e}$, which form a line.

Notice that $\vec{x}_1$ cannot be ``behind`` $\vec{e}_1$, because that would
imply that $\vec{X}$ is behind $\vec{C}_1$, which is not the case, because
$\vec{X}$ is visible in image 1. The same holds for camera 2.

In general, we want to be able to reconstruct 3-dimensional points from
2-dimensional point correspondences between $n$ images. This is done in the
following three steps,
\begin{enumerate}
    \item Compute the fundamental (or essential) matrix that relates the point
          correspondences, \[
              \transpose{\vec{x}_2} \mat{F} \vec{x}_1 = 0
              ,\]
          where $\mat{F}$ transforms $\vec{x}_1$ to the epipolar line of $\vec{x}_2$
          and $\transpose{\mat{F}}$ transforms $\vec{x}_2$ to the epipolar line of
          $\vec{x}_1$. This is done using the direct linear transform;

    \item From the fundamental matrix, compute the relative pose
          $(\vec{T},\mat{R})$ between the cameras;

    \item Using the point correspondences and the relative pose, backproject the
          2-dimensional image points into 3-dimensional space and find their
          intersection to find $\vec{X}$.

\end{enumerate}

\subsection{Correspondence geometry}

We need to find out in what way $\vec{x}_1$ constrains the location of
$\vec{x}_2$ in image 2, without knowing $\vec{X}$. $\vec{x}_1$ constrains the
point $\vec{X}$ to be on the projection line from $\vec{C}_1$ to $\vec{x}_1$,
as can be seen in \Cref{fig:image-1-constraint,fig:internal-camera}.

\begin{figure}[ht]
    \centering
    \incfig{image-1-constraint}
    \caption{Knowing $\vec{x}_1$ constrains its corresponding point $\vec{x}_2$
        to be on the epipolar line $\vec{\ell}_2$, and not behind $\vec{e}_2$,
        because then $\vec{X}$ would be behind camera 1, which is not the case.}
    \label{fig:image-1-constraint}
\end{figure}

The key insight into constraining $\vec{x}_2$ is that we do not need to know
the exact location of $\vec{X}$. We only need to find a second point on
$\vec{\ell}_2$ to compute it with the cross product (we already know
$\vec{e}_2$). Furthermore, the projection line emanating from $\vec{x}_1$ is on
the epipolar plane, thus any point on this projection line is also on the
epipolar plane. If we project that point onto image plane 2, we get a second
point on $\vec{\ell}_2$. We can compute a point on the projection line by using
the pseudo-inverse $\mat{P}_1^+$.\sidenote{The pseudo-inverse can be computed
    with SVD as $\mat{M}^+=\mat{V}\frac{1}{\mat{\Sigma}}\transpose{\mat{U}}$.
    $\mat{M}^+ \vec{x}$ maps $\vec{x}$ to some point $\vec{y}$ that mapped to
    $\vec{x}=\mat{M}\vec{y}$. However, this is not necessarily $\vec{y}$, since we
    lose one dimension of information in the $3\times 4$ matrix $\mat{P}$.} A
3-dimensional point $\vec{X}'$ on the projection line between $\vec{C}_1$ and
$\vec{x}_1$ can be computed as follows, \[
    \vec{X}' = \mat{P}_1^+ \vec{x}_1.
\]
The epipolar line $\vec{\ell}_2$ can be computed by the cross product between
the epipole $\vec{e}_2$ and this point on image plane 2, \[
    \vec{\ell}_2 = \mat{P}_2 \vec{C}_1 \times \mat{P}_2 \mat{P}_1^+ \vec{x}_1.
\]

\marginnote[-1.25cm]{The cross product matrix is the conversion of a vector to a matrix
    that would be equivalent to performing a cross product with that vector, \[
        [\vec{a}]_\times \doteq \begin{bmatrix}
            0    & -a_3 & a_2  \\
            a_3  & 0    & -a_1 \\
            -a_2 & a_1  & 0
        \end{bmatrix}.
    \]}

\begin{definition}[Fundamental matrix]
    \label{def:fundamental-matrix}

    The fundamental matrix $\mat{F}$ is defined as satisfying the
    \textit{epipolar constraint} for all point correspondences in image 1 and 2,
    \[
        \transpose{\vec{x}_2} \mat{F} \vec{x}_1 = 0. \margintag{$\mat{F}$ has 7 degrees of freedom, because it is a $3\times 3$ matrix, is defined up to scale, and $\det{\mat{F}}=0$ (rank 2).}
    \]
    The fundamental matrix relates points in one image plane with lines in the
    other. \Ie, $\vec{x}_2$ must be on the line $\mat{F} \vec{x}_1$ and
    $\vec{x}_1$ must be on the line $\transpose{\vec{x}_2} \mat{F} =
        \transpose{\mat{F}} \vec{x}_2$.

    It satisfies the following equations \wrt the camera matrices,
    \begin{align*}
        \mat{F}             & = [\mat{P}_2 \vec{C}_1]_\times \mat{P}_2 \mat{P}_1^+ \margintag{Note that these equations are not used for estimating the fundamental matrix, since we do not have access to the camera matrices.} \\
        \transpose{\mat{F}} & = [\mat{P}_1 \vec{C}_2]_\times \mat{P}_1 \mat{P}_2^+.
    \end{align*}
\end{definition}

\begin{definition}[Essential matrix]
    The essential matrix $\mat{E}$ relates points in one image plane with lines
    in the other by the epipolar constraint in the same way as the fundamental
    matrix (\cref{def:fundamental-matrix}), \[
        \transpose{\hat{\vec{x}}_2} \mat{E} \hat{\vec{x}}_1 = 0, \margintag{$\mat{E}$ has 5 degrees of freedom, because it has the same parameters as the fundamental matrix, minus the intrinsic camera parameters.}
    \]
    where $\hat{\vec{x}} = \inv{\mat{K}} \vec{x}$.

    The difference is that the essential matrix assumes that the cameras are
    calibrated. This means that the intrinsic camera parameters are known. We have
    the following relation between the two matrices, \[
        \mat{E} = \transpose{\mat{K}} \mat{F} \mat{K}.
    \]
\end{definition}

\paragraph{8-point algorithm.}

Given that we know 8 point correspondences,\sidenote{We assume that all our
    point correspondences are correct, \ie, they satisfy
    $\transpose{\vec{x}_1}\mat{F}\vec{x}_2=0$.} we can compute $\mat{F}$ using
direct linear transformation (DLT) derived from
$\transpose{\vec{x}'}\mat{F}\vec{x}=0$, \[
    \begin{bmatrix}
        x_1'x_1 & x'_1y_1 & x_1'   & y_1'x_1 & y_1'y_1 & y_1'   & x_1    & y_1    & 1      \\
        \vdots  & \vdots  & \vdots & \vdots  & \vdots  & \vdots & \vdots & \vdots & \vdots \\
        x_n'x_n & x'_ny_n & x_n'   & y_n'x_n & y_n'y_n & y_n'   & x_n    & y_n    & 1
    \end{bmatrix}
    \begin{bmatrix}
        f_{11} \\
        f_{12} \\
        f_{13} \\
        f_{21} \\
        f_{22} \\
        f_{23} \\
        f_{31} \\
        f_{32} \\
        f_{33}
    \end{bmatrix}
    = \vec{0}, \margintag{Let $\mat{A}$ denote the matrix, and $\vec{f}$ the vector of fundamental matrix entries.}
\]
after which we use the SVD trick to find the best algebraic solution.

However, the problem with this is that the values of the matrix grow very large,
because of the multiplications, while the last column vector remains 1. Using
DLT, this will always result in the best matrix being very similar to the
following, \[
    \begin{bmatrix}
        0 & 0 & 0 \\
        0 & 0 & 0 \\
        0 & 0 & 1
    \end{bmatrix}.
\]
Thus, importantly, we need to first normalize the coordinates of the images to
be in $[-1, 1]\times[-1, 1]$ before computing $\mat{F}$.

Now, we also need to enforce the \textit{singularity constraint} (rank 2,
$\det{\mat{F}}=0$) of $\mat{F}$, since DLT will give a rank 3 matrix. One
possible solution is to set the smallest singular value to equal 0 using
SVD. However, this does not necessarily yield the best results, because we are
adjusting it a posteriori.

\paragraph{7-point algorithm.}

We would like our algorithm to only be able to give rank-2 matrices. We could
use 7 point-correspondences, making $\mat{A}$ a $7\times 9$ matrix.\sidenote{We
    could also use more and then use the space corresponding to the two smallest
    singular values} Thus, this matrix has a 2-dimensional null space, which we
denote by $\vec{f}_1$ and $\vec{f}_2$ (with corresponding fundamental matrices
$\mat{F}_1$ and $\mat{F}_2$). This means that we have a family of solutions to
the problem such that $\mat{A} (\vec{f}_1 + \lambda \vec{f}_2) = \vec{0}$, \[
    \mat{F}_1 + \lambda \mat{F}_2.
\]
However, this does not necessarily give a rank-2 matrix. The solution to this
problem satisfies the following, \[
    \det{\mat{F}_1 + \lambda \mat{F}_2} = 0,
\]
which results in a cubic equation with 1 or 3 solutions for $\lambda$.

\paragraph{5-point algorithm.}

The essential matrix only has 5 degrees of freedom, thus we only need 5 point
correspondences to compute it. However, as in the 7-point algorithm, this
results in a 4-dimensional null space. Furthermore, we additionally have the
constraint that $\mat{E}$ has two equal non-zero singular values. This results
in 10 cubic polynomials that need to be solved for.

\subsection{Camera geometry}

From the essential matrix,\sidenote{Or the fundamental matrix, since they are
    closely related by \[ \mat{E} = \transpose{\mat{K}} \mat{F} \mat{K} ,\] as we
    have already seen.} we can compute the relative pose between the 2 cameras
\citep{longuet1981computer}. This is done by first decomposing $\mat{E} =
    \mat{U} \mat{\Sigma} \transpose{\mat{V}}$.\sidenote{We know the following, \[
        \mat{\Sigma} = \begin{bmatrix} 1 & 0 & 0 \\ 0 & 1 & 0 \\ 0 & 0 & 0 \end{bmatrix},
    \]
    because of the constraints on the essential matrix.} Then, we
can compute the four possible relative poses as the following,
\begin{align*}
    \vec{T} = \mat{U}_3,  & \hspace{0.2cm} \mat{R} = \mat{U}\mat{W}\transpose{\mat{V}}              \\
    \vec{T} = -\mat{U}_3, & \hspace{0.2cm} \mat{R} = \mat{U}\mat{W}\transpose{\mat{V}}              \\
    \vec{T} = \mat{U}_3,  & \hspace{0.2cm} \mat{R} = \mat{U}\transpose{\mat{W}}\transpose{\mat{V}}  \\
    \vec{T} = -\mat{U}_3, & \hspace{0.2cm} \mat{R} = \mat{U}\transpose{\mat{W}}\transpose{\mat{V}},
\end{align*}
where $\mat{W}$ is the following matrix, \[
    \mat{W} = \begin{bmatrix}
        0 & -1 & 0 \\
        1 & 0  & 0 \\
        0 & 0  & 1
    \end{bmatrix}.
\]

For each configuration, we then check for the \textit{Cheirality condition},
\ie, the reconstructed points must be in front of the cameras.

\subsection{Scene geometry}

From the relative camera pose and point correspondences, we can triangulate the
points to find them in three dimensions. A possible solution would be to
backproject the $\vec{x}_1$ and $\vec{x}_2$ and find their intersection.

We can do this by first finding the $x$ and $y$ lines that the points
$\vec{x}_1$ and $\vec{x}_2$ are within their respective image. Let $\vec{x}$ be
the point in the image, then we have the following relations,
\begin{align*}
    \vec{x}      & = \vec{\ell}_x \times \vec{\ell}_y                      \\
    \vec{\ell}_x & = \transpose{\begin{bmatrix} -1 & 0 & x \end{bmatrix}}  \\
    \vec{\ell}_y & = \transpose{\begin{bmatrix} 0 & -1 & y \end{bmatrix}}.
\end{align*}
Then, we can find the planes on which these lines are within the 3-dimensional
space,
\begin{align*}
    \vec{\Pi}_x & = \transpose{\mat{P}} \vec{\ell}_x \\
    \vec{\Pi}_y & = \transpose{\mat{P}} \vec{\ell}_y
    .\end{align*}
Lastly, to get $\vec{x}$'s backprojection, we have to find the intersection
between these planes, which is defined by the points $\vec{X}$ that satisfy the
following, \[
    \begin{bmatrix} \transpose{\vec{\Pi}_x} \\ \transpose{\vec{\Pi}_y} \end{bmatrix} \vec{X} = \vec{0}.
\]

We want to find the intersection of the two backprojections of $\vec{x}_1$ and
$\vec{x}_2$.\sidenote{We could also use $n > 2$ cameras, which relies on the
    same idea: finding the intersection of $n$ backprojections. However, more than
    4 cameras does not help.} We can find this by finding a solution for the two
backprojections that satisfies the following, \[
    \det{\begin{bmatrix} \transpose{\vec{\ell}_{x_1}} \mat{P}_1 \\ \transpose{\vec{\ell}_{y_1}} \mat{P}_1 \\ \transpose{\vec{\ell}_{x_2}} \mat{P}_2 \\ \transpose{\vec{\ell}_{y_2}} \mat{P}_2 \end{bmatrix}} = 0.
\]

\begin{figure}[t]
    \centering
    \incfig{mid-point-reprojection}
    \caption{Mid-point reprojection.}
    \label{fig:mid-point-reprojection}
\end{figure}

However, due to noise, the two backprojections will never truly intersect.
Thus, we need to find a best intersection. A natural best estimate is the
mid-point of the shortest line between the two backprojection lines. This
minimizes the 3-dimensional error, but not the reprojection error. We could
further optimize this estimate to minimize the reprojection error, \[
    d(\vec{x}_1, \mat{P}_1 \hat{\vec{X}})^2 + d(\vec{x}_2, \mat{P}_2 \hat{\vec{X}})^2.
\]
