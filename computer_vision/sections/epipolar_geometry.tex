\section{Epipolar geometry}

\begin{remark}
  In this section, all points are in projective space. They are not denoted by
  $\tilde{\vec{x}}$ and $\tilde{\vec{X}}$, but rather by $\vec{x}$ and
  $\vec{X}$.
\end{remark}

Epipolar geometry is the geometry of stereo vision, where two cameras view a
scene. When two cameras view a scene from distinct positions, there are
geometric relations between the 3-dimensional points and their 2-dimensional
projections on the image planes. These geometric relations lead to constraints
between the image points that we can make use of. These relations are derived
based on the assumption that the two cameras are approximated by the pinhole
camera model (\cref{sec:camera-model}).

\begin{figure}[h!]
    \centering
    \incfig{epipolar-geometry}
    \caption{Epipolar geometry.}
    \label{fig:epipolar-geometry}
\end{figure}

\Cref{fig:epipolar-geometry} shows the geometric relationships between the
3-dimensional points and its projections onto the images. $\vec{X}$ is the
3-dimensional point that is projected onto camera 1 as $\vec{x}_1$ and camera 2
as $\vec{x}_2$ (as seen in the pinhole camera model). Together, $\vec{C}_1$,
$\vec{C}_2$, and $\vec{X}$ form the \textit{epipolar plane} $\vec{\pi}$.
$\vec{x}_1$ and $\vec{x}_2$ must also be on this plane. More specifically, the
intersections between the epipolar plane and image planes are lines called the
\textit{epipolar lines} $\vec{\ell}_1$ and $\vec{\ell}_2$. $\vec{x}_1$ must be
on $\vec{\ell}_1$ and $\vec{x}_2$ must be on $\vec{\ell}_2$. 

We can also see this from a different perspective. The projections of
$\vec{C}_2$ onto image 1, and $\vec{C}_1$ onto image 2, are called the
\textit{epipoles} $\vec{e}_1$ and $\vec{e}_2$. These are computed as follows,
\begin{align*}
  \vec{e}_1 &= \mat{P}_1 \vec{C}_2 \\
  \vec{e}_2 &= \mat{P}_2 \vec{C}_1
,\end{align*}
where $\mat{P}_1$ and $\mat{P}_2$ are the camera matrices of cameras 1 and 2,
respectively. The line formed between $\vec{x}_1$ and $\vec{e}_1$ is the
epipolar line (same for $\vec{x}_2$ and $\vec{e}_2$). This gives us the
following constraint,
\begin{align*}
  \vec{\ell}_1 &= \vec{e}_1 \times \vec{x}_1 \\
  \vec{\ell}_2 &= \vec{e}_2 \times \vec{x}_2
.\end{align*}
Intuitively, this is because both $\vec{C}_1$ and $\vec{C}_2$ are on the
epipolar plane, thus so must be $\vec{e}_1$ and $\vec{e}_2$. The epipolar lines
are the intersection of the epipolar plane with the image planes, and we know
two points that are on both, $\vec{x}$ and $\vec{e}$, which form a line.

Notice that $\vec{x}_1$ cannot be ``behind`` $\vec{e}_1$, because that would
imply that $\vec{X}$ is behind $\vec{C}_1$, which is not the case, because
$\vec{X}$ is visible in image 1. The same holds for camera 2.

% In epipolar (two-view) geometry, there are three main questions that we need
% to answer to find the 3-dimensional position of points that two cameras
% depict,
% \begin{enumerate}
%   \item Given a point $\bm{x}$ in image 1, how does this constrain the
%     position of the corresponding point $\bm{x}'$ in image 2?
%   \item Given a set of corresponding image points $\{ \bm{x}_i
%     \leftrightarrow \bm{x}_i' \}_{i=1}^n$, what are the matrices $\mat{P}$ and
%     $\mat{P'}$ for the two cameras?
%   \item Given corresponding image points $\{ \bm{x}_i \leftrightarrow
%     \bm{x}_i' \}_{i=1}^n$ and cameras $\mat{P},\mat{P}'$, what is the position
%     of $\bm{X}$ in space.
% \end{enumerate}

\subsection{Correspondence geometry}

We need to find out in what way $\vec{x}_1$ constrains the location of
$\vec{x}_2$ in image 2, without knowing $\vec{X}$. $\vec{x}_1$ constrains the
point $\vec{X}$ to be on the projection line from $\vec{C}_1$ to $\vec{x}_1$,
as can be seen in \Cref{fig:image-1-constraint,fig:internal-camera}.

\begin{figure}[ht]
    \centering
    \incfig{image-1-constraint}
    \caption{Knowing $\vec{x}_1$ constrains its corresponding point $\vec{x}_2$
    to be on the epipolar line $\vec{\ell}_2$, and not behind $\vec{e}_2$,
    because then $\vec{X}$ would be behind camera 1, which is not the case.}
    \label{fig:image-1-constraint}
\end{figure}

The key insight into constraining $\vec{x}_2$ is that we do not need to know
the exact location of $\vec{X}$. We only need to find a second point on
$\vec{\ell}_2$ to compute it with the cross product (we already know
$\vec{e}_2$). Furthermore, the projection line emanating from $\vec{x}_1$ is on
the epipolar plane, thus any point on this projection line is also on the
epipolar plane. If we project that point onto image plane 2, we get a second
point on $\vec{\ell}_2$. We can compute a point on the projection line by using
the pseudo-inverse $\mat{P}_1^+$.\sidenote{The pseudo-inverse can be computed
with SVD as $\mat{M}^+=\mat{V}\frac{1}{\mat{\Sigma}}\transpose{\mat{U}}$.}
A 3-dimensional point $\vec{X}'$ on the projection line between $\vec{C}_1$ and
$\vec{x}_1$ can be computed as follows, \[
  \vec{X}' = \mat{P}_1^+ \vec{x}_1
.\]
The epipolar line $\vec{\ell}_2$ can be computed by the cross product between
the epipole $\vec{e}_2$ and this point on image plane 2, \[
  \vec{\ell}_2 = \mat{P}_2 \vec{C}_1 \times \mat{P}_2 \mat{P}_1^+ \vec{x}_1
.\]

\marginnote{The cross product matrix is the conversion of a vector to a matrix
that would be equivalent to performing a cross product with that vector, \[
  [\vec{a}]_\times \doteq \begin{bmatrix}
    0 & a_3 & -a_2 \\
    -a_3 & 0 & a_1 \\
    a_2 & -a_1 & 0
  \end{bmatrix}
.\]}

\begin{definition}[Fundamental matrix]
  \label{def:fundamental-matrix}

  The fundamental matrix $\mat{F}$ is defined as the following,\sidenote{The
  fundamental matrix has 7 degrees of freedom, because it is a $3\times 3$
  matrix, is defined up to scale, and $\det{\mat{F}}=0$.}
  \begin{align*}
    \mat{F} &\doteq [\mat{P}_2 \vec{C}_1]_\times \mat{P}_2 \mat{P}_1^+ \\
    \mat{F} &\doteq [\mat{P}_1 \vec{C}_2]_\times \mat{P}_1 \mat{P}_2^+
  .\end{align*}
  The fundamental matrix relates points in one image plane with lines in the
  other by the following equation, \[
    \transpose{\vec{x}_1} \mat{F} \vec{x}_2 = 0
  .\]
  \Ie, $\vec{x}_1$ must be on the line $\mat{F} \vec{x}_2$ and $\vec{x}_2$
  must be on the line $\transpose{\vec{x}_1} \mat{F} = \mat{F} \vec{x}_1$.
\end{definition}

\begin{definition}[Essential matrix]
  The essential matrix $\mat{E}$ relates points in one image plane with lines
  in the other by the following equation in the same way as the fundamental
  matrix (\cref{def:fundamental-matrix}),\sidenote{The essential matrix has 5
  degrees of freedom, because it has the same parameters as the fundamental
  matrix, minus the focal length and skew.} \[
    \transpose{\vec{x}_1} \mat{E} \vec{x}_2 = 0
  .\]
  The difference is that the essential matrix assumes that the cameras are
  calibrated. This means that the intrinsic camera parameters are known.
\end{definition}

\paragraph{Computing the fundamental matrix.}

Given that we know 8 point correspondences,\sidenote{We assume that all our
point correspondences are correct, \ie, they satisfy
$\transpose{\vec{x}_1}\mat{F}\vec{x}_2=0$.} we can compute $\mat{F}$ using
direct linear transformation (DLT) derived from
$\transpose{\vec{x}'}\mat{F}\vec{x}=0$, \[
  \begin{bmatrix}
    x_1'x_1 & x'_1y_1 & x_1' & y_1'x_1 & y_1'y_1 & y_1' & x_1 & y_1 & 1 \\
    \vdots & \vdots & \vdots & \vdots & \vdots & \vdots & \vdots & \vdots & \vdots \\
    x_n'x_n & x'_ny_n & x_n' & y_n'x_n & y_n'y_n & y_n' & x_n & y_n & 1
  \end{bmatrix}
  \begin{bmatrix}
    f_{11} \\
    f_{12} \\
    f_{13} \\
    f_{21} \\
    f_{22} \\
    f_{23} \\
    f_{31} \\
    f_{32} \\
    f_{33}
  \end{bmatrix}
  = \vec{0}
.\]
However, the problem with this is that the values of the matrix grow very
large, because of the multiplications. Using DLT, this will always result in
the best matrix being very similar to the following, \[
  \begin{bmatrix}
    0 & 0 & 0 \\
    0 & 0 & 0 \\
    0 & 0 & 1
  \end{bmatrix}
.\]
Thus, we need to first normalize the coordinates of the images to be in $[-1,
1]\times[-1, 1]$ before computing $\mat{F}$.

\paragraph{Singularity constraint.}

Enforce the rank 2 constraint of $\mat{F}$. ...

\subsection{Camera geometry}

\subsection{Scene geometry}
