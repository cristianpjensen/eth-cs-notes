\section{Regression}

In regression, we try to estimate a function $f: \R^d \to \R$ that best fits a given dataset $\{
    (\vec{x}_i, y_i) \}_{i=1}^n \subseteq \R^d \times \R$. \Ie, we try to minimize the following loss
function, \[
    \ell(\vec{\beta}) = \frac{1}{n} \sum_{i=1}^{n} \lft\| f(\vec{x}_i) - y_i \rgt\|^2.
\]

\subsection{Linear regression}

In linear regression, we assume the underlying data to be linear with a fixed noise, \[
    Y \mid \vec{X} = \vec{x} \sim \mathcal{N} \lft( \transpose{\vec{\beta}_\star} \vec{x}, \sigma^2 \rgt)
\]
for some ground truth $\vec{\beta}_\star$. We parametrize $f$ as a linear function, \[
    f(\vec{x}; \vec{\beta}) = \transpose{\vec{\beta}} \vec{x}.
\]
Under these assumptions, the minimizer of the loss function is analytically tractable---the
ordinary least squares estimator (OLSE), \[
    \hat{\vec{\beta}} = \inv{\lft( \transpose{\mat{X}} \mat{X} \rgt)} \transpose{\mat{X}} \vec{y},
\]
where $\mat{X} \in \R^{d \times n}$ is the design matrix with respective outputs $\vec{y} \in
    \R^n$.

In practice, it is important to remove outliers, because linear models can be heavily influenced by
them. Also, the data should be standardized, such that all features are on the same scale, because
differences in scale can make matrix inversion unstable.

TODO: Curse of dimensionality \citep{sur2019modern}, \Cref{fig:overestimation,fig:overconfidence}.

\begin{marginfigure}
    \begin{tikzpicture}
        \begin{axis}[width=\textwidth, height=\textwidth, xlabel=$i$, ylabel=$\beta_i$]
            \addplot [only marks, fill=black, draw=black] table [x=index, y=w_star, col sep=comma] {data/log_reg_estimator_indices.csv};
            \addplot [scatter, only marks] table [x=index, y=w_hat, col sep=comma] {data/log_reg_estimator_indices.csv};
        \end{axis}
    \end{tikzpicture}
    \caption{$\vec{\beta}^\star$ is shown as black marks and $\hat{\vec{\beta}}$ is indicated by the marks. As can be seen, $\hat{\vec{\beta}}$ is overestimated for indices where $\beta^\star_i \neq 0$.}
    \label{fig:overestimation}
\end{marginfigure}

\begin{marginfigure}
    \begin{tikzpicture}
        \begin{axis}[width=\textwidth, height=\textwidth, xlabel=$\sigma \lft( \transpose{\vec{\beta}_\star} \vec{x} \rgt)$, ylabel=$\sigma \lft( \transpose{\hat{\vec{\beta}}} \vec{x} \rgt)$]
            \addplot [scatter, only marks] table [x=true_prob, y=pred_prob, col sep=comma] {data/log_reg_estimator_overconfident.csv};
        \end{axis}
    \end{tikzpicture}
    \caption{True \vs predicted probability---the estimator is overconfident in its predictions.}
    \label{fig:overconfidence}
\end{marginfigure}

Further, if features are collinear, then the model might only learn the correlation with the target
variable of one of them and discard the other. In addition, some singular values will equal zero.
This makes matrix inversion unstable. This can easily be shown by considering the singular value
decomposition $\mat{X} = \mat{U} \mat{D} \transpose{\mat{V}}$, then
\begin{align*}
    \hat{\vec{\beta}} & = \inv{\lft( \transpose{\mat{X}} \mat{X} \rgt)} \transpose{\mat{X}} \vec{y}                                                                                     \\
                      & = \inv{\lft( \mat{V} \transpose{\mat{D}} \transpose{\mat{U}} \mat{U} \mat{D} \transpose{\mat{V}} \rgt)} \mat{V} \transpose{\mat{D}} \transpose{\mat{U}} \vec{y} \\
                      & = \inv{\lft( \mat{V} \transpose{\mat{D}} \mat{D} \transpose{\mat{V}} \rgt)} \mat{V} \transpose{\mat{D}} \transpose{\mat{U}} \vec{y}                             \\
                      & = \mat{V} \transpose{\lft( \inv{\mat{D}} \rgt)} \inv{\mat{D}} \transpose{\mat{V}} \mat{V} \transpose{\mat{D}} \transpose{\mat{U}} \vec{y}                       \\
                      & = \mat{V} \transpose{\lft( \inv{\mat{D}} \rgt)} \inv{\mat{D}} \transpose{\mat{D}} \transpose{\mat{U}} \vec{y}                                                   \\
                      & = \mat{V} \inv{\mat{D}} \transpose{\mat{U}} \vec{y}.
\end{align*}
The inversion of $\mat{D}$ is unstable if the singular values are small. The solution to this is to
remove collinear features and/or to add regularization.

Consider the mean-squared error loss. The risk (expected loss) can be decomposed as follows into
bias, variance, and noise terms, \[
    \E \lft[ \lft( f(\vec{x}) - y \rgt)^2 \rgt] = \underbrace{\lft( \E[f(\vec{x})] - \E[y] \rgt)^2}_{\text{squared bias}} + \underbrace{\Var[f(\vec{x})]}_{\text{variance}} + \underbrace{\E \lft[ \lft( \E[y] - y \rgt)^2 \rgt]}_{\text{noise}}.
\]
The OLSE is the minimum variance unbiased estimator.\sidenote{This is proven by the Gauss-Markov
    theorem.} However, this does not mean it is the best, because introducing some bias may
considerably decrease the variance. Bayesianism adds bias by introducing a prior---priors often
induce a regularizing term.\sidenote{\Eg, a Gaussian prior induces an $\ell_2$-norm regularizing
    term.}

\subsection{Polynomial regression}

In polynomial regression, we construct a feature function of all possible polynomials, \eg, \[
    \phi([x_1, x_2]) = [1, x_1, x_2, x_1^2, x_1 x_2, x_2^2, x_1^3, x_1^2 x_2, \ldots].
\]
We then perform linear regression in this space, \[
    \psi(\vec{x}; \vec{\beta}) = \langle \vec{\beta}, \phi(\vec{x}) \rangle.
\]
The problem is that this space is infinitely dimensional, and the inner product is ill-defined in
an infinitely dimensional space.\sidenote{This is due to $\langle \phi(\vec{x}), \phi(\vec{x}')
    \rangle = \infty$ for some $\vec{x}, \vec{x}'$, \eg, $\vec{x} = \vec{x}' = [1,1]$.} We can solve
this by transforming the vector such that inner products cannot diverge by introducing a
data-dependent scalar, \[
    \phi(\vec{x}) = \alpha(\vec{x}) \cdot \tilde{\phi}(\vec{x}), \quad \alpha(\vec{x}) > 0. \margintag{$\tilde{\phi}$ contains all polynomials in this equation.}
\]
An inner product \wrt this feature function is a valid inner product. Such transformations can have
a closed form for the inner product, called kernelization. Commonly, the radial basis function
(RBF) kernel is used, \[
    \langle \phi(\vec{x}), \phi(\vec{x}') \rangle = \exp \lft( -\frac{\| \vec{x} - \vec{x}' \|^2}{2 \sigma^2} \rgt), \quad \sigma \in \R.
\]
Let $\mat{\Phi} \in \R^{n \times \infty}$ contain all feature vectors, then the OLSE for polynomial
regression can be computed by \[
    \hat{\vec{\beta}} = \inv{\lft( \transpose{\mat{\Phi}} \mat{\Phi} \rgt)} \transpose{\mat{\Phi}} \vec{y}, \quad \mat{\Phi \in \R^{n\times \infty}}.
\]
However, $\transpose{\mat{\Phi}} \mat{\Phi}$ cannot be computed, because it is $\infty \times
    \infty$-dimensional. We can fix this by observing that $\mat{\Phi} \transpose{\mat{\Phi}} \in \R^{n
        \times n}$ and rewriting the OLSE as
\begin{align*}
    \hat{\vec{\beta}} & = \inv{\lft( \transpose{\mat{\Phi}} \mat{\Phi} \rgt)} \transpose{\mat{\Phi}} \vec{y}                                                                                                   \\
                      & = \inv{\lft( \transpose{\mat{\Phi}} \mat{\Phi} \rgt)} \transpose{\mat{\Phi}} \lft( \mat{\Phi} \transpose{\mat{\Phi}} \rgt) \inv{\lft( \mat{\Phi} \transpose{\mat{\Phi}} \rgt)} \vec{y} \\
                      & = \inv{\lft( \transpose{\mat{\Phi}} \mat{\Phi} \rgt)} \transpose{\mat{\Phi}} \mat{\Phi} \transpose{\mat{\Phi}} \inv{\lft( \mat{\Phi} \transpose{\mat{\Phi}} \rgt)} \vec{y}             \\
                      & = \transpose{\mat{\Phi}} \inv{\lft( \mat{\Phi} \transpose{\mat{\Phi}} \rgt)} \vec{y}.
\end{align*}
Let $\mat{K} = \mat{\Phi} \transpose{\mat{\Phi}}$, then it only contains kernel evaluations---$k_{ij} = \langle \phi(\vec{x}_i), \phi(\vec{x}_j) \rangle$. The next problem is that $\transpose{\mat{\Phi}}$ is still infinitely dimensional. However, this is not a problem, since when we want to make a prediction, we do the following,
\begin{align*}
    \psi(\hat{\vec{x}}) & = \langle \phi(\hat{\vec{x}}), \hat{\vec{\beta}} \rangle                        \\
                        & = \transpose{\phi(\hat{\vec{x}})} \transpose{\mat{\Phi}} \inv{\mat{K}} \vec{y}.
\end{align*}
Let $\vec{k}(\hat{\vec{x}}) = \transpose{\phi(\hat{\vec{x}})} \transpose{\mat{\Phi}}$, then it only contains kernel evaluations with the new point---$k_i(\hat{\vec{x}}) = \langle \phi(\hat{\vec{x}}), \phi(\vec{x}_i) \rangle$. In conclusion, we can make predictions by \[
    \psi(\hat{\vec{x}}) = \vec{k}(\hat{\vec{x}}) \inv{\mat{K}} \vec{y}, \quad \vec{k}(\hat{\vec{x}}) \in \R^{1 \times n}, \mat{K} \in \R^{n \times n}, \vec{y} \in \R^n.
\]
However, the problem with this approach is that it takes $O(n^3)$ to make a prediction---it scales
in the amount of data points.

\subsection{Regularization}

TODO: Ridge regression---$\ell_2$-norm.

TODO: LASSO---$\ell_1$-norm induces sparsity. Better interpretability properties.

TODO: Figure showing coefficient weights under different regularization factors.
