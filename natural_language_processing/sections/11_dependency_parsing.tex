\section{Dependency parsing}

\textit{Dependency parsing} is an alternative to constituency parsing. The
basic idea is to link every word with its \textit{syntactic head}.\sidenote{We
    encode this as an arc in a graph, see
    \Cref{fig:dependency-parse,fig:non-projective-dependency-tree}.}

\begin{figure}[h!]
    \centering
    \incfig{dependency-parse}
    \caption{Projective dependency tree of ``The boy eats Rösti.``}
    \label{fig:dependency-parse}
\end{figure}

In a dependency tree, only one word gets to be the root, and each word has a
single parent, called its \textit{syntactic head}. This allows for words to be
linked that have other words in between not part of this structure, in contrast
to constituency parsings.

There are two types of dependency trees, projective and non-projective.
Projective dependency trees do not allow for crossing arcs, which make them
closely related to constituency trees. Non-projective dependency trees do allow
for crossing arcs, which will be the focus of this text. An example of a
non-projective dependency tree can be found in
\Cref{fig:non-projective-dependency-tree}.

\begin{figure}[ht]
    \centering
    \incfig{non-projective-dependency-tree}
    \caption{Non-projective dependency tree, without labels, of ``Ryan ate
        Rösti, which was delicious.``}
    \label{fig:non-projective-dependency-tree}
\end{figure}

As always, we want to be able to parametrize a probability distribution over
non-projective spanning trees, given a sentence $\vec{w}$. So, we need to be
able to compute the normalizer $Z(\vec{w})$. However, there are $(N-1)^{N-2}$
spanning trees with the single root constraint, thus we need to design a more
efficient algorithm that makes use of its structure. We do this by decomposing
the scoring function over the edges, \[
    \score(\vec{t},\vec{w}) \doteq \score(r,\vec{w}) + \sum_{(i\to j)\in\vec{t}} \score(i,j,\vec{w})
    .\]
Thus, we only need a matrix $\mat{A}$, containing $\exp \score(i,j,\vec{w})$ and a
vector $\vec{\rho}$, containing $\exp \score(r,\vec{w})$.

\begin{theorem}[Kirchhoff's matrix-tree theorem \citep{kirchhoff1847ueber}]
    For an undirected unweighted graph $\mathcal{G}$ with $N$ vertices, let
    $\mat{L}$ be the graph Laplacian, \[
        \mat{L}_{ij} = \begin{cases}
            -\mat{A}_{ij}                & i \neq j            \\
            \sum_{k \neq i} \mat{A}_{kj} & \mathrm{otherwise},
        \end{cases}
    \]
    where $\mat{A}$ is the adjacency matrix, \ie, $\mat{A}_{ij}=1$ if $i \sim j$,
    otherwise $\mat{A}_{ij}=0$. Let $\hat{\mat{L}}_i \in \R^{(N-1)\times(N-1)}$ be the
    matrix created by removing the $i$-th row and column of $\mat{L}$. Then, we
    have \[
        N_T(\mathcal{G}) = \det{\hat{\mat{L}}_i}
        ,\]
    where $N_T(\mathcal{G})$ is the number of trees in $\mathcal{G}$.
\end{theorem}

\citet{tutte1948dissection} generalized Kirchhoff's MTT to directed trees,
which allows us to compute the normalizer $Z(\vec{w})$ in $\bigo{N^3}$ as
follows,\sidenote{Since computing the determinant can be done in $\bigo{N^3}$.}
\[
    Z(\bm{w}) = \det{\mat{L}}
    .\]
However, this does not account for the single-root constraint.
\citet{koo2007structured} further generalized Tutte's MTT by modifying the
graph Laplacian, \[
    \mat{L}_{ij} = \begin{cases}
        \vec{\rho}_j                 & i = 1               \\
        -\mat{A}_{ij}                & i \neq j            \\
        \sum_{k \neq i} \mat{A}_{kj} & \mathrm{otherwise}.
    \end{cases}
\]
So, we can now compute $Z(\vec{w}) = \det{\mat{L}}$ in $\bigo{N^3}$.

However, we are not able to semiringify this algorithm. Thus, we must design a
new algorithm for inference.

\subsection{Chu-Liu-Edmonds algorithm}

A valid dependency tree must adhere to the following three constraints,
\begin{itemize}
    \item All non-root nodes have exactly one incoming edge;
    \item No cycles;
    \item Only one outgoing edge from the root.
\end{itemize}

This is called an \textit{arborescence},\sidenote[][-0.75cm]{The first two
    constraints are satisfied by the maximum-weight spanning tree, which we could
    compute with Kruskal's algorithm. However, it does not adhere to the third
    constraint, because Kruskal's algorithm only works for undirected graphs.} and
we can find the maximum-weight arborescence with the Chu-Liu-Edmonds algorithm
\citep{chu1965shortest,edmonds1967optimum}, sped up to $\bigo{N^2}$ by
\citet{tarjan1977finding}.

The algorithm starts by constructing the \textit{greedy graph}, which is the
graph that takes the best incoming edge to each node, except the root. If the
greedy graph contains a cycle, we \textit{contract} the cycle into a single
node $c$, and break the cycle by reweighting the enter edges, which are the
edges that go into $c$.

\begin{figure*}[h!]
    \centering
    \incfig{clu-greedy}
    \caption{Chu-Liu-Edmonds algorithm. The second graph is the greedy graph.
        The third graph is the contracted graph, where we contract the cycle into
        its own node. Then, we construct the greedy graph again. We have two edges
        emanating from the root, so we must eliminate the one with the lowest swap
        score. Then, we expand by choosing all the edges in the cycle that are not
        canceled.}
    \label{fig:clu-greedy}
\end{figure*}

Then, we pick the greedy graph from the contracted graph. If there is more than
one edge emanating from the root, we need to delete edges outgoing from the root
to satisfy the single-root constraint. We remove the edge with the lowest
\textit{swap score}, which is the difference between the next-best incoming edge
and the current incoming edge of each node in the graph.

If there is still a cycle, contract again, and continue doing this recursively
until there are no more cycles. Then, expand all the cycles by picking the
edges that are not canceled by the greedy graph with the contracted node.
