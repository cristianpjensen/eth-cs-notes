\section{Finite-state automata} \label{sec:fsa}

\marginnote[-2cm]{An $n$-gram LM can be represented by a WFSA by setting the
states to be the history. The initial states are all $\bos$. We can only end on
the $\eos$ state. A state can go to another if the history can follow the other
and the weight is the probability of that happening. The initial and final
weights are all $\bm{1}$.}

\marginnote[1em]{A CRF can also be represented by a WFSA, where each POS tag is
a state. We can start and end on any state. The transitions then go from tag to
tag, where the weight is the score of the target tag following the source tag.
The initial weights are then the score of the tag following $\bos$, and the
target weights are the score of $\eos$ following the tag.}

A \textit{finite-state automaton} (FSA) is a computational device that
determines whether a string $\vec{s}\in\Sigma^*$ is an element of a given
language $L\subseteq\Sigma^*$. To check whether a string is part of a language
defined by an FSA, the FSA reads in letters of an input string
$\bm{s}\in\Sigma^*$. Then, it transitions from state to state according to the
transition function $\delta$ and the letters $a\in\bm{s}$. If there is a path
from an initial state to a final state while taking transitions as specified,
the FSA accepts the string and is part of its language.

\marginnote[0.5em]{The Kleene star of an alphabet $\Sigma$ is defined as \[ \Sigma^*
\doteq \bigcup_{n=0}^\infty \Sigma^n .\]}

\begin{definition}[Finite-state automaton]
  A finite-state automaton is a 5-tuple $\mathcal{A}=\langle \Sigma, Q, I, F,
  \delta \rangle$, such that
  \begin{itemize}
    \item $\Sigma$ is an alphabet;
    \item $Q$ is a finite set of states;
    \item $I\subseteq Q$ is the set of initial states;
    \item $F\subseteq Q$ is the set of final states;
    \item $\delta\subseteq Q \times (\Sigma\cup\{\epsilon\}) \times Q$ is a
      finite multi-set that defines the transitions between states. Let
      $\langle q_0,a,q_1 \rangle\in\delta$ be a transition, then, if we make
      that transition, we go to state $q_1$ from $q_0$ and concatenate $a$ to
      the current string.
  \end{itemize}
\end{definition}

\begin{marginfigure}
  \centering
  \incfig{example-fsa}
  \caption{The graph of a simple FSA, where initial states are denoted by an
  incoming arrow and final states are denoted by a double circle. This defines
  the language $L = \{ b(ca)^*cc, a(b)^*cb(ca)^*cc \}$.}
  \label{fig:example-fsa}
\end{marginfigure}

A \textit{weighted finite-state automation} (WFSA) adds weights from a semiring
to the transitions ($\delta\subseteq Q\times (\Sigma \cup \{ \epsilon \})
\times \mathbb{K} \times Q$), initial states ($\lambda: I\to\mathbb{K}$), and
final states  ($\rho: F\to\mathbb{K}$). Weights are added together using the
$\otimes$ operator.

A \textit{weighted finite-state transducer} (WFST) further adds an output
alphabet. The transitions then go from state to state while mapping input
characters to output characters. Formally, $\delta \subseteq Q \times (\Sigma
\cup \{\epsilon\}) \times (\Omega \cup \{\epsilon\}) \times \mathbb{K} \times
Q$, where $\Omega$ is the output alphabet.

\subsection{WFST composition}

WFST composition of two transducers $\mathcal{T}_1$ and $\mathcal{T}_2$ is a
common operation that involves mapping the inputs of $\mathcal{T}_1$ to the
outputs of $\mathcal{T}_2$. This requires the output alphabet of
$\mathcal{T}_1$ to be equal to the input alphabet of $\mathcal{T}_2$.
Intuitively, it is the same as first running the input through $\mathcal{T}_1$
then that output through $\mathcal{T}_2$, \[
  x \xRightarrow{\mathcal{T}_1} z \xRightarrow{\mathcal{T}_2} y
.\]
The weight of mapping $x$ to $y$ using the composition of two WFSTs is the
semiring-sum of the weights of all possible transformations of the above form.

\begin{definition}[WFST composition]
  Formally, the composition $\mathcal{T}=\mathcal{T}_1 \circ \mathcal{T}_2$
  of two WFSTs,
  \begin{align*}
    \mathcal{T}_1 &= \langle \Sigma,\Omega,Q_1,I_1,F_1,\delta_1,\lambda_1,\rho_1 \rangle \\
    \mathcal{T}_2 &= \langle \Omega,\Xi,Q_2,I_2,F_2,\delta_2,\lambda_2,\rho_2 \rangle
  ,\end{align*}
  is the WFST $\mathcal{T}=\langle
  \Sigma,\Xi,Q,I,F,\delta,\lambda,\rho \rangle$, such that \[
    \mathcal{T}(x,y) = \bigoplus_{z\in\Omega^*} \mathcal{T}_1(x,z) \otimes \mathcal{T}_2(z,y)
  .\]
\end{definition}

\begin{algorithm}
  \caption{Naive version of the algorithm for computing the composition of
  two WFSTs.}
  \label{fig:naive-composition}

  \begin{algorithmic}[1]
    \Function{NaiveComposition}{$\mathcal{T}_1,\mathcal{T}_2$}
      \State $\mathcal{T} \gets \langle \Sigma,\Omega,Q,I,F,\delta,\lambda,\rho \rangle$ \Comment{Create a new WFST}
      \For {$q_1,q_2\in Q_1\times Q_2$}
        \For {$q_1 \xrightarrow{\nicefrac{a:b}{w_1}} q_1', q_2 \xrightarrow{\nicefrac{c:d}{w_2}} q_2' \in E_1(q_1) \times E_2(q_2)$}
          \If {$b = c$}
            \State $Q \gets Q \cup \{(q_1,q_2),(q_1',q_2')\}$ \Comment{Add states}
            \State $\delta \gets \delta \cup \{ (q_1,q_2) \xrightarrow{\nicefrac{a:d}{w_1\otimes w_2}} (q_1',q_2') \}$ \Comment{Add arcs}
          \EndIf
        \EndFor
      \EndFor

      \For {$(q_1,q_2)\in Q$} \Comment{Initial and final weights}
        \State $\lambda((q_1,q_2)) \gets \lambda_1(q_1) \otimes \lambda_2(q_2)$
        \State $\rho((q_1,q_2)) \gets \rho_1(q_1) \otimes \rho_2(q_2)$
      \EndFor

      \State \Return $\mathcal{T}$
    \EndFunction
  \end{algorithmic}
\end{algorithm}

\subsection{Pathsum}

A path $\bm{\pi}\in\delta^*$ is an ordered set of \emph{consecutive}
transitions, \[
  \lft( q_1 \xrightarrow{\nicefrac{a_1:b_1}{w_1}} q_2, q_2 \xrightarrow{\nicefrac{a_2:b_2}{w_2}} q_3,\ldots, q_{N-1} \xrightarrow{\nicefrac{a_N:b_N}{w_N}} q_N \rgt)
.\]
The weight of this path is defined as \[
  w(\bm{\pi}) \doteq \lambda(q_1) \otimes \bigotimes_{n=1}^N w_n \otimes \rho(q_N)
.\]
The pathsum is then the semiring-sum over all paths in a WFST,
\begin{align*}
  Z(\mathcal{T}) &\doteq \bigoplus_{\bm{\pi}\in\Pi(\mathcal{A})} w(\bm{\pi}) \\
  &= \bigoplus_{\bm{\pi}\in\Pi(\mathcal{A})} \left( \lambda(q_1) \otimes \bigotimes_{n=1}^N w_n \otimes \rho(q_N) \right)
.\end{align*}

Under the real semiring, the pathsum computes \[
  \sum_{\bm{\pi}\in\Pi(\mathcal{A})} \lft( \lambda(q_1) \times \prod_{n=1}^N w_n \times \rho(q_N) \rgt)
,\]
which is the normalizer, while under the Viterbi semiring, the pathsum computes
\[
  \max_{\bm{\pi}\in\Pi(\mathcal{A})} \lft( \lambda(q_1) \times \prod_{n=1}^N w_n \times \rho(q_N) \rgt)
,\]
which is the maximum score of a path.

\subsection{Lehmann's algorithm}

Generally, there are an infinite amount of paths in a WFST, because of possible
cycles. Thus, we need to design an algorithm that can compute this potentially
infinite semiring-sum. Lehmann's algorithm \citep{lehmann1977algebraic}
computes the semiring-sum matrix $\mat{R}$ over all inner paths between all
pairs of nodes in $\bigo{|Q|^3}$ under a closed semiring. Then, we can use
these to compute the normalizer with the following equation, \[
  Z(\mathcal{T}) \doteq \bigoplus_{i,k \in Q} \lambda(q_i) \otimes \mat{R}_{ik} \otimes \rho(q_k)
.\]

\begin{algorithm}
  \caption{Lehmann's algorithm to compute the inner path semiring-sums.}
  \label{alg:lehmann}

  \begin{algorithmic}[1]
    \Function{Lehmann}{$\mat{W},\langle \mathcal{W},\oplus,\otimes,\bm{0},\bm{1} \rangle$}
      \LComment{$\mat{W}\in \mathcal{W}^{|Q| \times |Q|}$ is a matrix over a closed semiring}
      \State $\mat{R}^{(0)} \gets \mat{W}$
      \For {$j\gets 1,\ldots,|Q|$}
        \For {$i\gets 1,\ldots,|Q|$}
          \For {$k\gets 1,\ldots,|Q|$}
            \State $\mat{R}_{ik}^{(j)} \gets \mat{R}_{ik}^{(j-1)} \oplus \mat{R}_{ij}^{(j-1)} \otimes \lft( \mat{R}_{jj}^{(j-1)} \rgt)^* \otimes \mat{R}_{jk}^{(j-1)}$
          \EndFor
        \EndFor
      \EndFor

      \State \Return $\mat{I} \oplus \mat{R}^{(|Q|)}$
    \EndFunction
  \end{algorithmic}
\end{algorithm}

Lehmann's algorithm is a dynamic programming algorithm where
$\mat{R}_{ik}^{(j)}$ is the semiring-sum over all paths between $i$ and $k$
through $\{ 1,\ldots, j \}$. The base case is then the weight matrix $\mat{W}$
since that contains the weights of going from $i$ to $k$ directly.

Intuitively, the recurrence relationship, \[
  \mat{R}_{ik}^{(j)} = \mat{R}_{ik}^{(j-1)} \oplus \mat{R}_{ij}^{(j-1)} \otimes \lft( \mat{R}_{jj}^{(j-1)} \rgt)^* \otimes \mat{R}_{jk}^{(j-1)}
,\]
says that the sum over all paths through $\{ 1,\ldots,j \}$ is equal to the sum
over all paths through $\{ 1,\ldots,j-1 \}$ plus the sum over all paths through
$j$. The recurrence here is that we can use $\mat{R}_{ik}^{(j-1)}$ as the sum
over all paths through $\{ 1,\ldots,j-1 \}$. We can compute the sum over all
paths through $j$ by the following, \[
  \mat{R}_{ij}^{(j-1)} \otimes \lft( \mat{R}_{jj}^{(j-1)} \rgt)^* \otimes \mat{R}_{jk}^{(j-1)}
,\]
which is the weight of all paths from $i$ to $j$, $j$ to $j$ (cycles), and
finally $j$ to $k$.

\begin{algorithm}
  \caption{Floyd-Warshall algorithm to compute the shortest path distance
  between any two vertices in a graph without negative cycles. This is very
  similar to Lehmann's algorithm in the semiring $\langle \R, \min, +, \infty,
  0 \rangle$.}
  \label{alg:floyd-warshall}

  \begin{algorithmic}[1]
    \Function{Floyd-Warshall}{$\mat{W}$}
      \State $\mat{R}^{(0)} \gets \mat{W}$
      \For {$j\gets 1,\ldots,|Q|$}
        \For {$i\gets 1,\ldots,|Q|$}
          \For {$k\gets 1,\ldots,|Q|$}
            \State $\mat{R}_{ik}^{(j)} \gets \min \lft\{ \mat{R}_{ik}^{(j-1)}, \mat{R}_{ij}^{(j-1)} + \mat{R}_{jk}^{(j-1)} \rgt\}$
          \EndFor
        \EndFor
      \EndFor

      \State \Return $\min \lft\{ \mat{I}, \mat{R}^{(|Q|)} \rgt\}$
    \EndFunction
  \end{algorithmic}
\end{algorithm}

The Floyd-Warshall algorithm \citep{floyd1962algorithm,warshall1962theorem} has
the same intuition as Lehmann's algorithm. It is also a dynamic program with
the same state space. Also, the recurrence relationship is very similar, but it
assumes no negative cycles, thus it does not need the Kleene star to account
for cycles from $j$ to $j$.
