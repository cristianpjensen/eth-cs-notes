\section{Adversarial attacks}

In adversarial attacks, the attacker wants to make small changes to the input of the model such
that the model gives different results. This can have negative consequences in use cases such as
automated driving, where the traffic signs may be perturbed to give the wrong signals to the car.
Furthermore, it could also be used in medical classification and segmentation. Such attacks hint at
fundamental differences in human and machine vision.

\subsection{$p$-norm robustness}

Consider a multi-class classifier, \[
    f: \R^d \to \{ 1, \ldots, m \}.
\]
The goal of an adversarial attack is to find a perturbation $\vec{\eta}$ such that \[
    f(\vec{x} + \vec{\eta}) \neq f(\vec{x}), \quad \| \vec{\eta} \|_p \leq \epsilon,
\]
where \[
    \| \vec{x} \|_p \doteq \lft( \sum_{i=1}^{d} x_i^p \rgt)^{\nicefrac{1}{p}}, \quad \| \vec{x} \|_\infty \doteq \max_{i=1}^d |x_i|, \quad \| \vec{x} \|_0 \doteq |\{ i \mid x_i \neq 0 \}|. \margintag{Using $\infty$-norm, we get perturbation everywhere, whereas we only change a few pixels if we use $0$-norm.}
\]
We will focus on $p=2$ and an affine classifier that we wish to attack, \[
    f(\vec{x}) = \argmax_{i=1}^m f_i(\vec{x}), \quad f_i(\vec{x}) = \transpose{\vec{w}_i} \vec{x} + b_i.
\]
Further consider a binary classifier---$m=2$. Assume $\vec{x}$ is currently classified as the first
class, then we want $\vec{x} + \vec{\eta}$ to be classified as the second class. We need
$f_2(\vec{x} + \vec{\eta}) > f_1(\vec{x} + \vec{\eta})$, so we have the following convex program,
\begin{align*}
    \min              & \quad \frac{1}{2} \| \vec{\eta} \|^2_2                          \\
    \text{subject to} & \quad f_1(\vec{x} + \vec{\eta}) \leq f_2(\vec{x} + \vec{\eta}).
\end{align*}
This can be rewritten to
\begin{align*}
    \min              & \quad \frac{1}{2} \| \vec{\eta} \|_2^2                                               \\
    \text{subject to} & \quad \transpose{(\vec{w}_1 - \vec{w}_2)} (\vec{x} + \vec{\eta}) + b_1 - b_2 \leq 0.
\end{align*}
The Lagrangian is written as \[
    \mathcal{L}(\vec{\eta}, \lambda) = \frac{1}{2} \| \vec{\eta} \|_2^2 + \lambda \lft( \transpose{(\vec{w}_1 - \vec{w}_2)} (\vec{x} + \vec{\eta}) + b_1 - b_2 \rgt).
\]
By the KKT conditions, we have \[
    \grad{\mathcal{L}(\vec{\eta}, \lambda)}{\vec{\eta}} = \vec{\eta} + \lambda (\vec{w}_1 - \vec{w}_2) \condeq \vec{0}.
\]
Hence, \[
    \vec{\eta} = \lambda (\vec{w}_2 - \vec{w}_1).
\]
We only need to find the minimum $\lambda > 0$ such that
\begin{align*}
     &      & 0       & > f_1(\vec{x} + \lambda (\vec{w}_2 - \vec{w}_1)) - f_2(\vec{x} + \lambda(\vec{w}_2 - \vec{w}_1))        \\
     & \iff & 0       & > \transpose{(\vec{w}_1 - \vec{w}_2)} \lft( \vec{x} + \lambda (\vec{w}_2 - \vec{w}_1) \rgt) + b_1 - b_2 \\
     & \iff & 0       & > f_1(\vec{x}) - f_2(\vec{x}) - \lambda \| \vec{w}_1 - \vec{w}_2 \|_2^2                                 \\
     & \iff & \lambda & > \frac{f_1(\vec{x}) - f_2(\vec{x})}{\| \vec{w}_1 - \vec{w}_2 \|^2_2}.
\end{align*}
In conclusion the optimal $\vec{\eta}$ is the following, \[
    \vec{\eta} = \frac{f_1(\vec{x}) - f_2(\vec{x})}{\| \vec{w}_2 - \vec{w}_1 \|^2_2} \vec{w}_2 - \vec{w}_1.
\]
This can be generalized to any source class $i$ and target class $j$---if the target class does not
matter, you take the most easily confusable class. In the general case, we can linearize the model
and iteratively find the optimal adversarial perturbation \citep{moosavi2016deepfool}. This is done
by iteratively solving the following convex program,
\begin{align*}
    \min              & \quad \| \vec{\delta} \|_p                                                                                              \\
    \text{subject to} & \quad \transpose{\vec{\delta}} \lft( \grad{f_i(\vec{x})}{} - \grad{f_j(\vec{x})}{} \rgt) < f_i(\vec{x}) - f_j(\vec{x}).
\end{align*}
Then, update until $f_j(\vec{x} + \vec{\eta}_t) > f_i(\vec{x} + \vec{\eta}_t)$ \[
    \vec{\eta}_t = \vec{\eta}_{t-1} + \vec{\delta}, \quad \vec{\eta}_0 = \vec{0}.
\]

\subsection{Robust training}

Robust training is a systemic approach to making models robust to adversarial attacks. It works by
extending the loss function to neighborhoods of training points, \[
    \ell(\vec{x}) \mapsto \max_{\vec{\eta} : \| \vec{\eta} \|_p \leq \epsilon} \ell(\vec{x} + \vec{\eta}).
\]
This yields a two-player minimax game, where the adversary picks the worst perturbation and the
learner picks the best parameters in response, \[
    \argmin_{\vec{\theta} \in \Theta} \max_{\vec{\eta} : \| \vec{\eta} \|_p \leq \epsilon} \ell(\vec{x} + \vec{\eta}).
\]
The adversarial task can be solved with projected gradient ascent, \eg when $p=2$, we get \[
    \vec{\eta}_{t+1} = \epsilon \cdot \Pi \lft( \vec{\eta}_t + \alpha \cdot \grad{\ell(\vec{x} + \vec{\eta}_t)}{\vec{x}} \rgt), \quad \Pi(\vec{z}) \doteq \frac{\vec{z}}{\| \vec{z} \|_2}.
\]
And with $p=\infty$, we get \[
    \vec{\eta}_{t+1} = \epsilon \cdot \Pi \lft( \vec{\eta}_t + \alpha \cdot \mathrm{sgn} \lft( \grad{\ell(\vec{x} + \vec{\eta}_t)}{\vec{x}} \rgt) \rgt), \quad \Pi(\vec{z}) \doteq \frac{\vec{z}}{\| \vec{z} \|_\infty}.
\]
The fast gradient sign method \citep{goodfellow2014explaining} performs one iteration of projected
gradient descent with $p=\infty$, resulting in the following adversarial choice, \[
    \vec{\eta} = \epsilon \cdot \mathrm{sgn} \lft( \grad{\ell(\vec{x})}{\vec{x}} \rgt).
\]
