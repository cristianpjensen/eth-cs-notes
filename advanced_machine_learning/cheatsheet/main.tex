\documentclass[9pt]{extarticle}

\usepackage[a4paper, margin=0.5cm]{geometry}
\usepackage[utf8]{inputenc}
\usepackage[T1]{fontenc}
\usepackage{multicol}
\usepackage{amsmath,amsfonts,amsthm,amssymb,mathrsfs,bbm,mathtools,nicefrac,bm,centernot,colonequals,dsfont}
\usepackage{derivative}
\usepackage[skip=.5\baselineskip-3pt]{parskip}
\usepackage[mathspacing=normal, leadingfraction=0.8]{savetrees}
\usepackage[document]{ragged2e}
\usepackage{enumitem}

\usepackage{microtype}

\usepackage{mathpazo}
\usepackage[scaled=0.90]{helvet}
\usepackage[scaled=0.85]{beramono}
\usepackage[T1]{fontenc}
\usepackage{textcomp}

\usepackage{color,soul}
\usepackage{xcolor}

\usepackage{import}
\usepackage{xifthen}
\usepackage{pdfpages}
\usepackage{transparent}
\newcommand{\incfig}[1]{
    \def\svgwidth{\columnwidth}
    \import{./figures/}{#1.pdf_tex}
}

% Obscure delimiters
\DeclarePairedDelimiter{\floor}{\lfloor}{\rfloor}
\DeclarePairedDelimiter{\ceil}{\lceil}{\rceil}
\DeclarePairedDelimiter{\ip}{\langle}{\rangle}
\DeclarePairedDelimiter{\abs}{\lvert}{\rvert}
\DeclarePairedDelimiter{\norm}{\lVert}{\rVert}

% Better left and right
\newcommand{\lft}{\mathopen{}\left}
\newcommand{\rgt}{\aftergroup\mathclose\aftergroup{\aftergroup}\right}

% Common sets
\newcommand{\R}{\mathbb{R}}
\newcommand{\Q}{\mathbb{Q}}
\newcommand{\N}{\mathbb{N}}
\newcommand{\Z}{\mathbb{Z}}

% Expected value and variance
\newcommand{\E}{\mathbb{E}}
\newcommand{\Var}{\mathrm{Var}}

% Input and output space
\newcommand{\X}{\mathcal{X}}
\newcommand{\Y}{\mathcal{Y}}

% Vector, matrix, and tensor
\renewcommand{\vec}[1]{\bm{#1}}
\newcommand{\mat}[1]{\bm{#1}}
\newcommand{\tens}[1]{\bm{\mathsf{#1}}}

% Big-O notation
\newcommand{\bigo}[1]{\mathcal{O}\lft(#1\rgt)}

% Common functions
\newcommand{\transpose}[1]{#1^\top}
\newcommand{\inv}[1]{#1^{-1}}
\renewcommand{\det}[1]{\mathrm{det}\lft(#1\rgt)}
\newcommand{\tr}[1]{\mathrm{tr}\lft(#1\rgt)}
\newcommand{\diag}[1]{\mathrm{diag}\lft(#1\rgt)}

% Distributions
\newcommand{\sample}{\sim}
\newcommand{\sampleiid}{\overset{\mathrm{iid}}{\sim}}
\NewDocumentCommand{\Gauss}{O{0}O{1}o}{\mathcal{N}\lft(#1,#2 \IfValueT{#3}{\mid #3}\rgt)}
\newcommand{\MultiGauss}{\Gauss[\vec{0}][\mat{I}]}
\NewDocumentCommand{\Unif}{m}{\mathrm{Unif}\parentheses*{#1}}

% Math comment in smaller font size
\newcommand{\mathcomment}[1]{\mathrm{\footnotesize #1}}

% Gradient and Hessian operator
\newcommand{\grad}[2]{\vec{\nabla}_{\! #2} #1}
\newcommand{\hess}[2]{\vec{\nabla}^2_{\! #2} #1}
\newcommand{\jacob}[2]{\vec{D}_{#2} #1}

% List of symbols
\newlist{abbrv}{itemize}{1}
\setlist[abbrv,1]{label=,labelwidth=1in,align=parleft,itemsep=0.1\baselineskip,leftmargin=!}

% Add a period to the end of an abbreviation unless there's one
% already, then \xspace
\makeatletter
\DeclareRobustCommand\onedot{\futurelet\@let@token\@onedot}
\def\@onedot{\ifx\@let@token.\else.\null\fi\xspace}

\def\eg{\emph{e.g}\onedot} \def\Eg{\emph{E.g}\onedot}
\def\ie{\emph{i.e}\onedot} \def\Ie{\emph{I.e}\onedot}
\def\cf{\emph{c.f}\onedot} \def\Cf{\emph{C.f}\onedot}
\def\etc{\emph{etc}\onedot} \def\vs{\emph{vs}\onedot}
\def\wrt{w.r.t\onedot} \def\dof{d.o.f\onedot}
\makeatother

% Argument optima
\DeclareMathOperator*{\argmax}{argmax} % no space, limits underneath in displays
\DeclareMathOperator*{\argmin}{argmin} % no space, limits underneath in displays

% Margin tags
% \usepackage{marginfix}
\let\marginnotetemp\marginnote
\let\marginnote\relax
\usepackage{marginnote}
\let\marginnotepkg\marginnote

\NewDocumentCommand{\margintag}{O{0\baselineskip}m}{%
    \checkoddpage%
    \ifoddpage%
        {\marginnotepkg{\footnotesize #2}[#1]}%
    \else%
        {\reversemarginpar\marginnotepkg{\footnotesize #2}[#1]}%
    \fi}%
% \NewDocumentCommand{\margintagt}{O{0\baselineskip}m}{\marginpar{\vspace{#1}\footnotesize #2}}
% \NewDocumentCommand{\safefootnote}{om}{\footnotemark\margintag[#1]{\textsuperscript{\tiny\arabic{footnote}} \normalfont\justifying #2}}

\renewcommand\marginnote\marginnotetemp



\setlength{\columnseprule}{0.4pt}

\title{Deep Learning Cheatsheet}

\newenvironment{topic}[1]
{\textbf{\sffamily \colorbox{black}{\rlap{\textbf{\textcolor{white}{#1}}}\hspace{\linewidth}\hspace{-2\fboxsep}}}}
{}

% \newenvironment{subtopic}[1]
% {\begin{center}\textbf{\sffamily #1}\end{center}}
% {}

\newenvironment{subtopic}[1]
{\textbf{\sffamily #1:}}
{}

\renewcommand{\det}[1]{|#1|}
\renewcommand{\mat}[1]{\ensuremath{\mathbf{#1}}}
\renewcommand{\vec}[1]{\ensuremath{\mathbold{#1}}}
\renewcommand{\Var}{\mathbb{V}}

\begin{document}

\setlength{\columnsep}{0.1cm}

\begin{multicols*}{3}

    \textit{Projection}: $\mathrm{proj}_{\vec{b}} (\vec{a}) = \frac{\transpose{\vec{a}} \vec{b}}{\| \vec{b}
            \|^2} \vec{b}$.

    Bayes.\ has 3 steps: (1) def.\ prior, (2) def.\ likelihood, and (3) Bayes rule to compute
    posterior.

    \textit{Cholesky decomp}: If $\mat{A}$ is PD, then $\exists \mat{L}$ s.t.\ $\mat{A} = \transpose{\mat{L}} \mat{L}$.

    \textit{Geometric series}: $\sum_{k=0}^{\infty} ar^k = \nicefrac{a}{1-r}$ if $|r| < 1$.

    $1-x \leq \exp(-x) \implies (1-\epsilon)^n \leq \exp(-n \epsilon)$.

    \begin{subtopic}{Information theory}
        {\footnotesize\begin{align*}
                H(p)                        & = \E_p[-\log p(X)]                                                       \\
                D_{\mathrm{KL}}(p \;\|\; q) & = \E_p \lft[ \log \nicefrac{p(X)}{q(X)} \rgt]                            \\
                H(p,q)                      & = \E_p[-\log q(X)] = H(p) + D_{\mathrm{KL}}(p \;\|\; q)                  \\
                \mathrm{I}(X; Y)            & = \E \lft[ \log \nicefrac{p(X, Y)}{p(X) p(Y)} \rgt] = H(X) - H(X\mid Y).
            \end{align*}}

    \end{subtopic}

    \begin{subtopic}{Gaussian} \\
        $(2\pi)^{-\nicefrac{n}{2}} \det{\mat{\Sigma}}^{-\nicefrac{1}{2}} \exp \lft( -\frac{1}{2}\transpose{(\vec{x} - \vec{\mu})} \inv{\mat{\Sigma}}(\vec{x} - \vec{\mu}) \rgt)$.

        \textit{Conditional}: If $\begin{bmatrix} \vec{x}_1 \\ \vec{x}_2 \end{bmatrix} \sim \mathcal{N}\lft( \begin{bmatrix} \vec{\mu}_1 \\ \vec{\mu}_2 \end{bmatrix}, \begin{bmatrix} \mat{\Sigma}_{11} & \mat{\Sigma}_{12} \\ \mat{\Sigma}_{21} & \mat{\Sigma}_{22} \end{bmatrix} \rgt)$.
        Then: $\vec{x}_2 \mid \vec{x}_1 = \vec{z} \sim \mathcal{N}\lft( \bar{\vec{\mu}}, \bar{\mat{\Sigma}}
            \rgt)$, where $\bar{\vec{\mu}} = \vec{\mu}_2 + \mat{\Sigma}_{21} \inv{\mat{\Sigma}_{11}}
            (\vec{z} - \vec{\mu}_1)$ and $\bar{\mat{\Sigma}} = \mat{\Sigma}_{22} - \mat{\Sigma}_{21}
            \inv{\mat{\Sigma}_{11}} \mat{\Sigma}_{12}$.

        \textit{Information theory}:{\footnotesize
        \begin{align*}
            D_{\mathrm{KL}} & = \frac{1}{2} \lft[ \log \frac{\det{\mat{\Sigma}_2}}{\det{\mat{\Sigma}_1}} - d + \mathrm{tr}(\inv{\mat{\Sigma}_2} \mat{\Sigma}_1) + \transpose{(\vec{\mu}_2 - \vec{\mu}_1)} \inv{\mat{\Sigma}_2} (\vec{\mu}_2 - \vec{\mu}_1) \rgt] \\
            H               & = \frac{d}{2} \log(2\pi e) + \frac{1}{2} \log \det{\mat{\Sigma}}.
        \end{align*}}

    \end{subtopic}

    \begin{topic}{Paradigms of data science}
        Frequentism (optimize likelihood, MLE): $\vec{\theta}^\star \in \argmax_{\vec{\theta} \in \Theta} \sum_{i=1}^{n} \log p(\vec{x}_i \mid \vec{\theta})$.

        Bayesianism (optimize posterior, MAP): $\vec{\theta}^\star \in \argmax_{\vec{\theta} \in \Theta}
            \log p(\vec{\theta}) + \sum_{i=1}^{n} \log p(\vec{x}_i \mid \vec{\theta})$.

        Statistical learning (optimize risk): \\ $f^\star \in \argmax_{f \in \mathcal{F}} \mathcal{R}(f)
            \doteq \E_{X,Y}[\ell(Y, f(X))]$ \\ $\hat{f}_n \in \argmax_{f \in \mathcal{F}}
            \hat{\mathcal{R}}_n(f) \doteq \frac{1}{n} \sum_{i=1}^{n} \ell(y_i, f(\vec{x}_i))$.
    \end{topic}

    \begin{topic}{Anomaly detection}
        Objects $\mathcal{X} \subseteq \R^d$ with normal class $\mathcal{N} \subseteq \mathcal{X}$.
        Construct $\phi: \mathcal{X} \to \{ 0,1 \}$ such that $\phi(\vec{x}) =
            \mathbb{1}\{\vec{x} \not\in \mathcal{N}\}$. Anomaly is an ``unlikely event'' $\Rightarrow$
        Fit distribution to $\mathcal{X}$ and score according to $p(\vec{x})$.

        \begin{subtopic}{PCA}
            Proj.\ $\mathcal{X}$ to low-dim.\ $\Rightarrow$ $\Pi(\mathcal{N})$ is simpler.

            Linearly project $\R^d$ to $\R^{d^-}$ such that maximum variance is preserved. Base case $d^- = 1$:
            Find $\vec{u}$ with $\| \vec{u} \| = 1$ s.t.\ $\vec{x} \mapsto \transpose{\vec{u}} \vec{x}$. Sample
            mean and variance of reduced dataset: $\E[\transpose{\vec{u}} \vec{x}] = \transpose{\vec{u}}
                \E[\vec{x}]$ and $\Var[\transpose{\vec{u}} \vec{x}] = \transpose{\vec{u}} \mathrm{Cov}(\vec{x})
                \vec{u}$. We want maximum variance so we have: $\vec{u}^\star \in \argmax_{\| \vec{u} \|=1}
                \transpose{\vec{u}} \mathrm{Cov}(\vec{x}) \vec{u}$. Solvable by vanishing Lag.\ grad. Easy to find
            that $\vec{u}^\star$ is eigenvector with maximum eigenvalue. Then project it out ($\mathcal{X}_1 =
                \{ \vec{x} - \mathrm{proj}_{\vec{u}_1}(\vec{x}) \} = \{ \vec{x} - \transpose{\vec{u}_1} \vec{x}
                \cdot \vec{u}_1 \}$) and do the same for next dimension

        \end{subtopic}

        \begin{subtopic}{GMM}
            Lin.\ proj.\ onto low-dim.\ spaces resemble Gaussian dist.\ $\Rightarrow$ Fit
            GMM to $\Pi(\mathcal{X})$.

            Fit $p(\vec{x}; \vec{\theta}) = \sum_{j=1}^{k} \pi_j \mathcal{N}(\vec{x}; \vec{\mu}_j,
                \mat{\Sigma}_j)$ to data with EM algorithm. Can derive $\log p(\mat{X}; \vec{\theta}) = M(q,
                \vec{\theta}) + \E(q, \vec{\theta})$, where $M(q, \vec{\theta}) \doteq \E_q [\log
                    \nicefrac{p(\mat{X},z; \vec{\theta})}{q(z)}]$ and $E(q, \vec{\theta}) \doteq \E_q [\log
                    \nicefrac{q(z)}{p(z \mid \mat{X}; \vec{\theta})}]$. Properties: $\log p(\mat{X}; \vec{\theta}) \geq
                M(q, \vec{\theta})$ and $\log p(\mat{X}; \vec{\theta}) = M(q^\star, \vec{\theta})$ where $q^\star =
                p(\cdot \mid \mat{X}; \vec{\theta})$. Alg.: Iteratively $q^\star \in \argmin_q E(q,
                \vec{\theta}_{t-1})$ and $\vec{\theta}_t \in \argmax_{\vec{\theta}} M(q^\star, \vec{\theta})$.
            These can be done in closed form for GMM.

        \end{subtopic}

    \end{topic}

    \begin{topic}{Density estimation}
        MLE properties: (1) \textit{Consistency}: $\lim_{n\to\infty} \hat{\theta}^{\mathrm{MLE}}_n =
            \vec{\theta}$; (2) \textit{Equivariance}: If $\hat{\vec{\theta}}$ is the MLE of $\vec{\theta}$,
        then $g(\hat{\vec{\theta}})$ is the MLE of $g(\vec{\theta})$; (3) \textit{Asymptotically normal}:
        In the limit of $n$, $\nicefrac{\hat{\vec{\theta}} - \vec{\theta}}{\sqrt{n}}$ converges to
        $\mathcal{N}(\vec{0}, \inv{\mathcal{I}(\vec{\theta})})$; (4) \textit{Asymptotically efficient}: In
        the limit of $n$, MLE has smallest variance among unbiased estimators.

        Rao-Cram\'er bound: For any unbiased estimator: \[
            \Var[\hat{\theta}(\vec{y})] \geq \frac{(\pdv*{b_{\hat{\theta}}}{\theta} + 1)^2}{\mathcal{I}_n(\theta)} + b^2_{\hat{\theta}},
        \]
        where $\mathcal{I}_n(\theta) \doteq \E_{\vec{y} \mid \theta}[(\pdv*{\log p(\vec{y} \mid
                    \theta)}{\theta})^2]$ and $b_{\hat{\theta}} \doteq \E_{\vec{y}\mid \theta}[\hat{\theta}(\vec{y})] -
            \theta$. If unbiased: $\Var[\hat{\theta}(\vec{y})] \geq \nicefrac{1}{\mathcal{I}_n(\theta)}$. And
        MLE: $\lim_{n\to \infty} \Var[\hat{\theta}^{\mathrm{MLE}}(\vec{y})] =
            \nicefrac{1}{\mathcal{I}_n(\theta)}$.
    \end{topic}

    \begin{topic}{Regression}
        Minimize loss: $\ell(f) = \frac{1}{n} \sum_{i=1}^{n} (f(\vec{x}_i) - y_i)^2$.

        \begin{subtopic}{Linear regression}
            Assume $Y \mid \vec{X} = \vec{x} \sim \mathcal{N}(\transpose{\vec{\beta}_\star} \vec{x},
                \sigma^2)$. We parameterize $f(\vec{x}; \vec{\beta}) = \transpose{\vec{\beta}} \vec{x}$.

            OLSE: $\hat{\vec{\beta}} = \inv{(\transpose{\mat{X}} \mat{X})} \transpose{\mat{X}} \vec{y}$ s.t.\
            $\mat{X} \in \R^{n \times d}, \vec{y} \in \R^n$.

            Potential problems:
            \begin{enumerate}[left=0pt]
                \item Remove outliers, because linear models are heavily influenced by them;
                \item Standardize data, because features on different scales result in unstable matrix inversion;
                \item ``Curse of dimensionality'': In high dimensionality, logistic regression
                      outputs overconfident outputs, due to overestimation of weights;
                \item Collinear features result in unstable matrix inversion due to small eigenvalues.
            \end{enumerate}

            \textit{Risk decomposition}:\\\scalebox{0.85}{$\E[(\hat{f}(\vec{X}) - Y)^2] = (\E[\hat{f}(\vec{X})] - \E[y])^2 + \Var[\hat{f}(\vec{X})] + \Var[y]$}.
            \textit{Proof}: Use $Y = f(\vec{X}) + \epsilon$ and show $\E[\epsilon^2] = \Var[Y]$ where $\E[\epsilon] = 0$. Then $\pm \E[\hat{f}(\vec{X})]$ and finalize.

            \textit{Gauss-Markov}: $\Var[\transpose{\vec{a}} \hat{\vec{\beta}}] \leq
                \Var[\transpose{\vec{a}} \tilde{\vec{\beta}}]$ for any $\vec{a} \in \R^d$ and
            $\tilde{\vec{\beta}} = \mat{C}\vec{y}$ for $\mat{C} \in \R^{d \times n}$. (OLSE $\hat{\vec{\beta}}$ is unique min.-var.\
            unbiased linear estimator.) This does not mean it is best, because
            adding some bias may decrease variance considerably.
        \end{subtopic}

        \begin{subtopic}{Regularization}
            Ridge: Gaussian prior $\vec{\beta} \sim \mathcal{N}(\vec{0}, \lambda \mat{I})$. LASSO:
            Laplacian Gaussian $\vec{\beta} \sim \mathrm{Lap}(\vec{0}, \lambda \mat{I})$. \\ $\ell_1$
            results in sparse weights (better interpretation) and the sign of features remain.
        \end{subtopic}

        \begin{subtopic}{Polynomial regression}
            Feature map with all polynomials $\phi(\vec{x})$ and perform lin.\ reg.\ in this space: $\psi(\vec{x}; \vec{\beta}) = \transpose{\vec{\beta}} \phi(\vec{x})$.
            Problem: Infinitely dimensional $\Rightarrow$ Ill-defined inner product. Solution: Fix by
            data-dependent scalar, specifically: $\phi(\vec{x}) = \exp(-\nicefrac{\| \vec{x} \|^2}{2}) \lft[ \nicefrac{\prod_{i=1}^d x_i^{\alpha_i}}{\sqrt{\prod_{i=1}^d \alpha_i!}} \rgt]_{\vec{\alpha} \in \mathbb{N}^d}$. \\
            $\Rightarrow$ RBF kernel: $\langle \phi(\vec{x}), \phi(\vec{x}') \rangle = \exp (
                -\nicefrac{\| \vec{x} - \vec{x}' \|^2}{2} )$. Now compute OLSE in this space: $\hat{\vec{\beta}} =
                \inv{(\transpose{\mat{\Phi}} \mat{\Phi})} \transpose{\mat{\Phi}} \vec{y}, \quad \mat{\Phi} \in
                \R^{n \times \infty}$. Problem: Cannot compute $\transpose{\mat{\Phi}} \mat{\Phi} \in \R^{\infty
                    \times \infty}$. Solution: Rewrite OLSE: $\hat{\vec{\beta}} = \transpose{\mat{\Phi}}
                \inv{(\mat{\Phi} \transpose{\mat{\Phi}})} \vec{y}$. Prediction only contains kernel evaluations:
            $\psi(\vec{x}) = \transpose{\vec{k}(\vec{x})} \inv{\mat{K}} \vec{y}$. Problem: $\bigo{n^3}$
            runtime.

        \end{subtopic}

    \end{topic}

    \begin{topic}{Causality}
        Causal fallacies where one might conclude $X$ causes $Y$: (1) Reverse causality: $Y$ causes $X$;
        (2) Third-cause fallacy: $Z$ causes $X$ and $Y$; (3) Bidirectional causation: $X$ causes $Y$ and
        $Y$ causes $X$.

        \textit{Domain shift}: Test samples are drawn from different distribution than training set.

        \textit{Shortcut learning}: Spurious correlation between causal and non-causal features in the training
        depend on environment. \incfig{anti-causal-graph} \incfig{d-separation} Necessary conditions for
        counterfactual invariance: Anti-causal: $f(\mat{X}) \perp \mat{W} \mid Y$. Causal without selection
        (but possibly confounded): $f(\mat{X}) \perp \mat{W}$. Causal without confounded (but possibly
        selection): $f(\mat{X}) \perp \mat{W} \mid Y$ as long as $\mat{X} \perp Y \mid
            \mat{X}_{\mat{W}^{\perp}}, \mat{W}$.

    \end{topic}

    \begin{topic}{Gaussian processes}
        Outputs are modeled as $\vec{y} = \mat{X} \vec{\beta} + \vec{\epsilon}, \vec{\epsilon} \sim
            \mathcal{N}(\vec{0}, \mat{I})$. Thus: $\vec{y} \mid \mat{X}, \vec{\beta} \sim
            \mathcal{N}\lft(\mat{X} \vec{\beta}, \sigma^2 \mat{I}\rgt)$. BLR extends linear reg.\ with prior on
        $\vec{\beta}$: $\vec{\beta} \sim \mathcal{N}\lft(\vec{0}, \inv{\mat{\Lambda}}\rgt)$.

        Posterior: $\vec{\beta} \mid \mat{X}, \vec{y} \sim \mathcal{N}(\tilde{\vec{\mu}},
            \tilde{\mat{\Sigma}})$, where \[
            \tilde{\vec{\mu}} = -\frac{1}{\sigma^2} \tilde{\mat{\Sigma}} \transpose{\mat{X}} \vec{y}, \quad \tilde{\mat{\Sigma}} = \sigma^2 \inv{\lft( \transpose{\mat{X}} \mat{X} + \sigma^2 \mat{\Lambda} \rgt)}.
        \]
        Joint distribution over outputs (using prior): $\vec{y} \mid \mat{X} \sim \mathcal{N}\lft(\vec{0},
            \mat{X}\inv{\mat{\Lambda}} \transpose{\mat{X}} + \sigma^2 \mat{I}\rgt)$. Prediction: $y^\star \mid
            \vec{x}^\star, \mat{X}, \vec{y} \sim \mathcal{N} \lft( \vec{\mu}^\star, \mat{\Sigma}^\star \rgt)$,
        where $\vec{\mu}^\star = \transpose{\vec{k}} \inv{(\mat{K} + \sigma^2 \mat{I})} \vec{y}$ and
        $\mat{\Sigma}^\star = k - \transpose{\vec{k}} \inv{(\mat{K} + \sigma^2 \mat{I})} \vec{k}$. Problem:
        $\bigo{n^3}$ runtime.

        \begin{subtopic}{Kernels}
            Kernel $k$ must satisfy symmetry and PSD: \[
                \int \int f(\vec{x}) k(\vec{x}, \vec{x}') f(\vec{x}') \mathrm{d}\vec{x}\mathrm{d}\vec{x}' \geq 0, \quad \forall f \in L_2.
            \]
            Or there exists $\phi$ s.t.\ $k(\vec{x}, \vec{x}') = \transpose{\phi(\vec{x})} \phi(\vec{x}')$.

            \textit{Linear kernel}: $k(\vec{x}, \vec{x}') = \transpose{\vec{x}} \vec{x}'$;
            \textit{Polynomial kernel}: $k(\vec{x}, \vec{x}') = ( \transpose{\vec{x}} \vec{x}' + 1 )^p$;
            \textit{RBF kernel}: $k(\vec{x},\vec{x}') = \exp ( -\nicefrac{\| \vec{x} - \vec{x}' \|^2}{\ell^2})$;
            \textit{Sigmoid kernel}: $\tanh(\kappa \transpose{\vec{x}} \vec{x}') - b$.

            If $k_1$ and $k_2$ are valid kernels and $c > 0$, then the following are: $k_1 + k_2$, $k_1 \cdot
                k_2$, $c \cdot k_1$, and $\exp \circ k_1$.

        \end{subtopic}

    \end{topic}

    \begin{topic}{Uncertainty quantification}
        \begin{subtopic}{Statistical model validation}
            Methods to evaluate $\hat{f}$ (or algorithm $\mathcal{A}$) that is trained on data $\mathcal{Z}$:
            \textit{Cross-validation}: Partition $\mathcal{Z} = \bigcup_{k=1}^K \mathcal{Z}_k$ and produce $K$
            estimators $\hat{f}^{-k}$ from $\mathcal{Z} \backslash \mathcal{Z}_k$. Then estimate risk by
            $\mathcal{R}^{\mathrm{CV}}(\mathcal{A}) = \frac{1}{n} \sum_{i=1}^{n} \ell(y_i
                \hat{f}^{-k(i)}(\vec{x}_i))$, where $k$ maps $i$ to the partition such that $\vec{x}_i \in
                \mathcal{Z}_{k(i)}$.

            \textit{Bootstrap}: Used for measuring dist.\ over stat.\ params. Draw $B$
            bootstrap samples $\Rightarrow$ Compute parameter for each $\Rightarrow$ Compute statistics. Can
            also use for empirical risk: $\hat{\mathcal{R}}^{\mathrm{BS}}(\mathcal{A}) \doteq \frac{1}{n\cdot
                    B} \sum_{b=1}^{B} \sum_{i=1}^{n} \ell(y_i, \hat{f}^{*b}(\vec{x}_i))$. Problem: Overly optimistic.
            Solution: $\mathcal{R}^{\mathrm{BS}}(\mathcal{A}) \doteq \frac{1}{n} \sum_{i=1}^{n}
                \frac{1}{|\mathcal{C}^{-i}|} \sum_{b \in \mathcal{C}^{-i}} \ell(y_i, \hat{f}^{*b}(\vec{x}_i))$.
            Correct for optimism of $\hat{\mathcal{R}}^{\mathrm{BS}}$ by combining with
            $\mathcal{R}^{\mathrm{BS}}$: $\mathcal{R}^{(0.632)} = 0.368 \hat{\mathcal{R}}^{\mathrm{BS}} + 0.632
                \mathcal{R}^{\mathrm{BS}}$. $0.632$ is the prob.\ that a sample appears at least once in a
            bootstrap sample of size $n$.
        \end{subtopic}

        \begin{subtopic}{Uncertainty in linear models}
            OLSE has distribution over estimators: $\hat{\vec{\beta}} \sim \mathcal{N}(\vec{\beta}^\star, \sigma^2 \inv{(\transpose{\mat{X}} \mat{X})})$.
            Unbiased estimator of $\sigma^2$: $\hat{\sigma}^2 = \frac{1}{n-d} \sum_{i=1}^{n}
                (\transpose{\hat{\vec{\beta}}} \vec{x}_i - y_i)$. Then we have $1-\alpha$ confidence interval for
            $\beta_j^\star$: $\hat{\beta}_j \pm z_{\nicefrac{\alpha}{2}} \epsilon(\hat{\beta}_j)$,
            where $z_{\nicefrac{\alpha}{2}} = \inv{\Phi}(\nicefrac{\alpha}{2})$, $\Phi$ is standard Gaussian
            CDF, and $\epsilon(\hat{\beta}_j) = \hat{\sigma}^2 \inv{(\transpose{\mat{X}} \mat{X})}_{jj}$.

        \end{subtopic}

        \begin{subtopic}{Statistical testing}
            Null hypothesis: $H_0: \theta^\star \in \Theta$. Alternative hypothesis $H_1: \theta^\star \in
                \Theta$. We are given $n$ samples $x_1, \ldots, x_n \sim p(\cdot \mid \theta^\star)$ and a test
            statistic $t: \mathcal{X}^n \to \R$. The goal is to find a critical value $c \in \R$ such that
            $\mathbb{P}(t(X_1,\ldots,X_n) \geq c \mid \theta)$ is low when $\theta \in \Theta_0$ and high when
            $\theta \in \Theta_1$.

            We want to minimize the prob.\ of choosing $H_1$ when $H_0$ holds (worst possible situation). We
            quantify this notion of risk as $\alpha_c \doteq \sup_{\theta \in \Theta_0} \mathbb{P}(t(x_1,
                \ldots, x_n) \geq c \mid \theta)$. Problem: $\alpha_c \to 0$ as $c \to \infty$, so $c^\star \to
                \infty$ minimizes the risk, but then we never accept $H_1$. Solution: Run test on realization
            $t(x_1, \ldots, x_n)$ and compute risk of least risky critical value that would incorrectly reject
            $H_0$: $p = \inf_{c \in \R} \{ \alpha_c \mid t(x_1, \ldots, x_n) \geq c \}$.

            This is the $p$-value: \\ $p \doteq \sup_{\theta \in \Theta_0} \mathbb{P}(t(X_1, \ldots, X_n) \geq
                t(x_1, \ldots, x_n) \mid \theta)$. Intuition: Inverse prob.\ of $x_{1:n}$ being an outlier.

            \textit{Wald}: $W = \nicefrac{(\hat{\theta} - \theta_0)^2}{\hat{\sigma}^2}$;
            $H_0: \theta = \theta_0$, $H_1: \theta \neq \theta_0$.
        \end{subtopic}

        \begin{subtopic}{Bayesian neural networks}
            (S)GD only yields single point estimate of weights $\Rightarrow$ Define prior
            $\vec{\theta} \sim \mathcal{N}(\vec{0}, \sigma^2 \mat{I})$ and likelihood $p(\mathcal{Z}
                \mid \vec{\theta}) = \prod_{\vec{x},y \in \mathcal{Z}} p(y \mid \vec{x}, \vec{\theta})$
            $\Rightarrow$ Posterior with Bayes rule. Problem: $p(\mathcal{Z})$ is intractable.
            Solution: Variational inference with isotropic Gaussians and find
            \begin{align*}
                q^\star & \in \argmin_{\vec{\mu}, \sigma > 0} D_{\mathrm{KL}} (\mathcal{N}(\vec{\mu}, \sigma^2 \mat{I}) \;\|\; p(\vec{\theta} \mid \mathcal{Z})) \\
                        & = \argmin_{\vec{\mu}, \sigma > 0} \E_{\vec{\theta} \sim \mathcal{N}} [F(\vec{\mu}, \sigma, \vec{\theta})],
            \end{align*}
            where $F(\vec{\mu}, \sigma, \vec{\theta}) = \log \mathcal{N}(\vec{\theta}; \vec{\mu}, \sigma^2 \mat{I}) - \log p(\mathcal{Z} \mid \vec{\theta}) - \log p(\vec{\theta})$. Then, we can apply SGD with the following gradients:
            \begin{align*}
                \grad{}{\vec{\mu}} & = \E_{\vec{\epsilon}} [\grad{F(\vec{\mu}, \sigma, \vec{\theta})}{\vec{\theta}} + \grad{F(\vec{\mu}, \sigma, \vec{\theta})}{\vec{\mu}}]                           \\
                \grad{}{\sigma}    & = \E_{\vec{\epsilon}} [\transpose{\vec{\epsilon}} \grad{F(\vec{\mu}, \sigma, \vec{\theta})}{\vec{\theta}}] + \grad{F(\vec{\mu}, \sigma, \vec{\theta})}{\sigma}],
            \end{align*}
            where $\vec{\epsilon} \sim \mathcal{N}(\vec{0}, \mat{I})$ and $\vec{\theta} = \vec{\mu} + \sigma \vec{\epsilon}$.

        \end{subtopic}

        \begin{subtopic}{Information-based transductive learning}
            We are given domain $\mathcal{X}$ that contains safe area $\mathcal{S} \subseteq \mathcal{X}$ and
            area of interest $\mathcal{A} \subseteq \mathcal{X}$. We have an unknown $f^\star$ that we want to
            explore within $\mathcal{A}$, but we can only query (noisy) observations in $\mathcal{S}$: \[
                y_{\vec{x}} = f^\star(\vec{x}) + \epsilon_{\vec{x}}, \quad \E_{\epsilon_{\vec{x}}} = 0.
            \]
            We are given a history of points $\mathcal{D}_{n-1}$ and need to compute which point will give the
            most additional information. ITL selects the next point as: \[
                \vec{x}_n \in \argmax_{\vec{x} \in \mathcal{S}} \mathrm{I}(\vec{f}_{\mathcal{A}}; y_{\vec{x}} \mid \mathcal{D}_{n-1}).
            \]
            If $f \sim \mathrm{GP}(\mu, k)$, then \[
                \mathrm{I}(\vec{f}_{\mathcal{A}}; y_{\vec{x}} \mid \mathcal{D}_{n-1}) = \frac{1}{2} \log \frac{\Var[y_{\vec{x}} \mid \mathcal{D}_{n-1}]}{\Var[y_{\vec{x}} \mid \vec{f}_{\mathcal{A}}, \mathcal{D}_{n-1}]}.
            \]
            \textit{Proof}: Use entropy of Gaussian.

        \end{subtopic}

    \end{topic}

    \begin{topic}{Convex optimization and SVMs}
        \[
            \text{min} \quad f(\vec{x}), \quad \text{s.t.} \quad g_i(\vec{x}) = 0, h_j(\vec{x}) \leq 0.
        \]
        where $f$ and $h_j$ are convex and $g_i$ are affine.

        Lagrangian: $\mathcal{L}(\vec{x}, \vec{\lambda}, \vec{\alpha}) \doteq f(\vec{x}) + \sum_{i=1}^{n}
            \lambda_i g_i(\vec{x}) + \sum_{j=1}^{m} \alpha_j h_j(\vec{x})$. Lagrange dual function:
        $\theta(\vec{\lambda}, \vec{\alpha}) \doteq \inf_{\vec{x} \in \mathcal{X}} \mathcal{L}(\vec{x},
            \vec{\lambda}, \vec{\alpha})$.

        Weak duality: Let $\vec{x} \in \mathcal{C}$, $\vec{\alpha} \geq \vec{0}$, then
        $\theta(\vec{\lambda}, \vec{\alpha}) \leq f(\vec{x})$. Thus: $\max_{\vec{\lambda}, \vec{\alpha}
                \geq \vec{0}} \theta(\vec{\lambda}, \vec{\alpha}) \leq \min_{\vec{x} \in \mathcal{C}} f(\vec{x})$.

        If there is a Slater point (exists $\vec{x} \in \mathcal{C}$ such that $h_j(\vec{x}) < 0$ for all
        $j$) then strong duality: $\max_{\vec{\lambda}, \vec{\alpha} \geq \vec{0}} \theta(\vec{\lambda},
            \vec{\alpha}) = \min_{\vec{x} \in \mathcal{C}} f(\vec{x})$.

        If all $g_i$ and $h_j$ are differentiable, KKT conditions provide necessary (and sufficient for
        convex programming) conditions for strong duality: \[
            \alpha_j^\star h_j(\vec{x}^\star) = 0, \quad \grad{\mathcal{L}(\vec{x}^\star, \vec{\lambda}^\star, \vec{\alpha}^\star)}{\vec{x}} = \vec{0}.
        \]
        Or, condition 2: $\vec{x}^\star \in \argmin_{\vec{x} \in \mathcal{X}} \mathcal{L}(\vec{x},
            \vec{\lambda}^\star, \vec{\alpha}^\star)$.

        \begin{subtopic}{Support vector machine}
            We want to linearly separate a dataset with maximum margin $\Rightarrow$ Model as convex program
            with constraint for each data point: $f[\vec{w},b](\vec{x},y) = y (\transpose{\vec{w}}\vec{x} + b) \geq \epsilon > 0$.

            Margin ($\vec{x}^+$ and $\vec{x}^-$ are support vectors):
            \begin{align*}
                2 \cdot m(\vec{w},b) & = \| \mathrm{proj}_{\vec{w}}(\vec{x}^+) - \mathrm{proj}_{\vec{w}}(\vec{x}^-) \| \\
                                     & = |\transpose{\bar{\vec{w}}} (\vec{x}^+ - \vec{x}^-)|.
            \end{align*}
            Ill-posed problem because infinite number of solutions $\Rightarrow$ Only one solution satisfies \[
                \transpose{\vec{w}} \vec{x}^+ +b = 1, \quad \transpose{\vec{w}} \vec{x}^- + b = -1.
            \]
            Then, $m(\vec{w},b) = \nicefrac{1}{\| \vec{w} \|}$: \[
                \text{min} \quad \frac{1}{2} \| \vec{w} \|^2, \quad \text{s.t.} \quad 1 - y_i \lft( \transpose{\vec{w}}\vec{x}_i + b \rgt) \leq 0.
            \]
            $\vec{w}^\star = \sum_{i=1}^{n} \alpha^\star_i y_i \vec{x}_i$, $b^\star = -\frac{1}{2} \lft( \transpose{\vec{w}_{\star}} \vec{x}^+ + \transpose{\vec{w}_{\star}} \vec{x}^- \rgt)$,
            where $\vec{\alpha}^\star$ is the dual solution.

        \end{subtopic}

        \begin{subtopic}{SVM variations}
            Soft-margin introduces slackness in case data is not linearly separable:
            \begin{align*}
                \text{minimize}   & \quad \frac{1}{2} \| \vec{w} \|^2 + C \sum_{i=1}^{n} \xi_i                                \\
                \text{subject to} & \quad y_i \lft(\transpose{\vec{w}} \vec{x}_i +b \rgt) \geq 1 - \xi_i, \quad \xi_i \geq 0.
            \end{align*}
            Optimal slackness parameters: \[
                \xi_i^\star = \max \lft\{ 0, 1 - y_i \lft(\transpose{\vec{w}_{\star}} \vec{x}_i + b^\star\rgt) \rgt\}.
            \]

            If data is not linearly separable, use features and their kernels: $\transpose{\vec{w}_{\star}}
                \phi(\vec{x}) = \sum_{i=1}^{n} \alpha_i^\star y_i k(\vec{x}_i, \vec{x})$.

            We can generalize the margin notion to multi-class by introducing weights $\vec{w}_z$ per class.
            The margin is defined as the maximum $m \in \R$ s.t.\ \[
                m \leq \lft( \transpose{\vec{w}_{z_i}} \vec{y}_i + b_{z_i} \rgt) - \max_{z \neq z_i} \lft\{ \transpose{\vec{w}_z} \vec{y}_i + b_z \rgt\}.
            \]
            New optimization problem:
            \begin{align*}
                \text{min}  & \quad \frac{1}{2} \| \vec{w} \|^2 = \frac{1}{2} \sum_{z=1}^{k} \| \vec{w}_z \|^2                                                                \\
                \text{s.t.} & \quad \lft( \transpose{\vec{w}_{z_i}} \vec{y}_i + b_{z_i} \rgt) - \max_{z \neq z_i} \lft\{ \transpose{\vec{w}_z} \vec{y}_i + b_z \rgt\} \geq 1.
            \end{align*}

            Structural SVMs can have infinitely many classes. So, we need to define a joint feature map $\psi$
            such that $f_{\vec{w}}(\vec{x}, \vec{y}) = \transpose{\vec{w}} \psi(\vec{x}, \vec{y})$. This is
            used to perform classification: $c(\vec{x}) = \argmax_{\vec{y} \in \mathcal{Y}}
                f_{\vec{w}}(\vec{x}, \vec{y})$.

            We need to construct an algorithm to efficiently compute this argmax and an algorithm to compute
            the max in the below optimization problem. Some structures are closer than others $\Rightarrow$
            Introduce a loss function $\Delta$:
            \begin{align*}
                \text{min} & \quad \frac{1}{2} \| \vec{w} \|^2 \quad \text{s.t.} \quad \transpose{\vec{w}} \psi(\vec{x}_i, \vec{y}_i)                               \\
                           & \quad  - \max_{\vec{y} \neq \vec{y}_i} \lft\{ \transpose{\vec{w}} \psi(\vec{x}_i, \vec{y}) + \Delta(\vec{y}, \vec{y}_i) \rgt\} \geq 0.
            \end{align*}
        \end{subtopic}

    \end{topic}

    \begin{topic}{Ensembles}
        Average $B$ estimators into $\hat{f}$ $\Rightarrow$ avg.\ bias and: \[
            \Var[\hat{f}] = \frac{1}{B^2} \sum_{b=1}^{B} \Var[\hat{f}_b] + \frac{1}{B^2} \sum_{b=1}^{B} \sum_{b' \neq b}^{B} \mathrm{Cov}(\hat{f}_b, \hat{f}_{b'}).
        \]
        If the covariances are low, the variance is significantly decreased while the bias remains the
        same.

        \begin{subtopic}{Bagging}
            $B$ times take a bootstrap sample and train a classifier. This works well because
            covariances are small due to using different subsets for training and the variances are
            similar because each subsample behaves similarly on average.

            Random forests do this with (very deep) decision trees. Very deep because they have low bias and
            high variance, which is reduced by ensembling.

        \end{subtopic}

        \begin{subtopic}{AdaBoost}
            AdaBoost reduces cov.\ by using a different weighting for each estimator. The weights are
            determined by error of prev.\ classifiers.
            \begin{align*}
                w_i^{(b+1)} & = w_i^{(b)} \exp(\alpha_b \mathbb{1}\{ c_b(\vec{x}_i) \neq y_i \})                                   \\
                \alpha_b    & = \log \lft( \nicefrac{1-\epsilon_b}{\epsilon_b} \rgt)                                               \\
                \epsilon_b  & = \sum_{i=1}^{n} \frac{w_i^{(b)}}{\sum_{j=1}^{n} w_j^{(b)}} \mathbb{1}\{ c_b(\vec{x}_i) \neq y_i \}.
            \end{align*}
            Final classifier: $\hat{c}(\vec{x}) = \mathrm{sgn}(\sum_{b=1}^{B} \alpha_b c_b(\vec{x}))$.

            It can be shown that AdaBoost fits an additive model in base learners optimizing the exponential
            loss $\E[\exp(-yf(\vec{x}))]$ via Newton-like updates.

        \end{subtopic}

    \end{topic}

    \begin{topic}{Stable diffusion}

        \begin{subtopic}{Diffusion models}
            Iteratively denoise. Continuous:
            {\footnotesize \begin{align*}
                \mathrm{d}\vec{x}^+_t & = \vec{\mu}(\vec{x}_t, t) \mathrm{d}t + \sigma(\vec{x}_t, t) \mathrm{d}\omega_t                                                                                 \\
                \mathrm{d}\vec{x}^-_t & = \lft[ \vec{\mu}(\vec{x}_t, t) - \sigma^2(\vec{x}_t, t) \grad{\log p_t(\vec{x}_t)}{\vec{x}} \rgt] \mathrm{d}t + \sigma(\vec{x}_t, t) \mathrm{d}\bar{\omega}_t.
            \end{align*}}

            DDPM scheduler: $\vec{x}_{t+1} = \sqrt{1-\beta_t} \vec{x}_t + \sqrt{\beta_t} \vec{\epsilon}$, where
            $\vec{\epsilon} \sim \mathcal{N}(\vec{0}, \mat{I})$. Backward process: $\vec{x}_{t-1} =
                \frac{1}{\sqrt{\alpha_t}} \lft( \vec{x}_t - \frac{1-\alpha_t}{\sqrt{1-\bar{\alpha}_t}}
                \vec{\epsilon}_{\vec{\theta}}(\vec{x}_t, t) \rgt) + \sqrt{\beta_t} \vec{z}$, where $\alpha_t =
                1-\beta_t$ and $\bar{\alpha}_t = \prod_{\tau=1}^t \alpha_t$.

            Diffusion models are trained by sampling $\vec{x}_0 \sim p_0$, $t \sim \mathrm{Unif}(\{ 1, \ldots,
                T \})$, $\vec{\epsilon} \sim \mathcal{N}(\vec{0}, \mat{I})$ and performing gradient step on $\ell =
                \| \vec{\epsilon} - \vec{\epsilon}_{\vec{\theta}}(\vec{x}_t, t) \|^2$.
        \end{subtopic}

    \end{topic}

    \begin{topic}{Non-parametric Bayesian methods}
        Beta distribution ($x \in [0,1]$, $\alpha,\beta > 0$): $\mathrm{Beta}(x; \alpha, \beta) \propto
            x^{\alpha-1} (1-x)^{\beta-1}$. Dirichlet generalizes ($\vec{x} \in \Delta^{n-1}$):
        $\mathrm{Dir}(\vec{x}; \vec{\alpha}) \propto \prod_{k=1}^n x_k^{\alpha_k - 1}$.

        Problem: Need to know $K$ (\#clusters) beforehand.

        A Dirichlet process $\mathrm{DP}(\alpha, H)$ is a distribution over probability distributions on a
        space $\Theta$, where $\alpha$ is a concentration parameter. A sample $G \sim \mathrm{DP}(\alpha,
            H)$ is a function $G: \Theta \to \R_{\geq 0}$ such that $\int_{\Theta} G(\theta)\mathrm{d}\theta =
            1$. For every partition $(T_1, \ldots, T_k)$ of $\Theta$ and $G \sim \mathrm{DP}(\alpha, H)$, we
        have \[
            (G(T_1), ., G(T_k)) \sim \mathrm{Dir}(\alpha H(T_1), ., \alpha H(T_k)).
        \]
        Dir can be sampled recursively by stick-breaking:
        \begin{gather*}
            \beta_i \sim \mathrm{beta}\lft( \alpha_i, \prod_{k=i+1}^K \alpha_k \rgt), \quad \rho_i = \beta_i \prod_{j=1}^{i-1} (1-\beta_j) \\
            (\rho_{i+1}, \ldots, \rho_K) \sim \mathrm{Dir}(\alpha_{i+1}, \ldots, \alpha_K).
        \end{gather*}
        Still limited to fixed $K$. GEM distribution fixes this by fixing $\alpha$ such that
        $\beta_i \sim \mathrm{Beta}(1, \alpha)$ for all $i$. Recursion: $\beta_i \sim \mathrm{Beta}(1, \alpha)$ and \[
            \rho_i = \beta_i \prod_{j=1}^{i-1} (1-\beta_j), \quad \rho_K = \beta_k \lft( 1- \sum_{i=1}^{K-1} \rho_i \rgt).
        \]
        Keep sampling cluster probs until satisfied.

        If $(\rho_1, \rho_2, \ldots) \sim \mathrm{GEM}(\alpha)$ and $\theta_k \sim H$, then this is sample
        from $\mathrm{DP}(\alpha, H)$: $G(\theta) = \sum_{k=1}^{\infty} \rho_k \delta_{\theta_k}(\theta)$.

        \begin{subtopic}{Chinese restaurant process}
            \[
                P(\text{$n+1$ joins table $\theta$} \mid \mathcal{P}) = \begin{cases}
                    \frac{|\theta|}{\alpha+n} & \theta \in P \\
                    \frac{\alpha}{\alpha+n}   & \text{else}.
                \end{cases}
            \]
            Probability of partition $\mathcal{P}$ can be written as \[
                P(\mathcal{P}) = \alpha^{|\mathcal{P}|} \frac{\alpha!}{(N+\alpha)!} \prod_{\tau \in \mathcal{P}} (|\tau| -1)!.
            \]
            Problem is exchangeable. $\E[|\mathcal{P}|] \in \bigo{\alpha \log N}$.

        \end{subtopic}

        \begin{subtopic}{DPMM}
            Assume $\Theta = \R$ with $\mu \in \Theta$ corresponding to $\mathcal{N}(\mu, \sigma)$ for fixed
            $\sigma > 0$ and $H = \mathcal{N}(\mu_0, \sigma_0)$ for fixed $\mu_0,\sigma_0$. DPMM: Cluster probs
            are sampled from GEM: $(\rho_1, \rho_2, \ldots) \sim \mathrm{GEM}(\alpha)$. Cluster centers are
            sampled from base measure: $\mu_1, \mu_2, \ldots \sim \mathcal{N}(\mu_0, \sigma_0)$. Clusters are
            assigned: $z_i \sim \mathrm{Cat}(\rho_1, \rho_2, \ldots), \forall i \in [n]$. Data points are
            sampled: $x_i \sim \mathcal{N}(\mu_{z_i}, \sigma), \forall i \in [n]$. This process is
            exchangeable. To fit a DPMM, we use a collapsed Gibbs sampling formulation: $p(z_i = k \mid
                \vec{z}^{-i}, \vec{x}, \alpha, \vec{\mu}) \propto p(z_i = k \mid \vec{z}^{-i}, \alpha) p(x_i \mid
                \vec{x}^{-i}, z_i=k, \vec{z}^{-i}, \vec{\mu})$. Prior is as in CRP: \[
                p(z_i = k \mid \vec{z}^{-i}, \alpha) = \begin{cases}
                    \frac{N_k^{-i}}{\alpha+N-1} & \text{existing $k$} \\
                    \frac{\alpha}{\alpha+N-1}   & \text{else}.
                \end{cases}
            \]
            Likelihood (right term) is cond.\ on cluster $k$: \[
                \ell = \begin{cases}
                    p(x_i \mid \vec{x}_k^{-i}, \vec{\mu}) = \frac{p(x_i, \vec{x}_k^{-i} \mid \vec{\mu})}{p(\vec{x}_k^{-i} \mid \vec{\mu})} & \text{existing $k$} \\
                    p(x_i \mid \vec{\mu})                                                                                                  & \text{else}.
                \end{cases}
            \]

        \end{subtopic}

    \end{topic}

    \begin{topic}{Statistical learning theory}

        \begin{subtopic}{PAC learning}
            \textit{Definition}: A learning algorithm $\mathcal{A}$ can learn a concept $c \in
                \mathcal{C}$ if there exists $\mathrm{poly}(\cdot, \cdot, \cdot)$ such that for any
            distribution $p$ on $\mathcal{X}$ and $\epsilon,\delta \in (0,\nicefrac{1}{2})$, if
            $\mathcal{A}$ receives a sample of size $n \geq \mathrm{poly}(\nicefrac{1}{\epsilon},
                \nicefrac{1}{\delta}, \mathrm{size}(c))$, then $\mathcal{A}$ outputs $\hat{c}$ such
            that $\mathbb{P}(\mathcal{R}(\hat{c}) \leq \epsilon) \geq 1-\delta$.
            This probability is taken over the randomness of $\mathcal{Z}$ and $\mathcal{A}$.

            $\mathcal{C}$ is PAC learnable from $\mathcal{H}$ if there is an $\mathcal{A}$ that can learn any $c \in \mathcal{C}$.

            If $\mathcal{A}$ runs polynomial in only $\nicefrac{1}{\delta}$ and $\nicefrac{1}{\epsilon}$, then
            $\mathcal{C}$ is efficiently PAC learnable.

            In the stochastic setting, $y$ is also random and not deterministically decided by a concept $c \in
                \mathcal{C}$. Now the criterium is $\mathbb{P}_{\mathcal{Z} \sim p}(\mathcal{R}(\hat{c}) - \inf_{c
                    \in \mathcal{C}} \mathcal{R}(c) \leq \epsilon) \geq 1-\delta$.

        \end{subtopic}

        \begin{subtopic}{Vapnik-Chervonenkis}
            VC dimension is the cardinality of the largest set of points that $\mathcal{C}$ can shatter.

            \textit{Vapnik and Chervonenkis}: Assume a finite concept class and $\mathcal{R}(c^\star) = 0$ and define
            $c_n^\star \in \{ c \in \mathcal{C} \mid \hat{\mathcal{R}}_n(c) = 0 \}$. Then, for every $n \in
                \mathbb{N}$ and $\epsilon > 0$, \[
                \mathbb{P}(\mathcal{R}(\hat{c}^\star_n) > \epsilon) \leq |\mathcal{C}| \exp(-n \epsilon).
            \]
            And: $\E[\mathcal{R}(\hat{c}_n^\star)] \leq \frac{1 + \log |\mathcal{C}|}{n}$.

            \textit{VC inequality}: $\mathbb{P}(\mathcal{R}(\hat{c}_n^\star) - \inf_{c \in \mathcal{C}} \mathcal{R}(c) > \epsilon) \leq \mathbb{P}(\sup_{c \in \mathcal{C}} |\hat{\mathcal{R}}_n(c) - \mathcal{R}(c)| > \frac{\epsilon}{2})$.

            \textit{Hoeffding}: Let $X_i \in [a_i, b_i]$ be i.i.d.\ and $S_n = \sum_{i=1}^{n} X_i$, then for any $t >
                0$, \[
                \mathbb{P}(S_n - \E[S_n] \geq t) \leq \exp \lft( -\frac{2t^2}{\sum_{i=1}^{n} (b_i - a_i)^2} \rgt).
            \]
            Same bound for $\leq -t$.

            As a result: $\mathbb{P}(\tilde{S}_n - \E[\tilde{S}_n] \geq \epsilon) \leq \exp \lft( -\frac{2n
                    \epsilon^2}{\sum_{i=1}^{n} \nicefrac{(b_i - a_i)^2}{n}} \rgt)$, where $\tilde{S}_n =
                \nicefrac{S_n}{n}$.

            Assume $|\mathcal{C}| \leq N$, then for all $\epsilon > 0$, \[
                \mathbb{P}(\sup_{c \in \mathcal{C}} |\hat{\mathcal{R}}_n(c) - \mathcal{R}(c)| > \epsilon) \leq 2N\exp(-2n \epsilon^2).
            \]

            We can deal with infinite $|\mathcal{C}|$ by representing hypotheses by the classifications that
            they yield. Or measuring the VC dimension of $\mathcal{C}$.

        \end{subtopic}

    \end{topic}

\end{multicols*}

\end{document}
