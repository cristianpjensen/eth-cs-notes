\section{Projected gradient descent}

In constrained optimization, we want to solve the following optimization problem, \[
    \argmin_{\vec{x} \in \mathcal{X}} f(\vec{x}).
\]
In the previous section, we considered $\mathcal{X} = \R^d$. However, now we will assume that
$\mathcal{X} \subset \mathbb{R}^d$ is a closed convex set. The idea of projected gradient descent
is to project onto $\mathcal{X}$ after every step,
\begin{align*}
    \vec{y}_{t+1} & = \vec{x}_t - \gamma \grad{f(\vec{x}_t)}{}                                                                   \\
    \vec{x}_{t+1} & = \Pi_{\mathcal{X}}(\vec{y}_{t+1}) \doteq \argmin_{\vec{x} \in \mathcal{X}} \| \vec{x} - \vec{y}_{t+1} \|^2.
\end{align*}

\begin{marginfigure}
    \centering
    \incfig{constrained-optimization}
    \caption{Proof by picture of the properties of the projection step made in projected gradient descent.}
    \label{fig:constrained-optimization}
\end{marginfigure}

\begin{lemma}
    \label{lem:proj}

    Let $\mathcal{X} \subset \R^d$ be closed and convex, $\vec{x}\in \mathcal{X}, \vec{y}\in \R^d$,
    then
    \begin{enumerate}
        \item $\transpose{(\vec{x} - \Pi_{\mathcal{X}}(\vec{y}))} (\vec{y} - \Pi_{\mathcal{X}}(\vec{y})) \leq 0$;
        \item $\| \vec{x} - \Pi_{\mathcal{X}}(\vec{y}) \|^2 + \| \vec{y} - \Pi_{\mathcal{X}}(\vec{y}) \|^2 \leq \|\vec{x} - \vec{y}\|^2$.
    \end{enumerate}
\end{lemma}

\begin{proof}[Proof of 1]
    Let $d_{\vec{y}}(\vec{x}) \doteq \| \vec{x} - \vec{y} \|^2$, which is a differentiable convex
    function with gradient $\grad{d_{\vec{y}}(\vec{x})}{} = 2(\vec{x} - \vec{y})$. $\Pi_{\mathcal{X}}(\vec{y})$ is the minimizer of this function over $\mathcal{X}$.
    Thus, by the first-order optimality condition (\Cref{lem:optim}), we have
    \begin{align*}
         &      & \transpose{\grad{d_{\vec{y}}(\Pi_{\mathcal{X}}(\vec{y}))}{}} (\vec{x} - \Pi_{\mathcal{X}}(\vec{y})) & \geq 0 \\
         & \iff & 2\transpose{(\Pi_{\mathcal{X}}(\vec{y}) - \vec{y})}(\vec{x} - \Pi_{\mathcal{X}}(\vec{y}))           & \geq 0 \\
         & \iff & \transpose{(\vec{x} - \Pi_{\mathcal{X}}(\vec{y}))}(\vec{y} - \Pi_{\mathcal{X}}(\vec{y})) \leq 0.
    \end{align*}
\end{proof}

\begin{proof}[Proof of 2]
    This can easily be shown using the cosine theorem and the previous result,
    \begin{align*}
         &      & 2 \transpose{(\vec{x} - \Pi_{\mathcal{X}}(\vec{y}))}(\vec{y} - \Pi_{\mathcal{X}}(\vec{y}))                                                                & \leq 0 \margintag{Multiply both sides of (1) by 2.} \\
         & \iff & \| \vec{x} - \Pi_{\mathcal{X}}(\vec{y}) \|^2 + \| \vec{y} - \Pi_{\mathcal{X}}(\vec{y}) \| - \| \vec{x} - \vec{y} \|^2 \leq 0. \margintag{Cosine theorem.}
    \end{align*}
    The result follows by rearranging.
\end{proof}

The two properties equivalently say that the vectors $\vec{y} - \Pi_{\mathcal{X}}(\vec{y})$ and
$\vec{y} - \Pi_{\mathcal{X}}(\vec{y})$ form an obtuse angle---see
\Cref{fig:constrained-optimization}.

\begin{lemma}
    Let $f: \dom{f} \to \R$ be a differentiable convex function and $\mathcal{X} \subseteq \dom{f}$ a
    closed convex set. Then, during projected gradient descent of $f$ over $\mathcal{X}$, if
    $\vec{x}_{t+1} = \vec{x}_t$, we have an optimal solution.
\end{lemma}

\begin{proof}
    We assume $\vec{x}_{t+1} = \vec{x}_t$. Let $\vec{x} \in \mathcal{X}$, then, by \Cref{lem:proj}
    (1), we have
    \begin{align*}
         &      & \transpose{(\vec{x} - \vec{x}_t)}(\vec{y}_{t+1} - \vec{x}_t)     & \leq 0 \\
         & \iff & \transpose{(\vec{x} - \vec{x}_t)}(-\gamma \grad{f(\vec{x}_t)}{}) & \leq 0 \\
         & \iff & \transpose{\grad{f(\vec{x}_t)}{}} (\vec{x} - \vec{x}_t) \geq 0,
    \end{align*}
    which satisfies the first-order optimality condition (\Cref{lem:optim}) at $\vec{x}_t$.
\end{proof}

% \subsection{Lipschitz convex functions}
%
% Since we minimize $f$ over a closed and bounded convex set $\mathcal{X}$, we get the existence of a
% minimizer and bound $R$ for free. Furthermore, if we assume that $f$ is continuously
% differentiable, we also have a bound $B$ for the gradient norms over $\mathcal{X}$. This makes the
% following theorem a much more useful result than in the unconstrained case, since we need to make
% less assumptions about $f$.
%
% \begin{theorem}
%     Let $f: \dom{f} \to \R$ be convex and differentiable, $\mathcal{X} \subseteq \dom{f}$ closed and
%     convex, $\vec{x}^\star$ a minimizer of $f$ over $\mathcal{X}$. Furthermore, suppose that
%     $\|\vec{x}_0-\vec{x}^\star\| \leq R$ and $\|\grad{f(\vec{x})}{}\| \leq B$ for all
%     $\vec{x} \in \mathcal{X}$. Choosing the constant stepsize, \[
%         \gamma \doteq \frac{R}{B\sqrt{T}},
%     \]
%     with projected gradient descent yields the following bound, \[
%         \frac{1}{T} \sum_{t=0}^{T-1}  (f(\vec{x}_t) - f(\vec{x}^\star)) \leq \frac{RB}{\sqrt{T}}.
%     \]
% \end{theorem}
%
% \begin{proof}
%     We only need to show that the vanilla analysis still holds for the projected $\vec{x}_{t+1}$. We
%     know that for the non-projected iterate, we get the following equality from the vanilla analysis, \[
%         \transpose{\grad{f(\vec{x}_t)}{}} (\vec{x}_t - \vec{x}^\star) = \frac{1}{2 \gamma} \lft( \gamma^2 \| \grad{f(\vec{x}_t)}{} \|^2 + \| \vec{x}_t - \vec{x}^\star \|^2 - \| \vec{y}_{t+1} - \vec{x}^\star \|^2 \rgt).
%     \]
%     By the second property of projected gradient descent, we know $\| \vec{x}_{t+1} - \vec{x}^\star
%         \|^2 \leq \| \vec{y}_{t+1} - \vec{x}^\star \|^2$, hence we get \[
%         \transpose{\grad{f(\vec{x}_t)}{}} (\vec{x}_t - \vec{x}^\star) \leq \frac{1}{2 \gamma} \lft( \gamma^2 \| \grad{f(\vec{x}_t)}{} \|^2 + \| \vec{x}_t - \vec{x}^\star \|^2 - \| \vec{x}_{t+1} - \vec{x}^\star \|^2 \rgt).
%     \]
%     Then, we can prove the rest of the vanilla analysis as for unbounded gradient descent.
% \end{proof}
%
% This is the same bound as in the unconstrained case, thus the number of necessary iteration is of
% the order $\bigo{\nicefrac{1}{\epsilon^2}}$.

\subsection{Smooth functions}

\begin{lemma}[Projected sufficient decrease]
    Let $f: \dom{f} \to \R$ be differentiable and smooth with parameter $L$ over a closed and convex
    set $\mathcal{X} \subseteq \dom{f}$. Choosing stepsize \[
        \gamma \doteq \frac{1}{L},
    \]
    projected gradient descent satisfies \[
        f(\vec{x}_{t+1}) \leq f(\vec{x}_t) - \frac{1}{2L} \| \grad{f(\vec{x}_t)}{} \|^2 + \frac{L}{2} \| \vec{y}_{t+1} - \vec{x}_{t+1} \|^2.
    \]
\end{lemma}

\begin{proof}
    \begin{align*}
        f(\vec{x}_{t+1}) & \leq f(\vec{x}_t) + \transpose{\grad{f(\vec{x}_t)}{}} (\vec{x}_{t+1} - \vec{x}_t) + \frac{L}{2} \| \vec{x}_t - \vec{x}_{t+1} \|^2 \margintag{Smoothness.}                                                                       \\
                         & = f(\vec{x}_t) - L \transpose{(\vec{y}_{t+1} - \vec{x}_t)}(\vec{x}_{t+1} - \vec{x}_t) + \frac{L}{2} \| \vec{x}_t - \vec{x}_{t+1} \|^2 \margintag{Update rule: $\vec{y}_{t+1} = \vec{x}_t - \frac{1}{L} \grad{f(\vec{x}_t)}{}$.} \\
                         & = f(\vec{x}_t) - \frac{L}{2} \lft( \| \vec{y}_{t+1} - \vec{x}_t \|^2 + \| \vec{x}_{t+1} - \vec{x}_t \|^2 - \| \vec{y}_{t+1} - \vec{x}_{t+1} \| \rgt) \margintag{Cosine theorem.}                                                \\
                         & \qquad + \frac{L}{2} \| \vec{x}_t - \vec{x}_{t+1} \|^2                                                                                                                                                                          \\
                         & = f(\vec{x}_t) - \frac{L}{2} \| \vec{y}_{t+1} - \vec{x}_t \|^2 + \frac{L}{2} \| \vec{y}_{t+1} - \vec{x}_{t+1} \|^2                                                                                                              \\
                         & = f(\vec{x}_t) - \frac{1}{2L} \| \grad{f(\vec{x}_t)}{} \|^2 + \frac{L}{2} \|\vec{y}_{t+1} - \vec{x}_{t+1}\|^2. \margintag{$\vec{y}_{t+1} = \vec{x}_t - \frac{1}{L} \grad{f(\vec{x}_t)}{}$.}
    \end{align*}
\end{proof}

Thus, we also have a sufficient decrease lemma for this version of gradient descent, which has an
additional term in its bound. However, as we will see, this does not matter, because we can
compensate for it in the vanilla analysis.

\begin{theorem}
    Let $f: \dom{f} \to \R$ be convex and differentiable. Let $\mathcal{X} \subseteq \dom{f}$ be a closed convex set, and $\vec{x}^\star$ the minimizer of $f$ over $\mathcal{X}$. Furthermore, suppose that $f$ is smooth over $\mathcal{X}$ with parameter $L$. Choosing stepsize \[
        \gamma \doteq \frac{1}{L},
    \]
    projected gradient descent yields the following bound, \[
        f(\vec{x}_T) - f(\vec{x}^\star) \leq \frac{L}{2T} \|\vec{x}_0 - \vec{x}^\star\|^2.
    \]
\end{theorem}

\begin{proof}
    From \Cref{lem:proj} (2), we get the following inequality, \[
        \| \vec{x}_{t+1} - \vec{x}^\star \|^2 + \| \vec{y}_{t+1} - \vec{x}_{t+1} \|^2 \leq \| \vec{y}_{t+1} - \vec{x}^\star \|^2.
    \]
    Using this inequality, we get the following upper bound,
    \begin{align*}
        f(\vec{x}_t) - f(\vec{x}^\star) & \leq \transpose{\grad{f(\vec{x}_t)}{}} (\vec{x}_t - \vec{x}^\star) \margintag{Convexity.}                                                                                                                                                                                              \\
                                        & = \frac{1}{2 \gamma} \lft( \gamma^2 \| \grad{f(\vec{x}_t)}{} \|^2 + \|\vec{x}_t - \vec{x}^\star\|^2 - \|\vec{y}_{t+1}-\vec{x}^\star\|^2 \rgt) \margintag{See vanilla analysis, where $\vec{x}_{t+1}$ is substituted by $\vec{y}_{t+1}$, since that is the next unconstrained iterate.} \\
                                        & \leq \frac{1}{2 \gamma} \Big( \gamma^2 \| \grad{f(\vec{x}_t)}{} \|^2 + \|\vec{x}_t - \vec{x}^\star\|^2 - \|\vec{x}_{t+1} - \vec{x}^\star\|^2                                                                                                                                           \\
                                        & \qquad - \|\vec{y}_{t+1}-\vec{x}_{t+1}\|^2 \Big).
    \end{align*}
    We use projected sufficient decrease to bound the sum of gradients,
    \begin{align*}
        \frac{1}{2L} \sum_{t=0}^{T-1} \|\grad{f(\vec{x}_t)}{}\|^2 & \leq \sum_{t=0}^{T-1} f(\vec{x}_t) - f(\vec{x}_{t+1}) + \frac{L}{2} \| \vec{y}_{t+1} - \vec{x}_{t+1} \|^2 \margintag{Projected sufficient decrease.} \\
                                                                  & = f(\vec{x}_0) - f(\vec{x}_T) + \frac{L}{2} \sum_{t=0}^{T-1} \|\vec{y}_{t+1} - \vec{x}_{t+1}\|^2. \margintag{Telescoping sum.}
    \end{align*}
    Now, we can bound the summed error and we will see that the additional terms cancel,
    \begin{align*}
        \sum_{t=0}^{T-1} (f(\vec{x}_t) - f(\vec{x}^\star)) & \leq \frac{1}{2L} \sum_{t=0}^{T-1} \| \grad{f(\vec{x}_t)}{} \|^2 + \frac{L}{2} \|\vec{x}_0 - \vec{x}^\star\|^2          \\
                                                           & \qquad - \frac{L}{2} \sum_{t=0}^{T-1} \| \vec{y}_{t+1} - \vec{x}_{t+1} \|^2                                             \\
                                                           & \leq f(\vec{x}_0) - f(\vec{x}_T) + \frac{L}{2} \sum_{t=0}^{T-1} \| \vec{y}_{t+1} - \vec{x}_{t+1} \|^2                   \\
                                                           & \qquad + \frac{L}{2} \|\vec{x}_0 - \vec{x}^\star\|^2 - \frac{L}{2} \sum_{t=0}^{T-1} \|\vec{y}_{t+1} - \vec{x}_{t+1}\|^2 \\
                                                           & = f(\vec{x}_0) - f(\vec{x}_T) + \frac{L}{2} \|\vec{x}_0 - \vec{x}^\star\|^2,
    \end{align*}
    which results in \[
        \sum_{t=1}^{T} (f(\vec{x}_t) - f(\vec{x}^\star)) \leq \frac{L}{2} \|\vec{x}_0 - \vec{x}^\star\|^2.
    \]
    Due to sufficient decrease, we get
    \begin{align*}
        f(\vec{x}_T) - f(\vec{x}^\star) \leq \frac{L}{2T} \|\vec{x}_0 - \vec{x}^\star\|^2.
    \end{align*}
\end{proof}

This is the same bound as in the unconstrained case, thus the number of necessary iterations is of
the order $\bigo{\nicefrac{1}{\epsilon}}$. This does not take into account the time it takes to
compute the projection onto $\mathcal{X}$. However, for many problems, this can be computed
efficiently.

% \subsection{Smooth and strongly convex functions}
%
% \begin{theorem}
%     Let $f: \dom{f} \to \R$ be convex and differentiable. Let $\mathcal{X} \subseteq \dom{f}$ be a
%     nonempty closed and convex set and suppose that $f$ is smooth over $\mathcal{X}$ with parameter
%     $L$ and strongly convex over $\mathcal{X}$ with parameter $\mu > 0$. Choosing stepsize \[
%         \gamma \doteq \frac{1}{L},
%     \]
%     projected gradient descent satisfies the following two properties,
%     \begin{enumerate}
%         \item Squared distance to $\vec{x}^\star$ are geometrically decreasing, \[
%                   \| \vec{x}_{t+1} - \vec{x}^\star \|^2 \leq \lft( 1 - \frac{\mu}{L} \rgt) \|\vec{x}_t - \vec{x}^\star\|^2;
%               \]
%         \item The absolute error after $T$ iterations is exponentially small in $T$,
%               \begin{align*}
%                   f(\vec{x}_T) - f(\vec{x}^\star) & \leq \frac{L}{2} \lft( 1 - \frac{\mu}{L} \rgt)^T \| \vec{x}_0 - \vec{x}^\star \|^2                                                                          \\
%                                                   & \qquad + \underbrace{\|\grad{f(\vec{x}^\star)}{}\| \lft( 1 - \frac{\mu}{L} \rgt)^{\nicefrac{T}{2}} \|\vec{x}_0 - \vec{x}^\star\|}_{\text{additional term}}.
%               \end{align*}
%     \end{enumerate}
% \end{theorem}
%
% \begin{proof}
%     These proofs are similar to the normal gradient descent case, starting from the constrained vanilla bound, \[
%         \frac{1}{2 \gamma} \lft( \gamma^2 \| \grad{f(\vec{x}_t)}{} \|^2 + \|\vec{x}_t - \vec{x}^\star\|^2 - \|\vec{x}_{t+1}-\vec{x}^\star\|^2 - \|\vec{y}_{t+1}-\vec{x}_{t+1}\|^2 \rgt),
%     \]
%     which can be strengthened to \[
%         \frac{1}{2 \gamma} \lft( \gamma^2 \| \grad{f(\vec{x}_t)}{} \|^2 + \|\vec{x}_t - \vec{x}^\star\|^2 - \|\vec{x}_{t+1}-\vec{x}^\star\|^2 - \|\vec{y}_{t+1}-\vec{x}_{t+1}\|^2 \rgt) - \frac{\mu}{2} \|\vec{x}_t - \vec{x}^\star\|^2,
%     \]
%     on $f(\vec{x}_t) - f(\vec{x}^\star)$.
% \end{proof}
