\documentclass[justified,nobib]{tufte-handout}

\setcounter{secnumdepth}{2}
\setcounter{tocdepth}{2}

% Packages
\usepackage[utf8]{inputenc}
\usepackage[T1]{fontenc}
\usepackage{tikz}
\usetikzlibrary{calc,arrows,shapes,positioning}
\usepackage{xparse}
\usepackage{xcolor}
\usepackage{xspace}
\usepackage{amsmath,amsfonts,amsthm,amssymb,mathrsfs,bbm,mathtools,nicefrac,bm}
\usepackage{enumitem}
\usepackage{hyperref}
\usepackage[font+=footnotesize,justification=justified,singlelinecheck=false,labelfont=bf,labelsep=period]{caption}
\usepackage[justification=c]{subcaption}
\usepackage{float}
\usepackage{longtable}
\usepackage{derivative}
\usepackage[most]{tcolorbox}
\usepackage[parfill]{parskip}
\usepackage{graphicx}
\usepackage{cleveref}
\usepackage{booktabs}
\usepackage{mdframed}
\usepackage{algpseudocode}

% Figure support as explained in Gilles Castel's blog post
\usepackage{import}
\usepackage{xifthen}
\usepackage{pdfpages}
\usepackage{transparent}
\newcommand{\incfig}[1]{%
    \def\svgwidth{\columnwidth}
    \import{./figures/}{#1.pdf_tex}
}

% Fix some stuff
% http://tex.stackexchange.com/questions/76273/multiple-pdfs-with-page-group-included-in-a-single-page-warning
\pdfsuppresswarningpagegroup=1

% Example environment
\newcounter{example}
\newcommand\examplename{Example}
\makeatletter
\newcommand\listofexamples{%
    \section*{List of examples}%
    \@starttoc{loe}%
}
\renewcommand\theexample
     {\@arabic\c@example}
\def\fps@example{tbp}
\def\ftype@example{1}
\def\ext@example{loe}
\def\fnum@example{\examplename\nobreakspace\theexample}
\newenvironment{example}[1][htbp]
  {\begin{@tufte@float}[#1]{example}{}
    \begin{mdframed}[backgroundcolor=black!5,rightline=false,leftline=false]}
    {\end{mdframed}
  \end{@tufte@float}}
\let\l@example\l@figure
\makeatother

\theoremstyle{plain}
\newtheorem{theorem}{Theorem}

\theoremstyle{definition}
\newtheorem{definition}{Definition}

\theoremstyle{remark}
\newtheorem{remark}{Remark}

\theoremstyle{definition}
\newtheorem{properties}{Properties}

% Listing environment
\newcounter{listing}
\newcommand\listingname{Listing}
\makeatletter
\newcommand\listoflistings{%
    \section*{List of listings}%
%  \begin{fullwidth}%
    \@starttoc{loe}%
%  \end{fullwidth}%
}
\renewcommand\thelisting
     {\@arabic\c@listing}
\def\fps@listing{tbp}
\def\ftype@listing{1}
\def\ext@listing{loe}
\def\fnum@listing{\listingname\nobreakspace\thelisting}
\newenvironment{listing}[1][htbp]
  {\begin{@tufte@float}[#1]{listing}{}
    \begin{mdframed}[backgroundcolor=black!5,rightline=false,leftline=false]}
    {\end{mdframed}
  \end{@tufte@float}}
\let\l@listing\l@figure
\makeatother

% Algorithm environment
\newcounter{algorithm}
\newcommand\algorithmname{Algorithm}
\makeatletter
\newcommand\listofalgorithms{%
    \section*{List of algorithms}%
    \@starttoc{loe}%
}
\renewcommand\thealgorithm
     {\@arabic\c@algorithm}
\def\fps@algorithm{tbp}
\def\ftype@algorithm{1}
\def\ext@algorithm{loe}
\def\fnum@algorithm{\algorithmname\nobreakspace\thealgorithm}
\newenvironment{algorithm}[1][htbp]
  {\begin{@tufte@float}[#1]{algorithm}{}
    \begin{mdframed}[backgroundcolor=white,rightline=false,leftline=false]}
    {\end{mdframed}
  \end{@tufte@float}}
\let\l@algorithm\l@figure
\makeatother

\algdef{SE}[REPEATN]{RepeatN}{End}[1]{\algorithmicrepeat\ #1 \textbf{times}}{\algorithmicend}

% Obscure delimiters
\DeclarePairedDelimiter{\floor}{\lfloor}{\rfloor}
\DeclarePairedDelimiter{\ceil}{\lceil}{\rceil}
\DeclarePairedDelimiter{\ip}{\langle}{\rangle}
\DeclarePairedDelimiter{\abs}{\lvert}{\rvert}
\DeclarePairedDelimiter{\norm}{\lVert}{\rVert}

% Better left and right
\newcommand{\lft}{\mathopen{}\left}
\newcommand{\rgt}{\aftergroup\mathclose\aftergroup{\aftergroup}\right}

% Common sets
\newcommand{\R}{\mathbb{R}}
\newcommand{\Q}{\mathbb{Q}}
\newcommand{\N}{\mathbb{N}}
\newcommand{\Z}{\mathbb{Z}}

% Expected value and variance
\newcommand{\E}{\mathbb{E}}
\newcommand{\Var}{\mathrm{Var}}

% Input and output space
\newcommand{\X}{\mathcal{X}}
\newcommand{\Y}{\mathcal{Y}}

% Vector, matrix, and tensor
\renewcommand{\vec}[1]{\bm{#1}}
\newcommand{\mat}[1]{\bm{#1}}
\newcommand{\tens}[1]{\bm{\mathsf{#1}}}

% Big-O notation
\newcommand{\bigo}[1]{\mathcal{O}\lft(#1\rgt)}

% Common functions
\newcommand{\transpose}[1]{#1^\top}
\newcommand{\inv}[1]{#1^{-1}}
\renewcommand{\det}[1]{\mathrm{det}\lft(#1\rgt)}
\newcommand{\tr}[1]{\mathrm{tr}\lft(#1\rgt)}
\newcommand{\diag}[1]{\mathrm{diag}\lft(#1\rgt)}

% Distributions
\newcommand{\sample}{\sim}
\newcommand{\sampleiid}{\overset{\mathrm{iid}}{\sim}}
\NewDocumentCommand{\Gauss}{O{0}O{1}o}{\mathcal{N}\lft(#1,#2 \IfValueT{#3}{\mid #3}\rgt)}
\newcommand{\MultiGauss}{\Gauss[\vec{0}][\mat{I}]}
\NewDocumentCommand{\Unif}{m}{\mathrm{Unif}\parentheses*{#1}}

% Math comment in smaller font size
\newcommand{\mathcomment}[1]{\mathrm{\footnotesize #1}}

% Gradient and Hessian operator
\newcommand{\grad}[2]{\vec{\nabla}_{\! #2} #1}
\newcommand{\hess}[2]{\vec{\nabla}^2_{\! #2} #1}
\newcommand{\jacob}[2]{\vec{D}_{#2} #1}

% List of symbols
\newlist{abbrv}{itemize}{1}
\setlist[abbrv,1]{label=,labelwidth=1in,align=parleft,itemsep=0.1\baselineskip,leftmargin=!}

% Add a period to the end of an abbreviation unless there's one
% already, then \xspace
\makeatletter
\DeclareRobustCommand\onedot{\futurelet\@let@token\@onedot}
\def\@onedot{\ifx\@let@token.\else.\null\fi\xspace}

\def\eg{\emph{e.g}\onedot} \def\Eg{\emph{E.g}\onedot}
\def\ie{\emph{i.e}\onedot} \def\Ie{\emph{I.e}\onedot}
\def\cf{\emph{c.f}\onedot} \def\Cf{\emph{C.f}\onedot}
\def\etc{\emph{etc}\onedot} \def\vs{\emph{vs}\onedot}
\def\wrt{w.r.t\onedot} \def\dof{d.o.f\onedot}
\makeatother

% Argument optima
\DeclareMathOperator*{\argmax}{argmax} % no space, limits underneath in displays
\DeclareMathOperator*{\argmin}{argmin} % no space, limits underneath in displays

% Margin tags
% \usepackage{marginfix}
\let\marginnotetemp\marginnote
\let\marginnote\relax
\usepackage{marginnote}
\let\marginnotepkg\marginnote

\NewDocumentCommand{\margintag}{O{0\baselineskip}m}{%
    \checkoddpage%
    \ifoddpage%
        {\marginnotepkg{\footnotesize #2}[#1]}%
    \else%
        {\reversemarginpar\marginnotepkg{\footnotesize #2}[#1]}%
    \fi}%
% \NewDocumentCommand{\margintagt}{O{0\baselineskip}m}{\marginpar{\vspace{#1}\footnotesize #2}}
% \NewDocumentCommand{\safefootnote}{om}{\footnotemark\margintag[#1]{\textsuperscript{\tiny\arabic{footnote}} \normalfont\justifying #2}}

\renewcommand\marginnote\marginnotetemp



\newcommand{\ang}[1]{\langle #1 \rangle}

\title{Computational Intelligence Lab: Linear Algebra Recap}
\author{Cristian Perez Jensen}

\begin{document}

\pagenumbering{roman}
\maketitle
\newpage
\microtypesetup{protrusion=false}
\tableofcontents
\microtypesetup{protrusion=true}
\newpage
\section*{List of symbols}

\vspace{10mm}

\begin{abbrv}
    \item[$\doteq$] Equality by definition
    \item[$\condeq$] Conditional equality
    \item[$\approx$] Approximate equality
    \item[$\propto$] Proportional to
    \item[$\N$] Set of natural numbers
    \item[$\R$] Set of real numbers
    \item[$i:j$] Set of natural numbers between $i$ and $j$. \Ie, $\{i, i+1,\ldots,j\}$
    \item[$f: A \to B$] Function $f$ that maps elements of set $A$ to elements of set $B$
    \item[$\mathbb{1}\{\mathrm{predicate}\}$] Indicator function ($1$ if $\mathrm{predicate}$ is true, otherwise $0$)

    \item

    \item[$\vec{v} \in \R^n$] $n$-dimensional vector
    \item[$\mat{M} \in \R^{m\times n}$] $m \times n$ matrix
    \item[$\transpose{\mat{M}}$] Transpose of matrix $\mat{M}$
    \item[$\inv{\mat{M}}$] Inverse of matrix $\mat{M}$
    \item[$\det{\mat{M}}$] Determinant of $\mat{M}$

    \item

    \item[$\odv*{f(x)}{x}$] Ordinary derivative of $f(x)$ \wrt $x$ at point $x\in\R$
    \item[$\pdv*{f(\vec{x})}{x}$] Partial derivative of $f(\vec{x})$ \wrt $x$ at point $\vec{x}\in\R^n$
    \item[$\grad{f(\vec{x})}{\vec{x}} \in \R^n$] Gradient of $f: \R^n \to \R$ at point $\vec{x}\in\R^n$
    \item[$\hess{f(\vec{x})}{\vec{x}} \in \R^{n\times n}$] Hessian of $f: \R^n \to \R$ at point $\vec{x}\in\R^n$
\end{abbrv}

\newpage\cleardoublepage\pagenumbering{arabic}

\section{Inner product and norms}

\begin{definition}[Inner product]
    An inner product $\ang{\cdot,\cdot}: \mathcal{V} \times \mathcal{V} \to \R$ is an operation
    defined on a vector space $\mathcal{V}$ that satisfies the following properties $\forall \vec{x},\vec{y},\vec{z},a,b\in \R$,
    \begin{itemize}
        \item Commutativity, \[
                  \ang{\vec{x},\vec{y}} = \ang{\vec{y},\vec{x}};
              \]
        \item Linearity, \[
                  \ang{\vec{x},a \vec{y} + b \vec{z}} = a \ang{\vec{x},\vec{y}} + b\ang{\vec{x},\vec{z}};
              \]
        \item Positive definiteness,
              \begin{align*}
                  \vec{x} \neq \vec{0} & \implies \ang{\vec{x},\vec{x}} > 0 \\
                  \vec{x} = 0          & \iff \ang{\vec{x},\vec{x}} = 0;
              \end{align*}
        \item Bilinearity (follows from commutativity and linearity),
              \begin{align*}
                  \ang{\vec{x} + \vec{y}, \vec{z}} & = \ang{\vec{x},\vec{z}} + \ang{\vec{y},\vec{z}}  \\
                  \ang{\vec{x},\vec{y} + \vec{z}}  & = \ang{\vec{x},\vec{y}} + \ang{\vec{x},\vec{z}}.
              \end{align*}
    \end{itemize}
\end{definition}

\begin{corollary}
    $\ang{\vec{x}+\vec{y},\vec{x}+\vec{y}} = \ang{\vec{x},\vec{x}} + 2\ang{\vec{x},\vec{y}} + \ang{\vec{y},\vec{y}}$.
\end{corollary}

\begin{corollary}
    $\ang{\mat{A}\vec{x}, \vec{y}} = \ang{\vec{x}, \transpose{\mat{A}} \vec{y}}$.
\end{corollary}

The vector space $\mathcal{V}$, along with an inner product, defines an inner vector space. During
this course, we will assume that we always work with real vectors in Rn. An example of an inner
product is the dot product,\sidenote{Usually, this operation is what is meant by the inner
    product.} \[
    \vec{x}\cdot \vec{y} = \transpose{\vec{x}} \vec{y} \in \mathbb{R}.
\]

\begin{definition}[Norm]
    A norm $\|\cdot\|: \R^n \to \R$ is a function that can be thought of as a way of measuring the distance from the origin.
    Norms satisfy the following properties,
    \begin{itemize}
        \item Positive definiteness, $\vec{x} \neq \vec{0} \implies \|\vec{x}\| > 0$;
        \item Triangle inequality, $\|\vec{x}+\vec{y}\| \leq \|\vec{x}\| + \|\vec{y}\|$;
        \item Cauchy-Schwarz inequality, $|\ang{\vec{x},\vec{y}}| \leq \|\vec{x}\| \|\vec{y}\|$.
    \end{itemize}
\end{definition}

\begin{corollary}
    For the Euclidean norm, the following holds, \[
        \cos \theta = \frac{\ang{\vec{x},\vec{y}}}{\|\vec{x}\| \|\vec{y}\|},
    \]
    where $\theta$ is the angle between $x$ and $y$.
\end{corollary}

Each inner product defines a canonical norm $\|\vec{x}\| \doteq \sqrt{\ang{\vec{x},\vec{x}}}$. For
example, the Euclidean norm is defined by the dot product, \[
    \|\vec{x}\|_2 = \sqrt{\transpose{\vec{x}}\vec{x}} = \sqrt{\sum_{i=1}^{n} x_i^2}.
\]
The $p$-norm is a generalization of the Euclidean norm, \[
    \|\vec{x}\|_p = \sqrt{\sum_{i=1}^{n} |x_i|^p}.
\]

\section{Vector spaces}

The \textit{vector space} $\R^m$ consists of all column vectors with $m$ elements. For a set of
vectors $\mathcal{C} = \{ \vec{c}_1,\ldots,\vec{c}_n \mid \vec{c}_i \in \R^m \}$, we can define a
subspace spanned by this set, denoted by $\mathrm{span}(\mathcal{C})$. It is the set of all
possible linear combinations of elements of $\mathcal{C}$. If a set of vectors that span a subspace
are independent, they are called a \textit{basis}, $\vec{b}_1,\ldots,\vec{b}_k \in \R^m$. The
number of basis vectors defines the \textit{dimensionality} of the subspace.\sidenote{We know that
    the amount of basis vectors must be smaller than the amount of vectors that span the subspace,
    which must be smaller than the dimensionality of the space, \[
        k \leq m \leq n.
    \]}

\begin{observation}
    The following facts hold about subspaces,
    \begin{itemize}
        \item Every subspace contains the zero vector $\vec{0}$;
        \item If $\vec{x}$ and $\vec{y}$ are in the subspace, then $\vec{x}+\vec{y}$ is also in the subspace;
        \item If $\vec{x}$ is in the subspace and $a\in \R$, then $a\vec{x}$ is also in the subspace.
    \end{itemize}
\end{observation}

\begin{definition}[Orthogonal subspaces]
    Subspaces $\mathcal{V}$ and $\mathcal{W}$ are orthogonal when $\transpose{\vec{v}}\vec{w} = 0$ for
    all $\vec{v}\in \mathcal{V}, \vec{w}\in \mathcal{W}$.
\end{definition}

\section{Matrices}

The rank $r$ of a matrix $\mat{A}\in \R^{m\times n}$ is the dimensionality of its \textit{column
    space}. It is bounded by \[
    r \leq \min \{ m,n \}. \margintag{The matrix is full-rank if $r=\min \{ m,n \}$.}
\]

A matrix $\mat{A} \in \R^{m\times n}$ defines 4 fundamental subspaces,
\begin{itemize}
    \item Column space $\subseteq \R^m$ ($r$ dimensional), $\{ \vec{b} \mid \mat{A}\vec{x} = \vec{b} \}$;
    \item Null space $\subseteq \R^n$ ($n-r$ dimensional), $\{ \vec{x} \mid \mat{A}\vec{x} = \vec{0} \}$;
    \item Row space $\subseteq \R^n$ ($r$ dimensional), $\{ \vec{b} \mid \transpose{\mat{A}}\vec{x} = \vec{b}
              \}$;
    \item Left null space $\subseteq \R^m$ ($m-r$ dimensional), $\{ \vec{x} \mid \transpose{\mat{A}} \vec{x}
              = \vec{0} \}$.
\end{itemize}

The row space $\mathrm{row}(\mat{A})$ is the \textit{orthogonal complement} of the null space
$\mathrm{null}(\mat{A})$, thus $\mathrm{row}(\mat{A}) + \mathrm{null}(\mat{A}) = \R^n$. Similarly,
$\mathrm{col}(\mat{A}) + \mathrm{null}(\transpose{\mat{A}}) = \R^m$.

\begin{marginfigure}
    \centering
    \incfig{matrix-spaces}
    \caption{Illustration of the 4 spaces defined by a matrix $\mat{A}$. It shows the perpendicular spaces.
        Furthermore, it shows that $\mat{A}\vec{x}_r = \vec{b}$ for some $\vec{x}_r \in \mathrm{col}(\mat{A})$. Also,
        if you add a vector from the null space to the row vector, it still maps to the same $\vec{b}$,
        $\mat{A}(\vec{x}_r + \vec{x}_n) = \mat{A}\vec{x}_r + \mat{A}\vec{x}_n = \mat{A}\vec{x}_r = \vec{b}$.}
    \label{fig:matrix-spaces}
\end{marginfigure}

\subsection{Invertible matrices}

A matrix is only invertible if it is a square full-rank matrix, \ie, $r=m=n$.

\begin{properties}
    Let $\mat{A} \in \R^{n\times n}$ be a full-rank matrix and $k\in \R$, then
    \begin{align*}
        \inv{\mat{A}} \mat{A}   & = \mat{A} \inv{\mat{A}} = \mat{I} \\
        \inv{(k \mat{A})}       & = \frac{1}{k} \inv{\mat{A}}       \\
        \det{\inv{\mat{A}}}     & = \frac{1}{\det{\mat{A}}}         \\
        \inv{(\mat{A} \mat{B})} & = \inv{\mat{B}} \inv{\mat{A}}.
    \end{align*}
\end{properties}

The Moore-Penrose inverse (pseudo-inverse) is a generalization of the inverse. It is the solution
to the general least squares problem $\min_{\vec{x}} \|\mat{A}\vec{x} - \vec{b}\|_2$. For full-rank
matrices, the left pseudo-inverse can be computed by \[
    \mat{A}^+ = \inv{(\transpose{\mat{A}} \mat{A})} \transpose{\mat{A}}. \margintag{The left pseudo-inverse can be computed if $r=n$ and has the property $\mat{A}^+ \mat{A} = \mat{I}_{n\times n}$.}
\]
The right pseudo-inverse can be computed as \[
    \mat{A}^+ = \transpose{A} \inv{(\mat{A} \transpose{\mat{A}})}. \margintag{The right pseudo-inverse can be computed if $r=m$ and has the property $\mat{A}\mat{A}^+ = \mat{I}_{m\times m}$.}
\]

\subsection{Trace}

\begin{definition}[Trace]
    The trace of a square matrix $\mat{A}\in \R^{n\times n}$ is the sum of its diagonal, \[
        \tr{\mat{A}} = \sum_{i=1}^{n} a_{ii}.
    \]
\end{definition}

\begin{properties}
    Let $\mat{A},\mat{B},\mat{C}\in \R^{n\times n}, \vec{x},\vec{y}\in \R^n, c,d \in \R$, then
    \begin{align*}
        \tr{c \mat{A} + d \mat{B}}       & = c\cdot \tr{\mat{A}} + d\cdot \tr{\mat{B}} \margintag{Linearity.}                                                                \\
        \tr{\mat{A}}                     & = \tr{\transpose{\mat{A}}}                                                                                                        \\
        \tr{\mat{A}\mat{B}}              & = \tr{\mat{B}\mat{A}}                                                                                                             \\
        \tr{\mat{A}\mat{B}\mat{C}}       & = \tr{\mat{C}\mat{A}\mat{B}} \tr{\mat{B}\mat{C}\mat{A}} \margintag{Cyclic property.}                                              \\
        \tr{\transpose{\mat{A}} \mat{B}} & = \tr{\transpose{\mat{B}}\mat{A}} = \tr{\mat{A}\transpose{\mat{B}}} = \tr{\mat{B}\transpose{\mat{A}}}                             \\
        \transpose{\vec{x}}\vec{y}       & = \tr{\transpose{\vec{x}}\vec{y}} = \tr{\vec{x}\transpose{\vec{y}}}. \margintag{$\vec{x}\transpose{\vec{y}}$ is a rank-1 matrix.}
    \end{align*}
    Furthermore, the trace of a matrix is equal to the sum of the eigenvalues of the matrix.
\end{properties}

\subsection{Orthogonal projection}

\begin{marginfigure}
    \centering
    \incfig{vector-projection}
    \caption{Projection of $\vec{a}$ on $\vec{b}$, denoted $\vec{a}_1$, and the rejection of $\vec{a}$ from $\vec{b}$, denoted $\vec{a}_2$.}
    \label{fig:vector-projection}
\end{marginfigure}

The projection $\vec{a}_1$ of a vector $\vec{a}$ on another vector $\vec{b}$ can be computed as \[
    \vec{a}_1 = \frac{\transpose{\vec{b}}\vec{a}}{\transpose{\vec{b}}\vec{b}} \vec{b}.
\]
The rejection can then be computed as \[
    \vec{a}_2 = \vec{a} - \vec{a}_1.
\]

We can also project a vector $\vec{a}$ onto the column space of a matrix $\mat{B} \in \R^{m\times
        n}$, denoted $\vec{a}_{\mat{B}}$. This can be computed by a matrix multiplication, \[
    \mat{P} = \mat{B}\inv{(\transpose{\mat{B}}\mat{B})} \transpose{\mat{B}},
\]
which is equivalent to the left Moore-Penrose inverse.

\begin{definition}[Projection matrix]
    A square matrix $\mat{P} \in \R^{n\times n}$ is called a projection matrix if it is idempotent, \ie,
    $\mat{P}\mat{P} = \mat{P}$.
\end{definition}

\begin{properties}
    Let $\mat{P}$ be an orthogonal projection matrix, then $\mat{P} = \transpose{\mat{P}}$, \ie, $\mat{P}$
    is symmetric. Furthermore, the eigenvalues of a projection matrix are all ones and zeros, because of idempotency.
\end{properties}

\subsection{Special matrices}

\begin{definition}[Orthogonal matrix]
    An orthogonal matrix is an invertible matrix whose columns $\vec{q}_1,\ldots,\vec{q}_n$ are all
    orthogonal to each other and of unit length, \ie,
    \begin{align*}
        \transpose{\vec{q}_i} \vec{q}_j & = 0 \quad \forall i \neq j \in [n] \\
        \transpose{\vec{q}_i} \vec{q}_i & = 1 \quad \forall i \in [n].
    \end{align*}
\end{definition}

\begin{properties}
    Let $\mat{Q}\in \R^{n\times n}$ be an orthogonal matrix,
    \begin{align*}
        \transpose{\mat{Q}}   & = \inv{\mat{Q}}                         \\
        \ang{\vec{x},\vec{y}} & = \ang{\mat{Q}\vec{x}, \mat{Q}\vec{y}}.
    \end{align*}
\end{properties}

\newpage
\bibliography{main}
\bibliographystyle{plainnat}

\end{document}
