\section{Subgradient methods}

Until now, we have mostly assumed all functions to be differentiable and smooth. However, in
general this is not the case. In machine learning, non-differentiable functions arise everywhere:
\begin{itemize}
    \item Loss functions, such as the Hinge loss, $\max\{ 0, 1-x \}$ (SVM);
    \item Regularization, such as the $\ell_1$-norm (LASSO);
    \item Activation functions, such as ReLU.
\end{itemize}

This motivates the need for a more general notion of the gradient that can be applied to more
functions.

\begin{marginfigure}[3.5cm]
    \centering
    \incfig{subgradient}
    \caption{$\vec{g}$ is a subgradient of $f$ at $\vec{x}$ if the whole graph is above $\vec{x}$'s supporting hyperplane, parametrized by $\vec{g}$.}
    \label{fig:subgradient}
\end{marginfigure}

\begin{definition}[Subgradient]
    $\vec{g} \in \R^d$ is a subgradient of $f$ at $\vec{x}$ if \[
        f(\vec{y}) \geq f(\vec{x}) + \transpose{\vec{g}}(\vec{y} - \vec{x}), \quad \forall \vec{y} \in \dom{f}.
    \]
    We call $\partial f(\vec{x}) \subseteq \R^d$ the subdifferential, which is the set of subgradients
    of $f$ at $\vec{x}$.
\end{definition}

\begin{example}
    Consider $f(x) = |x|$, then $\partial f(0) = [-1, 1]$.
\end{example}

\begin{lemma}
    \label{lem:diff-subdiff}

    If $f$ is differentiable at $\vec{x} \in \dom{f}$, then $\partial f(\vec{x}) \subseteq \{
        \grad{f(\vec{x})}{} \}$.
\end{lemma}

\Cref{lem:diff-subdiff} means that if $f$ is differentiable at $\vec{x}$, then this is either the
only subgradient or there is no subgradient at all. There might not be any subgradient at all in this
case because it might be that the hyperplane is not below the entire function if the function is
non-convex.

\begin{lemma}[Convexity and subgradient]
    Let $f: \dom{f} \to \R$, then
    \begin{itemize}
        \item If $f$ is convex, then $\partial f(\vec{x}) \neq \emptyset$ for all $\vec{x}$ in the relative
              interior of $\dom{f}$ (so not necessarily on the edges);
        \item If $\dom{f}$ is convex and $\partial f(\vec{x}) \neq \emptyset$ for all $\vec{x} \in \dom{f}$, then
              $f$ is convex.
    \end{itemize}
\end{lemma}

\begin{lemma}[Subgradient optimality condition]
    Let $f: \dom{f} \to \R$ and $\vec{x} \in \dom{f}$. If $\vec{0} \in \partial f(\vec{x})$, then $\vec{x}$ is a global minimum.
\end{lemma}

\begin{proof}
    By definition of the subgradient with $\vec{g} = \vec{0} \in \partial f(\vec{x})$ gives \[
        f(\vec{y}) \geq f(\vec{x}) + \transpose{\vec{g}} (\vec{y} - \vec{x}) = f(\vec{x}), \quad \forall \vec{y} \in \dom{f}.
    \]
    Thus, $\vec{x}$ is a global minimum.
\end{proof}

\begin{lemma}[Subgradient calculus]
    We can use the following operations to work with subgradients:
    \begin{itemize}
        \item (Conic combination) Let $h(\vec{x}) \doteq \alpha f(\vec{x}) + \beta g(\vec{x})$ with $\alpha,\beta \geq
                  0$, then \[
                  \partial h(\vec{x}) = \alpha \cdot \partial f(\vec{x}) + \beta \cdot \partial g(\vec{x});
              \]
        \item (Affine transformation) Let $h(\vec{x}) \doteq f(\mat{A} \vec{x} + \vec{b})$, then \[
                  \partial h(\vec{x}) = \transpose{\mat{A}} \partial f(\mat{A} \vec{x} + \vec{b});
              \]
        \item (Pointwise maximum) Let $h(\vec{x}) \doteq \max_{i\in [m]} f_i(\vec{x})$, then \[
                  \partial h(\vec{x}) = \mathrm{conv}(\{ \partial f_i(\vec{x}) \mid f_i(\vec{x}) = h(\vec{x}) \}).
              \]
              Thus, at each point where we transition from one function to another, we get the convex hull of
              subgradients of the functions that transition. At all other points, we take the maximum function's
              subgradient.
    \end{itemize}
\end{lemma}

\subsection{Subgradient method}

In the subgradient method, the general update rule becomes \[
    \vec{x}_{t+1} = \Pi_{\mathcal{X}}(\vec{x}_t - \gamma_t \vec{g}_t), \quad \vec{g}_t \in \partial f(\vec{x}_t).
\]
If $f$ is convex and differentiable, gradient descent and projected gradient descent are special
cases of this update rule, where $\mathcal{X} = \R^d$ and $\mathcal{X} \subset \R^d$, respectively.
However, if $f$ is non-differentiable, we will see that this is technically not a descent method,
because the subgradient is not a descent direction in general.

\begin{lemma}[Subgradient method ``descent'' lemma]
    \label{lem:subgrad-descent}

    If $f$ is convex, then for any optimal solution $\vec{x}^\star$, \[
        \| \vec{x}_{t+1} - \vec{x}^\star \|^2 \leq \| \vec{x}_t - \vec{x}^\star \|^2 - 2 \gamma_t (f(\vec{x}_t) - f(\vec{x}^\star)) + \gamma_t^2 \| \vec{g}_t \|^2.
    \]
\end{lemma}

\begin{proof}
    \begin{align*}
        \| \vec{x}_{t+1} - \vec{x}^\star \|^2 & = \| \Pi_{\mathcal{X}}(\vec{x}_t - \gamma_t \vec{g}_t) - \vec{x}^\star \|^2 \margintag{Subgradient descent update rule.}                                                                                                                 \\
                                              & \leq \| \vec{x}_t - \gamma_t \vec{g}_t - \vec{x}^\star \|^2 \margintag{Projection is non-expansive.}                                                                                                                                     \\
                                              & = \| \vec{x}_t - \vec{x}^\star \|^2 - 2 \gamma_t \transpose{\vec{g}_t} (\vec{x}_t - \vec{x}^\star) + \gamma_t^2 \| \vec{g}_t \|^2 \margintag{Cosine theorem.}                                                                            \\
                                              & \leq \| \vec{x}_t - \vec{x}^\star \|^2 - 2 \gamma_t (f(\vec{x}_t) - f(\vec{x}^\star)) + \gamma_t^2 \| \vec{g}_t \|^2. \margintag{Subgradient: $f(\vec{x}^\star) \geq f(\vec{x}_t) + \transpose{\vec{g}_t} (\vec{x}^\star - \vec{x}_t)$.}
    \end{align*}
\end{proof}

\begin{theorem}[Convergence of the subgradient method]
    If $f$ is convex, then the subgradient method satisfies \[
        \min_{t \in [T]} f(\vec{x}_t) - f(\vec{x}^\star) \leq \frac{\| \vec{x}_0 - \vec{x}^\star \|^2 + \sum_{t=0}^{T-1} \gamma_t^2 \| \vec{g}_t \|^2}{2 \sum_{t=0}^{T-1} \gamma_t}.
    \]
\end{theorem}

\begin{proof}
    By \Cref{lem:subgrad-descent}, we have \[
        \| \vec{x}_{t+1} - \vec{x}^\star \|^2 \leq \| \vec{x}_t - \vec{x}^\star \|^2 - 2 \gamma_t (f(\vec{x}_t) - f(\vec{x}^\star)) + \gamma_t^2 \| \vec{g}_t \|^2.
    \]
    Rearranging yields \[
        \gamma_t (f(\vec{x}_t) - f(\vec{x}^\star)) \leq \frac{1}{2} \lft( \| \vec{x}_t - \vec{x}^\star \| - \| \vec{x}_{t+1} - \vec{x}^\star \| + \gamma_t^2 \| \vec{g}_t \|^2 \rgt).
    \]
    Summing over all timesteps and dividing by $\sum_{t=0}^{T-1} \gamma_t$ yields
    \begin{align*}
        \min_{t \in [T]} f(\vec{x}_t) - f(\vec{x}^\star) & \leq \frac{\sum_{t=0}^{T-1} \gamma_t (f(\vec{x}_t) - f(\vec{x}^\star))}{\sum_{t=0}^{T-1} \gamma_t}                                                             \\
                                                         & \leq \frac{\| \vec{x}_0 - \vec{x}^\star \|^2 - \| \vec{x}_T - \vec{x}^\star \|^2 + \sum_{t=0}^{T-1} \gamma_t^2 \| \vec{g}_t \|^2}{2 \sum_{t=0}^{T-1} \gamma_t} \\
                                                         & \leq \frac{\| \vec{x}_0 - \vec{x}^\star \|^2 + \sum_{t=0}^{T-1} \gamma_t^2 \| \vec{g}_t \|^2}{2 \sum_{t=0}^{T-1} \gamma_t}.
    \end{align*}
\end{proof}

Assuming bounded subgradient $\| \vec{g}_t \| \leq B$ for all steps $t$, we get the following
convergence rates under various stepsizes,
\begin{itemize}
    \item Constant stepsize ($\gamma_t = \gamma$): \[
              \lim_{t\to \infty} f(\vec{x}_t^{\mathrm{best}}) \leq f(\vec{x}^\star) + \frac{B^2 \gamma}{2};
          \]
    \item Scaled stepsize ($\gamma_t = \nicefrac{\gamma}{\| \vec{g}_t \|}$): \[
              \lim_{t \to \infty} f(\vec{x}_t^{\mathrm{best}}) \leq f(\vec{x}^\star) + \frac{B \gamma}{2};
          \]
    \item Square-summable stepsize ($\sum_{t=0}^{\infty} \gamma_t^2 < +\infty$, $\sum_{t=0}^{\infty} \gamma_t
              = +\infty$): \[
              \lim_{t \to \infty} f(\vec{x}_t^{\mathrm{best}}) = f(\vec{x}^\star);
          \]
    \item Diminishing stepsize ($\gamma_t \to 0$ and $\sum_{t=0}^{\infty} \gamma_t = +\infty$): \[
              \lim_{t \to \infty} f(\vec{x}_t^{\mathrm{best}}) = f(\vec{x}^\star).
          \]
\end{itemize}

\begin{corollary}
    Let $f$ be convex and $B$-Lipschitz continuous. Let $\mathcal{X}$ be convex compact with $R^2 = \max_{\vec{x},\vec{y} \in \mathcal{X}} \| \vec{x} - \vec{y} \|_2 < +\infty$. Setting \[
        \gamma \doteq \frac{R}{B \sqrt{T}},
    \]
    then the subgradient method satisfies \[
        \min_{t \in [T]} f(\vec{x}_t) - f(\vec{x}^\star) \leq \frac{BR}{\sqrt{T}}.
    \]
\end{corollary}

To achieve $\epsilon$-optimality, the subgradient method requires
$\bigo{\frac{B^2R^2}{\epsilon^2}}$ iterations.

\subsection{Strong convexity}

\begin{theorem}
    Let $f$ be $\mu$-strongly convex and $B$-Lipschitz continuous on $\mathcal{X}$. Setting \[
        \gamma_t \doteq \frac{2}{\mu (t+1)},
    \]
    then the subgradient method satisfies \[
        \min_{t \in [T]} f(\vec{x}_t) - f(\vec{x}^\star) \leq \frac{2B^2}{\mu (T+1)}.
    \]
\end{theorem}

\begin{proof}
    Adapting the proof of \Cref{lem:subgrad-descent} to use strong convexity in its last step, we
    get \[
        \| \vec{x}_{t+1} - \vec{x}^\star \|^2 \leq (1-\mu \gamma_t) \| \vec{x}_t - \vec{x}^\star \|^2 - 2 \gamma_t (f(\vec{x}_t) - f(\vec{x}^\star)) + \gamma_t^2 \| \vec{g}_t \|^2.
    \]
    Using this, we get the following,
    \begin{align*}
        f(\vec{x}_t) - f(\vec{x}^\star) & \leq \frac{1 - \mu \gamma_t}{2 \gamma_t} \|\vec{x}_t - \vec{x}^\star\|^2 - \frac{1}{2 \gamma_t} \|\vec{x}_{t+1} - \vec{x}^\star \|^2 + \frac{\gamma_t}{2} \|\vec{g}_t\|^2 \\
                                        & = \frac{\mu(t-1)}{4} \| \vec{x}_t - \vec{x}^\star \|^2 - \frac{\mu(t+1)}{4} \|\vec{x}_{t+1} - \vec{x}^\star\|^2                                                           \\
                                        & \qquad + \frac{1}{\mu(t+1)} \| \vec{g}_t \|^2 \margintag{$\gamma_t \doteq \nicefrac{2}{\mu(t+1)}$}.
    \end{align*}
    Now, it is easy to show the result by a telescoping sum.
\end{proof}

Hence, in the case of strong convexity, in order to achieve $\epsilon$-optimality, the subgradient
method requires $\bigo{\frac{B^2}{\mu \epsilon}}$ iterations.
