% Obscure delimiters
\DeclarePairedDelimiter{\floor}{\lfloor}{\rfloor}
\DeclarePairedDelimiter{\ceil}{\lceil}{\rceil}
\DeclarePairedDelimiter{\ip}{\langle}{\rangle}
\DeclarePairedDelimiter{\abs}{\lvert}{\rvert}
\DeclarePairedDelimiter{\norm}{\lVert}{\rVert}

% Better left and right
\newcommand{\lft}{\mathopen{}\left}
\newcommand{\rgt}{\aftergroup\mathclose\aftergroup{\aftergroup}\right}

% Common sets
\newcommand{\R}{\mathbb{R}}
\newcommand{\Q}{\mathbb{Q}}
\newcommand{\N}{\mathbb{N}}
\newcommand{\Z}{\mathbb{Z}}

% Expected value and variance
\newcommand{\E}{\mathbb{E}}
\newcommand{\Var}{\mathrm{Var}}

% Input and output space
\newcommand{\X}{\mathcal{X}}
\newcommand{\Y}{\mathcal{Y}}

% Vector, matrix, and tensor
\renewcommand{\vec}[1]{\bm{#1}}
\newcommand{\mat}[1]{\bm{#1}}
\newcommand{\tens}[1]{\bm{\mathsf{#1}}}

% Big-O notation
\newcommand{\bigo}[1]{\mathcal{O}\lft(#1\rgt)}

% Common functions
\newcommand{\transpose}[1]{#1^\top}
\newcommand{\inv}[1]{#1^{-1}}
\renewcommand{\det}[1]{\mathrm{det}\lft(#1\rgt)}
\newcommand{\tr}[1]{\mathrm{tr}\lft(#1\rgt)}
\newcommand{\diag}[1]{\mathrm{diag}\lft(#1\rgt)}

% Distributions
\newcommand{\sample}{\sim}
\newcommand{\sampleiid}{\overset{\mathrm{iid}}{\sim}}
\NewDocumentCommand{\Gauss}{O{0}O{1}o}{\mathcal{N}\lft(#1,#2 \IfValueT{#3}{\mid #3}\rgt)}
\newcommand{\MultiGauss}{\Gauss[\vec{0}][\mat{I}]}
\NewDocumentCommand{\Unif}{m}{\mathrm{Unif}\parentheses*{#1}}

% Math comment in smaller font size
\newcommand{\mathcomment}[1]{\mathrm{\footnotesize #1}}

% Notation environment if necessary
\newenvironment{notation}
  {\begin{fullwidth}
   \begin{longtable}{ll} \toprule
    \multicolumn{2}{c}{\centering \smallcaps{Summary of notation}} \\
    \midrule
    \vspace{0.5ex} \\
  }
  {\vspace{0.5ex} \\
   \bottomrule
   \end{longtable}
   \end{fullwidth}
  }

% Add a period to the end of an abbreviation unless there's one
% already, then \xspace
\makeatletter
\DeclareRobustCommand\onedot{\futurelet\@let@token\@onedot}
\def\@onedot{\ifx\@let@token.\else.\null\fi\xspace}

\def\eg{\emph{e.g}\onedot} \def\Eg{\emph{E.g}\onedot}
\def\ie{\emph{i.e}\onedot} \def\Ie{\emph{I.e}\onedot}
\def\cf{\emph{c.f}\onedot} \def\Cf{\emph{C.f}\onedot}
\def\etc{\emph{etc}\onedot} \def\vs{\emph{vs}\onedot}
\def\wrt{w.r.t\onedot} \def\dof{d.o.f\onedot}
\makeatother

% Argument optima
\DeclareMathOperator*{\argmax}{argmax} % no space, limits underneath in displays
\DeclareMathOperator*{\argmin}{argmin} % no space, limits underneath in displays
