\section{Bayesian optimization} \label{sec:bo}

\begin{marginfigure}
    \centering
    \incfig{bayesian-optimization}
    \caption{Illustration of Bayesian optimization. We pass an input $\vec{x}_t$
    into the unknown function $f^\star$ to obtain a noisy observation $y_t$.}
    \label{fig:bo}
\end{marginfigure}

In \textit{Bayesian optimization}, we do not only want to reduce uncertainty,
but we also want to maximize some objective.\sidenote{This is in contrast with
active learning. An example of this is automatically tuning the hyperparameters
of a machine learning model. In this scenario, we would both want to minimize
the uncertainty of the parameter space and maximize the performance of the
eventual model.} This means that we have to make a trade-off between
exploration (minimizing uncertainty) and exploitation (maximizing performance).

\begin{definition}[Regret]
  The cumulative regret for a time horizon $T$ associated with choices
  $\vec{x}_1,\ldots,\vec{x}_T$ is defined as \[
    R_T \doteq \sum_{t=1}^T \underbrace{\lft(\max_{\vec{x}} f(\vec{x}) - f(\vec{x}_t)\rgt)}_{\text{``instantaneous regret``}}
  .\]
  The regret can be interpreted as the additive loss with respect to the
  maximally achievable value $\max_{\vec{x}} f(\vec{x})$.
\end{definition}

The goal is to find algorithms that achieve a sublinear regret, \[
  \lim_{T\to\infty} \frac{R_T}{T} = 0
.\]
This leads to the algorithm converging on the maximum $\max_{\vec{x}}
f(\vec{x})$. Note that using an algorithm that explores forever will result in
the regret growing linearly, because we will never settle on a maximum value.
In contrast, if we use an algorithm that never explores and thus only
exploits, we might never find $\max_{\vec{x}} f(\vec{x})$. Thus, achieving
sublinear regret requires balancing exploration and exploitation.

\subsection{Acquisition functions}

\paragraph{Upper confidence bound.}

The principle of \textit{optimism in the face of uncertainty} naturally
suggests picking the point where we can hope for the optimal outcome. This
corresponds to maximizing the \textit{upper confidence bound}
(UCB),\sidenote{This assumes that the parameters of the model (\eg GP) are
known. However, this is not the case in practice. Thus, in practice, we can
alternate between learning the hyperparameters from the currently selected
data, and selecting the next datapoint. However, this introduces a danger of
overfitting. This can be solved by either placing a hyperprior on the
hyperparameters, or occasionally selecting some points at random.} \[
  \vec{x}_t = \argmax_{\vec{x}\in\mathcal{X}} \mu_{t-1}(\vec{x}) + \beta_t \sigma_{t-1}(\vec{x})
,\]
where $\beta_t$ regulates how confident we are in our current model. A high
$\beta_t$ leads the model to explore, while a low $\beta_t$ leads the model
to exploit. This acquisition function naturally trades exploitation by
preferring a large posterior mean with exploration by preferring a large
posterior variance.

If $f$ can be represented by our model and we choose $\beta_t$ ``correctly``, \[
  R_T \in \mathcal{O}^\star \lft( \sqrt{\frac{\gamma_T}{T}} \rgt)
,\]
where \[
  \gamma_T = \max_{|S|\leq T} I(f;\vec{y}_S)
,\]
quantifies the maximum information gain. Thus, the maximum information gain
$\gamma_T$ determines the regret of the UCB algorithm.

\begin{margintable}
  \centering
  \begin{tabular}{ll}
    \toprule
    \textbf{Kernel} & \textbf{$\gamma_T$ bound} \\
    \midrule
    Linear & $\bigo{d\log T}$ \\
    Gaussian & $\bigo{(\log T)^{d+1}}$ \\
    Mat\'ern ($\nu > \nicefrac{1}{2}$) & $\bigo{T^{\frac{d}{2\nu+d}} (\log T)^{\frac{2\nu}{2\nu+d}}}$ \\
    \bottomrule
  \end{tabular}

  \caption{Information gain bounds of common Gaussian process kernels. These
  guarantee sublinear regret, which means that they are guaranteed to converge
  to the maximum value of the function.}
  \label{tab:information-gain-bounds}
\end{margintable}

\paragraph{Thompson sampling.}

At every iteration $t$, Thompson sampling draws a function realization from the
Gaussian process, \[
  \tilde{f}_t \sim p(f\mid \vec{x}_{1:t},y_{1:t})
,\] and selects \[
  \vec{x}_{t+1} = \argmax_{\vec{x}\in\mathcal{X}} \tilde{f}_t(\vec{x})
.\]
The randomness in the realization of $\tilde{f}_t$ is sufficient to trade
exploration and exploitation. Similarly to UCB, it has sublinear regret bounds.
