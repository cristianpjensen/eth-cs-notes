\section{Camera model} \label{sec:camera-model}

The pinhole camera is a mathematical model of how perspective cameras work.
It assumes that the pinhole is infinitely small, so it only lets one lightray
through per point in the space.\sidenote[][-1.5cm]{If the pinhole camera is not
infinitely small, the image would become blurry. However, in reality, an
infinitely small pinhole is impossible and a very small pinhole would not let
enough light through to see anything. The solution to this are lenses that
map all light coming from a point to its corresponding point in the image.}

\subsection{Internal camera matrix}

\begin{figure}[ht]
    \centering
    \incfig{internal-camera}
    \caption{Illustration of how the intrinsic camera matrix works.}
    \label{fig:internal-camera}
\end{figure}

The coordinate frame of the pinhole camera is chosen such that the origin is in
the center of the pinhole. The image plane behind the camera is in the plane
where $z_c=-f$, where $f$ is the focal distance. The $z$-axis is called the
principal axis and goes in the direction where the camera is pointing toward. If
we know a three-dimensional point $\vec{X}_c$, then it is possible to calculate
the image coordinate $\vec{x}$ of that point projected onto the image plane: \[
  x=-f\frac{x_c}{z_c}, \;\;\; y=-f\frac{y_c}{z_c}
.\]
The minus-sign indicates that the projected image is upside down. To get rid of
this mirroring, we use a virtual pinhole camera in which the image plane is put
at $z_c=f$, \ie, in front of the pinhole. In this model, we have \[
  x=f\frac{x_c}{z_c}, \;\;\; y=f\frac{y_c}{z_c}
.\]
Using homogeneous coordinates, we can model this with the following, \[
  \tilde{\vec{x}} = \begin{bmatrix} f_x & s & c_x & 0 \\ 0 & f_y & c_y & 0 \\ 0
  & 0 & 1 & 0 \end{bmatrix}\tilde{\vec{X}}_c = \bm{K}\tilde{\vec{X}}_c
.\]
In practice, we often set $f_x=f_y$ and $s=0$. The translation $\bm{c}$ moves
the origin of the image from the middle to the top left of the image plane,
which makes computation easier.\sidenote{Computation becomes easier then,
because most libraries, such as \texttt{NumPy} and \texttt{PyTorch}, start the
coordinates from the top left.}

\subsection{External camera matrix}

\begin{figure}[ht]
    \centering
    \incfig{pinhole-in-space}
    \caption{$\mat{F}$ is the rigid transformation that the camera made within
    the world space. To convert world coordinates such that the camera is the
    origin, we transform them by $\inv{\mat{F}}$.}
    \label{fig:pinhole-in-space}
\end{figure}

The camera will generally not be in the origin of the three-dimensional world
space. So, we need the transformation that moved the camera from the origin
to its point. Let \[
  \mat{F}=\begin{bmatrix} \mat{R} & \vec{t} \\ \transpose{\vec{0}} & 1 \end{bmatrix}
\]
be this transformation, then the projection of a point $\vec{X}_w$ in
world space onto the image plane is given by the following,\sidenote{We take the inverse of
$\mat{F}$, because we want to move the origin of the points to the origin of
the camera.} \[
  \tilde{\vec{x}} = \mat{K}\inv{\mat{F}}\overline{\vec{X}}_w
.\]
We denote this matrix by $\mat{P} = \mat{K}\inv{\mat{F}}$,\sidenote{$\mat{P}$
has 11 degrees of freedom.} and can be estimated by DLT with 6
$(\vec{X}_w,\vec{x})$ correspondences. Again, DLT only minimizes the algebraic
error, which we can then perfect by minimizing a geometric error.
