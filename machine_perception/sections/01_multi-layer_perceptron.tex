\section{Deep learning basics}

\subsection{Multi-layer perceptron}

\begin{marginfigure}
    \centering
    \incfig{perceptron}
    \caption{Computation graph of a perceptron \citep{rosenblatt1958perceptron}, where $\sigma(x) = \mathbb{1}\{ x > 0 \}$.}
    \label{fig:perceptron}
\end{marginfigure}

The original perceptron \citep{rosenblatt1958perceptron} was a single layer perceptron with the
following non-linearity, \[
    \sigma(x) \doteq \mathbb{1}\{ x > 0 \}.
\]
The classification of a single point can then be written as \[
    \hat{y} = \mathbb{1}\{\transpose{\vec{w}}\vec{x} > 0\}.
\]
The learning algorithm then iteratively updates the weights for a data point that was classified
incorrectly, \[
    \vec{\theta} \gets \vec{\theta} + \eta \underbrace{(y_i - \hat{y}_i)}_{\text{residual}} \vec{x}_i,
\]
where $\eta$ is the learning rate. If the data is linearly separable, the perceptron converges in
finite time.

\begin{marginfigure}
    \centering
    \incfig{xor-problem}
    \caption{XOR problem. As can be seen, the data is not linearly separable, and thus not solvable by the perceptron.}
    \label{fig:xor-problem}
\end{marginfigure}

\begin{marginfigure}
    \centering
    \incfig{mlp}
    \caption{Example multi-layer perceptron architecture.}
    \label{fig:mlp}
\end{marginfigure}

The problem with the single-layer perceptron was that it could not solve the XOR problem; see
\Cref{fig:xor-problem}. This can be solved by introducing hidden layers, \[
    \hat{y} = \sigma(\mat{W}_k \sigma(\mat{W}_{k-1} \cdots \sigma(\mat{W}_1 \vec{x}))).
\]
We call this architecture a multi-layer perceptron (MLP); see \Cref{fig:mlp}. We then want to
estimate the parameters $\vec{\theta} = \{ \mat{W}_1,\ldots,\mat{W}_k,\vec{b}_1,\ldots,\vec{b}_k
    \}$, using an optimization algorithm such as gradient descent, which we call ``learning``.

\subsection{Loss functions}

We need an objective to optimize for. We typically call this objective function the loss function,
which we minimize. In classification, we typically optimize the maximum likelihood estimate (MLE),
\begin{align*}
    \argmax_{\vec{\theta}} p(\mathcal{D}\mid \vec{\theta}) & \overset{\mathrm{iid}}{=} \argmax_{\vec{\theta}} \prod_{i=1}^n p(y_i \mid \vec{\theta}) \\
                                                           & = \argmin_{\vec{\theta}} -\log \prod_{i=1}^n p(y_i\mid\vec{\theta})                     \\
                                                           & = \argmin_{\vec{\theta}} -\sum_{i=1}^n \log p(y_i\mid\vec{\theta}).
\end{align*}
If the model predicts the parameters of a Bernoulli distribution, MLE is equivalent to binary cross-entropy,
\begin{align*}
    \ell(\vec{\theta}) & = -\log \mathrm{Ber}(y_i \mid \hat{y}_i \doteq f(\vec{x}_i\mid\vec{\theta})) \margintag{$f$ is a model that outputs the Bernoulli parameter.} \\
                       & = -\log \hat{y}_i^{y_i} (1-\hat{y}_i)^{1-y_i}                                                                                                 \\
                       & = -y_i \log \hat{y}_i - (1-y_i) \log (1-\hat{y}_i).
\end{align*}
If we the choose the model to be Gaussian, we end up minimizing the mean-squared error. Furthermore,
the Laplacian distribution yields minimizing the $\ell_1$ norm.

If we have prior information about the weights, we could also optimize for the maximum a posteriori
(MAP),
\begin{align*}
    \argmax_{\vec{\theta}} p(\vec{\theta}\mid \mathcal{D}) & = \argmax_{\vec{\theta}} p(\vec{\theta}) p(\mathcal{D}\mid\vec{\theta})                                               \\
                                                           & \overset{\mathrm{iid}}{=} \argmax_{\vec{\theta}} p(\vec{\theta}) \prod_{i=1}^n p(y_i\mid\vec{\theta}) p(\vec{\theta}) \\
                                                           & = \argmin_{\vec{\theta}} -\log \lft( p(\vec{\theta}) \prod_{i=1}^n p(y_i\mid\vec{\theta}) \rgt)                       \\
                                                           & = \argmin_{\vec{\theta}} - \log p(\vec{\theta}) - \sum_{i=1}^{n} \log p(y_i\mid \vec{\theta})                         \\
\end{align*}
Note that MAP and MLE are equivalent if $p(\vec{\theta})$ is uniform over the domain of weights.
Assuming a Gaussian prior distribution over $\vec{\theta}$, MAP yields Ridge regression.

\subsection{Backpropagation}

Typically, we cannot find the optimal parameters $\vec{\theta}^\star$ in closed form, so we must
use an optimization algorithm. Optimization algorithms, such as gradient descent, typically require
computing the gradient \wrt the parameters. Backpropagation is an algorithm for computing the
gradient of any function, given that we have access to the derivatives of the primitive functions
that make up the function.\sidenote{For example, to compute the gradient of $f(\vec{x},\vec{y}) =
        \sigma(\transpose{\vec{x}}\vec{y})$, we would need access to $\odv*{\sigma(x)}{x}$,
    $\pdv*{\transpose{\vec{x}}\vec{y}}{\vec{x}}$, and $\pdv*{\transpose{\vec{x}}\vec{y}}{\vec{y}}$.} It
then computes the gradient by making use of dynamic programming, the chain rule, and sum rule.

Gradient descent iteratively updates the parameters by the following, \[
    \vec{\theta} \gets \vec{\theta} - \eta \grad{\mathcal{L}(\hat{\vec{y}},\vec{y})}{\vec{\theta}},
\]
until the gradient is small.

\subsection{Activation functions}

In MLPs, the activation function should be non-linear, or the resulting MLP is just an affine
mapping with extra steps. This is because the product of affine mappings are themselves affine
mappings.

\subsection{Universal approximation theorem}

\begin{theorem}[Universal approximation theorem]
    Let $\sigma: \R\to \R$ be a non-constant, bounded, and continuous activation function.
    Let $I_m$ denote the $m$-dimensional unit hypercube $[0,1]^m$ and the space of real-valued
    function on $I_m$ is denoted by $\mathcal{C}(I_m)$.

    Let $f \in \mathcal{C}(I_m)$ be any function in the hypercube. Let $\epsilon > 0, N\in \mathbb{N},
        v_i,b_i \in \R, \vec{w}_i \in \R^m$ for $i=1,\ldots,N$, then \[
        f(\vec{x}) \approx g(\vec{x}) = \sum_{i=1}^{N} v_i \sigma(\transpose{\vec{w}_i} \vec{x} + b_i),
    \]
    where $|f(\vec{x}) - g(\vec{x})| < \epsilon$ for all $\vec{x}\in I_m$.
\end{theorem}

The universal approximation theorem holds for any single hidden layer network. However, this hidden
layer may need to have infinite width to approximate $f$. In practice, deeper networks work better
than wider networks.
