\section{Part-of-speech tagging} \label{sec:pos-tagging}

\begin{figure}[ht]
    \centering
    \incfig{pos-graph}
    \caption{Example POS graph with $\mathcal{T} = \{ \cat{N}, \cat{V},
    \cat{Det} \}$. For inference, we want to find the best $N$-length path
    within this graph. For training, we want to compute the sum over all
    $N$-length paths within this graph. The dashed edges are the backpointers to
    the argmaxes, and the thick line is the best path.}
    \label{fig:pos-graph}
\end{figure}

In \textit{part-of-speech tagging} (POS tagging), we want to predict a POS tag
$t\in\mathcal{T}$ for every word of the input sentence $\vec{w}$ of length $N$.
In other words, given an $N$-dimensional input sequence of words $\vec{w} \in
\Sigma^N$, we want to output an $N$-dimensional sequence of tags $\vec{t} \in
\mathcal{T}^N$. This can be seen as searching through a POS graph as in
\Cref{fig:pos-graph}. The output space $\mathcal{T}^N$ is exponential, so we
need to design an algorithm to efficiently compute the normalizer $Z(\vec{w})$,
and find the maximum scoring tagging.

\subsection{Conditional random fields}

\textit{Conditional random fields} (CRF) are a conditional probabilistic model
for structured labelling. Whereas a classifier predicts a label for a single
sample without considering neighboring samples in the structure, a CRF does
take context into account. In other words, CRFs are a model for computing the
normalizer $Z$ in a structured labelling case.

In the sequence labelling case of POS tagging, we will assume that tags only
depend on their immediate neighbors, \[
  \score(\vec{t},\vec{w}) = \sum_{n=1}^N \score(\langle t_{n-1},t_n \rangle, \vec{w}, n)
,\]
which can be further decomposed into transition and emission scores, \[
  \score(\langle t_{n-1},t_n \rangle, \vec{w}, n) = \mathrm{transition}(\langle t_{n-1},t_n \rangle) + \mathrm{emission}(w_n, t_n)
.\]
This balances how likely $t_n$ is to follow $t_{n-1}$ and how likely word $w_n$
is to be assigned tag $t_n$.

Note that our bigram assumption does not mean that the current tag only depends
on the previous and current word, because the word representations can be
anything. \Eg, if we use a bidirectional RNN for the word representations, the
tags will still depend on the entire input sentence. However, a problem that
this can cause is that it cannot correctly tag garden-path sentences, since we
cannot change the start of the tagging after seeing the end of the
tagging.\sidenote{An example garden-path sentence is ``The horse raced past the
barn fell.``}

We can use the new decomposed scoring function to find an efficient algorithm
for computing the normalizer under a semiring,
\begin{align*}
  &\bigoplus_{\vec{t}\in\mathcal{T}^N} \bigotimes_{n=1}^N \exp \score(\langle t_{n-1},t_n \rangle, \vec{w}, n) \\
  &= \bigoplus_{t_1\in\mathcal{T}} \cdots \bigoplus_{t_n\in\mathcal{T}} \exp \score(\langle t_0,t_1 \rangle, \vec{w}, 1) \otimes \cdots \otimes \exp \score(\langle t_{N-1},t_N \rangle, \vec{w}, N) \\
  &= \bigoplus_{t_1\in\mathcal{T}} \exp \score(\langle t_0,t_1 \rangle, \vec{w}, 1) \otimes \lft( \cdots \otimes \lft( \bigoplus_{t_n\in\mathcal{T}} \exp \score(\langle t_{N-1},t_N \rangle, \vec{w}, N) \rgt) \rgt)
,\end{align*}
where distributivity is used in the last step. The backward and forward
algorithms (\Cref{alg:backward-algorithm,alg:forward-algorithm}) are direct results of
this rederivation, which compute the normalizer in $\bigo{N \cdot
|\mathcal{T}|^2}$.

\begin{algorithm}
  \caption{Backward algorithm that computes the semiring-sum over all taggings
  of a sentence $\vec{w}$.}
  \label{alg:backward-algorithm}

  \begin{algorithmic}[1]
    \Function{BackwardAlgorithm}{$\vec{w}, \score, \langle A,\oplus,\otimes,\vec{0},\vec{1} \rangle$}
      \State $\beta[N] \gets \bm{1}$

      \For {$n=N-1, \ldots, 0$}
        \For {$t_{n}\in\mathcal{T}$}
          \State $\beta[n,t_n] \gets \bigoplus_{t_{n+1}\in\mathcal{T}}
          \exp(\text{score}(\langle t_n,t_{n+1} \rangle), \vec{w}, n+1) \otimes
          \beta[n+1,t_{n+1}]$
        \EndFor
      \EndFor

      \State \Return $\beta[0, \smallcaps{BOT}]$
    \EndFunction
  \end{algorithmic}
\end{algorithm}

\begin{algorithm}
  \caption{Forward algorithm that computes the same thing as
  \Cref{alg:backward-algorithm}, but then in a different fashion. This version
  is more intuitive.}
  \label{alg:forward-algorithm}

  \begin{algorithmic}[1]
    \Function{ForwardAlgorithm}{$\vec{w}, \score, \langle A,\oplus,\otimes,\vec{0},\vec{1} \rangle$}
      \State $\alpha[0] \gets \bm{1}$

      \For {$n=1,\ldots N$}
        \For {$t_n\in\mathcal{T}$}
          \State $\alpha[n,t_n] \gets \bigoplus_{t_{n-1}\in\mathcal{T}} \exp(\text{score}(\langle t_{n-1},t_n \rangle), \vec{w}, n) \otimes \alpha[n-1,t_{n-1}]$
        \EndFor
      \EndFor

      \State \Return $\alpha[N, \smallcaps{EOT}]$
    \EndFunction
  \end{algorithmic}
\end{algorithm}

Under the real semiring, we can thus use these algorithms to compute the
normalizer $Z(\vec{w})$ during training. For inference, the Viterbi algorithm
is a version of the backward algorithm under the Viterbi semiring, where
backpointers to the argmax are kept such that the maximally scoring tagging can
be constructed by backtracking the backpointer.
